\documentclass{vcscores}

% Title-page text
\title  {Villancicos about Music from Seventeenth-Century Spain and New Spain}
\author {Edited by Andrew A. Cashner}
\date   {\WLSCMimprint{32}}

\renewcommand{\copyrightNotice}{%
    Copyright © \the\year\ Andrew A. Cashner
}

\renewcommand{\licenseNotice}{%
    Users may download editions, print them for personal use, and perform them
    in non-profit settings, provided proper acknowledgement is given to both the
    editor and to the Society for Seventeenth-Century Music. 
    Permission for performance in professional (for profit) settings must be
    negotiated directly between the performers or their agents and the editor.

    The editor remains the owner of all rights to the edition.

    \includegraphics{img/CC-by-nc-nd}

    This work is licensed under a Creative Commons
    Attribution-NonCommercial-NoDerivs 3.0 Unported License
    (\url{https://creativecommons.org/licenses/by-nc-nd/3.0/}).
}

\renewcommand{\headvolumeinfo}{%
    \worktitle{Villancicos about Music}, ed. Cashner, 2017
} 
\renewcommand{\WLSCMbyline}{%
    WEB LIBRARY OF SEVENTEENTH-CENTURY MUSIC 
    (\url{www.sscm-wlscm.org}), WLSCM no.\ 32%
}
\pagestyle{fancy}

% PDF metadate
\hypersetup{%
    pdftitle={%
        Villancicos about Music from Seventeenth-Century Spain and New Spain
    }, 
    pdfauthor={%
        Edited by Andrew A. Cashner
    }, 
    pdfkeywords={%
        villancico, Spain, New Spain, 17th century, religious music, devotional
        music, liturgical music, Christmas, Corpus Christi, Eucharist, Juan
        Gutiérrez de Padilla, Joan Cererols, Miguel de Irízar, Jerónimo de
        Carrión, José de Cáseda, Antonio de Salazar, Manuel de León Marchante,
        Madrid, Montserrat, Segovia, Puebla, Mexico City
    }, 
    pdfsubject={%
        Villancicos; Part-songs, Spanish, 1600-1700; Lyric poetry, Spanish,
        1600-1700; Spanish music; Latin American music
    }%
}
% XXX check Library of Congress subject headings

\hyphenation{
   Mont-se-rrat
   Na-cio-nal
   In-ves-ti-ga-ción
   Do-cu-men-ta-ción
   Chá-vez
   Gu-ti-érr-ez
   Pa-di-lla
}

\begin{document}

\maketitle

\frontmatter
\copyrightpage % XXX check with editor
\tableofcontents

\mainmatter % XXX check editor
% Each of these is a chapter
\edchapter{Preface}
\section{Acknowledgments}

This edition was produced with free and open-source software on the Debian 
GNU/Linux operating system, release 8.
The text was typeset with the \LaTeX{} document-preparation system, based on 
the work of Donald Knuth and Leslie Lamport.
The music was typeset with Lilypond, version 2.19.
I am grateful to the hundreds of volunteers who build and maintain these 
systems, and to those who provided specific help in programming this document.

The text typeface is EB Garamond, designed by Georg Duffner, based on 1592 type 
specimens by Claude Garamont.
The Spanish CZ metrical symbol used in the scores was traced manually in 
Inkscape from a villancico manuscript by Miguel de Irízar.

I am grateful to the following people and institutions for access to the 
primary sources on which these editions are based: 
the capitular archive of the Cathedral of Puebla de los Ángeles (P. Francisco 
Vázquez, rector; the Illmo. Sr. Carlos Ordaz, \term{canónigo archivista});
CENIDIM, the Mexican national center for music research in Mexico City;
the Biblioteca de Catalunya in Barcelona;
the parochial archive of the Church of Saints Peter and Paul, Canet de Mar, in 
the Archdiocese of Girona, Barcelona province;
the capitular archive of Segovia Cathedral (P. Bonifacio Bartolomé);
the Biblioteca Nacional in Madrid, and
the British Library in London.

Travel for archival research in Mexico and Spain was funded by 
\begin{anonymize}
    a Jacob K. Javits Fellowship from the United States Department of Education, 
    a Pre-Dissertation Research Fellowship from the Council for European Studies at 
    Columbia University, 
    a Eugene K. Wolf Research Travel Grant from the American Musicological Society, 
    and grants from the Department of Music and the Center for Latin American 
    Studies at the University of Chicago.
    Other research funding was provided by a Dissertation Completion Fellowship 
    from the American Council of Learned Societies and the Mellon Foundation.

    % Thank editor and reviewers XXX
    I thank my doctoral advisor, Robert Kendrick, and the other members of my 
    dissertation committee, Anne Walters Robertson, Martha Feldman, Frederick de 
    Armas, and María Gembero-Ustárroz.
    I am also grateful to the following people for specific help with the poetry 
    and music in this edition:
    Stephen Black,
    Anita Damjanovic, 
    Miguel Martínez, 
    Gustavo Mauleón Rodríguez,
    James Nemiroff, and
    Martha Tenorio.
    My wife Ann makes all of this possible and my children Ben and Joy make it
    worthwhile.
\end{anonymize}


% % Introduction
At the height of the Spanish Empire in the seventeenth century,
villancicos were one of the most widespread forms of religious expression and a 
central part of social life.%
  \begin{Footnote}
      For an introduction to the genre, see the entries for \quoted{villancico} in 
      \worktitle{Grove Music Online} and the \worktitle{Diccionario de la música 
      española e hispanoamericana}; \autocites{Laird:VC}{Knighton-Torrente:VCs};
      and the other studies and music editions listed in the bibliography below.
  \end{Footnote}
These poems in vernacular languages (usually Spanish or Portuguese) celebrated 
common Catholic beliefs, popular customs, and modes of devotion through 
metaphorical conceits both earthy and ingenious.
A few villancico poets are known by name from published collections, like
Vicente Sánchez (author of \worktitle{Qué música divina}, in this edition),
Manuel de León Marchante, and Sor Juana Inés de la Cruz; but most villancico
poems are anonymous, and in many cases were probably adapted and reworked by the
composers themselves.
In the form of printed leaflets of the poetry (\term{pliegos sueltos}) and in 
manuscript performing parts of the music they were passed from hand to hand 
across oceans in a global network of affiliated musicians and members of the 
literate elite.%
  \begin{Footnote}
      \Autocite{BNE:VCs17C}; see the bibliography for other catalogs and
      editions of villancico poetry.
  \end{Footnote}

The musical settings of these poems occupied the energies of every major 
chapelmaster and his ensemble on all the highest feast days of the year.
Though the genre originated as a form of courtly entertainment, by the 
beginning of the seventeenth century most villancicos were sacred both in their 
themes and in the venues and occasions of their performance.
Sacred villancicos were often composed in sets of eight so that they could be 
interspersed after or in place of the Responsory chants of the Matins liturgy, 
especially at Christmas and Corpus Christi.
Villancicos were also performed in church for Mass, Forty Hours' Devotion and 
other Eucharistic adoration, and outside of church in Corpus Christi 
processions and mystery plays (\term{autos sacramentales}).

Villancicos encompass a wide range of formal structures, but most feature an 
\term{estribillo}, a motet-like section for the full ensemble, and 
\term{coplas}, strophic verses often scored for soloists or a reduced ensemble.
Many of the musical sources specify that the \term{estribillo} was repeated
after the \term{coplas}.

Though many of the sources have been lost, the surviving repertoire is vast and 
rich, encompassing a broad range of Hispanic devotional life.
Many villancicos of this period were scored for large polychoral ensembles of 
voices, probably doubled by loud wind instruments and supported by organ, harp, 
and other plucked strings.
Others, sometimes called \term{tonos divinos}, are scored for a more intimate 
texture of a few voices with continuo.
The first type are public, festival pieces; the second offer more private, 
contemplative experiences.
With a variety of subgenres and topics from the comic to the learned, there was 
a villancico for everyone and nearly every occasion.

This edition offers performers and scholars a coherent set of newly edited 
villancicos, drawn from archives in Spain and Mexico, that share a common theme 
of \quoted{singing about singing}.
The vernacular poetry of these \soCalled{metamusical} pieces represents the act 
of hearing and making music.
The musical settings of these poems, then, become discourses about music, 
through the medium of music itself.%
  \begin{Footnote}
      These villancicos formed the primary corpus of study for \autocite{Cashner:PhD}.
      This edition corrects and supersedes the musical editions in the dissertation.
  \end{Footnote}
The thematic organization makes these pieces ideal both for concert programming 
and for scholarly study.

\edsection{Interpretive Themes}

The villancicos in this collection present a complex and multilayered discourse 
of music and theology.
Common tropes run through these pieces and demonstrate traditions of poetry and 
music about musical performance, or about music as an abstract concept.
The pieces embody a Neoplatonic theology of music in which listeners are 
invited to listen for echoes of higher music in the imperfect earthly music 
they hear.%
    \Autocite[108--132]{Cashner:PhD}
The following brief interpretive notes may serve as an initial guide to 
understanding these pieces.

\edsubsection{Cererols and Gutiérrez de Padilla: Christ as Singer and Song}

The first two pieces, \worktitle{Suspended, cielos, vuestro dulce canto} by Joan
Cererols and \worktitle{Voces, las de la capilla} by Juan Gutiérrez de Padilla,
are villancicos for Christmas that represent the newborn Christ as both singer
and song.%
    \Autocite[133--284]{Cashner:PhD}
Extending on an exegetical tradition going back to Bernard of Clairvaux and 
Augustine, these pieces celebrate Christ as the \term{Verbum infans}, the Word 
of God made flesh (John~1:1), but as an infant, unable to speak a word. 
Since Christ in his incarnate body is himself the Word, these pieces portray 
his inarticulate cries as a form of music, as the tuning note---the  
\quoted{sign of \term{A (la, mi, re)}}---to which the music of a renewed 
creation will be harmonized.
Joan Cererols, monk and chapelmaster of the choir school of the Benedictine
Abbey of Our Lady of Montserrat near Barcelona, has his ensemble bid the
heavenly spheres themselves to cease their imperfect music and \quoted{listen to
the newest consonance} of Christ.
Juan Gutiérrez de Padilla, priest and chapelmaster of Puebla Cathedral in New
Spain (Mexico), presents Christ as the heir of the musician-king David, the
masterpiece of the divine chapelmaster who puts God and Man in harmony through
his Incarnate body, which is made known through his infant voice.

The composers match the musical conceits of the poetry with the appropriate 
musical devices, such as the eight-voice fugue in strict counterpoint in late
sixteenth-century style (like that of Palestrina, Morales, and Guerrero) that
Cererols creates for \term{contrapunto celestial}.
Cererols even illustrates the idea of Christ as the \term{cantus firmus} for a 
restored heavenly music by developing the motive of a descending stepwise fifth 
throughout the estribillo, culminating in a concluding section in the style of 
a cantus-firmus motet.
Cererols illustrates \quoted{the newest consonance} by setting the word 
\mentioned{consonancia} on a prominent, unprepared, and repeated dissonance.
By drawing listeners' attention to the imperfection of worldly music through 
this ironic symbol, Cererols points them in Neoplatonic fashion past the 
sounding music, to listen for an unhearable, higher music of Christ the divine 
Word.

Padilla also creates musical devices to illustrate the arcane music-theoretical 
and theological references in his poem.
He quotes the plainchant \term{tonus peregrinus} on the words \term{peregrino 
tono}.
Padilla has half his ensemble exhort the other half to \quoted{keep count with 
what is sung} while they are literally counting their rests.
Then the other chorus sings about \quoted{awaiting the thirty-three} (a 
reference to Christ's Passion) with exactly thirty-three notes.
Both choirs join together to represent the celebration of heavenly beings, 
humans, and beasts singing in the manger, in the style of a madrigal, scored 
for voices \quoted{three by three, two by two, one by one}.
Padilla's \term{estribillo} climaxes with an epitome of Catholic belief about
Christ's Incarnation, \quoted{Everything in man is to ascend, and everything in
God is to descend}.
Padilla sets the first line to an ascending line in normal triple meter and 
juxtaposes this against the second phrase, which he sets as a long descending 
line in \gloss{sesquialtera}{hemiola}, written using all blackened noteheads.
Thus the theological and musical are closely linked in both pieces, so that 
one's knowledge of theology informs understanding of the musical structure, and 
one's knowledge of music theory and ability to perceive musical-rhetorical 
devices gives insight into theological conceptions of Incarnation, voice, and
hearing.

\edsubsection{Irízar, Carrión, Cáseda: Hearing and Faith}

Next are two settings of the villancico poem, \worktitle{Si los sentidos queja 
forman del Pan Divino}, by successive chapelmasters at Segovia Cathedral in the 
later seventeenth century.%
    \Autocite[285--338]{Cashner:PhD}
The poem, attributed to Vicente Sánchez of Zaragoza, presents a contest 
of the senses, to be judged by their merits in relationship to faith.
The contest is similar to that in Pedro Calderón de la Barca's Corpus Christi 
play \worktitle{En nuevo palacio del Retiro} of 1634.%
    \Autocites{Calderon:Retiro}[52--107]{Cashner:PhD}
The coplas articulate commonly held beliefs about the powers of the senses and 
emphasize that the mystery of the Eucharist confounds every sense.
Hearing is given the first prize because only through believing in what is 
heard, and not through the other senses, can one rightly perceive Christ's 
presence in the sacrament.
The poem uses music to exemplify the sense of hearing.
Irízar's festival setting evokes the contest musically through polychoral
dialogue and perhaps evoking the keyboard genre of \term{batalla}.%
    \Autocite{Sutton:IberianBatalla}
Carrión's continuo song, by contrast, invites a more personal reflection on the 
nature of sensation.

José de Cáseda's setting of \worktitle{Qué música divina} intersects 
both with the metamusical conceits of the pieces by Padilla and Cererols and 
with the discourse on sensation in the Irízar and Carrión villancicos.
The central conceit of this piece for Eucharistic devotion presents Christ in 
his Passion as a \term{vihuela}.%
    \Autocite[375--405]{Cashner:PhD}
The poem applies patristic allegorical traditions of the \term{cithara} and 
\term{lira} to a distinctly Spanish instrument.
The music played on this instrument is \quoted{not for the senses}; it 
\quoted{elevates the senses} and \quoted{confounds the mind's powers}.
If it could be heard it would sound \quoted{false}---dissonant, out of tune, or 
as \term{musica ficta}. 
Similar to Cererols evoking divine consonance through earthly dissonance, 
Cáseda appears to employ deliberate solecisms to represent this \quoted{false}
music, like the parallel fifths and direct octaves on the 
word \quoted{tuneful}, or the cadential patterns on \quoted{various cadences}
that tempt singers to add accidentals in the wrong places.
He evokes the seven-course vihuela in several ways through the vocal texture, 
most notably through the strumming texture at the end of the estribillo.

Though Cáseda lived and worked in Zaragoza, this piece survives only in the
collection of the Conceptionist Convento de la Santísima Trinidad in Puebla.%
    \Autocite{Favila:PhD}
In performing this piece, the chorus of nuns whose names are preserved in the 
parts would in a sense \emph{become} a vihuela, embodying an instrument while 
presenting that instrument as a symbol of Christ's body.

This piece demonstrates a strain of villancico composition quite removed from 
the popularizing, folkloric types of villancicos that have become better known,
such as the pieces that follow in this edition.  
Instead this is an exercise in contemplative devotion worked out through a
musical craft that emphasizes both ingenuity and affective power.

\edsubsection{Gutiérrez de Padilla and Salazar: Singing in Christ's Stable}

The last two pieces in this edition return to the stable in Bethlehem to unite 
humans and angels in the music of Christ's Incarnation.
In a piece for the new cathedral of Puebla (consecrated three years earlier in 
1649), Juan Gutiérrez de Padilla and his ensemble call up a colorful host of
characters \quoted{to the most blessed stable} to sing and dance for the baby
Jesus.%
    \Autocite[406--467]{Cashner:PhD}
This \term{ensaladilla} is a potpourri of different song and dance styles, 
probably referencing pre-existing music known to the hearers.
A group of shepherds sing something called the \quoted{New Trojan} to the music 
of \quoted{tempered panpipes}.
A buffoon mule-skinner's mule barges into the stable in search of straw; the 
befuddled candy vendor tries to excuse himself before the Christ-child, whom he 
obsequiously calls \quoted{Sir Baby}, while struggling to control his mule---a 
struggle evoked through disorderly rhythm.
Next a group of \quoted{mountain folk}, whose language marks them as 
agricultural laborers, dances a gentle \quoted{Papalotillo}.
This name is derived from a Nahuatl word, and these characters may be meant to 
represent indigenous people.

The final section of the piece is a complete, self-contained \term{negrilla} or 
\soCalled{black villancico}, a common subgenre.
Here Padilla's ensemble of Spaniards and Spaniard-descended \term{criollos} 
presented caricatures of Africans and their music, in a mocking imitation of 
African speech.
In the midst of a pseudo-African dance, the black characters are suddenly 
joined by a chorus of angels in singing \term{Gloria}---but the blacks sing in 
the ternary meter typical of villancicos while the angels sing in the duple
meter more commonly used for Latin-texted liturgical music, and probably evoke a
plainchant intonation.

Padilla's \quoted{little salad} tosses together characters from different 
racial and economic strata to present an idealistic vision of the whole 
colonial society united around the body of Christ.
This composer, who was both a university-educated priest and a slave-owner,
brings the highest and lowest beings together in harmony while paradoxically
keeping them apart, reflecting a Neoplatonic concept of the social hierarchy.%
    \Autocite{Mauleon:PadillaPalafox}
While this piece and other \soCalled{ethnic villancicos} have much to teach 
about how Spanish elites perceived their relationships to the other groups 
under their control, performers should consider seriously how it might be 
possible to present such a piece today in an ethically responsible manner.%
  \begin{Footnote}
      \Autocites{Baker:EthnicVC}{Baker:PerformancePostColonial}. 
      For further on the relationship between Spanish representations of Africans 
      and their actual situation, see
      \autocites{Molinero:Negros}{Lipski:AfroHispanic}
      {Fromont:DancingKingCongo}.
  \end{Footnote}

The last piece in the collection is a typical representation of angelic music 
at Christmas, by Antonio de Salazar, who became chapelmaster of Mexico City 
Cathedral.%
    \Autocite[29--34]{Cashner:PhD}
This delicate villancico, with its lilting rhythms, is from the same convent 
collection as the Cáseda piece, in Puebla de los Ángeles, the original
\quoted{city of the angels}.  
Salazar uses imitative counterpoint to represent the angelic chorus coming down 
to earth, much as Cererols did in his celestial fugue. 
The convent sisters who sang this piece at Christmas embodied and incited the 
affects of wonder and joy that theologians considered most characteristic of 
this feast.



% \section{Select Bibliography}

\subsection{Studies of Villancicos}
\nocite{Rubio:Forma}
\nocite{Laird:VC}
\nocite{Torrente:PhD}
\nocite{Tenorio:SorJuana}
\nocite{CaberoPueyo:PhD}
\nocite{Illari:Polychoral}
\nocite{Knighton-Torrente:VCs}
\nocite{Cashner:Cards}
\printbibliography[heading=none,filter=villancico-studies]

\subsection{Musical Editions of Villancicos}
\nocite{Cererols:MEM-VC}
\nocite{Stevenson:Christmas}
\nocite{Ruimonte:Parnaso}
\nocite{Padilla:Tello}
\nocite{Ezquerro:MME55}
\nocite{RuizSamaniego:MME63}
\nocite{Ezquerro:MME59}
\nocite{Ezquerro:MME65}
\nocite{Fernandez:Cancionero}
\nocite{Torrejon:VCs}
\printbibliography[heading=none,filter=villancico-editions]

\subsection{Catalogs of Villancico Poetry Imprints}
\nocite{BNE:VCs17C}
\nocite{BNE:VCs18C}
\nocite{UK:VCs}
\nocite{US:VCs}
\printbibliography[heading=none,filter=villancico-imprint-catalogs]



% \section*{Editorial Policies}

\paragraph{Sources}
The sources for the each poem and its musical setting are listed in the 
critical notes.
The music is preserved in individual manuscript performing parts in looseleaf 
sets or bound partbooks.
For the villancico by Miguel de Irízar, the composer's draft score also 
survives.

The texts and translations are based on the poetic text in the musical settings.
They have been annotated and sometimes corrected in comparision with the 
surviving poetry imprints of the same or related villancico poems.
The poems are generally anonymous, but are often adapted from existing poems or 
poetic types.

The manuscript parts were practical tools for performers.
They all bear evidence of frequent use over a long period: they are soiled 
along the creases in the paper where performers held them up, and they include 
the names of multiple performers, corrections in different hands, and added 
accidentals and barlines.
Aspects of notation that seem ambiguous to a modern scholar were not, 
apparently, impediments to effective performance from the originals.
The goal of this edition, in keeping with the nature of its sources, is to 
enable the practical performance and study of these villancicos through a clear
and consistent notation.


\paragraph{Orthography}
Spelling and punctuation have been modernized and standardized.
Though in doing this some information about historic local pronunciation is
lost, a standard orthography allows performers to present the works in a way
that will be most intelligible to their audiences.%
\begin{Footnote}
    The phonetic orthography in the performing parts does suggest that
    \mentioned{ci} and \mentioned{ce} were pronounced like \mentioned{si} and
    \mentioned{se} in New Spain and Catalonia, rather than with the TH sound in
    modern peninsular Spanish (as in \mentioned{thick}).
\end{Footnote}
The exception to this rule is in the \term{negrilla} of Padilla's \term{Al 
establo más dichoso}, in which it seemed more responsible to present the 
pseudo-African dialect in its original orthography.
Possible equivalents in proper Spanish are given in the footnotes.

\paragraph{Voice and Instruments}
The original names for voices and instruments have been preserved. 
\mentioned{Tiple} refers to a treble singer, usually a boy.
Several terms are used for continuo parts, such as \term{Acompañamiento},
\term{General}, or \term{Guión}.

The edition preserves indications of solo and instrumental parts when they
appear in the original.
Original figured bass is preserved, but continuo realizations are left to the
discretion and creativity of the performer.
Separate instrumental parts and realized keyboard parts are available on request
from the editor.

\paragraph{Editorial Text}
Italic text indicates editorial underlay, usually where there are signs
(\MSrepeat{}) in the sources that specify that the preceding text should
be repeated.
Other textual additions by the editor, such as standardized section headings, 
are enclosed in square brackets.

\paragraph{Pitch Level}
All pieces are transcribed at their original notated pitch level.
The preparatory staves at the beginning of each piece show the original clefs, 
signatures, and the first note.

\paragraph{Accidentals}
Accidental placement in the partbooks is contextual and sometimes ambiguous to 
a modern reader.
The original notation has no \na{} symbol, using B\sh{} and E\sh{} instead.
In a few cases, indicated in the critical notes, scribes use a \sh{} sign as a 
cautionary accidental.
One common use was to warn the singer \emph{not} to apply a sharp according 
to \term{musica ficta} conventions.%
  \autocites{Harran:Cautionary1}{Harran:Cautionary2}

The edition presents the pitches with their accidental inflections when 
unambiguously specified in at least one source.
According to modern convention, these accidentals are valid until the next 
barline.
Thus repeated accidentals in the source are omitted if the modern convention 
does not require them; and in a few cases accidentals are added where modern 
notation demands.
Editorial suggestions for other accidentals, mostly according to \term{musica 
ficta} conventions, are set above the staff.

\paragraph{Repeats}
Some of the sources indicate repeated sections barlines with dots (like modern
repeats), or by giving the incipit of the music and text to be repeated; often
there is also a \term{signum congruentiae} at the point of repetition or a
textual note.
In most cases, the estribillo was reprised after the last copla was sung (more
like a psalm antiphon than a \quoted{refrain} as the term might imply).
Some pieces call for a reprise after each copla or after certain groups of
coplas.
In many sources, the repeat of the estribillo is not specified, and it is
possible that it was not always reprised, especially as villancicos became
longer and more complex.%
    \Autocite{Torrente:Estribillo}

This edition uses modern repeat barlines for short repeated sections and
indications of \quoted{D.C. al Fine} or \quoted{D.S. al Fine}. % XXX continue




\paragraph{Rhythm, Meter, Tempo}
The original music was written in mensural notation, with few barlines in the 
performing parts.%
\begin{Footnote}
    Spanish composers like Miguel de Irízar did use barlines when they notated in 
    score format.
    Irízar writes two \term{compases} per bar in both triple and duple meters,
    occasionally squeezing in a third \term{compás} if there was an odd number
    of groups.
    Cerone advises students who wish to write out a score from parts to write 
    barlines every two \term{compases}; \textcite[745]{Cerone:Melopeo}.
\end{Footnote}
The duple-meter sections of these pieces were written in mensural \meterC{} 
meter, which the seventeenth-century Spanish theorists Pedro Cerone and Andrés 
Lorente refer to as \term{tiempo menor imperfecto} or \term{compasillo}.%
  \autocites[537]{Cerone:Melopeo}[156, 210]{Lorente:Porque}
In this meter, the \term{compás} or \term{tactus} consisted of a semibreve 
divided into two minims.%
  \autocites{GonzalezValle:MusicaTexto}{GonzalezValle:CompasCabezon}

The other common meter for seventeenth-century villancicos was notated with the 
symbol \meterCZorig{}, a cursive \meterCZ{}.
Lorente says that this is a shorthand for \meterCThreeTwo{} or \meterCThree{},
where these signs all indicate \term{tiempo menor de proporción menor}, a 
proportion of C meter.%
  \autocite[165]{Lorente:Porque}
The \term{compás} consists of one perfect semibreve which is divided into three 
minims, instead of the two minims of C.

In the sources, deviations from the normal ternary groups are indicated through 
coloration. 
When noteheads in CZ meter are blackened, this often indicates a shift to 
\term{sesquialtera} or hemiola.
In \term{sesquialtera} two groups of three minims are exchanged for three 
groups of two minims; and three imperfect semibreves take the place of two
perfect semibreves.

The edition presents the rhythms of the sources according to modern conventions 
of meter and barlines.
The music has been notated in the \meterC{} for duple meter and \meterCThree{}
for triple meter.
The original meter signs are shown in preparatory staves or above the staff.
The original note values have not been reduced.
Mensural coloration is indicated with short rectangular brackets above the 
staff.
Ligatures are indicated by long rectangular brackets.
Beaming is unchanged.

Regarding tempo, the theoretical $3:2$ proportion of minims between
\meterCThreeTwo{} and \meterC{} does not necessarily imply the same proportion of
\emph{tempo}.  
In actual practice, a $3:1$ tempo relationship often makes more musical sense,
so that three minims in triple meter together take the same amount of time as
one minim in duple meter.
Thus two \term{compases} of CZ would have about the same duration as one 
\term{compás} of C.

\section*{Performance Suggestions}

\paragraph{Spanish Pronunciation}
Spanish-speaking ensembles should feel free to pronounce the Spanish according
to their own accent.
Other ensembles should work with local native speakers and experts whenever
possible to shape their pronunciation and understanding, so that they can
perform these pieces in a way that Spanish-speaking audience members will
understand and recognize as a part of their own cultural heritage.

\paragraph{Instrumentation and Voicing}
These villancicos are scored for an ensemble of voices with instrumental bass 
or continuo groups.
Vocal ensembles varied in size, from one-to-a-part groups to much larger 
polychoral forces.
Most of the pieces also feature prominent solo parts, particularly in the 
\term{coplas}.

The lowest voice parts in these pieces are meant to be performed on instruments. 
They are only provided with short incipits of the text to orient the 
performer, and in several cases instruments like \gloss{bajón}{dulcian, bass 
curtal} or organ are specified.
Though there is need for more research into the specific instrumentation of 
Spanish musical ensembles, it is plausible that the bass line was performed in 
most cases by a continuo group of \term{bajón} doubled by harp, organ, and 
possibly other instruments like the \term{vihuela de mano}.%
\begin{Footnote}
    On the changing instrumentation in one Spanish institution, see 
    \autocite{Torrente:PhD}.
\end{Footnote}
In pieces without figured bass, continuo players---which could include any
polyphonic instruments like keyboard or plucked strings---likely improvised
harmonies to match the other voices.

The upper voices could have been doubled on \term{bajoncillos},
\gloss{chirimías}{shawms}, \gloss{sacabuches}{sackbuts}, and other instruments
according to local resources and suited to the occasion.
There is as yet no clear evidence, though, that church ensembles of 
seventeenth-century Spain or Spanish America included percussion instruments.%
\begin{Footnote}
  For a critique of exoticizing practices in recent villancico performances,  
  see \autocite{Baker:PerformancePostColonial}.
\end{Footnote}

Ensembles should not be deterred by the lack of early instruments or by by vocal
ranges outside their resources.
It would be entirely within the spirit of the performing traditions that these
sources represent, for a school or community chorus to substitute modern
instruments for their historic relatives.

At a minimum, it is appropriate to use a contrapuntal instrument like keyboard
for the continuo and a melodic instrument like the historic \term{bajón} or
modern bassoon or cello for the instrumental bass lines.
If more instruments are available, add other instruments to the continuo
section like historic \term{bajón}, \term{vihuela}, Spanish harp, and organ;
or modern alternatives like bassoon, cello, modern harp, guitar, digital
organ (with a good sample of an 8$'$ flue stop), or piano.
Double vocal parts freely with historic or modern reeds, winds, and brass, or
possibly bowed strings.

If possible, use soloists or a reduced ensemble for the first chorus in
polychoral pieces, and for the coplas.
A chorus of more modest ability, such as a high school choir, could be paired
with more advanced soloists, such as college students or adult community
members.
Make sure there is at least one singing voice per chorus to present the text;
the other voices could be played instrumentally.

Instrumental parts and continuo realizations are available from the editor upon
request.

\paragraph{Pitch Level}
One of the features of seventeeth-century Spanish choral music that is most
surprising to those more familiar with other repertoires is how high the vocal
ranges are.
Many of the Tiple (treble) parts have tessituras above \pitch{F}{5}; and none of
the pieces have texted bass parts, these parts being played instrumentally
instead.
Either Spanish ensembles performed these pieces at a lower pitch level than
notated (because of a lower general pitch, or through transposition), or Spain
cultivated a lost art of angelically high singing.
Modern ensembles should sing the pieces at a pitch level or transposition that
works for them.

Transposed scores are available from the editor upon request.

\paragraph{Rhythm, Meter, and Tempo}
The meters and barlines in the edition are only conveniences to make the pieces
plainly legible and performable.
Performers should not take the barlines as guides to accentuation, nor should 
they assume the music lacks natural accentuation.
Mensural meters do not necessarily imply any particular pattern of rhythmic 
accentuation.
Most of the time poetic declamation should be the primary guide for pacing
and emphasis.
In other cases, when a set style of dance or song seems to be evoked, a regular
rhythmic pattern may win out over poetic nuances.
Regarding such rhythmic patterns, it should be noted that no one has yet
provided conclusive evidence for the presence of African or American indigenous
rhythms in villancicos.

The sign \meterC{} indicates a duple meter that should be felt \quoted{in
two}.
The sign \meterCThree{} indicates a ternary meter that should be felt
\quoted{in one}.
Often triple meter is syncopated or altered by hemiola (also called
\term{sesquialtera}) to create patterns of accentuation that differ from the
normal ternary groupings indicated by the barlines.

In most cases, it seems appropriate to maintain a tempo relationship of three
minims in \meterCThree{} to one minim in \meterC{}.
By maintaining this tempo relationship it is usually possible to maintain a
consistent pulse throughout the whole piece.
A resting heart rate of about sixty beats per minute generally makes a good
tempo, such that in \meterC{}, $\musMinim{} = 60$; while in \meterCThree,
$\musSemibreveDotted = 60$ and $\musMinim{} = 180$.


\paragraph{Authenticity and Flexibility}
In the editor's opinion, an authentic performance of a seventeenth-century
villancico would be one that is meaningful to present-day performers and their
audience, but which also opens a window to experiencing what made the piece
meaningful to its original performers and hearers.
Performers should seek out the distinctive character and significance of each
piece, but should also feel free to adapt the pieces to suit their own resources
and social context.
It would be better to have a spirited, respectful, musically sensitive
performance with modern instruments than to have no performance at all because
historic instruments were not available.

The sources for this edition are performing parts that, on the one hand,
were used as practical tools for performance in a particular place, and, on the
other hand, represent traditions of performance that cannot be completely fixed
in place or time.
In other words, even within one institution, such as the Conceptionist convent
in Puebla from which come the parts for Salazar's \worktitle{Angélicos coros}
and Cáseda's \worktitle{Qué música divina}, these parts were used and reused
possibly over generations. 
In some cases, later performers made corrections, added barlines, sewed in new
lines of lyrics or even new music to replace certain strophes.
There is no single way that these pieces were performed throughout their terms
of service as part of the local repertoire.

Moreover, these pieces represent single instances of a repertoire that
circulated around the globe. 
José de Cáseda lived in Zaragoza and set a poem by a composer from his same
region, Vicente Sánchez; but his setting is only known from the surviving parts
in the Puebla convent.
The spelling in those parts reflects New Spanish, not Zaragozan pronunciation
(e.g., \mentioned{consonancias} is spelled \mentioned{consonansias} in the
Puebla parts, even though the final C would probably have been pronounced like an
English TH sound in Zaragoza).
The piece may have been rearranged or adapted for female ensemble from a lost
original with different scoring.
On some occasions, a particular sister may have fallen ill and her vocal line
may have been played instrumentally.

The starting point for considering modern performance of these pieces, then, is
that historic performers made these pieces their own and performed them in a way
that fit their local needs in terms of personnel, instrumentation, rehearsal
time, acoustic space, and other factors; and in a way that was intelligible and
meaningful to them and to their hearers.

\paragraph{Ethical Responsibility}
While some amount of adaptation seems appropriate for this repertoire,
performers are urged never to lose site of the religious, social, and political
contexts of these pieces in their early modern origins.
These pieces cannot be cleanly separated from the social values of the colonial
era that this music both reflected and reinforced.
A piece like Juan Gutiérrez de Padilla's \worktitle{Al establo más dichoso}
bears the imprint of imperial Spain's racial hierarchy: it is documented that
the composer himself owned an Angolan slave, and the representation of
\quoted{Angolans} in the piece caricatures their bodies and voices as deformed
and deficient, even as it perhaps strives to present them in a sympathetic light
as offering devotion to Christ and joining with the angelic chorus.

It would be ethically irresponsibly to perform such a piece merely as an exotic
curiosity, or worse, as though it were a twenty-first century celebration of
ethnic diversity. 
Indeed, performers, scholars, and community members ought to engage in serious
discussions about what performing a piece might mean in a contemporary context.
In the right setting, such as a community workshop with appropriate
opportunities for critique, response, and discussion, the piece might be used 
effectively to raise issues of tremendous contemporary relevance; but in the
wrong context the piece could actually perpetuate the negative racial
stereotypes that are built into it.



\section{Abbreviations}

\begin{tabu} to \textwidth{lZ}
    A. & Alto, \term{Altus}\\
    A\octave{4} & The note A, fourth octave (above middle C) \\
    Ac. & \term{Acompañamiento}: Accompaniment, \term{basso continuo}\\
    B. & \term{Bajo}, \term{Bassus}\\
    Ch. & Chorus\\
    CN & See critical notes\\
    Corr. & Editorial correction\\
    DRAE & Real Academía Española, \worktitle{Diccionario de la lengua española}, 
    23rd ed.\\
    Imprints & Reading based on consensus of extant poetry imprints\\
    Leg. & \term{Legajo} (archival folder)\\
    OED & \worktitle{Oxford English Dictionary Online} (accessed 2017)\\
    S. & Soprano; Used in part listings (e.g., \quoted{SSAT}) for the highest voice 
    part, usually for \term{Tiple}, to avoid confusion between \quoted{Ti.} for 
    Tiple and \quoted{T.} for Tenor\\
    Sugg. & Editorial suggestion\\
    T. & Tenor\\
    Ti. & \term{Tiple}: Treble, boy soprano\\
    Ti. I-1 & Chorus 1, First Tiple\\
\end{tabu}

\subsection{Archival Sigla}

\begin{tabu} to \textwidth{llZ}
    \textsc{Siglum} & \textsc{Country} & \textsc{Archive}\\
    E-Bbc & Spain & Barcelona, Biblioteca de Catalunya\\
    E-CAN &  & Canet de Mar, Arxiu Parròquia de Sant Pere i Sant Pau de Canet de 
    Mar, Bisbat de Girona, Fons Capella de Música\\
    E-Mn & &  Madrid, Biblioteca Nacional de España\\
    E-SE & & Segovia, Catedral, Archivo Capitular\\
    MEX-Pc & Mexico &  Puebla, Catedral, Archivo Capitular\\
    MEX-Mcen & Mexico & Mexico City, CENIDIM (Centro Nacional de Investigación, 
    Documentación e Información Musical Carlos Chávez)\\
    GB-Lbl & United Kingdom & London, British Library\\
\end{tabu}




% \section{Acknowledgments}

This edition was produced with free and open-source software on the Debian 
GNU/Linux operating system, release 8.
The text was typeset with the \LaTeX{} document-preparation system, based on 
the work of Donald Knuth and Leslie Lamport.
The music was typeset with Lilypond, version 2.19.
I am grateful to the hundreds of volunteers who build and maintain these 
systems, and to those who provided specific help in programming this document.

The text typeface is EB Garamond, designed by Georg Duffner, based on 1592 type 
specimens by Claude Garamont.
The Spanish CZ metrical symbol used in the scores was traced manually in 
Inkscape from a villancico manuscript by Miguel de Irízar.

I am grateful to the following people and institutions for access to the 
primary sources on which these editions are based: 
the capitular archive of the Cathedral of Puebla de los Ángeles (P. Francisco 
Vázquez, rector; the Illmo. Sr. Carlos Ordaz, \term{canónigo archivista});
CENIDIM, the Mexican national center for music research in Mexico City;
the Biblioteca de Catalunya in Barcelona;
the parochial archive of the Church of Saints Peter and Paul, Canet de Mar, in 
the Archdiocese of Girona, Barcelona province;
the capitular archive of Segovia Cathedral (P. Bonifacio Bartolomé);
the Biblioteca Nacional in Madrid, and
the British Library in London.

Travel for archival research in Mexico and Spain was funded by 
\begin{anonymize}
    a Jacob K. Javits Fellowship from the United States Department of Education, 
    a Pre-Dissertation Research Fellowship from the Council for European Studies at 
    Columbia University, 
    a Eugene K. Wolf Research Travel Grant from the American Musicological Society, 
    and grants from the Department of Music and the Center for Latin American 
    Studies at the University of Chicago.
    Other research funding was provided by a Dissertation Completion Fellowship 
    from the American Council of Learned Societies and the Mellon Foundation.

    % Thank editor and reviewers XXX
    I thank my doctoral advisor, Robert Kendrick, and the other members of my 
    dissertation committee, Anne Walters Robertson, Martha Feldman, Frederick de 
    Armas, and María Gembero-Ustárroz.
    I am also grateful to the following people for specific help with the poetry 
    and music in this edition:
    Stephen Black,
    Anita Damjanovic, 
    Miguel Martínez, 
    Gustavo Mauleón Rodríguez,
    James Nemiroff, and
    Martha Tenorio.
    My wife Ann makes all of this possible and my children Ben and Joy make it
    worthwhile.
\end{anonymize}


\edchapter{Editorial Report}
\section{Abbreviations}

\begin{tabu} to \textwidth{lZ}
    A. & Alto, \term{Altus}\\
    A\octave{4} & The note A, fourth octave (above middle C) \\
    Ac. & \term{Acompañamiento}: Accompaniment, \term{basso continuo}\\
    B. & \term{Bajo}, \term{Bassus}\\
    Ch. & Chorus\\
    CN & See critical notes\\
    Corr. & Editorial correction\\
    DRAE & Real Academía Española, \worktitle{Diccionario de la lengua española}, 
    23rd ed.\\
    Imprints & Reading based on consensus of extant poetry imprints\\
    Leg. & \term{Legajo} (archival folder)\\
    OED & \worktitle{Oxford English Dictionary Online} (accessed 2017)\\
    S. & Soprano; Used in part listings (e.g., \quoted{SSAT}) for the highest voice 
    part, usually for \term{Tiple}, to avoid confusion between \quoted{Ti.} for 
    Tiple and \quoted{T.} for Tenor\\
    Sugg. & Editorial suggestion\\
    T. & Tenor\\
    Ti. & \term{Tiple}: Treble, boy soprano\\
    Ti. I-1 & Chorus 1, First Tiple\\
\end{tabu}

\subsection{Archival Sigla}

\begin{tabu} to \textwidth{llZ}
    \textsc{Siglum} & \textsc{Country} & \textsc{Archive}\\
    E-Bbc & Spain & Barcelona, Biblioteca de Catalunya\\
    E-CAN &  & Canet de Mar, Arxiu Parròquia de Sant Pere i Sant Pau de Canet de 
    Mar, Bisbat de Girona, Fons Capella de Música\\
    E-Mn & &  Madrid, Biblioteca Nacional de España\\
    E-SE & & Segovia, Catedral, Archivo Capitular\\
    MEX-Pc & Mexico &  Puebla, Catedral, Archivo Capitular\\
    MEX-Mcen & Mexico & Mexico City, CENIDIM (Centro Nacional de Investigación, 
    Documentación e Información Musical Carlos Chávez)\\
    GB-Lbl & United Kingdom & London, British Library\\
\end{tabu}




\section*{Editorial Policies}

\paragraph{Sources}
The sources for the each poem and its musical setting are listed in the 
critical notes.
The music is preserved in individual manuscript performing parts in looseleaf 
sets or bound partbooks.
For the villancico by Miguel de Irízar, the composer's draft score also 
survives.

The texts and translations are based on the poetic text in the musical settings.
They have been annotated and sometimes corrected in comparision with the 
surviving poetry imprints of the same or related villancico poems.
The poems are generally anonymous, but are often adapted from existing poems or 
poetic types.

The manuscript parts were practical tools for performers.
They all bear evidence of frequent use over a long period: they are soiled 
along the creases in the paper where performers held them up, and they include 
the names of multiple performers, corrections in different hands, and added 
accidentals and barlines.
Aspects of notation that seem ambiguous to a modern scholar were not, 
apparently, impediments to effective performance from the originals.
The goal of this edition, in keeping with the nature of its sources, is to 
enable the practical performance and study of these villancicos through a clear
and consistent notation.


\paragraph{Orthography}
Spelling and punctuation have been modernized and standardized.
Though in doing this some information about historic local pronunciation is
lost, a standard orthography allows performers to present the works in a way
that will be most intelligible to their audiences.%
\begin{Footnote}
    The phonetic orthography in the performing parts does suggest that
    \mentioned{ci} and \mentioned{ce} were pronounced like \mentioned{si} and
    \mentioned{se} in New Spain and Catalonia, rather than with the TH sound in
    modern peninsular Spanish (as in \mentioned{thick}).
\end{Footnote}
The exception to this rule is in the \term{negrilla} of Padilla's \term{Al 
establo más dichoso}, in which it seemed more responsible to present the 
pseudo-African dialect in its original orthography.
Possible equivalents in proper Spanish are given in the footnotes.

\paragraph{Voice and Instruments}
The original names for voices and instruments have been preserved. 
\mentioned{Tiple} refers to a treble singer, usually a boy.
Several terms are used for continuo parts, such as \term{Acompañamiento},
\term{General}, or \term{Guión}.

The edition preserves indications of solo and instrumental parts when they
appear in the original.
Original figured bass is preserved, but continuo realizations are left to the
discretion and creativity of the performer.
Separate instrumental parts and realized keyboard parts are available on request
from the editor.

\paragraph{Editorial Text}
Italic text indicates editorial underlay, usually where there are signs
(\MSrepeat{}) in the sources that specify that the preceding text should
be repeated.
Other textual additions by the editor, such as standardized section headings, 
are enclosed in square brackets.

\paragraph{Pitch Level}
All pieces are transcribed at their original notated pitch level.
The preparatory staves at the beginning of each piece show the original clefs, 
signatures, and the first note.

\paragraph{Accidentals}
Accidental placement in the partbooks is contextual and sometimes ambiguous to 
a modern reader.
The original notation has no \na{} symbol, using B\sh{} and E\sh{} instead.
In a few cases, indicated in the critical notes, scribes use a \sh{} sign as a 
cautionary accidental.
One common use was to warn the singer \emph{not} to apply a sharp according 
to \term{musica ficta} conventions.%
  \autocites{Harran:Cautionary1}{Harran:Cautionary2}

The edition presents the pitches with their accidental inflections when 
unambiguously specified in at least one source.
According to modern convention, these accidentals are valid until the next 
barline.
Thus repeated accidentals in the source are omitted if the modern convention 
does not require them; and in a few cases accidentals are added where modern 
notation demands.
Editorial suggestions for other accidentals, mostly according to \term{musica 
ficta} conventions, are set above the staff.

\paragraph{Repeats}
Some of the sources indicate repeated sections barlines with dots (like modern
repeats), or by giving the incipit of the music and text to be repeated; often
there is also a \term{signum congruentiae} at the point of repetition or a
textual note.
In most cases, the estribillo was reprised after the last copla was sung (more
like a psalm antiphon than a \quoted{refrain} as the term might imply).
Some pieces call for a reprise after each copla or after certain groups of
coplas.
In many sources, the repeat of the estribillo is not specified, and it is
possible that it was not always reprised, especially as villancicos became
longer and more complex.%
    \Autocite{Torrente:Estribillo}

This edition uses modern repeat barlines for short repeated sections and
indications of \quoted{D.C. al Fine} or \quoted{D.S. al Fine}. % XXX continue




\paragraph{Rhythm, Meter, Tempo}
The original music was written in mensural notation, with few barlines in the 
performing parts.%
\begin{Footnote}
    Spanish composers like Miguel de Irízar did use barlines when they notated in 
    score format.
    Irízar writes two \term{compases} per bar in both triple and duple meters,
    occasionally squeezing in a third \term{compás} if there was an odd number
    of groups.
    Cerone advises students who wish to write out a score from parts to write 
    barlines every two \term{compases}; \textcite[745]{Cerone:Melopeo}.
\end{Footnote}
The duple-meter sections of these pieces were written in mensural \meterC{} 
meter, which the seventeenth-century Spanish theorists Pedro Cerone and Andrés 
Lorente refer to as \term{tiempo menor imperfecto} or \term{compasillo}.%
  \autocites[537]{Cerone:Melopeo}[156, 210]{Lorente:Porque}
In this meter, the \term{compás} or \term{tactus} consisted of a semibreve 
divided into two minims.%
  \autocites{GonzalezValle:MusicaTexto}{GonzalezValle:CompasCabezon}

The other common meter for seventeenth-century villancicos was notated with the 
symbol \meterCZorig{}, a cursive \meterCZ{}.
Lorente says that this is a shorthand for \meterCThreeTwo{} or \meterCThree{},
where these signs all indicate \term{tiempo menor de proporción menor}, a 
proportion of C meter.%
  \autocite[165]{Lorente:Porque}
The \term{compás} consists of one perfect semibreve which is divided into three 
minims, instead of the two minims of C.

In the sources, deviations from the normal ternary groups are indicated through 
coloration. 
When noteheads in CZ meter are blackened, this often indicates a shift to 
\term{sesquialtera} or hemiola.
In \term{sesquialtera} two groups of three minims are exchanged for three 
groups of two minims; and three imperfect semibreves take the place of two
perfect semibreves.

The edition presents the rhythms of the sources according to modern conventions 
of meter and barlines.
The music has been notated in the \meterC{} for duple meter and \meterCThree{}
for triple meter.
The original meter signs are shown in preparatory staves or above the staff.
The original note values have not been reduced.
Mensural coloration is indicated with short rectangular brackets above the 
staff.
Ligatures are indicated by long rectangular brackets.
Beaming is unchanged.

Regarding tempo, the theoretical $3:2$ proportion of minims between
\meterCThreeTwo{} and \meterC{} does not necessarily imply the same proportion of
\emph{tempo}.  
In actual practice, a $3:1$ tempo relationship often makes more musical sense,
so that three minims in triple meter together take the same amount of time as
one minim in duple meter.
Thus two \term{compases} of CZ would have about the same duration as one 
\term{compás} of C.

\section*{Performance Suggestions}

\paragraph{Spanish Pronunciation}
Spanish-speaking ensembles should feel free to pronounce the Spanish according
to their own accent.
Other ensembles should work with local native speakers and experts whenever
possible to shape their pronunciation and understanding, so that they can
perform these pieces in a way that Spanish-speaking audience members will
understand and recognize as a part of their own cultural heritage.

\paragraph{Instrumentation and Voicing}
These villancicos are scored for an ensemble of voices with instrumental bass 
or continuo groups.
Vocal ensembles varied in size, from one-to-a-part groups to much larger 
polychoral forces.
Most of the pieces also feature prominent solo parts, particularly in the 
\term{coplas}.

The lowest voice parts in these pieces are meant to be performed on instruments. 
They are only provided with short incipits of the text to orient the 
performer, and in several cases instruments like \gloss{bajón}{dulcian, bass 
curtal} or organ are specified.
Though there is need for more research into the specific instrumentation of 
Spanish musical ensembles, it is plausible that the bass line was performed in 
most cases by a continuo group of \term{bajón} doubled by harp, organ, and 
possibly other instruments like the \term{vihuela de mano}.%
\begin{Footnote}
    On the changing instrumentation in one Spanish institution, see 
    \autocite{Torrente:PhD}.
\end{Footnote}
In pieces without figured bass, continuo players---which could include any
polyphonic instruments like keyboard or plucked strings---likely improvised
harmonies to match the other voices.

The upper voices could have been doubled on \term{bajoncillos},
\gloss{chirimías}{shawms}, \gloss{sacabuches}{sackbuts}, and other instruments
according to local resources and suited to the occasion.
There is as yet no clear evidence, though, that church ensembles of 
seventeenth-century Spain or Spanish America included percussion instruments.%
\begin{Footnote}
  For a critique of exoticizing practices in recent villancico performances,  
  see \autocite{Baker:PerformancePostColonial}.
\end{Footnote}

Ensembles should not be deterred by the lack of early instruments or by by vocal
ranges outside their resources.
It would be entirely within the spirit of the performing traditions that these
sources represent, for a school or community chorus to substitute modern
instruments for their historic relatives.

At a minimum, it is appropriate to use a contrapuntal instrument like keyboard
for the continuo and a melodic instrument like the historic \term{bajón} or
modern bassoon or cello for the instrumental bass lines.
If more instruments are available, add other instruments to the continuo
section like historic \term{bajón}, \term{vihuela}, Spanish harp, and organ;
or modern alternatives like bassoon, cello, modern harp, guitar, digital
organ (with a good sample of an 8$'$ flue stop), or piano.
Double vocal parts freely with historic or modern reeds, winds, and brass, or
possibly bowed strings.

If possible, use soloists or a reduced ensemble for the first chorus in
polychoral pieces, and for the coplas.
A chorus of more modest ability, such as a high school choir, could be paired
with more advanced soloists, such as college students or adult community
members.
Make sure there is at least one singing voice per chorus to present the text;
the other voices could be played instrumentally.

Instrumental parts and continuo realizations are available from the editor upon
request.

\paragraph{Pitch Level}
One of the features of seventeeth-century Spanish choral music that is most
surprising to those more familiar with other repertoires is how high the vocal
ranges are.
Many of the Tiple (treble) parts have tessituras above \pitch{F}{5}; and none of
the pieces have texted bass parts, these parts being played instrumentally
instead.
Either Spanish ensembles performed these pieces at a lower pitch level than
notated (because of a lower general pitch, or through transposition), or Spain
cultivated a lost art of angelically high singing.
Modern ensembles should sing the pieces at a pitch level or transposition that
works for them.

Transposed scores are available from the editor upon request.

\paragraph{Rhythm, Meter, and Tempo}
The meters and barlines in the edition are only conveniences to make the pieces
plainly legible and performable.
Performers should not take the barlines as guides to accentuation, nor should 
they assume the music lacks natural accentuation.
Mensural meters do not necessarily imply any particular pattern of rhythmic 
accentuation.
Most of the time poetic declamation should be the primary guide for pacing
and emphasis.
In other cases, when a set style of dance or song seems to be evoked, a regular
rhythmic pattern may win out over poetic nuances.
Regarding such rhythmic patterns, it should be noted that no one has yet
provided conclusive evidence for the presence of African or American indigenous
rhythms in villancicos.

The sign \meterC{} indicates a duple meter that should be felt \quoted{in
two}.
The sign \meterCThree{} indicates a ternary meter that should be felt
\quoted{in one}.
Often triple meter is syncopated or altered by hemiola (also called
\term{sesquialtera}) to create patterns of accentuation that differ from the
normal ternary groupings indicated by the barlines.

In most cases, it seems appropriate to maintain a tempo relationship of three
minims in \meterCThree{} to one minim in \meterC{}.
By maintaining this tempo relationship it is usually possible to maintain a
consistent pulse throughout the whole piece.
A resting heart rate of about sixty beats per minute generally makes a good
tempo, such that in \meterC{}, $\musMinim{} = 60$; while in \meterCThree,
$\musSemibreveDotted = 60$ and $\musMinim{} = 180$.


\paragraph{Authenticity and Flexibility}
In the editor's opinion, an authentic performance of a seventeenth-century
villancico would be one that is meaningful to present-day performers and their
audience, but which also opens a window to experiencing what made the piece
meaningful to its original performers and hearers.
Performers should seek out the distinctive character and significance of each
piece, but should also feel free to adapt the pieces to suit their own resources
and social context.
It would be better to have a spirited, respectful, musically sensitive
performance with modern instruments than to have no performance at all because
historic instruments were not available.

The sources for this edition are performing parts that, on the one hand,
were used as practical tools for performance in a particular place, and, on the
other hand, represent traditions of performance that cannot be completely fixed
in place or time.
In other words, even within one institution, such as the Conceptionist convent
in Puebla from which come the parts for Salazar's \worktitle{Angélicos coros}
and Cáseda's \worktitle{Qué música divina}, these parts were used and reused
possibly over generations. 
In some cases, later performers made corrections, added barlines, sewed in new
lines of lyrics or even new music to replace certain strophes.
There is no single way that these pieces were performed throughout their terms
of service as part of the local repertoire.

Moreover, these pieces represent single instances of a repertoire that
circulated around the globe. 
José de Cáseda lived in Zaragoza and set a poem by a composer from his same
region, Vicente Sánchez; but his setting is only known from the surviving parts
in the Puebla convent.
The spelling in those parts reflects New Spanish, not Zaragozan pronunciation
(e.g., \mentioned{consonancias} is spelled \mentioned{consonansias} in the
Puebla parts, even though the final C would probably have been pronounced like an
English TH sound in Zaragoza).
The piece may have been rearranged or adapted for female ensemble from a lost
original with different scoring.
On some occasions, a particular sister may have fallen ill and her vocal line
may have been played instrumentally.

The starting point for considering modern performance of these pieces, then, is
that historic performers made these pieces their own and performed them in a way
that fit their local needs in terms of personnel, instrumentation, rehearsal
time, acoustic space, and other factors; and in a way that was intelligible and
meaningful to them and to their hearers.

\paragraph{Ethical Responsibility}
While some amount of adaptation seems appropriate for this repertoire,
performers are urged never to lose site of the religious, social, and political
contexts of these pieces in their early modern origins.
These pieces cannot be cleanly separated from the social values of the colonial
era that this music both reflected and reinforced.
A piece like Juan Gutiérrez de Padilla's \worktitle{Al establo más dichoso}
bears the imprint of imperial Spain's racial hierarchy: it is documented that
the composer himself owned an Angolan slave, and the representation of
\quoted{Angolans} in the piece caricatures their bodies and voices as deformed
and deficient, even as it perhaps strives to present them in a sympathetic light
as offering devotion to Christ and joining with the angelic chorus.

It would be ethically irresponsibly to perform such a piece merely as an exotic
curiosity, or worse, as though it were a twenty-first century celebration of
ethnic diversity. 
Indeed, performers, scholars, and community members ought to engage in serious
discussions about what performing a piece might mean in a contemporary context.
In the right setting, such as a community workshop with appropriate
opportunities for critique, response, and discussion, the piece might be used 
effectively to raise issues of tremendous contemporary relevance; but in the
wrong context the piece could actually perpetuate the negative racial
stereotypes that are built into it.



\section{Select Bibliography}

\subsection{Studies of Villancicos}
\nocite{Rubio:Forma}
\nocite{Laird:VC}
\nocite{Torrente:PhD}
\nocite{Tenorio:SorJuana}
\nocite{CaberoPueyo:PhD}
\nocite{Illari:Polychoral}
\nocite{Knighton-Torrente:VCs}
\nocite{Cashner:Cards}
\printbibliography[heading=none,filter=villancico-studies]

\subsection{Musical Editions of Villancicos}
\nocite{Cererols:MEM-VC}
\nocite{Stevenson:Christmas}
\nocite{Ruimonte:Parnaso}
\nocite{Padilla:Tello}
\nocite{Ezquerro:MME55}
\nocite{RuizSamaniego:MME63}
\nocite{Ezquerro:MME59}
\nocite{Ezquerro:MME65}
\nocite{Fernandez:Cancionero}
\nocite{Torrejon:VCs}
\printbibliography[heading=none,filter=villancico-editions]

\subsection{Catalogs of Villancico Poetry Imprints}
\nocite{BNE:VCs17C}
\nocite{BNE:VCs18C}
\nocite{UK:VCs}
\nocite{US:VCs}
\printbibliography[heading=none,filter=villancico-imprint-catalogs]







% These all go in one chapter (command is in Cererols file)
\noteshead{Joan Cererols, \worktitle{Suspended, cielos, vuestro dulce canto}}

\begin{notesources}

\item[CAN]
\source{\signature{E-CAN}{AU/0116}}

\item[Bbc]
\source{\signature{E-Bbc}{M/765/25}}

\item[MEM]
\source{Modern edition: \fullcite{Cererols:MEM-VC}}
\end{notesources}

% \begin{criticalnotes}
% \end{criticalnotes}

\noteshead{Juan Gutiérrez de Padilla, \worktitle{Voces, las de la capilla}}

\begin{notesources}

\item[P.]
\source{\signature{MEX-Pc}{Leg.~3/3}, In manuscript performing partbooks, \worktitle{Navidad del año de 1657}}
\annotation{A 6/ Padilla}
\parts{SAT, ATB; B.~II is instrumental (\term{bajón} and other continuo instruments)}

\item[H.]
\source{Edition by Nelson Hurtado and Aurelio Tello, \autocite{Padilla:Tello}}

\end{notesources}

This piece is from the complete cycle of villancicos composed by Juan Gutiérrez de Padilla for the cathedral of Puebla de los Ángeles and performed at Christmas 1657.
The single source is a set of eight partbooks labeled \worktitle{Navidad del año de 1657} on the cover of each (the Tiple I partbook has the additional marking \quoted{en 8 quadernos}, confirming the total of eight notebooks.
The partbooks include all the villancicos needed for performance at Matins for Christmas and Epiphany of the 1657--1658 liturgical year, plus the hymn \worktitle{Christus natus est nobis}.

This villancico does not include the Tiple or Bassus of chorus II.
The other partbooks include the piece with the heading \quoted{A 6}.
Altus I and Tenor II include the composer's name, \quoted{Padilla}.

The bass part is in the partbook of Bassus, Chorus I, but this part plays with the voices of Chorus II throughout the piece.
Typical of Padilla's scores and many other contemporary villancicos, this part only includes brief textual incipits to help the player coordinate with the ensemble.
It is meant to be played on \term{bajón} and probably doubled with other continuo instruments.
As always, the other voices could also be doubled instrumentally.

The handwriting, ink, and paper are consistent with that used in Padilla's other extant Christmas cycles in the cathedral archive.
In each set between 1651 set and the last one in 1659, a year before Padilla signed a power-of-attorney document citing his failing health\autocite{Mauleon:PadillaCivil} (he died in 1664), there is a pronounced decline in the quality of the handwriting.
It seems reasonable to believe this to be Padilla's own hand.

The partbooks show signs of repeated use over many years: fingerprints, multiple performers' names written in the parts, corrections, and added barlines.
The barlines suggest that the pieces were still being performed into the eighteenth century, when performers were becoming less familiar with the old mensural notation.

This piece was previously edited by Nelson Hurtado and Aurelio Tello. 
That edition contains a serious error based on the misreading of rests and repeat signs, as discussed below. % XXX check

\subsection*{Structure of Repeated Sections}

% why respuesta is sung twice
% why estribillo is repeated after both coplas

\begin{criticalnotes}
1 & Ti. I & 
Spanish notation of the period lacks a natural sign, so sharps are used as naturals on Es and Bs. 
Here the E\sh{} tells the singer not to apply to \foreign{una nota super la} rule.\\
30 & Ti. I & 
Lyric underlay of \quoted{su-- a-- ve\undertie y} as three syllables is clear.\\
39 & T. II & Lyrics: \quoted{y aguarda} instead of \quoted{y aguardan} as in all the other parts.\\
40 & A. II & The second F would be natural by default in the MS; the natural is added in conformity with modern accidental notation.\\
50 & A. II & 
The first rest, a perfect semibreve (modern dotted whole note), was inadvertently omitted from the MS. 
The T. II and B. I, who have the same gesture as the A. II here, both have the missing rest. 
The rest may have been accidentally erased when another correction was made to the MS at the end of the preceding phrase.
Hurtado and Tello mistakenly delete the rest from T. II and B. I rather than adding it to A. I, producing a collision between choirs in the succeeding phrase.\\
72 & T. II & Last note is breve with fermata instead of semibreve.\\
73--74 & A. I &
Sharp signs on F in MS have been interpreted as cautionary accidentals---, as signs to the singer \emph{not} to follow \term{ficta} conventions of sharping the F in the gesture G--F--G. 
In other words, the sharps are used as natural signs.%
  \autocites{Harran:Cautionary1}{Harran:Cautionary2}. 
In m.~74, the Alto F must be natural to match the F\na\octave{5} in the Tiple.
By the same logic, the F in m.~73 should also be natural, and this avoids a B\fl{}/F\sh{} sonority, which would be unusual for Padilla.\\
74 & A. I & On the phrase \term{peregrino tono}, the Alto outlines the final cadence of the plainchant \term{tonus peregrinus} transposed in \term{cantus mollis}: G--B\fl{}--A--G.\\
86 & A. II & In a widespread convention, the slur is written over the group of notes without a clearly specified beginning and end, but the text underlay makes the slur placement clear. (Likewise in m. 55, T. II.)\\
\end{criticalnotes}



\noteshead{Miguel de Irízar, \worktitle{Si los sentidos queja forman del Pan Divino}}

\begin{notesources}

\item (P)
\source{\signature{E-SE}{5/32}, Manuscript performing parts (copyist's hand)}
\annotation{Al \oldabbrev{SS}{mo} a 8. Si los sentidos}
\parts{SSAT, SATB, \foreign{Acompañamiento}}

\item (S)
\source{\signature{E-SE}{18/19}, Manuscript draft score in Irízar's hand for Corpus Christ 1674 at Segovia Cathedral}
\annotation{Fiesta del SSantissimo de este año del 1674}
\parts{SATB, SATB, continuo only in coplas}

\end{notesources}

In a rare case of a surviving draft score of a villancico, Irízar composed the piece in one of his makeshift notebooks made from received letters, with the music on the backsides and in the margins of the letters.
The score (S) is drafted with written barlines every two \term{compases} both in duple and triple meter; when a single odd compás is left at the end of a section (e.g., \measure{17} in this edtion) Irízar groups it with what follows.
When a colored (imperfected) semibreve extends across a barline, Irízar centers the note on the line (since mensural notation did not allow for ties).

The performing parts (P) appear to be in the hand of a professional copyist, and correspond closely with the score.
The score agrees with the parts in pitches and rhythms in the estribillo, differing only in a few cases of accidentals, where \term{musica ficta} practice made the notation of some accidentals optional.

The score lacks the \term{General} continuo part in the estribillo.
In the coplas, Irízar originally composed separate four-voice settings of the first two coplas, but then at the bottom of the page drafted the setting for Tiple solo and continuo that appears in P, with a slightly different beginning to the continuo part.
It may have been a later idea to combine the solo setting with the end of the four-voice setting for the \term{Respuesta a las coplas} on the \soCalled{tag line}, \quoted{No se den por sentidos los sentidos}.
This is the first use of the continuo, suggesting that Irízar decided after composing the rest of the piece to add the continuo part (\term{General} in P).

\notesection{Lyrics}

In S, Irízar simply wrote the lyric text out in a single line underneath each system, with no text underlay in the individual voices.
Thus all text underlay in this edition is based on P.
P includes only textual incipits for the Bajo II and \term{General} parts, a common convention indicating that these were played instrumentally; the incipits are only there to help orient the player.
The Bajo II part would probably have been played on \term{bajón}, and the \term{General}, by a continuo ensemble including harp, plucked and bowed strings, and small organ.
Figures only appear in the coplas.

The lyrics correspond closely with those later attributed to Vicente Sánchez in the \worktitle{Lira Poetica} (Zaragoza, 1689), 171--172.
Irízar died in 1684, so he either had access to an earlier version of Sánchez's text through his correspondence network, or Sánchez's text is an improvement on a pre-existing poem that Irízar used.
Irízar does not include one of Sánchez's coplas and arranges the strophes differently.

The notated melody for the coplas does not fit every stanza equally well.
The singer was apparently expected to adapt the rhythm to fit the poetry for the subsequent stanzas.

\criticalnotesheader

\begin{criticalnotes}
7  & Ti. I-1 & C\sh\ in S only \\
9  & Ti. I-2, T. II & S has F\na\ in both voices; P has F\sh\ in T.~II only, possibly a cautionary accidental (i.e., performed as F\na)\\
12 & Ti. I-1, I-2 & Slur in P only\\
14 & A. I & C\sh\ in P only\\
27 & T. II & C\sh\ in P only\\
44 & T. II & F\sh\ in P only\\
49 & T. I & First three notes slurred in S\\
\end{criticalnotes}

\noteshead{Jerónimo de Carrión, \worktitle{Si los sentidos queja forman del Pan Divino}}

\notesource{E-SE:~28/25, Manuscript performing parts (\foreign{Solo}, \foreign{Acompañamiento})}

The lyrics correspond closely with those attributed to Vicente Sánchez in the \worktitle{Lira Poetica} (Zaragoza, 1689), 171--172.

The notated melody for the coplas does not fit every stanza equally well.
The singer was apparently expected to adapt the rhythm to fit the poetry for the subsequent stanzas.

As in many later seventeenth-century villancicos, there is no sign indicating that the estribillo should be repeated, as was customary with earlier villancicos.
The recurring tag line at the end of each copla may have been made it unnecessary to repeat the whole estribillo.

The suggested tempo relationship between the music in (C)Z meter and the music in C is only approximate. 
As with other Spanish villancicos from the later seventeenth century, the music in C seems to call for a slower tempo, with a feel closer to modern \musfig{4}{4}.
\noteshead{José de Cáseda, \worktitle{Qué música divina}}

\begin{notesources}

\begin{source}
\sourcedescription{\signature{MEX-Mcen}{CSG.256}, Manuscript performing parts from collection of the Convento de la Santísima Trinidad, Puebla}
\annotation{A 4/ \oldabbrev{D}{n} Joseph de Caseda}
\parts{SSATB; B. is instrumental}
\end{source}

\end{notesources}

Like Salazar's \worktitle{Angélicos coros} (in this edition), this piece is in the Colección Sánchez Garza at CENIDIM, the Mexican national music research center, in Mexico City.
The collection is originally from the Convento de la Santísima Trinidad, a Conceptionist convent in Puebla.
There are numerous works in the collection ascribed to José de Cáseda and his father Diego, who were both chapelmasters in Zaragoza.%
  \footnote{\autocites{Calahorra:Zaragoza2}{Stevenson:CasedaD}, and the relevant entries in the \worktitle{Diccionario de la música española e hispanoamericana}, and \autocite{Stevenson:CasedaD}.}

This is a set of individual performing parts. 
A tear along the fold at the bottom obscures a few of the notes.

The parts bear the names of the convent sisters listed below.
The name of the Alto, Madre Belona, also appears in Salazar's \worktitle{Angélicos coros}.

\begin{tabular}{ll}
Tiple 1 & Tomasita\\
Tiple 2 & María de Jesús\\
Alto & Madre Belona\\
\quoted{Thenor} & Rosa María de Jesús\\
Bajo & (no name)\\
\end{tabular}

The bass part is instrumental: it has only incipits of the lyrics and includes figured bass.
Given the piece's central conceit of Christ as a \term{vihuela}, that instrument would seem to be an apt choice for the continuo group.

\notesection{Coplas}

In the original version of this manuscript, coplas 1, 4, and 6 are scored for the full ensemble and the music for them is written out only once.
Coplas 2, 3, and 5 are sung by soloists with the same accompaniment part for each; thus the Bajo part for the solo strophes is only written out once in the MS.
The full-ensemble coplas actually require small adjustments for the different text underlay, so they are all written out in full in the edition.

The MS includes repeat signs after the first phrase in every copla. 
In the solo coplas, the accompaniment has the repeat sign placed one semiminim later than the vocal lines, indicating a \quoted{first ending}.
The edition simply writes out the repeated music in the solo coplas.

The Tenor part has an alternate setting of copla 4 sewn onto the paper.%
  \footnote{Many of the pieces in the Sánchez Garza collection have alternate versions, most commonly of lyrics, sewn or pasted in. These alterations provide evidence for the repeated use of these pieces for varying occasions and according to changing aesthetics and devotional needs.}
By lifting the sheet it is still possible to see most of the original setting of the solo copla 5, except for a few passages obscured by the stitching at the top.
The music for copla 5, as edited here, appears to be identical to that for copla 2; the obscured passages are indicated with brackets in the edition.

The alternate setting uses a G2 (treble) clef instead of the C3 clef of the Tenor part, and duplicates the Tiple 2 line in the original full-ensemble setting of copla 4, except that the repeated high G\octave{5} in \measures{107--108} is replaced with D\octave{5}.
This version appears to be intended as a solo setting of copla 4, with the female Tenor switching to a higher register for this copla. 
Perhaps it was used in an abridged version of the piece with fewer coplas, or in a version arranged for reduced voices.


\notesection{Solecisms}

In \measures{3--4}, there are parallel fifths in the outer voices, which cannot be avoided through \term{musica ficta} or any simple editorial correction. 
This is either intended by the composer, or was copied incorrectly.
If intentional, it may be an error or an aspect of personal style; but given that the lyric here is \gloss{acorde}{tuneful} this may be a deliberately ironic gesture.

In \measures{44-45}, there appears to be a cross-relation between the Tiple 2 (B\fl) and the Tenor (B\na).
On its own, the Ti. 2 would sing all B flats in this phrase, as no accidentals are included.
To match the motive used throughout this section, though, the Ti. 2 would sing B flats in \measure{44} and then B natural in \measure{45}.

But the Tenor has a sharp on the B in \measure{44}, normally indicating B natural; this would produce a cross relation.
The Tenor B sharp is probably not a cautionary accidental because the phrase would not normally call for a \term{ficta} alteration.

Thus the most likely solution seems to be (as indicated in the editorial accidentals) for the Ti. 2 to break the motive and sing all B naturals.
The possibilities here are similar to the parallel fifths in the opening phrase: either they are intentional solecisms or they are simple errors of copying or composition.

The condition of the manuscript indicates frequent use, so any solution to these problems must account for the fact that the sisters actually performed the piece from these manuscripts, in their current form.

\criticalnotesheader
\begin{criticalnotes}
3--4 & Ti. 1, B. & See discussion of solecisms above.\\
25 & Ti. 1 
  & The last four notes are beamed together in the MS, but the text underlay requires two groups of two.\\
37 & All 
  & The tempo marking \mentioned{aspacio} occurs in all the parts except the Ti. 1 (it is written \mentioned{espacio} in the T.)\\
44--45 & Ti. 2, T. & See discussion of solecisms above.\\
45 & B. & Notes nearly destroyed by tear in MS, but the left side of the G and right side of the C are still visible on either side.\\
\end{criticalnotes}
  


\noteshead{Juan Gutiérrez de Padilla, \worktitle{Al establo más dichoso}}

\begin{notesources}

\begin{source}
\sourcedescription{\signature{MEX-Pc}{Leg.~1/3}, In manuscript partbooks, 
\worktitle{Navidad del año de 1652}}
\annotation{Ensaladilla}
\parts{SATB, SATB; both bass parts are instrumental, with indications for 
\term{bajón}}
\end{source}

\end{notesources}

This piece is part of Padilla's cycle of villancicos for Christmas 1652 at 
Puebla Cathedral, copied as a set into individual partbooks, possibly in 
Padilla's own hand.%
    \Autocites{Cashner:PadillaRhythm-AMS}[406--462]{Cashner:PhD}
This is the earliest of this composer's extant Christmas cycles for which all 
the partbooks survive.

The partbooks bear the names of some of the Puebla chapel performers in various 
places.
The name of Francisco Rodríguez is in the Tiple I part, and that of Sr. Nicolás 
Griñón is in the Tenor II.

The Puebla cathedral chapel was usually organized in two choirs, but since this 
piece does not utilize polychoral textures, the edition presents the voices in 
a single-choir layout.

This musical \quoted{salad} features multiple sections with contrasting styles
with the names of songs or dances, which were likely based on specific music or
types of music already known to the audience.
Because of these multiple sections and large amount of repetition within 
sections, the parts are written in an abbreviated manner.
This edition writes out most of the reprises and other repeated material to 
achieve a more straightforward presentation for performers.
Examples of this include the stanzas of the \term{Nuevo Troyano} and the 
\term{responsión} reprise of the \term{Papalotillo}.

The piece has been recorded once prior to this edition.%
  \autocite{Padilla:1652ChristmasCD}

\notesection{Bass Parts}

Both bass parts are intended for instrumental performance.
They have only incipits of the text to help orient the player.
The Bassus I part contains this marking after the \term{Nuevo Troyano} and 
before the \term{Papalotillo}: \quotedgloss{antes del papalotillo diçe el 
harriero con el otro bajon}{Before the \term{papalotillo} the mule skinner 
\quoted{speaks} with the other \term{bajón}}.
This implies that both bass parts were played on the \term{bajón}, not to 
exclude other continuo instruments like harp.
The sections labeled \term{Dúo} (the \term{Arriero} and the beginning of the 
\term{Negrilla}) are actually, in modern terms, vocal solos with accompaniment, 
perhaps intended for a single vocalist with solo \term{bajón}.

If a vocal solo with \term{bajón} is a \term{Dúo}, then the section marked 
\term{Papalotillo Solo} would seem to be a true solo without any accompaniment 
at all.
The scribe has only written a four-bar accompaniment pattern in the bass, with 
unspecific indications to repeat.
This edition therefore includes the bass line for the \term{Papalotillo} only 
for the \term{responsión}.
But it is also plausible that the bass should repeat the same phrase as 
accompaniment for the coplas.

\notesection{The \soCalled{Gloria}}

In the midst of the \term{Negrilla}, which caricatures people of African
descent, an \soCalled{Angolan} character (T.~II) sings, \quoted{Listen, for we
are singing like the angels}.
Then the two upper voices of the first chorus (Tiple and Altus I) sing the 
angels' song from Luke~2 in Spanish, \quoted{Gloria en las alturas y en la 
tierra, paz}, continuing in the ternary meter of the preceding section.
This section is labeled \quoted{A 3} in the Tiple I part.

The Tiple II and Altus II parts contain only one phrase of notated music for 
the \term{ensaladilla}.
Both are labeled \quoted{A 3 de la ensaladilla}, and contain music for the 
Spanish \quoted{Gloria}, but in C meter instead of the CZ meter of the other 
voices.
The Tiple II melody quotes a common plainchant intonation of 
the \worktitle{Gloria in excelsis} of the Mass.

The Tiple II part actually includes an earlier version in CZ that has been 
crossed out and replaced with one in C.
The only way to align these voices with those of Chorus I is to maintain the 
theoretical $3:2$ proportion of minims between CZ and C meter so that the 
perfect semibreve in CZ is equal in time to the semibreve in C.

The reason for the marking \quoted{A 3}, when there is notated music for four 
voices, is unclear.
Given the crossed-out and corrected music in the Tiple II, it is also possible 
that \quoted{A 3} functions primarily as a rehearsal marking, and that Padilla 
changed his mind about the scoring after writing out the Chorus I parts, but 
left the marking intact.

\criticalnotesheader
\begin{criticalnotes}
    108, 112, 132, 136
    & T.~I
    & Semibreve--minim 
    & As written; but possibly an error; cf. other voices minim--semibreve
    \\

    113--128
    & B.~I 
    & Tacet 
    & MS provides only phrase in mm. 105--12, without clearly indicating
    repeats; that phrase could be repeated to accompany the coplas 
    \\
    
    129--136
    & Chorus I, all 
    & Responsión reprise after each pair of coplas
    & Repeat structure unclear; it is possible the \term{responsión} is only
    meant to be reprised after the final copla
    \\
    
    136
    & All 
    & Fermata, all voices 
    & Fermata only in T. I; given the following text (\quoted{Hush!}) it is
    possible but speculative (as in the recording by the Angelicum de Puebla) that 
    this voice only is meant to hold past the cutoff of the other voices
    \\
    
    215--223
    & All 
    & Polymetric Gloria, C meter vs. CZ
    & As written, with heading \quoted{A 3} of unclear meaning; see discussion
    of \soCalled{Gloria} above 
    \\
\end{criticalnotes}

\noteshead{Antonio de Salazar, \worktitle{Angélicos coros con gozo cantad}}

\begin{notesources}

    \begin{source}
        \sourcedescription{\signature{MEX-Mcen}{CSG.256}, Manuscript performing parts 
        from collection of the Convento de la Santísima Trinidad, Puebla}
        \annotation{A 8 de Navidad/ \oldabbrev{M}{o} Salazar}
        \parts{SSA, SATB, \term{Guión}; B. II specified as \term{Órgano}}
    \end{source}

\end{notesources}

Like Cáseda's \worktitle{Qué música divina}, this piece is in the Colección 
Jesús Sánchez Garza at CENIDIM, from the Convento de la Santísima Trinidad in 
Puebla.%
    \Autocite[123--125]{Cashner:PhD}
The collection includes many works by Salazar, who may have been trained in 
Puebla before becoming chapelmaster at Mexico City Cathedral.%
  \Autocites{Koegel:Salazar}[109--157]{Goldman:Responsory}
The performing parts bear the names of the convent sisters who performed them. 
The name added in a later hand to the Tiple I-1 part, \quoted{Belona}, would 
appear to be the same person who performed the Alto in Cáseda's \worktitle{Qué 
música divina} (in this edition).

\begin{tabular}{lll}
    Tiple & I-1 & Madre Andrea, [different hand:] belona\\
    Tiple & I-2 & Madre Assumpsion\\
    Alto & I & Madre Sacramento\\
    Tiple & II & Madre Thomasa, [on verso, different hand:] Alphonsa de 
    \oldabbrev{S}{ta} crus [cruz]\\
    Alto & II & Madre Rosa\\
    Tenor & II & Ynesica Baeza\\
    Bajo & II & Madre Mariana\\
\end{tabular}

The \term{Guión} part is in a different, less mature, hand than the others. 
It is written in portrait orientation rather than landscape, and includes 
barlines in the C meter sections.
The part may have been recopied at a later date to replace a worn original.

\criticalnotesheader
\begin{criticalnotes}
    67
    & Ti. II, n. 2
    & \pitch{F}{4}
    & \pitch{G}{4}; cf. other voices F (correction)
    \\

    68
    & Guión 
    & Fermata 
    & No fermata
    \\
   
    75--79
    & A. I 
    & \foreign{derribado busca, viena a edificar}
    & \foreign{lo derribado busca quien viene a edificar}; too many syllables for
    the notes (correction)
    \\
\end{criticalnotes}


% These all go in one chapter (command in Cererols file)
% Suspended, cielos, vuestro dulce canto
% Critical poetry edition
% New version 2016/06/08

\edchapter{Texts and Translations}

\begin{poemtitleblock}
\poemhead{%
    \worktitle{Suspended, cielos, vuestro dulce canto} (Montserrat,
    \circa{1660})%
}
\poemsource{%
    Anonymous text from setting by Joan Cererols (\signature{E-CAN}{AU/0116});
    variant versions in seven poetry imprints after 1651%
}
\end{poemtitleblock}

\begin{poemtranslation}
\begin{original}

\StanzaSection{13}[\add{Estribillo}]
Suspended, cielos, &
vuestro dulce canto; &
tened, parad, escuchad &
la más nueva consonancia &
que forman en su distancia &
lo eterno y lo temporal. &
Escuchad, &
que entonan las jerarquías &
en sonoras armonías &
contrapunto celestial. &
Y con sollozos tiernos &
\critnote{un niño soberano}
  {In place of ll.~11--12, the Eucharistic \worktitle{Bbc} version has 
    \quotedgloss{y desde un pan divino/ un hombre soberano}%
    {and through divine bread, a sovereign man}.} &
a los ángeles lleva el canto llano.
\SectionBreak

\StanzaSection{4}[Coplas]
1. \critnote{Las fugas que el}
  {\worktitle{CAN} has \textquote{Las fugas del}, but all the poetry 
    imprints have \textquote{que el}.}
     primer hombre &
formó en desatentos pasos &
al compás ajusta un Niño &
de las perlas de su llanto. \&

\Stanza{4}
2. \critnote{Qué mucho si}
  {Corrected after poetry imprints; 
    \worktitle{CAN} has \textquote{Qué mucho que}.} 
a los \critnote{despeños}
{Probably a musical term (for ornamentation?).} &
que le ocasionó un engaño, &
bella corriente de aljófar, &
\critnote{grillos le previene blandos}
{Translation uncertain.}. \&

\Stanza{4}
3. Una voz que ha dado el cielo, &
de metal más soberano &
a ordenar entra sonora &
la disonancia del barro. \&

\Stanza{4}
4. Concierto tan soberano &
sólo pudo ser reparo, &
con una voz tan humilde, &
de \critnote{un desentono tan vano}
  {\worktitle{CAN}: Tiple I-1 has \quotedgloss{desatento}{inattentiveness} 
    instead of \quotedgloss{desentono}{untunefulness}; both vocal parts 
    have \quoted{tan grande} instead of the metrically correct 
    \quoted{tan vano} in the poetry imprints.}. \&

\Stanza{4}
5. En las pajas \critnote{sustenido}
  {\worktitle{CAN}: Tiple I-1 and 2 have 
    \quotedgloss{susteniendo}{sustaining/sharping}; but Altus I and Tenor I 
    have \textquote{sustenido}, in agreement with the poetry imprints.} &
dulcemente se ha escuchado &
ligar en pajas lo eterno, &
reducir \critnote{lo inmenso a espacio}
  {All the poetry imprints have this text; the \worktitle{CAN} partbooks 
  have \textquote{lo inmenso spacio}, most likely a contraction for the same.}. \&

\Stanza{4}
6. Divina cláusula sea &
deste eterno canto llano, &
que forma en su movimiento &
de cada punto un milagro. \&

\end{original}

\begin{translation}
\StanzaSection{13}
Suspend, O heavens, &
your sweet chant. &
Hold, stop, and listen &
to the newest consonance &
that the eternal and the temporal &
are forming in their distance. &
Listen, &
for the hierarchies are intoning &
in resounding harmonies &
celestial counterpoint. &
And with tender sobs, &
a sovereign baby boy &
bears the plainsong to the angels. \&

\Stanza{4}
1. The flight/fugue that the first man &
made in heedless paces &
is set aright by a baby boy to the measure &
of the pearls of his crying. \&

\Stanza{4}
2. What wonder, if from the falls &
that a deceit caused him, &
the lovely mother-of-pearl stream &
gently restrains him with shackles. \&

\Stanza{4}
3. A voice that heaven has given, &
of the most sovereign timbre, &
to bring order, enters resounding &
into the dissonance of the clay. \&

\Stanza{4}
4. So sovereign a concord/concerto & 
could only be a resolution, &
with so humble a voice, &
of so vain a discord. \&

\Stanza{4}
5. Upon the straw \critnote{sustained}
  {Musically, \quoted{sharp}.} &
sweetly he has been heard &
\critnote{binding}{Musically, \quoted{tying} or forming a ligature.} 
  in straw the eternal, &
reducing the immense \critnote{to this space}
  {Musically, \quoted{slowly}.}. \&

\Stanza{4}
6. Let there be a divine cadence &
of this eternal plainsong, &
which forms in its movement &
a miracle from each \critnote{note}{Literally, \quoted{point}.}. \&

\end{translation}
\end{poemtranslation}


% Voces las de la capilla
% Critical poetry edition
% New version 2016/06/03

\begin{poemtitle}
\poemhead{\worktitle{Voces, las de la capilla} (Puebla, 1657)}
Anonymous, from musical setting by Juan Gutiérrez de Padilla, \worktitle{Navidad del año de 1657} (\signature{MEX-Pc}{Leg.~3/3})
\end{poemtitle}

\begin{poemtranslation}
\begin{original}

\StanzaSection{6}[\add{Introducción}]
1. Voces, las de la capilla, &
\critnote{cuenta}{Pay attention to.} con lo que se canta, &
que es músico el rey, y \critnote{nota}{Takes note of.} &
las más leves disonancias &
a lo de Jesús infante &
y a lo de David monarca.
\SectionBreak

\StanzaSection{4}[Respuesta]
Puntos ponen a sus letras &
los siglos de sus hazañas. &
La clave que sobre el hombro &
para el treinta y tres se aguarda.
\SectionBreak

\StanzaSection{6}[\add{Introducción} cont.]
2. Años antes la divisa, &
la destreza en la esperanza, &
por sol comienza una gloria, &
por mi se canta una gracia, &
y a medio compás la noche &
remeda quiebros del alba
\SectionBreak[\add{Respuesta rep.}]

\StanzaSection{15}[\add{Estribillo}]
Y a trechos las distancias &
en uno y otro coro, &
grave, suave y sonoro, &
hombres y brutos y Dios, &
tres a tres y dos a dos, &
uno a uno, &
y aguardan tiempo oportuno, &
quién antes del tiempo fue. &
Por el signo a la mi re, &
puestos los ojos en mi, &
a la voz del padre oí &
cantar por puntos de llanto. &
\hphantom{uno a uno,} ¡O qué canto! &
tan de oír y de admirar, &
tan de admirar y de oír. \&

\Stanza{2}
Todo en el hombre es subir &
y todo en Dios es bajar.
\SectionBreak

\StanzaSection{4}[Coplas]
1. Daba un niño peregrino &
tono al hombre y subió tanto &
que en sustenidos de llanto &
dió octava arriba en un trino. \&

\Stanza{4}
2. Hizo alto en lo divino &
y de la máxima y breve &
composición en que pruebe &
de un hombre y Dios consonancias. \&

\end{original}

\begin{translation}
\StanzaSection{6}
1. Voices, those of the chapel choir &
keep count with what is sung, &
for the king is a musician, and notes &
even the most venial dissonances, &
in the manner of Jesus \critnote{the infant prince}{\term{Infante} means both infant and prince.}, &
as in the manner of David the monarch. \&

\StanzaSection{4}
The centuries of his heroic exploits &
are putting notes to his lyrics. &
The \critnote{key}{Or clef.} that upon his shoulder &
awaits the thirty-three. \&

\StanzaSection{6}
2. Years before the sign, &
\critnote{dexterity in hope}
  {In Golden Age literature \term{destreza} connotes heroic skill in combat, particularly in \term{esgrima} or sworsdmanship. 
  Musically, the term suggests virtuosity. 
  The whole phrase sounds like a heraldic device (\term{divisa}) or motto, summing up Christ's mission.} &
\critnote{with the sun}
  {Here begins a series of musical plays on words: \term{sol} and \term{mi} are solmization syllables with double meanings; \term{gloria} and \term{gracia} probably refer to the songs of Christmas in both history and liturgy like the \term{Gloria in excelsis}.}
  [on \term{sol}] a \textquote{glory} begins, &
upon me [\term{mi}] a \textquote{grace} is sung, &
and at the half-measure, the night &
imitates the trills of the dawn. \&

\StanzaSection{15}
And from afar, the \critnote{intervals}
  {Both musical intervals and astronomical distances between planetary spheres.} &
in one choir and then the other, &
solemn, mild, and resonant, &
men, animals, and God, &
three by three and two by two, &
one by one, &
they all await the opportune time, &
the one who was before all time. &
Upon the sign of \term{A (la, mi, re)}, &
with eyes placed on me [\term{mi}] &
at the voice of the Father I heard &
singing in tones of weeping--- &
\hphantom{one by one,} Oh, what a song! &
as much to hear as to admire, &
as much to admire as to hear! \&

\Stanza{2}
Everything in Man is to ascend &
and everything in God is to descend. \&

% COPLAS
\StanzaSection{4}
1. A baby gave a \critnote{wandering song}
  {Or \textquote{pilgrim song}, or the musical \term{tonus peregrinus}.} &
to the Man, and ascended so high &
that in \critnote{sustained weeping}
  {Musically, \textquote{sharps of weeping}} &
\critnote{he went up the eighth \add{day} into the triune.}
  {Musically, \textquote{he went up the octave in a trill.}} \&

\Stanza{4}
2. From \critnote{on high}
  {\term{Alto} also denotes the musical voice part.} in divinity, &
\critnote{of the greatest and least}
  {A play on the name of very long and short music notes.}, &
he made a composition in which to \critnote{prove}{Or \textquote{test}.} &
the consonances of a Man and God. \&
\end{translation}
\end{poemtranslation}


\include{poems/Si_los_sentidos-Sanchez}
% Qué música divina (Cáseda)
% Critical poetry edition
% 2016-06-01  New version begun

\begin{poemtitleblock}
\poemhead{\worktitle{Qué música divina} (Zaragoza and Puebla, \circa{1700})}
\poemsource{Anonymous text from setting by José de Cáseda (\signature{MEX-Mcen}{CSG.256}); coplas attrib. Vicente Sánchez, \worktitle{Lyra poética} (Zaragoza, 1688), 191}
\end{poemtitleblock}

\begin{poemtranslation}
\begin{original}
\StanzaSection{10}[\add{Estribillo}]
Qué música divina, &
acorde y soberana &
afrenta de las aves &
con tiernas, armoniosas consonancias, &
en quiebros \critnote{suaves, sonoros y graves}
  {Cf. \worktitle{Voces, las de la capilla} (Padilla, in this edition), \quoted{grave, suave y sonoro}.}%
  , &
acordes accentos &
ofrece a los vientos &
y en cláusulas varias &
sentidos eleva, &
\critnote{potencias}
  {Powers or faculties of the \gloss{anima sensitiva}{sensitive soul}, such as the intellective, cogitative, imaginative factulties, and the memory.
  See \fullcite[439--484]{LuisdeGranada-Balcells:SimboloPtI}.}%
   \ desmaya.
\SectionBreak

\StanzaSection{4}[Coplas]
1. Suenen las dulces cuerdas &
de esa \critnote{divina cítara y humana}
  {The central conceit of the coplas connect Christ, in his Passion, to a string instrument. 
  The Spanish \term{vihuela} is linked to an older symbolic tradition of the \term{cítara} and the \term{lira}.
  The seven- and eleven-syllable lines throughout may allude to the poetic form of the \term{lira}, which consists of particular combinations of such lines, even though this poem does not follow the typical \term{lira} form.}%
  , &
que \critnote{aún sol}
  {Sánchez edition has \gloss{un son}{a sound}.}
    que es de los cielos, &
forma unida la alta con la baja. \&

\Stanza{4}
2. De la fe es instrumento &
y al oído su música regala &
donde hay por gran misterio &
en cada punto entera consonancia. \&

\Stanza{4}
3. De el lazo a este instrumento &
sirve la unión que sus extremos ata: &
tres clavos son clavijas &
y puente de madera fue una tabla. \&

\Stanza{4}
4. Misteriosa vihuela, &
al \critnote{herirle}
  {Sánchez: \mentioned{herirla}.} 
    sus cuerdas una lanza, &
su sagrada armonía &
\critnote{se vió allí}
  {Sánchez omits \mentioned{allí}, preserving the pattern of eleven-syllable lines.}\
  \critnote{de siete órdenes formada}
    {The seven-course \term{vihuela} as metaphor for the seven sacraments, signified by the blood and water coming from Christ's pierced side (John 19:34).}%
    . \&

\Stanza{4}
5. No son a los sentidos &
lo que suenan sus voces soberanas &
porque de este instrumento &
\critnote{cuantas}
  {Sánchez: \mentioned{quantos}.}
     ellos percibían serían \critnote{falsas}
      {The term may refer to notes that are out of tune, out of temperament, incorrect, or that use \term{musica ficta} accidentals.}%
        . \&

\Stanza{4}
6. Su primor misterioso, &
que a los cielos eleva al que \critnote{lo}{Sánchez: \mentioned{le}.} alcanza &
no lo come el sentido &
porque es pasto su música del alma. \&
\end{original}

%**** TRANSLATION ****
\begin{translation}
\StanzaSection{10}
What divine music, &
tuneful and sovereign, &
rivals that of the birds &
with tender, harmonious consonances, &
in trills mild, sonorous and solemn; &
it offers tuneful accents &
to the winds, &
and in varying cadences &
elevates the senses, &
confounds the \add{mind's} powers. \&

\StanzaSection{4}
1. Let the sweet strings sound &
of that divine and human \term{cithara}, &
who, the very sun/\term{sol} who is in the heavens, &
forms the high \add{string} and the low in unity. \&

\Stanza{4}
2. Of faith he is the instrument, &
and his music regales the ear &
when, by a great mystery, there is &
in every point a perfect consonance. \&

\Stanza{4}
3. Serving as the string on this instrument &
is the union that ties together his extremes: &
three nails are the pegs &
and a crossing of wood was a soundboard. \&

\Stanza{4}
4. Mysterious \term{vihuela}, &
when a lance wounded/plucked your strings, &
your sacred harmony & 
was seen there, formed of seven orders. \&

\Stanza{4}
5. They are not for the senses, &
that which your sovereign notes sound, &
for, of this instrument &
as many notes as they perceived will be false. \&

\Stanza{4}
6. Your mysterious virtuosity, which &
elevates to the heavens the one who achieves it: &
sensation does not eat it, &
for your music is fodder for the soul. \&

\end{translation}
\end{poemtranslation}

% Al establo más dichoso
% Critical poetry edition
% New version 2016/06/17

\begin{poemtitle}
\poemhead{\worktitle{Al establo más dichoso} (Puebla, 1652)}
Anonymous, from musical setting by Juan Gutiérrez de Padilla, \worktitle{Navidad del año de 1652} (\signature{MEX-Pc}{Leg.~1/3})
\end{poemtitle}

\begin{poemtranslation}
\begin{original}

\StanzaSection{4}[\add{Prologue} a 4]
Al establo más dichoso, &
donde triunfa la victoria, &
principio a siglos de gracia, &
la noche más venturosa, \&

\Stanza{4}
\critnote{Buena noche y la más buena}
  {\mentioned{La Nochebuena} is Christmas Eve.}, &
pues a pesar de las sombras &
en su mitad amanece &
quién con tanta luz entolda. \&

\Stanza{4}
Un zagal de aquel contorno, &
en su templada zampoña, & 
tocando el Nuevo Troyano, & 
cantó en la pajiza choza:
\SectionBreak

\StanzaSection{4}[\add{Nuevo Troyano} Solo y a 4]
En Belén cantando están, &
todo es gloria, todo es cielo, &
y en un portalico pobre &
se ha estrechado él que es inmenso. \&

\Stanza{4}
Fuego derrite la nieve, &
y entre tanta nieve el fuego &
a cada llama bosteza, & 
lo acendrado deste estremo. \&

\Stanza{4}
Míranse por todos lados, &
en cada paja un lucero, &
una antorcha a cada viso &
y un Dios grande aunque pequeño.
\SectionBreak

%*** ARRIERO *****
\StanzaSection{4}[\add{Prologue a 4}]
Después Bartholo, él de marras, &
arriero \critnote{de cala y gorra}
  {The manuscripts have \foreign{cala} clearly, but the meaning would be unclear. Possibly a mistake for \gloss{de capa y gorra}{in plain, simple clothes} (RAE).} &
que fue espadachín de antaño, &
y hoy mercader de \critnote{panochas}
  {In Mexican Spanish, slabs of hard brown sugar or candies made from them (RAE). Cf. English \metioned{penuche}, derived from this usage (OED).}, \&

\Stanza{4}
En busca de una mulilla &
que se le fue por \critnote{tramoya}
  {Scheme, or a piece of stage machinery.}, &
a darse una buena noche & 
en las pajas misteriosas. \&

\Stanza{4}
Al portal con los pastores & 
se entró arrojando bramonas &
y a quién ocupa el pesebre, &
dice como que se entona:
\SectionBreak

%*******
\StanzaSection{4}[El Arriero: \critnote{Responsión a Dúo}{Scored for solo Tenor with \term{Bajón} accompaniment.} \add{Solo, Ac.}]
1. \critnote{Señor niño, voto a San}
  {Bartholo refers to the Christ-child as \quoted{Sir} or \quoted{Lord}, and addresses him with formal \foreign{Usted} forms, but in the same breath begins to curse.}--- &
ya lo dije, y esto sobra &
para que entienda que vengo &
puesto a lo de aquí fue \critnote{Troya}
  {Idiom for a disastrous mess; with double meaning here since Bartholo is brought unawares into the manger by his mule; also reference back to the \quoted{Nuevo Troyano} just performed, with its references to fire.}. \&

\Stanza{4}
2. No se me asuste le digo &
ni de inocente se ponga, &
cuando me dicen que sabe &
lo que su padre no ignora. \&

\Stanza{4}
3. Es bueno que de mis mulas, &
la más lucia y la más gorda &
me la traiga a este pesebre &
sin decir esta es mi boca. \&

\Stanza{4}
4. Y yo sin haber vendido &
las cargas de mis melcochas, &
\critnote{ande en flores y con flores}
  {\foreign{Andar en flores} is an idiom for refusing to get into an argument; \foreign{con flores} might refer to flowers Bartholo sells from his cart.} &
pregonándola a mi costa. \&

\Stanza{4}
5. Si \critnote{arrobar}
  {Play on \gloss{a robar}{to steal}.}
    viene a los hombres, &
paréceme cosa impropia &
dar principio con mi mula, &
si no ha de ocupar carroza. \&

\Stanza{4}
6. Pero ya he considerado, &
si mi decir no le enoja, &
que por la escarcha pretende &
el aliento de su boca. \& 

\Stanza{4}
7. Y por vida de Bartholo, &
que en aquestas y en esotras, &
cuando por esto la quiera, &
que aquí se las traiga todas. \&

\Stanza{4}
8. Abra esa boca de perlas &
con que tanto me enamora, &
y pida que \critnote{estos serranos}
  {The next group to perform.} &
no pretenden otra cosa. \&

\Stanza{4}
9. Un baile quieren hacerle, &
que \critnote{\term{papalotillo}}
  {Diminutive of \gloss{papalote}{kite or paper toy}, derived from Nahuatl \gloss{papalotl}{butterfly} (RAE).}
   nombran &
y como cantemos todos, &
más que rueden las panochas. 
\SectionBreak

%*** PAPALOTILLO *****
\StanzaSection{2}[Papalotillo: Solo]
Ven y verás un donoso chiquito. &
Míralo bien, que en sus ojos me miro.
\SectionBreak

\StanzaSection{2}[Responsión a 4]
Ven y verás un donoso chiquito. &
Míralo bien, que en sus ojos me miro.
\SectionBreak

\StanzaSection{2}[Coplas]
i. Míralo bien, como llora y suspira, &
siendo del padre la misma alegría. \&

\Stanza{2}
ii. Míralo bien entre pobres alajas, &
grano fecundo escondido entre pajas. \&

\Stanza{2}
i. Míralo bien que aunque agora se estrecha, &
nos ha de dar una fértil cosecha. \&

\Stanza{2}
ii. Míralo bien con terneza y cuidado, &
que ha de ser pasto y pastor desvelado. \&

\Stanza{2}
i. Míralo bien, corderito amoroso, &
que ha de huir de las garras del lobo. \&

\Stanza{2}
ii. Míralo bien, pequeñito pastor, &
pues cuando grande será labrador.
\SectionBreak

\StanzaSection{2}[Responsión a 4]
Ven y verás un donoso chiquito. &
Míralo bien, que en sus ojos me miro.
\SectionBreak


%*** NEGRILLA *****
\StanzaSection{4}[\add{Prologue a 4}]
\critnote{El Angola Minguelillo}
  {Diminutive of Miguel, identified as an African of the Angolan \soCalled{nation} or brand, likely a slave.} &
acaudillando su tropa, &
no quiere ser el postrero &
en la fiesta en que se goza. \&

\Stanza{4}
Dejando \critnote{el tumba catumba} 
  {Apparently a nonsense word, possibly imitating African drumming and the sounds of Angolan languages like Kikongo.
  Cf. the refrain of Padilla's 1651 \term{ensaladilla}, \quoted{Tumbu cutu, cutu, cutu}.} &
y gruñendo a lo de Angola, &
desenvainó con la voz, &
\critnote{de su tizón la tizona.}[desenvainó \dots\ de su tizón la tizona]
  {Mocking the voice and singing of this African character. 
  \foreign{Tizona} means sword (after the Cid's weapon), playing on the idea of Minguelillo leading a quasi-military \soCalled{troop}; \foreign{tizón} means a charred log or piece of coal, referring to Minguelillo's dark-skinned, muscular neck, and to the perceived dark, gravelly sound of his voice.}
\SectionBreak

\StanzaSection{4}[Negrilla: \add{Introducción} Dúo y a 6]
i. Diga plimo donde sa? &
la niño, de nacimenta &
pluque samo su palenta &
y la venimo a \critnote{busca}[Diga plimo \dots\ busca]
  {Pseudo-African dialect Spanish, original orthography and punctuation. 
   Possible equivalent in proper Spanish: \quoted{Diga primo, ¿dónde está/ el niño de nacimiento?/ porque sabemos sus parientes/ y lo venimos a buscar}}. \&

\Stanza{5}
ii. \critnote{Ayta}
  {Also written in MS as \foreign{aytá} and \foreign{a\'yta}; probably for \gloss{ahí está}{there he is}, answering the question \gloss{donde sa?}{where is he?}.}%
    , ayta, &
cundiro entle pajita &
su ojo como treyita &
y uno buey y uno mulita &
con su baho, \critnote{cayenta.}[Ayta, \dots\ cayenta]
  {\quoted{Ahí está,/ candela entre pajitas,/ su ojo como estrellita,/ y un buey y una mulilla/ con su bajo callentar}.}
\SectionBreak

\StanzaSection{2}
Turu turu yega, &
ayta \critnote{ayta}[Turu turu yega,/ ayta, ayta]
  {Possibly, \quoted{Todos, todos llegan,/ ahí está}; or pseudo-African nonsense}.
\SectionBreak

\StanzaSection{4}
Caya, caya, mila no panta &
que duelme la siguetito. &
Sesu, Sesu, que bonito, &
\critnote{sucucha}
  {Accentuated in the setting as \term{sucuchá}.}%
    , que cantamo lo \critnote{angelito:}[Caya, \dots\ lo angelito]
  {Possible equivalent: \quoted{Calla, calla, mira, no le espanta,/ que duerme el chiquitito,/ Jesús, Jesús, qué bonito,/ esuchar, que cantamos a lo del angelito} or \quoted{a lo angélico}.}
\SectionBreak

\StanzaSection{1}[\critnote{A 3}
  {Two additional voices join here in a contrasting musical meter, with music based on the plainchant \worktitle{Gloria in excelsis}.}]
Gloria en las alturas y en la tierra paz.
\SectionBreak

\StanzaSection{4}[\add{Estribillo a 6}]
\critnote{Valamindioso que lindo canta}
  {Dubious possible equivalent: \quoted{Para mi Dios, O qúe lindo cantar}.}, &
ayta, ayta, &
sucucha, sucucha, &
ayta, ayta, ayta.
\SectionBreak

\StanzaSection{5}[Coplas a 6]
1. Caya, caya, chiquito, \emph{ayta}. &
Que tlaemo plecente, \emph{ayta}. &
Mantiya pañalito, \emph{ayta}. &
Y uno \critnote{papagayito}
  {\foreign{Flor de Nochebuena}, poinsettia, native to Central America.}, \emph{ayta}. &
Que savemo \critnote{habra}[Caya, \dots\ savemo habra]
  {\quoted{Calla, calla, chiquito,/ que traemos un presente,/ una mantilla, un pañalito,/ y un papagayito,/ que sabemos habrá}.}.
\SectionBreak[\add{Repeat negrilla estribillo}]

\Stanza{6}
2. Mi siñol Manuele, \emph{ayta}. &
ese papa he sablosa, \emph{ayta}.  &
pluque sa linda cosa, \emph{ayta}.  &
mantequiya con mele, \emph{ayta}. &
ay, Sesu, le, le, le, le, \emph{ayta}. &
ro, ro, ro, ro, \critnote{caya}[Mi siñol \dots\ caya]
  {\quoted{Mi señor Manuel [Emmanuel]/, esa papa, qué sabrosa,/ porque está linda cosa,/ mantequilla con mel,} and then nonsense lullaby words.}.
\SectionBreak[\add{Repeat negrilla estribillo}]

\end{original}

%**********************************************
\begin{translation}
\StanzaSection{4}
1. At the most blessed stable, &
where victory triumphs, &
beginning of centuries of grace, &
the most fortunate night, \&

\Stanza{4}
2. A merry eve, the best, &
since despite the shadows &
at its midpoint dawns &
one who with so much light overwhelms it. \&

\Stanza{4}
3. A shepherd-boy from that scene, &
on his tempered panpipes, &
playing the \quoted{New Trojan}, &
sang in the straw-filled hutch: \&

\StanzaSection{4}
1. In Bethlehem they are singing, &
all is glory, all is heaven, &
and in a poor little stable &
he who is immense has confined himself. \&

\StanzaSection{4}
2. Fire melts the snow, &
and among so much snow, the fire &
yawns to each flame, &
that which is purified from this extreme. \&

\StanzaSection{4}
3. Look around on all sides: &
in each bit of straw, a blazing star, &
a torch at each glance &
and a God who is great, though little. \&

%*** ARRIERO
\StanzaSection{4}
1. Next Bartholo---you know the one--- &
a mule skinner in plain clothes, &
who was a swordsman \critnote{in days gone by}
  {Or perhaps in a previous villancico?}, &
and now, a vendor of candies. \&

\Stanza{4}
2. In search of a little mule &
who went off from him in a scheme &
to give himself a merry eve &
in the mysterious straw. \&

\Stanza{4}
3. Into the stable with the shepherds &
he entered, braying up a storm, &
and to the one who occupies the manger, &
he says as it is intoned: \&

%*****
\StanzaSection{4}
1. Mr. Baby, I swear to Saint--- &
well now I've said it, and it's more than enough &
for you to understand that I come &
on account of all this \quoted{Troy}/mess. \&

\Stanza{4}
2. Don't be afraid of me, I tell you, Sir, &
or play innocent &
when they tell me that you know &
whatever is not unknown to your father. \&

\Stanza{4}
3. It's just about right that of all my mules, &
the dirtiest and the fattest &
should bring me to this manger &
without knowing her mouth from a hole in the wall, \&

\Stanza{4}
4. And that I, without having sold &
all my stock of sweets, &
should have to \critnote{be so polite}
  {Or, \quoted{give up the struggle}.}, carrying all these flowers, &
hawking it at my own expense. \&

\Stanza{4}
5. If you come to enrapture men &
it seems to me an improper thing &
to have my mule go first, &
if she's not even going to carry the wagon. \&

\Stanza{4}
6. But now I've been thinking, &
if my saying so doesn't make you mad, &
that on account of the frost you ought to have &
the feed from her mouth. \&

\Stanza{4}
7. And upon the life of Bartholo, &
whether in these things or any others, &
if you should want anything, &
they should all be brought here for you. \&

\Stanza{4}
8. Open that mouth of pearls, &
with which I am so enamored, &
and request that these mountain folk &
don't try anything else. \&

\Stanza{4}
9. They want to do a dance for you, &
that they call \term{papalotillo}, &
and so, let us all sing, &
and let the candies go round all the more. \&

%*** PAPALOTILLO *****

\StanzaSection{2}
Come and you will see a genteel little boy. &
Look on him well, for in his eyes I see myself. \&

\StanzaSection{2}
Come and you will see a genteel little boy. &
Look on him well, for in his eyes I see myself. \&

\Stanza{2}
1. Look on him well, how he cries and sighs, &
which at the same time is his father's joy. \&

\Stanza{2}
2. Look on him well: jewels among poor things, &
a fertile seed hidden in the straw. \&

\Stanza{2}
3. Look on him well, for though now he confines himself, &
he will give us a fertile harvest. \&

\Stanza{2}
4. Look on him well, with tenderness and care, &
for he will be revealed as both \critnote{pasture and pastor}
  {\foreign{Pasto} is livestock feed (and anagogically, the Eucharist); \foreign{pastor} is both shepherd and religious minister.}. \&

\Stanza{2}
5. Look on him well, a little lamb full of love, &
for he will flee from the claws of the wolf. \&

\Stanza{2}
6. Look on him well, the tiny shepherd, &
for when he is big he will be a laborer. \&

\StanzaSection{2}
Come and you will see a genteel little boy. &
Look on him well, for in his eyes I see myself. \&

%*** NEGRILLA *****
\StanzaSection{4}
Little Miguel the Angolan, &
marshalling his troop, &
does not wish to be the last one &
at the party that is being enjoyed. \&

\Stanza{4}
Leaving the \quoted{tumbacatumba} &
and grunting like the Angolans do &
he unsheathed his voice, &
like pulling a sword from his charred log. \&

\StanzaSection{4}
Tell me, cousin, where is &
the baby who was born? &
for we know his relatives &
and we come to seek him. \&

\Stanza{5}
There he his, &
a candle among the straw, &
his eye like a little star, &
and an ox and a little mule &
with its belly to warm him. \&

\StanzaSection{2}
\add{Everybody come,} &
there he is. \&

\StanzaSection{4}
Hush, hush, look, don't startle him, &
for the tiny boy is sleeping. &
Jesu, Jesu, how lovely, &
listen, for we are singing like the angels: \&

\StanzaSection{1}
Glory in the heights and on earth, peace. \&

\StanzaSection{4}
For my God, O what a lovely song, &
there he is, &
listen, &
there he is. \&

\StanzaSection{5}
1. Hush, hush, baby boy, &
for we are bringing you a present: &
a little blanket, a diaper, &
and a little poinsettia, &
for we know how things go [with babies]. \&

\StanzaSection{6}
2. My Lord Emmanuel, &
this potato, how tasty, &
since this is a nice thing, &
butter with honey, &
ay, Jesu, lulla, lulla, &
ro, ro, ro, ro, hush. \&

\end{translation}

\end{poemtranslation}


% Angelicos coros con gozo cantad
% Critical poetry edition
% New version 2016/04/20

\begin{poemtitle}
\poemhead{\worktitle{Angélicos coros con gozo cantad} (Puebla, \circa{1680})}
\poemsource{Anonymous, from musical setting by Antonio de Salazar, Puebla, Convento de la Santísima Trinidad (\signature{MEX-Mcen}{CSG.256})}
\end{poemtitle}

\begin{poemtranslation}
\begin{original}

\StanzaSection{12}[Estribillo]
Angélicos coros &
con gozo cantad &
la gloria a Belén, &
que es \critnote{casa de pan}
  {\term{Bethlehem} means \quoted{House of Bread} in Hebrew; linking Incarnation and Eucharist.}%
  . &
Celestes esferas, &
estrellas y luces, &
bajad, bajad, &
y \critnote{el cielo de la tierra}
  {Perhaps, the sky, as opposed to supernatural Heaven (\term{cielo Empyreo}).} &
de gloria llenad. &
Que solo aquel lugar &
que el mundo desprecia &
de Dios es capaz.
\SectionBreak

\StanzaSection{4}[Coplas]
1. Para nacer Dios hombre, &
escoge este portal, &
que él sólo es digno alcázar &
de tanta majestad. \&

\Stanza{4}
2. No puede en los palacios &
nacer su inmensidad, &
porque Dios sólo cabe &
en \critnote{él de la humildad}
  {The stable or the Christian.}%
  . \&

\Stanza{4}
3. Aquestas \critnote{ruinas}
  {The stone ruins of Christ's stable, as in contemporary paintings; and the contrite Christian.}%
   \ quiere &
porque con caridad &
lo derribado busca, &
quién viene a edificar. \&

\Stanza{4}
4. Naced, Señor divino, &
que la justicia ya &
del cielo está mirando, &
que nace la Verdad. \&
\end{original}

\begin{translation}
\StanzaSection{12}
Angelic choirs, &
joyfully sing &
\quoted{Glory} to Bethlehem, &
the \quoted{House of Bread}. &
Celestial spheres, & 
stars and lights, &
descend, descend, &
and fill earth's heaven &
with glory. &
For only that place &
which the world discounts &
is capacious for God. 
\SectionBreak

\StanzaSection{4}
1. For God to be born as a man &
he chooses this stable, &
for it alone is a worthy palace &
for such great majesty. \&

\Stanza{4}
2. Not in the palaces &
can his immensity be born, &
because there is only room for God &
in the one that is humble. \&

\Stanza{4}
3. He favors these ruins &
because with compassionate love &
he seeks that which is torn down, &
since he comes to build. \&

\Stanza{4}
4. Be born, divine Lord, &
for lo, justice &
is looking down from heaven, &
for the Truth is born. \&
\end{translation}
\end{poemtranslation}
 
% Each of the scores is a chapter
\includescore{Cererols-Suspended_cielos}
[Cererols, \worktitle{Suspended, cielos}]
{Joan Cererols, \worktitle{Suspended, cielos, vuestro dulce canto}}

\includescore{Padilla-Voces_las_de_la_capilla}
[Gutiérrez de Padilla, \worktitle{Voces, las de la capilla}]
{Juan Gutíerrez de Padilla, \worktitle{Voces, las de la capilla}}

\includescore{Irizar-Si_los_sentidos}
[Irízar, \worktitle{Si los sentidos}]
{Miguel de Irízar, \worktitle{Si los sentidos queja forman del Pan divino}}

\includescore{Carrion-Si_los_sentidos}
[Carrión, \worktitle{Si los sentidos}]
{Jerónimo de Carrión, \worktitle{Si los sentidos queja forman del Pan divino}}

\includescore{Caseda-Que_musica_divina}
[Cáseda, \worktitle{Qué música divina}]
{José de Cáseda, \worktitle{Qué música divina}}

\includescore{Padilla-Al_establo_mas_dichoso}
[Gutiérrez de Padilla, \worktitle{Al establo más dichoso}]
{Juan Gutiérrez de Padilla, \worktitle{Al establo más dichoso}}

\includescore{Salazar-Angelicos_coros}
[Salazar, \worktitle{Angélicos coros}]
{Antonio de Salazar, \worktitle{Angélicos coros con gozo cantad}}

\end{document}
