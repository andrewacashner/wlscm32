\noteshead{José de Cáseda, \worktitle{Qué música divina}}

\begin{notesources}

\begin{source}
\sourcedescription{\signature{MEX-Mcen}{CSG.256}, Mexico City, CENIDIM, Colección Jesús Sánchez Garza; Manuscript performing parts from collection of the Convento de la Santísima Trinidad, Puebla}
\annotation{A 4/ \oldabbrev{D}{n} Joseph de Caseda}
\parts{SSATB; B. is instrumental}
\end{source}

\end{notesources}

Like Salazar's \worktitle{Angélicos coros} (in this edition), this piece is in the Colección Sánchez Garza at CENIDIM, from the Convento de la Santísima Trinidad in Puebla.
There are numerous works in the collection ascribed to José de Cáseda and his father Diego, successive chapelmasters at El Pilar in Zaragoza.

This is a set of individual performing parts. 
A tear along the fold at the bottom obscures a few of the notes.

The parts bear the names of the convent sisters listed below.
The name of the Alto, Madre Belona, was also written into the Tiple I-1 part of Salazar's \worktitle{Angélicos coros} after that of Madre Andrea, possibly for a later performance.

\begin{tabular}{ll}
Tiple 1 & Tomasita\\
Tiple 2 & María de Jesús\\
Alto & Madre Belona\\
\quoted{Thenor} & Rosa María de Jesús\\
Bajo & (no name)\\
\end{tabular}

The bass part is instrumental: it has only incipits of the lyrics and includes figured bass.
Given the piece's central conceit of Christ is a \term{vihuela}, that instrument would seem to be an apt choice for the continuo group.

\notesection{Coplas}

In the original version of this manuscript, coplas 1, 4, and 6 are scored for the full ensemble and the music for them is written out only once.
Coplas 2, 3, and 5 are sung by soloists with the same accompaniment part for each; thus the Bajo part for the solo strophes is only written out once in the MS.
The full-ensemble coplas actually require small adjustments for the different text underlay, so they are all written out in full in the edition.

The MS includes repeat signs after the first phrase in every copla. 
In the solo coplas, the accompaniment has the repeat sign placed one semiminim later than the vocal lines, indicating a \quoted{first ending}.
The edition simply writes out the repeated music in the solo coplas.

The Tenor part has an alternate setting of copla 4 sewn onto the paper.%
  \footnote{Many of the pieces in the Sánchez Garza collection have alternate versions, most commonly of lyrics, sewn or pasted in, showing the repeated use of these pieces for varying occasions and according to changing aesthetics and devotional needs.}
By lifting the sheet it is still possible to see most of the original setting of the solo copla 5, except for a few passages obscured by the stitching at the top.
The music for copla 5, as edited here, appears to be identical to that for copla 2; the obscured passages are indicated with brackets in the edition.

The alternate setting uses a G2 (treble) clef instead of the C3 clef of the Tenor part, and duplicates the Tiple 2 line in the original full-ensemble setting of copla 4, except that the repeated high G\octave{5} in \measures{107--108} are replaced with D\octave{5}.
This version appears to be intended as a solo setting of copla 4, with the female Tenor switching to a higher register for this copla. 
Perhaps it was used in an abridged version of the piece with fewer coplas, or in a version arranged for reduced voices.


\notesection{Solecisms}

In \measures{3--4}, there are parallel fifths in the outer voices, which cannot be avoided through ficta or any simple editorial correction. 
This is either intended by the composer, or was copied incorrectly.
If intentional, it may be an error or an aspect of personal style; but given that the lyric here is \gloss{acorde}{tuneful} this may be a deliberately ironic gesture.

In \measures{44-45}, there appears to be a cross-relation between the Tiple 2 (B\fl) and the Tenor (B\na).
On its own, the Ti. 2 would sing all B flats in this phrase, as no accidentals are included.
To match the motive used throughout this section, though, the Ti. 2 would sing B flats in \measure{44} and then B natural in \measure{45}.

But the Tenor has a sharp on the B in \measure{44}, normally indicating B natural; this would produce a cross relation.
The Tenor B sharp is probably not a cautionary accidental because the phrase would not normally call for a \term{ficta} alteration.

Thus the most likely solution seems to be (as indicated in the editorial accidentals) for the Ti. 2 to break the motive and sing all B naturals.
The possibilities here are similar to the parallel fifths in the opening phrase: either they are intentional solecisms or they are simple errors of copying or composition.

The condition of the manuscript indicates frequent use, so any solution to these problems must account for the fact that the sisters actually performed the piece from these manuscripts, in their current form.

\criticalnotesheader
\begin{criticalnotes}
3--4 & Ti. 1, B. & See discussion of solecisms above.\\
25 & Ti. 1 
  & The last four notes are beamed together in the MS, but the text underlay requires two groups of two.\\
37 & All 
  & The tempo marking \mentioned{aspacio} occurs in all the parts except the Ti. 1 (it is written \mentioned{espacio} in the T.)\\
44--45 & Ti. 2, T. & See discussion of solecisms above.\\
45 & B. & Notes nearly destroyed by tear in MS, but the left side of the G and right side of the C are still visible on either side.\\
\end{criticalnotes}
  

