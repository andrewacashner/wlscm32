\noteshead{José de Cáseda, \worktitle{Qué música divina}}

\begin{notesources}

    \begin{source}
        \sourcedescription{\signature{MEX-Mcen}{CSG.154}, Manuscript performing
        parts from collection of the Convento de la Santísima Trinidad, Puebla}
        \annotation{A 4/ \oldabbrev{D}{n} Joseph de Caseda}
        \parts{SSATB; B. is instrumental}
    \end{source}

\end{notesources}

Like Salazar's \worktitle{Angélicos coros} (in this edition), this piece is in 
the Colección Sánchez Garza at CENIDIM, the Mexican national music research 
center, in Mexico City.%
    \Autocite[375--403]{Cashner:PhD}
The collection is originally from the Convento de la Santísima Trinidad, a 
Conceptionist convent in Puebla.
There are numerous works in the collection ascribed to José de Cáseda and his 
father Diego, who were both chapelmasters in Zaragoza.%
\begin{Footnote}
    \Autocites{Calahorra:Zaragoza2}{Stevenson:CasedaD}, and the 
    relevant entries in the \worktitle{Diccionario de la música española e
    hispanoamericana}.
\end{Footnote}

This is a set of individual performing parts. 
A tear along the fold at the bottom obscures a few of the notes.
The parts bear the names of the convent sisters listed below.
The name of the Alto, Madre Belona, also appears in Salazar's 
\worktitle{Angélicos coros}.

\begin{inlinetable}
\begin{tabular}{ll}
    Tiple 1 & Tomasita\\
    Tiple 2 & María de Jesús\\
    Alto & Madre Belona\\
    \quoted{Thenor} & Rosa María de Jesús\\
    Bajo & (no name)\\
\end{tabular}
\end{inlinetable}

The bass part is instrumental: it has only incipits of the text and includes 
figured bass.
Given the piece's central conceit of Christ as a \term{vihuela}, that 
instrument would seem to be an apt choice to feature in the continuo group.

\notesection{Coplas}

In the original version of this manuscript, coplas 1, 4, and 6 are scored for 
the full ensemble and the music for them is written out only once.
Coplas 2, 3, and 5 are sung by soloists with the same accompaniment part for 
each; thus the Bajo part for the solo strophes is only written out once in the 
MS.
The full-ensemble coplas actually require small adjustments for the different 
text underlay, so they are all written out in full in the edition.

The MS includes repeat signs after the first phrase in every copla. 
In the solo coplas, the accompaniment has the repeat sign placed one semiminim 
later than the vocal lines, indicating a \quoted{first ending}.
The edition simply writes out the repeated music in the solo coplas.

The Tenor part has an alternate setting of copla 4 written on a separate strip
of paper and sewn onto the original performing part to cover the original
music (figure~\ref{Caseda-Coplas}).%
  \begin{Footnote}
      Many of the pieces in the Sánchez Garza collection have alternate 
      versions, most commonly of poetic text, sewn or pasted in. 
      These alterations provide evidence for the repeated use of these pieces
      for varying occasions and according to changing aesthetics and devotional
      needs.
  \end{Footnote}
By lifting the sheet it is still possible to see most of the original setting 
of the solo copla 5, except for a few passages obscured by the stitching at the 
top.
The music for copla 5, as edited here, appears to be identical to that for 
copla 2; the obscured passages are indicated with brackets in the edition.

The alternate setting on the sewn-in sheet uses a G2 (treble) clef instead of
the C3 clef of the Tenor part, and duplicates the Tiple 2 line in the original
full-ensemble setting of copla 4, except that the repeated high \pitch{G}{5} in 
\measures{138--139} is replaced with \pitch{D}{5}.
This version appears to be intended as a solo setting of copla 4, with the 
female Tenor switching to a higher register for this copla. 
Perhaps it was used in an abridged version of the piece with fewer coplas, or 
in a version arranged for reduced voices.


\notesection{Solecisms}

There are several instances of what appear to be compositional mistakes; 
I have argued, though, that these are intentional solecisms meant to communicate
a conceit of musical \quoted{falsehood} (see copla 5).%
    \Autocite[375--403]{Cashner:PhD}
The condition of the manuscript indicates frequent use, so any solution to 
these problems must account for the fact that the sisters actually performed 
the piece from these manuscripts, in their current form.

In \measures{3--4}, there are parallel fifths followed by direct octaves in the
outer voices, which cannot be avoided through \term{musica ficta} or any simple
editorial correction. 
This is either intended by the composer, or was copied incorrectly.
If intentional, it may be an error or an aspect of personal style.
Since the text here is \gloss{acorde}{tuneful}, and in light of other such
oddities described below, this may be a deliberately ironic gesture.

In \measures{69--70}, there appears to be a cross-relation between the Tiple 2 
(B\fl) and the Tenor (B\na).
On its own, the Ti. 2 would sing all B flats in this phrase, as no accidentals 
are included.
To match the motive used throughout this section, though, the Ti. 2 would sing 
B flats in \measure{69} and then B natural in \measure{70}.

The Tenor, though, has a sharp on the B in \measure{44}, normally indicating B 
natural (figure~\ref{Caseda-Estribillo}); this would produce a cross relation.
The Tenor B sharp is probably not a cautionary accidental because the phrase 
would not normally call for a \term{ficta} alteration, and it would seem more
obvious to use a flat symbol for this purpose.
Thus the most likely solution seems to be (as indicated in the editorial 
accidentals) for the Ti. 2 to break the motive and sing all B naturals.


\criticalnotesheader
\begin{criticalnotes} 
    3--4 
    & Ti. 1, B. 
    & Parallel fifths, direct octaves
    & As written; See discussion of solecisms above
    \\

    32
    & Ti. 1
    & Slur, n. 3--4, 5--6
    & Slur, n. 3--6; cf. text underlay
    \\
    
    56
    & All
    & Tempo \mentioned{a espacio}
    & Ti. 1 No marking; Ti. 2, A., B. \mentioned{aspacio}; 
    T. \mentioned{espacio}
    \\

    69--70
    & Ti. 2, T. 
    & Ti. 2 B\na{} ficta, T. B\na{}
    & Ti. 2 B\fl{} assumed without accidentals vs. T. explicit B\sh{};
    intentional cross-relation?
    \\

    70--71
    & B. 
    & \pitch{G}{3}--\pitch{C}{4}
    & Obscured by tear; part of \pitch{G}{3} and \pitch{C}{4} still visible
    \\

    124
    & A.
    & Cf. m. 109, Ti. 1
    & Obscured by sewn-in sheet
    \\

    147--148, 150--151
    & T.
    & Cf. mm. 101--102, 104--105, Ti. 1
    & Obscured by sewn-in sheet
    \\

\end{criticalnotes}


