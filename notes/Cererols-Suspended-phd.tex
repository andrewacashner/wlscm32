% Editorial notes for critical edition of Cererols, Suspended cielos
% Andrew Cashner diss., 2013-07-18

\section{%
\ti{Suspended, cielos}, Setting by Joan Cererols
}

\subsection*{%
Sources
}

This critical edition is based on two manuscript sources and an early twentieth-century edition based on the first manuscript source.

\begin{enumerate}
\item{\ti{CAN}}
	\begin{itemize}
	\item{%
	E-CAN:~AU/0116, Canet de Mar, Arxiu parròquia de Sant Pere i Sant Pau de Canet de Mar, Bisbat de Girona, Fons capella de música
	} \item{%
	Title page: \q{Villancico al SSmo. Sto. / Suspendet cielos a 8 / Cererols}
	} \item{%
	Complete set of manuscript partbooks with estribillo and coplas, performance parts SSAT, SATB, \q{acompañamiento}
	} \item{%
	See \fullcite[60--61]{Bonastre:CanetCatalog}
	}
	\end{itemize}
%
\item{\ti{Bbc}}
	\begin{itemize}
	\item{%
	E-Bbc:~M/765/25, Barcelona, Biblioteca de Catalunya, previously unattributed 
	} \item{%
	Title page: \q{Villansico a 8}
	} \item{%
	Incomplete set of manuscript partbooks with estribillo only (no coplas), some lyrics altered for a Eucharistic dedication
	} \item{%
	SSA, SATB, missing TI and accompaniment
	}
	\end{itemize}
%
\item{\ti{MEM}}
	\begin{itemize}
	\item{%
	\fullcite[xxv (lyrics), 221--236 (music)]{Cererols:MEM-VC}
	} \item{%
	Based exclusively on \ti{CAN}, with some errors
	}
	\end{itemize}
\end{enumerate}

%************************************************************

\subsection*{%
Editorial Policies
}

This new edition takes \ti{CAN} as its primary source, since this source preserves all the performing parts and includes both coplas and estribillo, and since \ti{Bbc} is an adaptation to fit a Eucharistic function.
Both manuscripts are probably copies based with some degree of separation on an original exemplar by Cererols at his Abbey of Montserrat, which would have been lost when the library burned in the nineteenth century.

The transcription follows \ti{CAN}, but \ti{Bbc} provides added support for a number of the editorial \term{musica ficta} accidentals and text underlay.
\ti{Bbc} adds the unique dynamic markings \term{eco} and \term{falsete}; these are shown in parentheses.

%********************
\subsubsection*{
Gaps and Errors
}

In the acompañamiento partbook of \ti{CAN}, some of the notes written along the paper fold have eaten through the paper. 
In almost every case, however, the original rhythmic and pitch values are unambiguous from the shape of the hole and from the musical context.
Notes supplied to fill these gaps are demarcated with square brackets.
The accompaniment parts is the same for all each pair of coplas, so \ti{CAN} only writes it out once.

%************************************************************
\subsection*{%
Specific Editorial Notes}

\subsubsection*{%
Estribillo}

Both \ti{CAN} and \ti{Bbc} contain the estribillo, so differences between the two are noted below.

\bigskip
\begin{hangparas}{\myparindent}{1}
%
\emph{Measure 1, Alto I: } 
\ti{Bbc} starts coloration on C, matching other Chorus I parts.

\emph{14, Tenor II, third note: } 
\ti{Bbc} has F\octave{4}; \ti{CAN} has A\octave{4}, which matches the contour of the Alto II (mm.~13--14), and of Tiple I-2 and Alto I in mm.~12--13.

\emph{17, Chorus I, lyrics, second and third note: } 
There is disagreement between versions and between partbooks within each version regarding the lyrics.
In \ti{CAN}, Tenor I has \q{tened} and the other voices have repeat signs indicating the repeat of \q{tened}.
In \ti{Bbc}, Tiple I-1 and Alto I have \q{parad}, while Tiple I-1 has \q{tened}, and Tenor has sign for repeat of \q{tened}.

\emph{33, Tenor II: }
\ti{Bbc} starts coloration in m.~33 (like the other Chorus II parts); \ti{CAN} starts coloration on \q{las jerarquías} in m.~32.

\emph{40, Bajo II, third note: }
F is sharped in \ti{CAN} only. 
This figure is the subject of the eight-voice fugue, and none of the other entrances have an F-sharp. 
The sharp is probably a cautionary accidental warning the player of the Bajo II part \emph{not} to sharp the F.

\emph{45--46, Tiple I-2: }
\ti{Bbc} has slur between last note of m.~45 and first note of m.~46 (D--E).

\emph{47--59, All voices, lyrics: }
\ti{Bbc} has the following Eucharist lyrics in place of those from \ti{CAN} in the transcription: \q{y desde un pan divino/ un hombre soberano}. 

\emph{58, Tenor I, first note: }
\ti{CAN} (the only source of the Tenor I voice) has B\octave{3}, which must be an error. Correction to A follows \ti{MEM}.

\emph{77--80, Tiple I-1: }
\ti{Bbc} has different conclusion, shown in small staff above.

\emph{78, Tiple I-2, second note: }
In \ti{Bbc}, G is a semibreve, which does not align rhythmically with the other voices. It is a minim in \ti{CAN}.
%
\end{hangparas}

%********************
\subsubsection*{%
Coplas}

Coplas are in \ti{CAN} only. 
The poetic text of the coplas has been amended in several places to align with the surviving poetry imprints of the poem, wherever the change clarifies the meaning or improves the grammar of the text as given in \ti{CAN}.

\bigskip
\begin{hangparas}{\myparindent}{1}
%
\emph{M. 81, Lyrics: }
\q{Las fugas del primer hombre formó} amended to \q{Las fugas que el primer hombre formó}.
 
\emph{94, Lyrics: }
\q{Qué mucho que} amended to \q{Qué mucho si}.

\emph{96, Tiple I-2: } 
Third rest (beat 5) in \ti{CAN} is semiminim rest, which must be an error. 
Corrected to minim rest, following \ti{MEM}.

\emph{110, All voices: }
The rhythmic values for the first three beats of this bar in \ti{CAN} are as follows: semibreve, semiminim rest, two flagged minims. 
The flagged minims are normally transcribed as seminimims or quarter notes, but with the rest this does not fit within the tactus. 
The copyist is apparently trying to accomodate the text to the rhythm of copla 1 (m.~83), but here there is an extra syllable. 
So either the flagged minims must be rendered as corcheas or eighth notes, which would be quite difficult to sing, or the flagged minims are transcribed normally and the rest must be omitted or treated like a breath mark. 
This edition follows \ti{MEM} in transcribing the flagged notes as semiminims.

\emph{116--117, Alto I, lyrics: }
\q{Las disonancias} amended to singular in accord with the other voices and the imprints.

\emph{131--135, Lyrics: }
Tiple I-1 \q{desatento} amended to \q{desentono}, which accords with the Tiple I-2 and with the poetry imprint S1 that contains this copla.
Both voices \q{tan grande} amended to \q{tan vano}, as in all the poetry imprints; \q{grande} only appears in this source and breaks the metrical pattern of romance.

\emph{137, All voices: }
Same problem as m.~111.

\emph{142--146, Lyrics: }
\q{Lo inmenso spacio} amended to \q{lo inmensio a espacio} following the poetry imprints, though in practice there would be little discernible difference in pronunciation.

\emph{135--137, Lyrics: } 
\q{Susteniendo} in the two Tiple parts amended to \q{sustenido}, in accord with the Alto and Tenor parts and with the poetry imprints.

\emph{152, Tiple I-2: }
A semibreve rest is missing (compare the corresponding places in the other coplas (mm.~98, 125).

\emph{160--161, Tiple I-2: }
The notes transcribed as quarters are written (per usual practice) in \ti{CAN} as flagged minims. 
The four are beamed together, which usually indicates a melisma, but the text underlay clearly places the words as shown in the transcription.
%
\end{hangparas}

%****************************************
\subsection*{
Differences from the \ti{MEM} Edition}

There is much to praise about the careful edition by Dom Pujols, which was among the first published modern transcriptions of villancicos, made at a time when almost nothing was known about the genre.
The most serious error is a missing rest in the fugato section, causing one of the fugal voices to enter at the wrong moment.

\ti{MEM} does not indicate rhythmic coloration, and it does not clearly distinguish editorial additions such as lyrics, accidentals, and slurs.
\ti{MEM} underlays lyrics to the Bajo II part, though only the first two words are written in the manuscripts (both \ti{CAN} and \ti{Bbc}).
In keeping with common practice across the Hispanic world, this part was played instrumentally, probably on the bajón.
\ti{MEM} omits the figured bass in the acompañamiento part.
Finally, there are discrepancies between \ti{MEM}'s transcription of the lyrics in the text-only part of the edition and in the musical score itself; and sometimes neither corresponds to the manuscript.
Specific errors are noted below (measure numbers refer to this edition, not \ti{MEM}).

\bigskip
\begin{hangparas}{\myparindent}{1}
%
\emph{12, Tiple I-2, first note: }
\ti{MEM} has E; both manuscripts have F.

\emph{39, Alto I: }
\ti{MEM} omits the breve rest in m.~39, shifting the Alto fugue entrance one bar earlier through m.~47. 
Both manuscripts match the new edition.

\emph{39, Acomp.: }
\ti{CAN} has sharp over B (beat 4), indicating a natural (counter to normal ficta practice). 
\ti{MEM} puts the sharp on the C (beat 2) and omits the natural. 
The C was probably sharped, nevertheless, as the editorial accidental indicates.

\emph{61ff, Lyrics: }
\ti{MEM} has correct text underlay in the musical score, but in the lyrics section (\ti{MEM} p. xxv), has \q{el canto llano de los ángeles lleva}.

\emph{85, Lyrics: }
\ti{MEM} score has \q{el compás}; \ti{MEM} lyrics section has \q{al compás}. \ti{CAN} has \q{al compás}.

\emph{100, Tiple I-2, first note: }
\ti{MEM} has C\sh; \ti{CAN} does not. 
The note should probably be sung sharp, however.

\emph{101--103, Tiple I-2: }
\ti{MEM} omits slur in \ti{CAN} on last three notes of m.~102, and therefore underlays the words differently.

\emph{104--106, Lyrics: }
\ti{MEM} has \q{previenen} instead of \q{previene}, which is clear in \ti{CAN} and in the poetry imprint S1.

\emph{114, Alto I: }
\ti{MEM} adds slur between first two notes, not present in \ti{CAN}.

\emph{128--130, Tiple I-2: }
Lyric repeat signs in \ti{CAN} are ambiguous, and \ti{MEM} underlays the lyrics differently. 
That the notes transcribed as quarter notes (flagged minims in the original) are not beamed together normally would mean that different syllables were sung to each note, as in the underlay in this edition.

\emph{132--134, Lyrics: }
\ti{MEM} has \q{tan grave} in score, but \q{grande} in lyrics section. \ti{CAN} clearly has \q{grande} (though as noted earlier, this itself is probably an error).

\emph{136, Lyrics: }
\ti{MEM} amplifies the lyrical disagreement between upper and lower voices (\q{susteniendo} vs. \q{sustenido}) by using \q{sostenido} for all voices in the musical score but \q{sosteniendo} in the lyrics section.

\emph{141, Alto I: }
\ti{MEM} adds slur on first two notes; no slur in \ti{CAN}.

\emph{142--146, Lyrics: }
\ti{MEM} changes \q{spacio} in \ti{CAN} to \q{espacio} and changes text underlay to fit this. 
All the poetry imprints have \q{a espacio}.

\emph{148--150, Lyrics: }
\ti{MEM} has \q{clausura} in score but \q{cláusula} in lyrics section. \ti{CAN} has \q{claúsula}.

\emph{155--156, Tiple I-2, sixth note: }
\ti{MEM} omits \q{en}, which is clearly indicated in \ti{CAN}.

\emph{160--161, Tiple I-2: }
\ti{MEM} respects the slur in \ti{CAN} and thus shifts the beginning of \q{milagro} to the C\sh\ (beat 4). 
But the spacing of the text underlay in \ti{CAN} matches the current edition, and in most other places this cadential figure is sung on one syllable (cf. mm.~91, 102, 118, 133--134, 145, 156).
%
\end{hangparas}
