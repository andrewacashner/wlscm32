\noteshead{Miguel de Irízar, \worktitle{Si los sentidos queja forman del Pan Divino}}

\begin{notesources}

\item (P)
\source{\signature{E-SE}{5/32}, Manuscript performing parts (copyist's hand)}
\annotation{Al \oldabbrev{SS}{mo} a 8. Si los sentidos}
\parts{SSAT, SATB, \foreign{Acompañamiento}}

\item (S)
\source{\signature{E-SE}{18/19}, Manuscript draft score in Irízar's hand for Corpus Christ 1674 at Segovia Cathedral}
\annotation{Fiesta del SSantissimo de este año del 1674}
\parts{SATB, SATB, continuo only in coplas}

\end{notesources}

In a rare case of a surviving draft score of a villancico, Irízar composed the piece in one of his makeshift notebooks made from received letters, with the music on the backsides and in the margins of the letters.
The score (S) is drafted with written barlines every two \term{compases} both in duple and triple meter; when a single odd compás is left at the end of a section (e.g., \measure{17} in this edtion) Irízar groups it with what follows.
When a colored (imperfected) semibreve extends across a barline, Irízar centers the note on the line (since mensural notation did not allow for ties).

The performing parts (P) appear to be in the hand of a professional copyist, and correspond closely with the score.
The score agrees with the parts in pitches and rhythms in the estribillo, differing only in a few cases of accidentals, where \term{musica ficta} practice made the notation of some accidentals optional.

The score lacks the \term{General} continuo part in the estribillo.
In the coplas, Irízar originally composed separate four-voice settings of the first two coplas, but then at the bottom of the page drafted the setting for Tiple solo and continuo that appears in P, with a slightly different beginning to the continuo part.
It may have been a later idea to combine the solo setting with the end of the four-voice setting for the \term{Respuesta a las coplas} on the \soCalled{tag line}, \quoted{No se den por sentidos los sentidos}.
This is the first use of the continuo, suggesting that Irízar decided after composing the rest of the piece to add the continuo part (\term{General} in P).

\notesection{Lyrics}

In S, Irízar simply wrote the lyric text out in a single line underneath each system, with no text underlay in the individual voices.
Thus all text underlay in this edition is based on P.
P includes only textual incipits for the Bajo II and \term{General} parts, a common convention indicating that these were played instrumentally; the incipits are only there to help orient the player.
The Bajo II part would probably have been played on \term{bajón}, and the \term{General}, by a continuo ensemble including harp, plucked and bowed strings, and small organ.
Figures only appear in the coplas.

The lyrics correspond closely with those later attributed to Vicente Sánchez in the \worktitle{Lira Poetica} (Zaragoza, 1689), 171--172.
Irízar died in 1684, so he either had access to an earlier version of Sánchez's text through his correspondence network, or Sánchez's text is an improvement on a pre-existing poem that Irízar used.
Irízar does not include one of Sánchez's coplas and arranges the strophes differently.

The notated melody for the coplas does not fit every stanza equally well.
The singer was apparently expected to adapt the rhythm to fit the poetry for the subsequent stanzas.

\criticalnotesheader

\begin{criticalnotes}
7  & Ti. I-1 & C\sh\ in S only \\
9  & Ti. I-2, T. II & S has F\na\ in both voices; P has F\sh\ in T.~II only, possibly a cautionary accidental (i.e., performed as F\na)\\
12 & Ti. I-1, I-2 & Slur in P only\\
14 & A. I & C\sh\ in P only\\
27 & T. II & C\sh\ in P only\\
44 & T. II & F\sh\ in P only\\
49 & T. I & First three notes slurred in S\\
\end{criticalnotes}
