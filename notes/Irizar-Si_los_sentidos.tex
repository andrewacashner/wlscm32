\noteshead{Miguel de Irízar, \worktitle{Si los sentidos queja forman del Pan Divino}}

\begin{notesources}
\item E-SE:~5/32, Manuscript performing parts (copyist's hand): Tiple 1, Tiple 2, Alto, and Tenor of Chorus I; Tiple, Alto, Tenor, and Bajo of Chorus II; \foreign{Acompañamiento}
\item E-SE:~18/19, Manuscript draft score in Irízar's hand
\end{notesources}

The Bajo II part has no text underlay and would have been played instrumentally, probably on \term{bajón}.

The lyrics correspond closely with those later attributed to Vicente Sánchez in the \worktitle{Lira Poetica} (Zaragoza, 1689), 171--172.
Irízar died in 1684, so he either had access to an earlier version of Sánchez's text through his correspondence network, or Sánchez's text is an improvement on a pre-existing poem that Irízar used.
Irízar does not include one of Sánchez's coplas and arranges the strophes differently.

The notated melody for the coplas does not fit every stanza equally well.
The singer was apparently expected to adapt the rhythm to fit the poetry for the subsequent stanzas.
