\noteshead{Miguel de Irízar, \worktitle{Si los sentidos queja forman del Pan 
Divino}}

\begin{notesources}

    \begin{source}
        \sourcedescription[P]{%
            \signature{E-SE}{5/32}, Manuscript performing parts (copyist's
            hand)} \annotation{Al \oldabbrev{SS}{mo} a 8. Si los sentidos%
        }
        \parts{SSAT, SATB, \foreign{Acompañamiento}}
    \end{source}

    \begin{source}
        \sourcedescription[S]{%
            \signature{E-SE}{18/19}, Manuscript draft score in Irízar's hand for
            Corpus Christ 1674 at Segovia Cathedral} 
            \annotation{Fiesta del SSantissimo de este año del 1674}
        \parts{SATB, SATB, continuo only in coplas}
    \end{source}

\end{notesources}

In a rare case of a surviving draft score of a villancico, Irízar composed the 
piece in one of his makeshift notebooks made from received letters, with the 
music on the reverse sides and in the margins of the letters.%
  \Autocites[285--338]{Cashner:PhD}
  {LopezCalo:Segovia}{Olarte:Irizar}
  {LopezCalo:IrizarLetters1}{Rodriguez:Networks}
The score (S) is drafted with written barlines every two \term{compases} both 
in duple and triple meter; when a single odd compás is left at the end of a
section Irízar groups it with what follows.
When a colored (imperfected) semibreve extends across a barline, Irízar centers 
the note on the line, since mensural notation did not allow for ties.

The performing parts (P) appear to be in the hand of a professional copyist, 
and correspond closely with the score.
The score agrees with the parts in pitches and rhythms in the estribillo, 
differing only in a few cases of accidentals, where \term{musica ficta} 
practice made the notation of some accidentals optional.
Generally, the edition uses accidentals that are present in either source,
weighting the parts more heavily since these were actually used for
performance.

The score lacks the \term{General} continuo part in the estribillo.
In the coplas, Irízar originally composed separate four-voice settings of the 
first two coplas, but then at the bottom of the page drafted the setting for 
Tiple solo and continuo that appears in P, with a slightly different beginning 
to the continuo part.
It may have been a later idea to combine the solo setting with the end of the 
four-voice setting for the \term{Respuesta a las coplas} on the \soCalled{tag 
line}, \quoted{No se den por sentidos los sentidos}.
This is the first use of the continuo, suggesting that Irízar decided after 
composing the rest of the piece to add the continuo part (\term{General} in P).

\notesection{Lyrical Text}

In S, Irízar simply wrote the poetic text out in a single line underneath each 
system, with no text underlay in the individual voices.
Thus all text underlay in this edition is based on P.
Figures only appear in the coplas.

The text corresponds closely with a poem later attributed to Vicente Sánchez in 
the \worktitle{Lyra Poética} (Zaragoza, 1688), 171--172.
Irízar died in 1684, so either he had access to an earlier version of Sánchez's 
text through his correspondence network, or Sánchez's text is an improvement on 
a pre-existing poem that Irízar used.
Irízar does not include one of Sánchez's coplas and arranges the strophes 
differently.

The notated melody for the coplas does not fit every stanza equally well.
The singer was apparently expected to adapt the rhythm to fit the poetry for 
the subsequent stanzas.

\criticalnotesheader

\begin{criticalnotes}
    7  
    & Ti. I-1 
    & C\sh{}
    & P: C\na{}; S: C\sh{} 
    \\
   
    9  
    & T. II 
    & F\na{}
    & P: F\sh{}; S: F\na{}; cf. Ti. I-2 F\na{}in P and S (cautionary accidental in P?)
    \\

    15
    & Ti. I-1, I-2 
    & Slur
    & P: Slur; S: No slur; cf. text underlay
    \\
   
    18
    & A. I 
    & C\sh{} 
    & P: C\sh{}; S: C\na{}
    \\
    
    34
    & T. II 
    & C\sh{}
    & P: C\sh{}; S: C\na{}; cf. Ti. I-1 C\sh{}
    \\

    44 
    & Gn., n. 3
    & \pitch{A}{2}
    & P: \pitch{G}{2}; cf. B. II (corr.)
    \\
   
    51
    & T. II 
    & F\sh{}
    & P: F\sh{}; S: F\na{}; cf. Ti. I-2 F\sh{}
    \\
   
    56
    & T. I 
    & Nn. 1--3, Slur
    & P: No slur; S: Slur; cf. text underlay
    \\

\end{criticalnotes}
