\noteshead{Juan Gutiérrez de Padilla, \worktitle{Voces, las de la capilla}}

\begin{notesources}

    \begin{source}
        \sourcedescription{%
            \signature{MEX-Pc}{Leg.~3/3}, In manuscript partbooks, 
            \worktitle{Navidad del año de 1657}%
        }
        \annotation{A 6/ Padilla}
        \parts{SAT, ATB; B.~II is instrumental (\term{bajón} and other continuo 
        instruments)}
    \end{source}

    \begin{source}
        \sourcedescription{Modern edition: \fullcite{Padilla:Tello}}
    \end{source}

\end{notesources}

This piece is from the complete cycle of villancicos composed by Juan Gutiérrez 
de Padilla for the cathedral of Puebla de los Ángeles and performed at 
Christmas 1657.%
    \Autocite[133--226]{Cashner:PhD}
This composer, whom the manuscripts refer to simply as Padilla, was born near
Málaga in southern Spain around 1590.%
    \Autocites{Mauleon:PadillaPalafox}{Hurtado:Padilla}{Stevenson:Padilla}
He served as chapelmaster of churches in Jérez de la Frontera and Cádiz before
emigrating to New Spain around 1622.
In 1628 he was named assistant to Gaspar Fernández, the ailing chapelmaster of
the cathedral of Puebla de los Ángeles.
His earliest dated villancicos survive from that year.%
    \Autocite{Cashner:Cards}
Padilla succeeded to the post after Fernández died in 1629, and continued as
chapelmaster at Puebla until his own death in 1664.

The only source for this villancico is a set of partbooks, each labeled
\worktitle{Navidad del año de 1657}.
  \Autocites{Puebla:Microfilm}{Stanford:Catalog}
The Tiple I partbook has the additional marking \quoted{en 8 quadernos}, 
confirming the total of eight notebooks for the whole cathedral ensemble, 
typically organized in two choirs of four voices each.
The partbooks include all the villancicos needed for performance at Matins for 
Christmas and Epiphany of the 1657--1658 liturgical year, plus the hymn 
\worktitle{Christus natus est nobis}.

Only six partbooks contain the music for \worktitle{Voces, las de la capilla}, 
scored (as the parts indicate) \term{a 6}.
Tiple and Bassus of Chorus II are not included.
The Altus I and Tenor II parts for this piece include the composer's name, 
\quoted{Padilla}.
The partbooks show signs of repeated use over many years.

The bass part is in the partbook of Bassus, Chorus I, but this part plays with 
the voices of Chorus II throughout the piece.
Typical of Padilla's scores, this part only includes brief textual
incipits to help an instrumentalist coordinate with the ensemble.

The handwriting, ink, and paper are consistent with that used in the composer's
other extant Christmas cycles in the cathedral archive, and it seems reasonable
to believe this to be his own hand.
There is a pronounced decine in the quality of the handwriting across each sets
from 1651 to 1659, consistent with a physical decline in Padilla's later
years.
In 1660, he signed a power-of-attorney document citing his failing 
health\autocite{Mauleon:PadillaCivil}, and he died in 1664.

The one previous edition of this piece contains a serious error based on the 
misreading of rests, as discussed below.%
    \Autocite{Padilla:Tello}
The piece has been recorded once prior to this edition, with an erroneous 
reading of the repeats.%
  \autocite{Padilla:HabanaCD}

The setting is rich in musical symbols and puns that match the conceits about
music in the poem.%
    \Autocite[167--178]{Cashner:PhD}
These are some of the less obvious examples:
\begin{itemize}
    \item On \term{cuenta} (mm. 6--7), the voices sing a long, offbeat note that
        requires special counting, and this whole strophe is sung by Chorus I
        while Chorus II counts rests.
    \item In the \term{respuesta} (mm. 28--44), which mentions \quoted{the
        thirty-three}---Christ's traditional age at his crucifixion---the voice
        parts have thirty-three notes in the original notation.
    \item Padilla evokes madrigal style with literal word painting throughout; 
        in mm. 45--59 he depicts singing at the Christmas stable \quoted{in one
        choir and the other} through polychoral texture and imitation, and
        \quoted{three by three, two by two, one by one} by the number of voices.
    \item For \quoted{the sign of \term{A (la, mi, re)}} (mm. 67--69)---a
        reference to Christ as \quoted{\term{alpha} and \term{omega}} in musical
        terms---the voices sing pitches corresponding to the named syllables as
        they sing them; likewise for \quoted{his eyes set on \term{mi}} (m. 72).
    \item On the words \foreign{peregrino tono} (mm. 128--130), the A. I sings
        G--B\fl--A--G, which is the final cadence of the plainchant \term{tonus
        peregrinus} in the transposed \term{cantus mollis} of the villancico.
    \item To illustrate the words \foreign{máxima y breve}, the
        T. II sings the word \foreign{máxima} on a breve (m. 153).
\end{itemize}

\notesection{Structure of Repeated Sections}

Like many of villancicos by Juan Gutiérrez de Padilla, this one begins with an
introductory section for a portion of the ensemble.
In the edition this is labeled \term{introducción} after the model of 
contemporary poetry imprints.
Next the full ensemble sings the \gloss{estribillo}{refrain}.
Two \gloss{coplas}{verses} follow, the first sung by Chorus I and the second by 
Chorus II. 
Finally the estribillo is repeated.

The repeat of the estribillo is indicated clearly.
In all the parts there is a \term{signum congruentiae} at the beginning of the 
estribillo (\measure{45} in the edition).
In every part except Tiple I, after the end of the coplas, the scribe has 
written the first few notes of the estribillo and the \term{signum} to point 
back to the estribillo.

The \term{introducción} has two parts: the first section has two stanzas of 
poetry and is sung by Chorus I; the second is a \gloss{respuesta}{response} 
section sung by Chorus II.
The one recording of this piece errs in performing both stanzas of the Chorus I 
section first and then performing the respuesta once.
While that arrangement might seem logical, the parts all clearly indicate 
through the \term{signum congruentiae}, notated rests, incipits, and 
\term{custodes} that the first part of the introducción (with the first stanza 
of lyrics) proceeds immediately into the respuesta, and then the whole section 
is repeated from the beginning with the second stanza of lyrics.
This means that the respuesta is sung twice, once after each stanza of the 
introducción.

\notesection{Related Sources}

A 1649 catalog entry from the collection of Portuguese King John IV appears to 
indicate an earlier setting of the same or similar text as Padilla's.
The citation is listed under the works of Francisco de Santiago, who was 
chapelmaster of Seville Cathedral until his death in 1644: \quoted{Vozes las de 
la capilla. solo. Ya trechos las distancias. a 9}.%
  \autocite[caixão 26, no.~674]{JohnIV:Catalog}

Compare the related poem, \worktitle{Cantores de la capilla}, performed at 
Seville Cathedral, Epiphany 1647.
This was probably set to music (now probably lost) by Santiago's successor Luis
Bernardo Jalón.% 
\begin{Footnote}
    \worktitle{Villancicos que se cantaron en la S.~Iglesia 
    Metropolytana de Sevilla, en los Maytines de los Santos Reyes. En este año de 
    mil y seiscientos y quarenta y siete} (Puebla de los Ángeles, private 
    collection, courtesy of Gustavo Mauleón Rodríguez); 
    edition of poem in \autocite[209]{Cashner:PhD}.
\end{Footnote}

\criticalnotesheader
% Critical notes for Padilla, Voces, las de la capilla
% Revised 2017/02/17

\begin{criticalnotes}
    2
    & Ti. I
    & E\na 
    & Cautionary E\sh{} to counteract \term{una nota super la} rule
    \\
    
    8
    & Ti. I
    & E\na
    & Cautionary E\sh{}, cf. m. 2
    \\

    44
    & T. II
    & aguardan
    & aguarda; cf. all other voices (corr.)
    \\

    82
    & A. II
    & Bar rest 
    & Omitted; cf. correct rests in T. II, B. I; 
    cf. error here in Palacios and Tello ed.
    (corr.)
    \\

    92--93
    & T. II
    & Slur
    & Slur extent unclear, cf. text underlay
    \\

    121 
    & B. I
    & E\fl{}
    & E\na; cf. explicit E\fl in A. I; \term{una nota super la}, cf. m. 123
    \\

    126
    & T. II
    & Semibreve
    & Breve (corr.)
    \\

    128
    & A. I
    & F\na{}
    & F\sh{}, likely a cautionary accidental indicating F\na{}; 
    counteracts \term{ficta} tendency to sharp the F in G--F--G gesture, 
    avoids B\fl/F\sh{} sonority unusual for Padilla; 
    cf. mm. 130--131
    \\

    130--131
    & A. I
    & F\na{}
    & F\sh{}, certainly a cautionary accidental indicating F\na{}; cf.
    simultaneous F\na{} in Ti. I; cf. m. 128

   

\end{criticalnotes}


