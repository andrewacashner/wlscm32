\noteshead{Juan Gutiérrez de Padilla, \worktitle{Voces, las de la capilla}}

\begin{notesources}

\item[P.]
\source{\signature{MEX-Pc}{Leg.~3/3}, In manuscript performing partbooks, \worktitle{Navidad del año de 1657}}
\annotation{A 6/ Padilla}
\parts{SAT, ATB; B.~II is instrumental (\term{bajón} and other continuo instruments)}

\item[H.]
\source{Edition by Nelson Hurtado and Aurelio Tello, \autocite{Padilla:Tello}}

\end{notesources}

This piece is from the complete cycle of villancicos composed by Juan Gutiérrez de Padilla for the cathedral of Puebla de los Ángeles and performed at Christmas 1657.
The single source is a set of eight partbooks labeled \worktitle{Navidad del año de 1657} on the cover of each (the Tiple I partbook has the additional marking \quoted{en 8 quadernos}, confirming the total of eight notebooks.
The partbooks include all the villancicos needed for performance at Matins for Christmas and Epiphany of the 1657--1658 liturgical year, plus the hymn \worktitle{Christus natus est nobis}.

This villancico does not include the Tiple or Bassus of chorus II.
The other partbooks include the piece with the heading \quoted{A 6}.
Altus I and Tenor II include the composer's name, \quoted{Padilla}.

The bass part is in the partbook of Bassus, Chorus I, but this part plays with the voices of Chorus II throughout the piece.
Typical of Padilla's scores and many other contemporary villancicos, this part only includes brief textual incipits to help the player coordinate with the ensemble.
It is meant to be played on \term{bajón} and probably doubled with other continuo instruments.
As always, the other voices could also be doubled instrumentally.

The handwriting, ink, and paper are consistent with that used in Padilla's other extant Christmas cycles in the cathedral archive.
In each set between 1651 set and the last one in 1659, a year before Padilla signed a power-of-attorney document citing his failing health\autocite{Mauleon:PadillaCivil} (he died in 1664), there is a pronounced decline in the quality of the handwriting.
It seems reasonable to believe this to be Padilla's own hand.

The partbooks show signs of repeated use over many years: fingerprints, multiple performers' names written in the parts, corrections, and added barlines.
The barlines suggest that the pieces were still being performed into the eighteenth century, when performers were becoming less familiar with the old mensural notation.

This piece was previously edited by Nelson Hurtado and Aurelio Tello. 
That edition contains a serious error based on the misreading of rests and repeat signs, as discussed below. % XXX check

\subsection*{Structure of Repeated Sections}

% why respuesta is sung twice
% why estribillo is repeated after both coplas

\begin{criticalnotes}
1 & Ti. I & 
Spanish notation of the period lacks a natural sign, so sharps are used as naturals on Es and Bs. 
Here the E\sh{} tells the singer not to apply to \foreign{una nota super la} rule.\\
30 & Ti. I & 
Lyric underlay of \quoted{su-- a-- ve\undertie y} as three syllables is clear.\\
39 & T. II & Lyrics: \quoted{y aguarda} instead of \quoted{y aguardan} as in all the other parts.\\
40 & A. II & The second F would be natural by default in the MS; the natural is added in conformity with modern accidental notation.\\
50 & A. II & 
The first rest, a perfect semibreve (modern dotted whole note), was inadvertently omitted from the MS. 
The T. II and B. I, who have the same gesture as the A. II here, both have the missing rest. 
The rest may have been accidentally erased when another correction was made to the MS at the end of the preceding phrase.
Hurtado and Tello mistakenly delete the rest from T. II and B. I rather than adding it to A. I, producing a collision between choirs in the succeeding phrase.\\
72 & T. II & Last note is breve with fermata instead of semibreve.\\
73--74 & A. I &
Sharp signs on F in MS have been interpreted as cautionary accidentals---, as signs to the singer \emph{not} to follow \term{ficta} conventions of sharping the F in the gesture G--F--G. 
In other words, the sharps are used as natural signs.%
  \autocites{Harran:Cautionary1}{Harran:Cautionary2}. 
In m.~74, the Alto F must be natural to match the F\na\octave{5} in the Tiple.
By the same logic, the F in m.~73 should also be natural, and this avoids a B\fl{}/F\sh{} sonority, which would be unusual for Padilla.\\
74 & A. I & On the phrase \term{peregrino tono}, the Alto outlines the final cadence of the plainchant \term{tonus peregrinus} transposed in \term{cantus mollis}: G--B\fl{}--A--G.\\
86 & A. II & In a widespread convention, the slur is written over the group of notes without a clearly specified beginning and end, but the text underlay makes the slur placement clear. (Likewise in m. 55, T. II.)\\
\end{criticalnotes}


