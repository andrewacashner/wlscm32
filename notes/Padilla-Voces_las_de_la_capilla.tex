\noteshead{Juan Gutiérrez de Padilla, \worktitle{Voces, las de la capilla}}

\begin{notesources}

\begin{source}
\sourcedescription{\signature{MEX-Pc}{Leg.~3/3}, In manuscript partbooks, \worktitle{Navidad del año de 1657}}
\annotation{A 6/ Padilla}
\parts{SAT, ATB; B.~II is instrumental (\term{bajón} and other continuo instruments)}
\end{source}

\begin{source}
\sourcedescription{Modern edition: \fullcite{Padilla:Tello}}
\end{source}

\end{notesources}

This piece is from the complete cycle of villancicos composed by Juan Gutiérrez de Padilla for the cathedral of Puebla de los Ángeles and performed at Christmas 1657.%
  \autocites{Puebla:Microfilm}{Stanford:Catalog}
  {Mauleon:PadillaPalafox}{Hurtado:Padilla}{Stevenson:Padilla}
The only source is a set of partbooks, each labeled \worktitle{Navidad del año de 1657}.
The Tiple I partbook has the additional marking \quoted{en 8 quadernos}, confirming the total of eight notebooks for the whole cathedral ensemble, typically organized in two choirs of four voices each.
The partbooks include all the villancicos needed for performance at Matins for Christmas and Epiphany of the 1657--1658 liturgical year, plus the hymn \worktitle{Christus natus est nobis}.

Only six partbooks contain the music for \worktitle{Voces, las de la capilla}, scored (as the parts indicate) \emph{a 6}.
Tiple and Bassus of Chorus II are not included.
The Altus I and Tenor II parts for this piece include the composer's name, \quoted{Padilla}.

The bass part is in the partbook of Bassus, Chorus I, but this part plays with the voices of Chorus II throughout the piece.
Typical of Padilla's scores and many other contemporary villancicos, this part only includes brief textual incipits to help the player coordinate with the ensemble.
The manuscripts of other Padilla villancicos, such as \worktitle{Al establo más dichoso} of 1652, specify that the instrument used for these bass parts was the \term{bajón}.
It was probably also doubled with other continuo instruments.
As always, the other voices could also be doubled instrumentally.

The handwriting, ink, and paper are consistent with that used in Padilla's other extant Christmas cycles in the cathedral archive.
In each set between 1651 set and the last one in 1659, a year before Padilla signed a power-of-attorney document citing his failing health\autocite{Mauleon:PadillaCivil} (he died in 1664), there is a pronounced decline in the quality of the handwriting.
It seems reasonable to believe this to be Padilla's own hand.

The partbooks show signs of repeated use over many years: fingerprints, multiple performers' names written in the parts, corrections, and added barlines.
The barlines suggest that the pieces were still being performed into the eighteenth century, when performers were becoming less familiar with the old mensural notation.

The one previous edition of this piece contains a serious error based on the misreading of rests, as discussed below.
The piece has been recorded once prior to this edition, with an erroneous reading of the repeats.%
  \autocite{Padilla:HabanaCD}

\notesection{Structure of Repeated Sections}

Like many of Padilla's villancicos, this one begins with an introductory section for a portion of the ensemble (labeled here \term{introducción} after the model of contemporary poetry imprints).
Next the full ensemble sings the \gloss{estribillo}{refrain}.
Two \gloss{coplas}{verses} follow, the first sung by Chorus I and the second by Chorus II. 
Finally the estribillo is repeated.

The repeat of the estribillo is indicated clearly.
In all the parts there is a \term{signum congruentiae} at the beginning of the estribillo (\measure{24} in the edition).
In every part except Tiple I, after the end of the coplas, the scribe has written the first few notes of the estribillo and the \term{signum} to point back to the estribillo.

The \term{introducción} has two parts: the first section has two stanzas of poetry and is sung by Chorus I; the second is a \gloss{respuesta}{response} section sung by Chorus II.
The one recording of this piece errs in performing both stanzas of the Chorus I section first and then performing the respuesta once.
While that arrangement might seem logical,the parts all clearly indicate that the first part of the introducción (with the first stanza of lyrics) proceeds immediately into the respuesta, and then the whole section is repeated from the beginning with the second stanza of lyrics.
This means that the respuesta is sung twice, once after each stanza of the introducción.

\notesection{Related Sources}

A 1649 catalog entry from the collection of Portuguese King John IV appears to indicate an earlier setting of the same or similar text as Padilla's.
The citation is listed under the works of Francisco de Santiago, who was chapelmaster of Seville Cathedral until his death in 1644: \quoted{Vozes las de la capilla. solo. Ya trechos las distancias. a 9}.%
  \autocite[caixão 26, no.~674]{JohnIV:Catalog}

Compare the related poem, \worktitle{Cantores de la capilla}, performed at Seville Cathedral, Epiphany 1647, probably in a setting by Santiago's successor Luis Bernardo Jalón.%
  \footnote{\worktitle{Villancicos que se cantaron en la S.~Iglesia Metropolytana de Sevilla, en los Maytines de los Santos Reyes. En este año de mil y seiscientos y quarenta y siete} (Puebla de los Ángeles, private collection, courtesy of Gustavo Mauleón Rodríguez).}

\criticalnotesheader
\begin{criticalnotes}
1 & Ti. I & Sharps on E (and B) are used as naturals according to Spanish conventions.\\
39 & T. II & Lyrics: \quoted{aguarda} corrected to \quoted{aguardan} as in all the other parts.\\
50 & A. II & 
The first rest, a perfect semibreve (modern dotted whole note), was inadvertently omitted from the MS. 
The T. II and B. I, who have the same gesture as the A. II here, both have the missing rest. 
The rest may have been accidentally erased when another correction was made to the MS at the end of the preceding phrase.
The Palacios and Tello edition mistakenly deletes the rest from T. II and B. I rather than adding it to A. I, producing a collision between choirs in the succeeding phrase.\\
72 & T. II & Last note is breve with fermata instead of semibreve.\\
73--74 & A. I &
Sharp signs on F in MS have been interpreted as cautionary accidentals, as signs to the singer \emph{not} to follow \term{ficta} conventions of sharping the F in the gesture G--F--G. 
In other words, the sharps are used as natural signs.%
  \autocites{Harran:Cautionary1}
In m.~74, the Alto F must be natural to match the F\na\octave{5} in the Tiple.
By the same logic, the F in m.~73 should also probably be natural, and this avoids a B\fl{}/F\sh{} sonority, which would be unusual for Padilla.\\
74 & A. I & On the phrase \term{peregrino tono}, the Alto outlines the final cadence of the plainchant \term{tonus peregrinus} transposed in \term{cantus mollis}: G--B\fl{}--A--G.\\
86 & A. II & In a widespread convention, the slur is written over the group of notes without a clearly specified beginning and end, but the text underlay makes the slur placement clear. (Likewise in \measure{55}, T. II.)\\
\end{criticalnotes}


