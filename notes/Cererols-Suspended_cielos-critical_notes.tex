% Critical notes for Cererols, Suspended cielos
% Revised 2017/02/08

\begin{criticalnotes}
    2       
    & A. I
    & Coloration starts n. 1
    & CAN: Same; 
    Bbc: Starts on n. 2; cf. Ti. I-1, Ti. I-2, T. I \\
%
%    10
%    & Ti. I-2
%    & N. 1, C\sh{} \term{ficta}
%    & CAN, Bbc: C;
%    C\sh{} avoids cross-relation with Ti. II \\
%
%    23--24
%    & T. II
%    & M. 24, n. 2 \& m. 25, n. 1, F\sh{} \term{ficta}
%    & CAN, Bbc: F; cf. mm. 21--22 contour, harmony \\
%
%    23--24 
%    & A. II
%    & Tened, \emph{tened, tened}    
%    & CAN: tened \MSrepeat{} parad; 
%    Bbc: tened \MSrepeat{} tened;
%    cf. mm. 21 (Ti. I-2, A. I), m. 23--24 (T. II) \\
%
%    24
%    & T. II
%    & N. 3, \pitch{A}{4}
%    & CAN: Same; 
%    Bbc: \pitch{F}{4}; 
%    cf. m. 21 (A. I), m. 24 (A. II) \\
%
%    25
%    & T. II
%    & N. 4, cautionary F\na{}
%    & CAN, Bbc: F; 
%    F\na{} avoids cross-relations with Ti. I-2, A. II entrances \\
%
%    26--28
%    & Ch. I
%    & Tened, tened, parad
%    & CAN, A. I: tened tened parad; 
%    CAN: Ti. I-1, Ti. I-2, T. I: tened \MSrepeat{} parad; 
%    Bbc, Ti. I-1, A. I: tened parad parad; 
%    Bbc, Ti. I-2: parad tened parad \\
%
%    29--30
%    & Ti. I-2
%    & M. 29, n. 2, F\sh{} \term{ficta}; m. 30, n. 2, F\na{} \term{ficta}
%    & CAN, Bbc: Both F\na;
%    cf. imitation m. 32 (Ti.  I-2), mm. 77-78 (A. II); 
%    contrapuntal motion into octave vs. Ac. \\
%
%    39 
%    & Ti. II
%    & N. 3, cautionary C\na{}
%    & CAN, Bbc: C; descending motion \\
%
%    41
%    & B. II
%    & N. 1, B\fl{}
%    & CAN, Bbc: B;
%    cf. explicit B\fl{} in Ac.; \term{una nota super la} \\
%
%    44
%    & Ac.
%    & N. 1, C\sh{}
%    & CAN, Bbc: C;
%    cf. explicit C\sh{} in B. II; ascending motion \\
%
%    49
%    & Ti. II
%    & N. 2, C\sh{}
%    & Bbc: Same;
%    CAN: C; cadence \\
%
%    50
%    & Ti. I-2
%    & \term{Ficta} F\na{}--F\sh{}--F\sh{}
%    & CAN, Bbc: F--F--F; 
%    first F resolves cadence; next Fs lead up to G (sugg.) \\
%
%    55--56
%    & T. II
%    & Coloration starts m. 56, n. 1
%    & Bbc: Same; 
%    CAN: Starts m. 55, n. 1; 
%    cf. Ti. II, A. II, B. II, Ac. \\
%
%    57--58
%    & Ti. I-2
%    & \term{Ficta}
%    & Cf. mm. 29--30 \\
%
%    67 
%    & Ti. I-1
%    & N. 4, G\sh{}
%    & Bbc: Same, 
%    CAN: G; contrapuntal motion into octave vs. Ac; inversion of fugue subject \\
%
%    67 
%    & Ac. 
%    & N. 2, C\sh{}
%    & CAN, Bbc: C; 
%    cf. explicit C\sh{} in T. I \\
%
%    67
%    & Ac.
%    & N. 4, B\na{}
%    & Bbc: Same;
%    CAN: B\sh{}; 
%    Cautionary sharp to indicate natural (cancels \term{una nota super la}) \\
%
%    68
%    & B. II
%    & N. 3, F\na{}
%    & Bbc: Same;
%    CAN: F\sh{};
%    Cautionary sharp to indicate natural, cf. fugue subject, unison Ac. \\
%
%    73--74
%    & Ti. I-2
%    & M. 74, nn. 1--2, slur
%    & CAN: Same;
%    Bbc: Slur, m. 73, n. 4--m. 74, n. 1 \\
%
%    75--86
%    & All
%    & Y con sollozos tiernos, un niño soberano
%    & CAN, Imprints: Same; 
%    Bbc: Y desde un pan divino, un hombre soberano \\
%
%    85
%    & A. II
%    & Nn. 1--2, slur
%    & CAN: Same;
%    Bbc: Omit \\
%
%    86 
%    & T. I
%    & N. 1, \pitch{A}{3}
%    & CAN: \pitch{B}{3} 
%    Bbc: Missing voice part;
%    cf. B. II, Ac.; MEM (corr.)\\
%
%    89
%    & Ti. I-2
%    & Nn. 3--5, C\sh{} \term{ficta}
%    & CAN, Bbc: C; 
%    cf. motive, m. 90 (T. II); explicit C\sh{} in Bbc, m. 92 (T. II) \\
%
%    92
%    & T. II
%    & Nn. 3--5, C\sh{}
%    & Bbc: Same; 
%    CAN: C; 
%    cf. m. 89 \\
%
%    94 
%    & A. I
%    & N. 2, B\fl{} \term{ficta}
%    & CAN, Bbc: B; 
%    cf. motive, mm. 89, 90; descending motion \\
%
%    95 
%    & T. I
%    & Nn. 3--5, C\sh{} \term{ficta}
%    & CAN, Bbc: C;
%    cf. mm. 89, 92 \\
%
%    97 
%    & T. I-1
%    & Nn. 3--5, F\sh{}
%    & Bbc: Same; 
%    CAN: F; 
%    cf. mm. 90, 93 \\
%
%    100
%    & T. II
%    & Nn. 1--3, C\sh{}
%    & Bbc: Same;
%    CAN: C;
%    cf. mm. 89, 92, 95 \\
%
%    105--108
%    & T. I-1
%    & \pitch{G}{5}--\pitch{G}{5}--\pitch{D}{5}--\pitch{D}{5}
%    & CAN: Same;
%    Bbc: \pitch{G}{5}--\pitch{G}{5}--\pitch{A}{5}--\pitch{B}[\fl]{5}--\pitch{A}{5} \\
%
%    106 
%    & Ti. I-2 
%    & N. 1, minim 
%    & CAN: Same; Bbc: Semibreve (corr.) \\
%
%    107
%    & Ti. I-2, A. II
%    & F\sh{} \term{ficta}
%    & CAN, Bbc: F;
%    cadence, anticipating explicit final F\sh{} in m. 108;
%    cf. motive m. 1, 29--30, 130--131 (sugg.) \\
%
%    109--114
%    & All
%    & Las fugas que el primer hombre formó
%    & Imprints: Same;
%    CAN: Las fugas del primer hombre formó;
%    Bbc: Coplas missing \\
%
%    111
%    & Ac.
%    & N. 1, C\sh{}
%    & CAN: C;
%    cf. explicit C\sh{} in T. I \\
%
%    120
%    & A. I
%    & N. 1, B\fl{} \term{ficta}
%    & CAN: B; 
%    \term{una nota super la} \\
%
%    130
%    & Ti. I-1
%    & N. 2, F\sh{} \term{ficta}
%    & CAN: F; 
%    cadence; anticipate explicit final F\sh{} in m. 131; 
%    cf. motive m. 1, 29--30, 107 (sugg.) \\
%
%    131
%    & Ti. I-2
%    & Fermata
%    & CAN: Omit; 
%    cf. Ti. I-1, Ti. I-2, T. I, Ac. (corr.) \\ 
%
%     132--138
%    & All
%    & Qué mucho si a los despeños
%    & Imprints: Same;
%    CAN: Qué mucho que a los despeños \\
%
%    134
%    & Ti. I-1
%    & N. 2, F\sh{} \term{ficta}
%    & CAN: F; 
%    cf. 137 (sugg.)\\ 
%
%    137
%    & Ti. I-1
%    & N. 2, F\sh{} \term{ficta}
%    & CAN: F;
%    cf. explicit F\sh{} in repeated passage, m. 243 \\
%
%    139
%    & Ti. I-1
%    & N. 6, F\sh{} \term{ficta}
%    & CAN: F;
%    cf. explicit F\sh in repeated passage, m. 192 \\
%
%    146 
%    & Ti. I-2
%    & Nn. 1--2, C\sh{} \term{ficta}
%    & CAN: C;
%    cf. explicit C\sh{} in repeated passage, m. 199; 
%    voice exchange in m. 148 \\
%
%    151
%    & Ti. I-2
%    & N. 2, C\sh{} \term{ficta}
%    & CAN: C;
%    cf. m. 130 (sugg.) \\
%
%    154 
%    & Ti. I-1
%    & N. 1, C\sh{} \term{ficta}
%    & CAN: C;
%    cf. motive m. 155 (Ti. I-2), m. 158 (Ti. I-1);
%    explicit C\sh{} in imitation, m. 159 (Ti. I-2), \& repeated passage, m. 196 \\
%
%    155
%    & Ti. I-2
%    & N. 1, F\sh{} \term{ficta}
%    & CAN: F;
%    cf. explicit F\sh{} in parallel passage, m. 208 (Ti. I-2) \\
%
%    157
%    & Ac.
%    & N. 2, \pitch{F}{3}
%    & CAN: Notehead missing bc. of tear; 
%    cf. MEM \\
%
%    158
%    & Ti. I-1
%    & N. 1, F\sh{} \term{ficta}
%    & CAN: F;
%    cf. m. 155 \\
%
%    160
%    & Ti. I-2
%    & N. 2, F\sh{} \term{ficta}
%    & CAN: F; 
%    cf. m. 130, 151 \\
%
%    164
%    & Ti. I-1
%    & N. 2, F\sh{} \term{ficta}
%    & Cf. explicit F\sh{} in m. 111 \\
%
%    164
%    & Ac. 
%    & N. 1, C\sh{}
%    & Cf. m. 111 \\
%
%    166
%    & Ch. I
%    & Breath mark after n. 1
%    & CAN, Ch. I: Semiminim rest after n. 1 (cf. m. 113);
%    cf. MEM (corr.)\\ 
%
%    173
%    & A. I
%    & N. 1, B\fl{} \term{ficta}
%    & Cf. m. 120 \\
%
%    178--180
%    & A. I
%    & La disonancia
%    & CAN: las disonancias;
%    cf. Ti. I-1, Ti. I-2, T. I, Imprints \\
%
%    182
%    & Ti. I-2
%    & N. 1, C\sh{} \term{ficta}
%    & Cf. m. 129 \\
%
%    183
%    & Ti. I-1
%    & N. 2, F\sh{} \term{ficta}
%    & Cf. mm. 130, 151 \\
%
%    187
%    & Ti. I-1
%    & N. 2, F\sh{} \term{ficta}
%    & Cf. m. 134 \\
%
%    190
%    & Ti. I-1
%    & N. 2, F\sh{} \term{ficta}
%    & Cf. m. 137 \\
%
%    196
%    & Ti. I-1
%    & N. 1, F\sh{} \term{ficta}
%    & Cf. explicit F\sh{} in m. 143 \\
%
%    205
%    & Ti. I-2
%    & N. 1, C\sh{} \term{ficta}
%    & Cf. explicit C\sh{} in m. 152 \\
%
%    206--214
%    & Ti. I-1
%    & desentono
%    & Imprints: Same; 
%    CAN: desatento; 
%    cf. Ti. I-2 \\
%
%    207
%    & Ti. I-1
%    & N. 1, C\sh{} \term{ficta}
%    & Cf. m. 154, explicit C\sh{} in m. 159 \\
%
%    208--214
%    & All
%    & tan vano
%    & Imprints: Same;
%    CAN: tan grande;
%    Meter is \term{romance} in \term{a--o} (corr.)\\
%
%    211
%    & Ti. I-1
%    & N. 1, F\sh{} \term{ficta}
%    & Cf. m. 158, explicit F\sh{} in m. 208 (Ti. I-1) \\
%
%    213
%    & Ti. I-1
%    & N. 2, F\sh{} \term{ficta}
%    & Cf. m. 130, 151, 160 \\
%
%    216--219 
%    & All
%    & Sustenido
%    & Imprints: Same;
%    CAN, A. I, T. I: Same;
%    CAN, Ti. I-1, Ti. I-2: susteniendo \\
%
%    217
%    & Ac.
%    & N. 1, F\sh{} \term{ficta}
%    & Cf. explicit F\sh{} in m. 111 \\
%
%    219
%    & All
%    & Breath mark after n. 1
%    & Cf. m. 166 \\
%
%    219
%    & Ti. I-1
%    & N. 1, F\sh{} \term{ficta} 
%    & Cf. explicit F\sh{} in m. 113 \\
%
%    220
%    & Ti. I-1
%    & N. 5, F\sh{} \term{ficta}
%    & Cf. mm. 114, 167 \\
%
%    227
%    & A. I
%    & N. 1, B\fl{} \term{ficta}
%    & Cf. mm. 120, 173 \\
%
%      229--236
%    & All
%    & Lo inmenso a espacio
%    & Imprints: Same;
%    CAN: lo inmenso spacio \\
%
%    236
%    & Ti. I-1
%    & N. 2, F\sh{} \term{ficta}
%    & Cf. mm. 130, 151, 160, 213 \\
%
%    249
%    & Ti. I-2
%    & Semibreve rest
%    & CAN: Omit;
%    cf. mm. 143, 196 (corr.)\\
%
%    260
%    & Ti. I-1
%    & N. 1, C\sh{} \term{ficta}
%    & Cf. m. 154, 159, 207, explicit C\sh{} in m. 265 (Ti. I-2) \\
%
%    263 
%    & Ti. I-1
%    & N. 2, B\fl{} \term{ficta}
%    & Cf. motive m. 260 (Ti. I-1); \term{una nota super la} \\
%
%    264
%    & Ti. I-1
%    & N. 1, F\sh{} \term{ficta}
%    & Cf. m. 155, 158 \\
%
%    266
%    & Ti. I-1
%    & N. 2, F\sh{} \term{ficta}
%    & Cf. m. 130, 151, 160, 213 160 \\
%
%    267
%    & Ti. I-1
%    & N. 1, F\sh{} \term{ficta}
%    & CAN: F; 
%    cf. explicit F\sh{} in mm. 108, 131, 161, 184, 214, 237;
%    unless F\na{} is deliberate before repeat of estribillo \\

\end{criticalnotes}









