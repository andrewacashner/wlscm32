\noteshead{Juan Gutiérrez de Padilla, \worktitle{Al establo más dichoso}}

\begin{notesources}

\begin{source}
\sourcedescription{\signature{MEX-Pc}{Leg.~1/3}, In manuscript partbooks, \worktitle{Navidad del año de 1652}}
\annotation{Ensaladilla}
\parts{SATB, SATB; both bass parts are instrumental, with indications for \term{bajón}}
\end{source}

\end{notesources}

This piece is part of Padilla's cycle of villancicos for Christmas 1652 at Puebla Cathedral, copied as a set into individual partbooks, probably in Padilla's own hand.
This is the earliest of this composer's extant Christmas cycles for which all the partbooks survive.

The partbooks bear the names of some of the Puebla chapel performers in various places.
The name of Francisco Rodríguez is in the Tiple I part, and that of Sr. Nicolás Griñón is in the Tenor II.

The Puebla cathedral chapel was usually organized in two choirs, but since this piece does not utilize polychoral textures, the edition presents the voices in a single-choir arrangement.

Because of these multiple sections and large amount of repetition within sections, the parts are written in an abbreviated manner.
This edition writes out most of the reprises and other repeated material to achieve a more straightforward presentation for performers.
Examples of this include the stanzas of the \term{Nuevo Troyano} and the \term{responsión} reprise of the \term{Papalotillo}.

The piece has been recorded once prior to this edition.%
  \autocite{Padilla:1652ChristmasCD}

\notesection{Bass Parts}

Both bass parts are intended for instrumental performance.
They have only incipits of the lyrical text to help orient the player.
The Bassus I part contains this marking after the \term{Nuevo Troyano} and before the \term{Papalotillo}: \quotedgloss{antes del papalotillo diçe el harriero con el otro bajon}{Before the \term{papalotillo} the mule skinner \quoted{speaks} with the other \term{bajón}}.
This implies that both bass parts were played on the \term{bajón}, not to exclude other continuo instruments like harp.
The sections labeled \term{Dúo} (the \term{Arriero} and the beginning of the \term{Negrilla}) are actually, in modern terms, vocal solos with accompaniment, perhaps intended for a single vocalist with solo \term{bajón}.

If a vocal solo with \term{bajón} is a \term{Dúo}, then the section marked \term{Papalotillo Solo} would seem to be a true solo without any accompaniment at all.
The scribe has only written a four-bar accompaniment pattern in the bass, with unspecific indications to repeat.
This edition therefore includes the bass line for the \term{Papalotillo} only for the \term{responsión}.
But it is also plausible that the bass should repeat the same phrase as accompaniment for the coplas.

\notesection{The \soCalled{Gloria}}

In the midst of the \term{Negrilla}, an \soCalled{Angolan} character (T.~II) sings, \quoted{Listen, for we are singing like the angels}.
Then the two upper voices of the first chorus (Tiple and Altus I) sing the angels' song from Luke~2 in Spanish, \quoted{Gloria en las alturas y en la tierra, paz}.
This section is labeled \quoted{A 3} in the Tiple I part.

The Tiple II and Altus II parts contain only one phrase of notated music for the \term{ensaladilla}.
Both are labeled \quoted{A 3 de la ensaladilla}, and contain music for the Spanish \quoted{Gloria}, but in C meter instead of the CZ meter of the other voices.
The Tiple II part actually includes an earlier version in CZ that has been crossed out and replaced with one in C.
(The Tiple II melody would seem to quote the common plainchant intonation of the \worktitle{Gloria in excelsis} of the Mass.)

The only way to align these voices with those of Chorus I is to maintain the theoretical $3:2$ proportion of minims between CZ and C meter so that the perfect semibreve in CZ is equal in time to the semibreve in C.
For practical performance, the edition puts all four voices in \meter{6}{2}, and preserves the proportion of the second-choir voices simply by adding dots.
A more literal transcription of the original notation would be as follows:

\includemusic{scores/Padilla-Al_establo_mas_dichoso/Padilla-Al_establo_mas_dichoso-Gloria}

The reason for the marking \quoted{A 3}, when there is notated music for four voices, is unclear.
The Altus I and Altus II voices have the same pitches but are in different meters; so perhaps Padilla considered these to be a single voice, or perhaps one of these voices should be omitted.
Given the crossed-out and corrected music in the Tiple II, it is also possible that \quoted{A 3} functions primarily as a rehearsal marking, and that Padilla changed his mind about the scoring after writing out the Chorus I parts, but left the marking intact.

\criticalnotesheader
\begin{criticalnotes}
75, 77 & T.~I, notes 4--5 
  & MS has semibreve, minim, in contrast to the other three voices. Possibly an error.\\
78--85 & B.~I 
  & The phrase in \measures{74--77} could be repeated to accompany the coplas. MS does not specify repeats here with sufficient clarity.\\
86--89 & Chorus I, all 
  & It is possible the \term{responsión} is only meant to be reprised after the final copla, not after each pair.\\
119 & All & Only the T.~I has a fermata here. Given the following text (\quoted{Hush!}) it is possible but speculative (as in the one recording) that this voice only is meant to hold past the cutoff of the other voices. But it is also common to omit fermatas inadvertently from some of the parts.\\
129 & All & See discussion of \soCalled{Gloria} above.\\
\end{criticalnotes}
