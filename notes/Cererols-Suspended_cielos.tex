\edchapter{Critical Notes}

\noteshead{Joan Cererols, \worktitle{Suspended, cielos, vuestro dulce canto}}

\begin{notesources}

\begin{source}
    \sourcedescription[CAN]{%
        \signature{E-CAN}{AU/0116}, Canet de Mar, Arxiu Parròquia de Sant Pere i
        Sant Pau de Canet, Bisbat de Girona, Fons capella de música; Manuscript
        performing parts of complete piece with coplas%
    }
    \annotation{Villancico al \oldabbrev{SS}{mo} \oldabbrev{S}{to}/ Suspendet 
    cielos a 8/ Cererols}
    \parts{SSAT, SATB, \term{Acompañamiento}; B.~II is instrumental}
\end{source}

\begin{source}
    \sourcedescription[Bbc]{\signature{E-Bbc}{M/765/25}, Manuscript performing 
    parts, previously unattributed}
    \annotation{Villansico a 8}
    \parts{SSA, SATB, missing T.~I and Acomp.; estribillo only, no coplas; alternate
    text}
\end{source}

\begin{source}
    \sourcedescription[MEM]{Modern edition: \fullcite[xxv, 
    221--236]{Cererols:MEM-VC}, based on CAN}
\end{source}

\end{notesources}

Joan Cererols (1618--1680) was a monk at the Benedictine Abbey of Our Lady of
Montserrat, a pilgrimage site at the top of a mountain north of Barcelona and
home of Europe's oldest continually established singing school for boys, the
Escolania de Montserrat.%
\begin{Footnote}
    \Autocite[227--284]{Cashner:PhD}.
    This biographical sketch is based on 
    \autocites{Estrada:CererolsBio}{Balanza:CererolsFamily}.
\end{Footnote}
Joan Pau Cererols Fornell was baptized in 1618 in the nearby village of
Martorell, the youngest child of Jaume Cererols, a tailor.
His mother died when he was ten, and only a few months later Joan entered the
boarding school as a chorister at the Escolania.
He entered the novitiate of the Monsterrat Benedictines at age 18, in 1636, and
remained at the monastery until his death in 1680.
He served for many years as chapelmaster of the Escola (the choir of boys and
men), teacher in the Escolania, and sacristan of the abbey church.
According to a monastery chronicle, Cererols was also an excellent poet and
theologian, and his pupils included distinguished chapelmasters and organists
throughout Catalonia and the rest of Spain.

This critical edition is based on two manuscript sources of this villancico, 
CAN and Bbc, the latter previously unknown.
It replaces the earlier edition by David Pujols, which was based on CAN only 
and included some errors.
Most critically, MEM mistakenly omits the breve rest in this edition's measure 
39, thus shifting the Alto fugue entrance one bar earlier through \measure{47}.
This edition is not only based on an additional manuscript source but also 
improves on the earlier Montserrat edition by indicating editorial additions 
and mensural coloration, correcting text underlay, and including the 
manuscript's figured bass.

The two manuscript sources present minor variants of the same piece of music.
Both sources are probably copies based with some degree of separation on an 
original exemplar by Cererols at the Abbey of Montserrat.
The original was most likely lost when the abbey library burned in the 
nineteenth century.

Only CAN includes all the voice parts and the coplas, and thus it is the 
primary source for this edition (figure~\ref{Cererols-CAN}).%
  \autocite[60--61]{Bonastre:CanetCatalog}
In comparing the parts that survive in both versions, there are only a few 
significant differences in Bbc: 
\begin{enumerate}
    \item One phrase of text is changed to make the piece fit a Eucharistic
        dedication.  
        The text of CAN makes more sense as a Christmas piece, despite the
        dedication to the Blessed Sacrament on the cover leaf of the group of
        partbooks.
    \item The highest voice part has a different final phrase in the estribillo.
    \item Bbc includes the performance instructions \term{eco} and \term{falsete}.
    \item Bbc differs in the use of accidentals on C, F, and B; in most cases
        it is more explicit, writing out accidentals in situations where
        \term{musica ficta} practice would suggest them anyway.
\end{enumerate}
This edition primarily follows CAN, but does include the dynamic markings from 
Bbc in parentheses.
Bbc writes out more accidentals explicitly, clarifying the usage of \term{musica
ficta}.


\notesection{Related Sources}

This villancico is the only complete musical setting yet found of one of the 
most popular villancico poems of the seventeenth century.
The poem as set by Cererols is one variant of a textual tradition extending 
back as early as a Royal Chapel performance in 1651.%
\begin{Footnote}
    \signature{E-Mn}{R/34199/27}, omitted from \autocite{BNE:VCs17C}; see
    figure~\ref{Cererols-Poem}.
\end{Footnote}
A distinct branch of later variant versions may be traced to the work of Manuel 
de León Marchante from 1675.

This family of villancico poems is attested in the following imprints:

\begin{inlinetable}
\begin{tabular}{lll}
  1651 & Madrid & \signature{E-Mn}{R/34199/27}\\
  1668 & Calatayud & \signature{GB-Lbl}{11450.dd.8~(54)}\\
  1675 & Alcalá & Reprinted in Marchante, \worktitle{Obras poéticas} 
    (Madrid, 1733), 139\\
  1680 & Seville & \signature{E-Mn}{VE/83/10}\\
  1681 & Seville & \signature{E-Mn}{VE/79/7}\\
  1683 & Zaragoza & \signature{E-Mn}{VE/129/2}, 
    \signature{GB-Lbl}{1073.k.22~(07)}\\
  1689 & Madrid & \signature{E-Mn}{VE/88/80}\\
\end{tabular}
\end{inlinetable}

Cererols's text incorporates aspects of both the early Royal Chapel tradition 
and the versions influenced by Marchante.
In a few passages, indicated in the notes below, the text of the coplas in the 
Canet manuscript departs from the consensus of the other poetic imprints from 
this villancico tradition.
This edition preserves the text in CAN, but corrects a few phrases in which the 
consensus reading of the other imprints makes more sense both poetically and 
grammatically.

\clearpage
\criticalnotesheader
% Critical notes for Cererols, Suspended cielos
% Revised 2017/02/08
\begin{criticalnotes}
    2       
    & A. I
    & Coloration starts n. 1
    & CAN: Same; 
    Bbc: Starts on n. 2; cf. Ti. I-1, Ti. I-2, T. I \\

    10
    & Ti. I-2
    & N. 1, C\sh{} \term{ficta}
    & CAN, Bbc: C;
    C\sh{} avoids cross-relation with Ti. II \\

    23--24
    & T. II
    & M. 24, n. 2 \& m. 25, n. 1, F\sh{} \term{ficta}
    & CAN, Bbc: F; cf. mm. 21--22 contour, harmony \\

    23--24 
    & A. II
    & Tened, \emph{tened, tened}    
    & CAN: tened \MSrepeat{} parad; 
    Bbc: tened \MSrepeat{} tened;
    cf. mm. 21 (Ti. I-2, A. I), m. 23--24 (T. II) \\

    24
    & T. II
    & N. 3, \pitch{A}{4}
    & CAN: Same; 
    Bbc: \pitch{F}{4}; 
    cf. m. 21 (A. I), m. 24 (A. II) \\

    25
    & T. II
    & N. 4, cautionary F\na{}
    & CAN, Bbc: F; 
    F\na{} avoids cross-relations with Ti. I-2, A. II entrances \\

    26--28
    & Ch. I
    & Tened, tened, parad
    & CAN, A. I: tened tened parad; 
    CAN: Ti. I-1, Ti. I-2, T. I: tened \MSrepeat{} parad; 
    Bbc, Ti. I-1, A. I: tened parad parad; 
    Bbc, Ti. I-2: parad tened parad \\

    29--30
    & Ti. I-2
    & M. 29, n. 2, F\sh{} \term{ficta}; m. 30, n. 2, F\na{} \term{ficta}
    & CAN, Bbc: Both F\na;
    cf. imitation m. 32 (Ti.  I-2), mm. 77-78 (A. II); 
    contrapuntal motion into octave vs. Ac. \\

    39 
    & Ti. II
    & N. 3, cautionary C\na{}
    & CAN, Bbc: C; descending motion \\

    41
    & B. II
    & N. 1, B\fl{}
    & CAN, Bbc: B;
    cf. explicit B\fl{} in Ac.; \term{una nota super la} \\

    44
    & Ac.
    & N. 1, C\sh{}
    & CAN, Bbc: C;
    cf. explicit C\sh{} in B. II; ascending motion \\

    49
    & Ti. II
    & N. 2, C\sh{}
    & Bbc: Same;
    CAN: C; cadence \\

    50
    & Ti. I-2
    & \term{Ficta} F\na{}--F\sh{}--F\sh{}
    & CAN, Bbc: F--F--F; 
    first F resolves cadence; next Fs lead up to G (sugg.) \\

    55--56
    & T. II
    & Coloration starts m. 56, n. 1
    & Bbc: Same; 
    CAN: Starts m. 55, n. 1; 
    cf. Ti. II, A. II, B. II, Ac. \\

    57--58
    & Ti. I-2
    & \term{Ficta}
    & Cf. mm. 29--30 \\
\end{criticalnotes}

\begin{criticalnotes}
    67 
    & Ti. I-1
    & N. 4, G\sh{}
    & Bbc: Same, 
    CAN: G; contrapuntal motion into octave vs. Ac; inversion of fugue subject \\

    67 
    & Ac. 
    & N. 2, C\sh{}
    & CAN, Bbc: C; 
    cf. explicit C\sh{} in T. I \\

    67
    & Ac.
    & N. 4, B\na{}
    & Bbc: Same;
    CAN: B\sh{}; 
    Cautionary sharp to indicate natural (cancels \term{una nota super la}) \\

    68
    & B. II
    & N. 3, F\na{}
    & Bbc: Same;
    CAN: F\sh{};
    Cautionary sharp to indicate natural, cf. fugue subject, unison Ac. \\

    73--74
    & Ti. I-2
    & M. 74, nn. 1--2, slur
    & CAN: Same;
    Bbc: Slur, m. 73, n. 4--m. 74, n. 1 \\

    75--86
    & All
    & Y con sollozos tiernos, un niño soberano
    & CAN, Imprints: Same; 
    Bbc: Y desde un pan divino, un hombre soberano \\

    85
    & A. II
    & Nn. 1--2, slur
    & CAN: Same;
    Bbc: Omit \\

    86 
    & T. I
    & N. 1, \pitch{A}{3}
    & CAN: \pitch{B}{3} 
    Bbc: Missing voice part;
    cf. B. II, Ac.; MEM (corr.)\\

    89
    & Ti. I-2
    & Nn. 3--5, C\sh{} \term{ficta}
    & CAN, Bbc: C; 
    cf. motive, m. 90 (T. II); explicit C\sh{} in Bbc, m. 92 (T. II) \\

    92
    & T. II
    & Nn. 3--5, C\sh{}
    & Bbc: Same; 
    CAN: C; 
    cf. m. 89 \\

    94 
    & A. I
    & N. 2, B\fl{} \term{ficta}
    & CAN, Bbc: B; 
    cf. motive, mm. 89, 90; descending motion \\

    95 
    & T. I
    & Nn. 3--5, C\sh{} \term{ficta}
    & CAN, Bbc: C;
    cf. mm. 89, 92 \\

    97 
    & T. I-1
    & Nn. 3--5, F\sh{}
    & Bbc: Same; 
    CAN: F; 
    cf. mm. 90, 93 \\

    100
    & T. II
    & Nn. 1--3, C\sh{}
    & Bbc: Same;
    CAN: C;
    cf. mm. 89, 92, 95 \\

    105--108
    & T. I-1
    & \pitch{G}{5}--\pitch{G}{5}--\pitch{D}{5}--\pitch{D}{5}
    & CAN: Same;
    Bbc: \pitch{G}{5}--\pitch{G}{5}--\pitch{A}{5}--\pitch{B}[\fl]{5}--\pitch{A}{5} \\

    106 
    & Ti. I-2 
    & N. 1, minim 
    & CAN: Same; Bbc: Semibreve (corr.) \\

    107
    & Ti. I-2, A. II
    & F\sh{} \term{ficta}
    & CAN, Bbc: F;
    cadence, anticipating explicit final F\sh{} in m. 108;
    cf. motive m. 1, 29--30, 130--131 (sugg.) \\
\end{criticalnotes}

\begin{criticalnotes}
    109--114
    & All
    & Las fugas que el primer hombre formó
    & Imprints: Same;
    CAN: Las fugas del primer hombre formó;
    Bbc: Coplas missing \\

    111
    & Ac.
    & N. 1, C\sh{}
    & CAN: C;
    cf. explicit C\sh{} in T. I \\

    120
    & A. I
    & N. 1, B\fl{} \term{ficta}
    & CAN: B; 
    \term{una nota super la} \\

    130
    & Ti. I-1
    & N. 2, F\sh{} \term{ficta}
    & CAN: F; 
    cadence; anticipate explicit final F\sh{} in m. 131; 
    cf. motive m. 1, 29--30, 107 (sugg.) \\

    131
    & Ti. I-2
    & Fermata
    & CAN: Omit; 
    cf. Ti. I-1, Ti. I-2, T. I, Ac. (corr.) \\ 

     132--138
    & All
    & Qué mucho si a los despeños
    & Imprints: Same;
    CAN: Qué mucho que a los despeños \\

    134
    & Ti. I-1
    & N. 2, F\sh{} \term{ficta}
    & CAN: F; 
    cf. 137 (sugg.)\\ 

    137
    & Ti. I-1
    & N. 2, F\sh{} \term{ficta}
    & CAN: F;
    cf. explicit F\sh{} in repeated passage, m. 243 \\

    139
    & Ti. I-1
    & N. 6, F\sh{} \term{ficta}
    & CAN: F;
    cf. explicit F\sh in repeated passage, m. 192 \\

    146 
    & Ti. I-2
    & Nn. 1--2, C\sh{} \term{ficta}
    & CAN: C;
    cf. explicit C\sh{} in repeated passage, m. 199; 
    voice exchange in m. 148 \\

    151
    & Ti. I-2
    & N. 2, C\sh{} \term{ficta}
    & CAN: C;
    cf. m. 130 (sugg.) \\

    154 
    & Ti. I-1
    & N. 1, C\sh{} \term{ficta}
    & CAN: C;
    cf. motive m. 155 (Ti. I-2), m. 158 (Ti. I-1);
    explicit C\sh{} in imitation, m. 159 (Ti. I-2), \& repeated passage, m. 196 \\

    155
    & Ti. I-2
    & N. 1, F\sh{} \term{ficta}
    & CAN: F;
    cf. explicit F\sh{} in parallel passage, m. 208 (Ti. I-2) \\

    157
    & Ac.
    & N. 2, \pitch{F}{3}
    & CAN: Notehead missing bc. of tear; 
    cf. MEM \\

    158
    & Ti. I-1
    & N. 1, F\sh{} \term{ficta}
    & CAN: F;
    cf. m. 155 \\

    160
    & Ti. I-2
    & N. 2, F\sh{} \term{ficta}
    & CAN: F; 
    cf. m. 130, 151 \\

    164
    & Ti. I-1
    & N. 2, F\sh{} \term{ficta}
    & Cf. explicit F\sh{} in m. 111 \\

    164
    & Ac. 
    & N. 1, C\sh{}
    & Cf. m. 111 \\

    166
    & Ch. I
    & Breath mark after n. 1
    & CAN, Ch. I: Semiminim rest after n. 1 (cf. m. 113);
    cf. MEM (corr.)\\ 

    173
    & A. I
    & N. 1, B\fl{} \term{ficta}
    & Cf. m. 120 \\

    178--180
    & A. I
    & La disonancia
    & CAN: las disonancias;
    cf. Ti. I-1, Ti. I-2, T. I, Imprints \\

    182
    & Ti. I-2
    & N. 1, C\sh{} \term{ficta}
    & Cf. m. 129 \\

    183
    & Ti. I-1
    & N. 2, F\sh{} \term{ficta}
    & Cf. mm. 130, 151 \\

    187
    & Ti. I-1
    & N. 2, F\sh{} \term{ficta}
    & Cf. m. 134 \\

    190
    & Ti. I-1
    & N. 2, F\sh{} \term{ficta}
    & Cf. m. 137 \\

    196
    & Ti. I-1
    & N. 1, F\sh{} \term{ficta}
    & Cf. explicit F\sh{} in m. 143 \\
\end{criticalnotes}

\begin{criticalnotes}
    205
    & Ti. I-2
    & N. 1, C\sh{} \term{ficta}
    & Cf. explicit C\sh{} in m. 152 \\

    206--214
    & Ti. I-1
    & desentono
    & Imprints: Same; 
    CAN: desatento; 
    cf. Ti. I-2 \\

    207
    & Ti. I-1
    & N. 1, C\sh{} \term{ficta}
    & Cf. m. 154, explicit C\sh{} in m. 159 \\

    208--214
    & All
    & tan vano
    & Imprints: Same;
    CAN: tan grande;
    Meter is \term{romance} in \term{a--o} (corr.)\\

    211
    & Ti. I-1
    & N. 1, F\sh{} \term{ficta}
    & Cf. m. 158, explicit F\sh{} in m. 208 (Ti. I-1) \\

    213
    & Ti. I-1
    & N. 2, F\sh{} \term{ficta}
    & Cf. m. 130, 151, 160 \\

    216--219 
    & All
    & Sustenido
    & Imprints: Same;
    CAN, A. I, T. I: Same;
    CAN, Ti. I-1, Ti. I-2: susteniendo \\

    217
    & Ac.
    & N. 1, F\sh{} \term{ficta}
    & Cf. explicit F\sh{} in m. 111 \\

    219
    & All
    & Breath mark after n. 1
    & Cf. m. 166 \\

    219
    & Ti. I-1
    & N. 1, F\sh{} \term{ficta} 
    & Cf. explicit F\sh{} in m. 113 \\

    220
    & Ti. I-1
    & N. 5, F\sh{} \term{ficta}
    & Cf. mm. 114, 167 \\

    227
    & A. I
    & N. 1, B\fl{} \term{ficta}
    & Cf. mm. 120, 173 \\

      229--236
    & All
    & Lo inmenso a espacio
    & Imprints: Same;
    CAN: lo inmenso spacio \\

    236
    & Ti. I-1
    & N. 2, F\sh{} \term{ficta}
    & Cf. mm. 130, 151, 160, 213 \\

    249
    & Ti. I-2
    & Semibreve rest
    & CAN: Omit;
    cf. mm. 143, 196 (corr.)\\

    260
    & Ti. I-1
    & N. 1, C\sh{} \term{ficta}
    & Cf. m. 154, 159, 207, explicit C\sh{} in m. 265 (Ti. I-2) \\

    263 
    & Ti. I-1
    & N. 2, B\fl{} \term{ficta}
    & Cf. motive m. 260 (Ti. I-1); \term{una nota super la} \\

    264
    & Ti. I-1
    & N. 1, F\sh{} \term{ficta}
    & Cf. m. 155, 158 \\

    266
    & Ti. I-1
    & N. 2, F\sh{} \term{ficta}
    & Cf. m. 130, 151, 160, 213 160 \\

    267
    & Ti. I-1
    & N. 1, F\sh{} \term{ficta}
    & CAN: F; 
    cf. explicit F\sh{} in mm. 108, 131, 161, 184, 214, 237;
    unless F\na{} is deliberate before repeat of estribillo \\

\end{criticalnotes}










