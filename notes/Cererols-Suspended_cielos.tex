\noteshead{Joan Cererols, \worktitle{Suspended, cielos, vuestro dulce canto}}

\begin{notesources}

\begin{source}
\sourcedescription[CAN]{\signature{E-CAN}{AU/0116}, Canet de Mar, Arxiu Parròquia de Sant Pere i Sant Pau de Canet, Bisbat de Girona, Fons capella de música; Manuscript performing parts of complete piece with coplas}
\annotation{Villancico al \oldabbrev{SS}{mo} \oldabbrev{S}{to}/ Suspendet cielos a 8/ Cererols}
\parts{SSAT, SATB, \term{Acompañamiento}; B.~II is instrumental; see \fullcite[60--61]{Bonastre:CanetCatalog}}
\end{source}

\begin{source}
\sourcedescription[Bbc]{\signature{E-Bbc}{M/765/24}, Manuscript performing parts, previously unattributed}
\annotation{Villansico a 8}
\parts{SSA, SATB, missing T.~I and Acomp.; estribillo only, no coplas}
\end{source}

\begin{source}
\sourcedescription[MEM]{Modern edition: \fullcite[xxv, 221--236]{Cererols:MEM-VC}, based on \term{CAN}}
\end{source}

\end{notesources}

This critical edition is based on two manuscript sources of this villancico, \term{CAN} and \term{Bbc}, the latter previously unknown.
It replaces the earlier edition by David Pujols, which was based on \term{CAN}, with some errors.
Most critically, \term{MEM} mistakenly omits the breve rest in this edition's measure 39, thus shifting the Alto fugue entrance one bar earlier through \measure{47}.
This edition is not only based on an additional manuscript source but also improves on the earlier Montserrat edition by indicating editorial additions and mensural coloration, correcting lyrical underlay, and including the manuscript's figured bass.

The two manuscript sources present minor variants of the same piece of music.
Both sources are probably copies based with some degree of separation on an original exemplar by Cererols at the Abbey of Montserrat, where he was monk and chapelmaster until his death in 1680.
The original was most likely have been lost when the abbey library burned in the nineteenth century.

Only \term{CAN} includes all the voice parts and the coplas, and thus it is the primary source for this edition.
There are only a few significant differences in \term{Bbc}: 
\begin{enumerate}
\item One phrase of lyrics is changed to make the piece fit a Eucharistic dedication.
The lyrics of \term{CAN} make more sense as a Christmas piece, despite the dedication to the Blessed Sacrament on the cover leaf of the group of partbooks.
\item The highest voice part has a different final phrase in the estribillo.
\item \term{Bbc} includes the dynamic markings \term{eco} and \term{falsete}.
\item \term{Bbc} differs in the usage of accidentals on C, F, and B; in most cases it is more explicit, writing out accidentals in situations where \term{musica ficta} practice would suggest them anyway.
\end{enumerate}
This edition follows \term{CAN}, but does include the dynamic markings from \term{Bbc} in parentheses.
The placement of accidentals on the staff, as opposed to advisory accidentals above it, follows \term{CAN} as well, but it should be noted that many of the advisory accidentals are written out explicitly in \term{Bbc}.


\notesection{Related Sources}

This villancico is the only complete musical setting yet found of one of the most popular villancico poems of the seventeenth century.
The poem as set by Cererols is one variant of a textual tradition extending back as early as a Royal Chapel performance in 1651.
A distinct branch of later variant versions may be traced to the work of Manuel de León Marchante from 1675.

This family of villancico poems is attested in the following imprints:

\begin{tabular}{lll}
  1651 & Madrid & \signature{E-Mn}{R/34199/27}\\
  1668 & Calatayud & \signature{GB-Lbl}{11450.dd.8~(54)}\\
  1675 & Alcalá & Reprinted in Marchante, \worktitle{Obras poéticas} (Madrid, 1733), 139\\
  1680 & Seville & \signature{E-Mn}{VE/83/10}\\
  1681 & Seville & \signature{E-Mn}{VE/79/7}\\
  1683 & Zaragoza & \signature{E-Mn}{VE/129/2}, \signature{GB-Lbl}{1073.k.22~(07)}\\
  1689 & Madrid & \signature{E-Mn}{VE/88/80}\\
\end{tabular}

Cererols's text incorporates aspects of both the early Royal Chapel tradition and the versions influenced by Marchante.
In a few passages, the text of the coplas in the Canet manuscript departs from the consensus of the other poetic imprints from this villancico tradition.
This edition preserves the text in \term{CAN}, but corrects a few phrases in which the consensus reading of the other imprints makes more sense both poetically and grammatically.


\criticalnotesheader[(Estribillo)]
\begin{criticalnotes}

1 & A. I 
  & \term{Bbc} starts coloration on C, matching other Chorus I parts.\\

13 & T. II, third note 
  & \term{Bbc} has F\octave{4}; edition follows \term{CAN} and puts A\octave{4}, matching the contour of the A. II (\measures{13--14}) and of Ti. I-2 and A. I in \measures{12--13}.\\

13--14 & A. II, lyrics 
  & \term{CAN} has \quoted{tened \MSrepeat{} parad}, but the other two voices (Ti. II and T. II) have \quoted{tened} for the final word in this phrase.\\

16--18 & Chorus I, lyrics 
  & \term{Bbc} has \quoted{tened parad parad escuchad} in Ti.~I-1 and A.~I, and \quoted{parad parad parad escuchad} in Ti.~I-2.
  Edition follows \term{CAN}.\\

19--20 & Ti. I.-2 
  & This is one of numerous places with this rising and falling stepwise motive, in which the usage of \term{ficta} accidentals is uncertain.\\

33 & T. II 
  & \term{Bbc} starts coloration in \measure{33} like the other Chorus II parts in both sources.\\

40 & B. II, third note 
  & F is sharped in \term{CAN} only; certainly a cautionary accidental (that is, it means F natural), as \term{MEM} also determines. 
  This figure, with F natural, is the subject of the eight-voice fugue, and the Ti.~I-2 has F natural on the same beat.\\

45--46 & Ti. I-2 
  & \term{Bbc} places beginning and end of slur one note earlier.\\

47--59 & All voices, lyrics 
  & \term{Bbc} substitutes the following Eucharistic lyrics in place of those in \term{CAN} and in the other poetic sources: \quoted{y desde un pan divino/ un hombre soberano}.\\

57 & A. II 
  & Slur in \term{CAN} only.\\

58 & T. I, first note 
  & \term{CAN} (the only source of the T. I voice) has B\octave{3}, which must be an error. 
  Correction to A\octave{3} follows \term{MEM}.\\

77--80 & Ti. I-1 
  & \term{Bbc} has different conclusion shown in small staff above.\\

78 & Ti. I-2, first note 
  & In \term{Bbc}, the D\octave{5} is a semibreve, resulting in an extra minim and misalignment with the other voices. Edition follows \term{CAN}, which has a minim.\\

79 & Ti. 1-2, A. II 
  & MSS do not indicate F sharp until the final note in \measure{80}, but perhaps both of these Fs should be sharp as well.\\

\end{criticalnotes}

\criticalnotesheader[(Coplas)]

\begin{criticalnotes}
81 & Ac. 
  & The accompaniment part is the same for each pair of coplas, so \term{CAN} only writes it out once.\\

81 & Lyrics 
  & \quoted{Las fugas del primer hombre formó} amended to \quoted{Las fugas que el primer hombre formó}, after consensus of poetry imprints\\

92 & Ti. I-2 
  & \term{CAN} omits fermata.\\

94 & Lyrics 
  & \quoted{Qué mucho que} amended to \quoted{Qué mucho si} after imprints.\\

96 & Ti. I-1, last two notes 
  & The first of several places where the scribe confuses rhythmic values for the two-seminimim pickup gesture. 
  The scribe notes the pickup as two \term{corcheas}, which are too fast to sing in a reasonable tempo; the values do not add up to a full measure.
  Edition follows \term{MEM} in treating these pickup notes as semiminims.\\

105 & Ac., fourth note 
  & Note obscured by a hole in the paper in \term{CAN}, on the fold.\\

110 & All voices 
  & Cf. \measure{96}: scribe writes a semiminim rest and two corcheas.
  The breath mark preserves the sense of phrasing indicated by the rest.\\

116--117 & A. I, lyrics 
  & \quoted{Las disonancias} amended to singular in accord with the other voices and the imprints.\\

131--135 & Lyrics
  & \quoted{Desatento} in Ti. I-1 amended to \quoted{desentono}, to match the Ti. I-2 and 1651 Madrid poetry imprint.
  \quoted{Tan grande} amended to \quoted{tan vano} after the poetry imprint, preserving the poetic meter of \term{romance} in \term{-a -o}.\\

137 & All voices 
  & Same problem as \measure{96} and \measure{110}.\\

142--146 & Lyrics 
  & \quoted{Lo inmenso spacio} amended to \quoted{Lo inmenso a espacio} following the poetry imprints.\\

135--137 & Lyrics 
  & \quoted{Susteniendo} in the two Tiple parts amended to \quoted{sustenido}, in accord with A. I, T. I, and poetry imprints.\\

152 & Ti. I-2 
  & \term{CAN} omits the semibreve rest (compare \measures{98, 125}).\\

161 & Ti. I-1, last note 
  & The last note of all the even-numbered coplas is explicitly written as F\sh, but here the sharp is omitted.
  Either the sharp is assumed, or it is a deliberate F\na, to create a contrast with F\sh{} upon repeat of the estribillo.\\
\end{criticalnotes}
