\noteshead{%
    Jerónimo de Carrión, 
    \worktitle{Si los sentidos queja forman del Pan Divino}%
}

\begin{notesources}
    \begin{source}
        \sourcedescription{\signature{E-SE}{28/25}, Manuscript performing parts}
        \annotation{%
            \oldabbrev{Villan}{co} Al Santissimo Sacramento. Solo. 
        Si los Sentidos quexa forman./ 
        \oldabbrev{M}{ro} \oldabbrev{D}{n} Geronimo de Carrion%
        }
        \parts{Solo, \term{Acompañamiento}}
    \end{source}
\end{notesources}

Jerónimo de Carrión (1660--1721), who succeeded Miguel de Irízar as 
chapelmaster of Segovia Cathedral after Irízar's death in 1684, sets a version 
of the same villancico poem as Irízar's \worktitle{Si los sentidos}.%
    \Autocites[70--84, 331--336]{Cashner:PhD}
    {LopezCalo:Segovia}
Carrión's text corresponds more closely to the version attributed to Vicente 
Sánchez in 1688.
He follows Sánchez's ordering of the coplas but does not include two of 
Sánchez's coplas.
Like Irízar, Carrión apparently expects the soloist to adapt the rhythm of the 
coplas to fit the poetry of the subsequent stanzas.

As in many later seventeenth-century villancicos, there is no sign indicating
that the estribillo should be repeated.
This repetition was customary with earlier villancicos, but the recurring tag 
line at the end of each copla may have been made it unnecessary to repeat the 
whole estribillo.%
    \Autocite{Torrente:Estribillo}

Compared to the other villancicos in this volume, the music in duple meter seems
to call for a slower tempo relative to the music in C3, with a feel closer to
modern \meter{4}{4}.
