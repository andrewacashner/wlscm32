\noteshead{Jerónimo de Carrión, \worktitle{Si los sentidos queja forman del Pan Divino}}

\notesource{E-SE:~28/25, Manuscript performing parts (\foreign{Solo}, \foreign{Acompañamiento})}

The lyrics correspond closely with those attributed to Vicente Sánchez in the \worktitle{Lira Poetica} (Zaragoza, 1689), 171--172.

The notated melody for the coplas does not fit every stanza equally well.
The singer was apparently expected to adapt the rhythm to fit the poetry for the subsequent stanzas.

As in many later seventeenth-century villancicos, there is no sign indicating that the estribillo should be repeated, as was customary with earlier villancicos.
The recurring tag line at the end of each copla may have been made it unnecessary to repeat the whole estribillo.

The suggested tempo relationship between the music in (C)Z meter and the music in C is only approximate. 
As with other Spanish villancicos from the later seventeenth century, the music in C seems to call for a slower tempo, with a feel closer to modern \musfig{4}{4}.