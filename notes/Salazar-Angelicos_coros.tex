\noteshead{Antonio de Salazar, \worktitle{Angélicos coros con gozo cantad}}

\begin{notesources}

\begin{source}
\sourcedescription{\signature{MEX-Mcen}{CSG.256}, Manuscript performing parts from collection of the Convento de la Santísima Trinidad, Puebla}
\annotation{A 8 de Navidad/ \oldabbrev{M}{o} Salazar}
\parts{SSA, SATB, \term{Guión}; B. II specified as \term{Órgano}}
\end{source}

\end{notesources}

Like Cáseda's \worktitle{Qué música divina}, this piece is in the Colección Jesús Sánchez Garza at CENIDIM, from the Convento de la Santísima Trinidad in Puebla.
The collection includes many works by Salazar, who may have been trained in Puebla before becoming chapelmaster at Mexico City Cathedral.%
  \autocite{Koegel:Salazar}[109--157]{Goldman:Reponsory}
The performing parts bear the names of the convent sisters who performed them. 
The name added in a later hand to the Tiple I-1 part, \quoted{Belona}, would appear to be the same person who performed the Alto in Cáseda's \worktitle{Qué música divina}.

\begin{tabular}{lll}
Tiple & I-1 & Madre Andrea, [different hand:] belona\\
Tiple & I-2 & Madre Assumpsion\\
Alto & I & Madre Sacramento\\
Tiple & II & Madre Thomasa, [on verso, different hand:] Alphonsa de \oldabbrev{S}{ta} crus [cruz]\\
Alto & II & Madre Rosa\\
Tenor & II & Ynesica Baeza\\
Bajo & II & Madre Mariana\\
\end{tabular}

The \term{Guión} part is in a different, less mature, hand than the others. 
It is written in portrait orientation rather than landscape, and includes barlines in the C meter sections.
The part may have been recopied at a later date to replace a worn original.

\criticalnotesheader

\begin{criticalnotes}
45 & Ti. II 
  & Second pitch in MS is G\octave{4}; corrected to F\octave{4}.\\
46 & Guión 
  & No fermata in MS.\\
54--57 & A. I 
  & The notated lyrics are \quoted{lo derribado busca quien viene a edificar}, which has too many syllables for the notes.\\
\end{criticalnotes}
