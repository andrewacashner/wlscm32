\noteshead{Antonio de Salazar, \worktitle{Angélicos coros con gozo cantad}}

\begin{notesources}

\item[P.]
\source{\signature{MEX-Mcen}{CSG.256}, Manuscript performing parts}
\annotation{A 8 de Navidad/ \oldabbrev{M}{o} Salazar}
\parts{SSA, SATB, \term{Guión}; B. II specified as \term{Órgano}}

\end{notesources}

This piece is in the Colección Jesús Sánchez Garza at CENIDIM, the Mexican national music research center, in Mexico City.
The collection is originally from the Convento de la Santísima Trinidad, a Conceptionist convent in Puebla.
The performing parts bear the names of the convent sisters who performed them:

\begin{tabular}{ll}
Tiple I-1 & Madre Andrea, [different hand:] belora [?]\\
Tiple I-2 & Madre Assumpsion\\
Alto I & Madre Sacramento\\
Tiple II & Madre Thomasa, [on verso, different hand:] Alphonsa de \oldabbrev{S}{ta} crus [cruz]\\
Alto II & Madre Rosa\\
Tenor II & Ynesica Baeza\\
Bajo II & Madre Mariana\\
\end{tabular}

The \term{Guión} part is in a different, less mature, hand than the others. 
It is written in portrait orientation rather than landscape, and includes barlines in the C meter sections.
The part may have been recopied at a later date to replace a worn original.

\begin{criticalnotes}
45 & Ti. II & Second pitch in MS is G\octave{4}; corrected to F\octave{4}\\
46 & Guión & No fermata in MS\\
54--57 & A. I & The notated lyrics are \quoted{lo derribado busca quien viene a edificar}, which has too many syllables for the notes
\end{criticalnotes}
