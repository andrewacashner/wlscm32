\section*{%
\emph{Voces, las de la capilla} (Puebla, 1657)}

Source: (Music) \emph{Navidad del año de 1657}, no. 4 (MEX-Pc: Leg. 3/3)\\
Poet anonymous; Music by Juan Gutiérrez de Padilla

\begin{tabular}{rll}
%
& [INTRODUCCIÓN] a 6 & \\ [1ex]
%
& 1. (i.) Voces, las de la capilla,	&	1. Voices, those of the chapel:%
	%
	\footnote{%
	As in the \q{Capilla Real}, the Royal Chapel.
	}
	%
	\\
& cuenta con lo que se canta,		&	keep count with what is sung,%
	%
	\footnote{%
	\fq{Cuenta \Dots{} nota}: Or, \q{he keeps count}. 
	The subject could either be \q{el Rey} (as translated here), or \q{la capilla} (admonishing the chorus, \q{keep count \dots{} take note of \dots{}}).
	}
	%
	\\
& que es músico el Rey, y nota		&	for the King is a musician, and notes\\
& las más leves disonancias		&	even the least dissonances\\
5 & a lo de Jesús infante			&	in the manner of Jesus, the infant [prince]%
	%
	\footnote{%
	\fq{Infante} has both meanings. 
	\fq{A lo de}: In the style of, in that which concerns (King David was a musician and founded the first \socalled{chapel} in the Hebrew temple; his descendent, the \fq{infante} Christ will be no less a musical taskmaster).
	}
	%
	\\
& y a lo de David monarca.		&	and in that of David, the monarch.\\ [2ex]
%
& RESPUESTA a 3 (ii.) & \\ [1ex]
%
& Puntos ponen a sus letras			&	They put notes to his lyrics,\\
& los siglos de sus hazañas,		&	the centuries of his heroic exploits,\\
& la clave que sobre el hombro		&	the key/clef that upon his shoulder\\
10 & para el treinta y tres se aguarda.	&	is preserved for the thirty-three.\\ [2ex]
%
& [INTRODUCCIÓN cont.] & \\ [1ex]
%
& 2. (i.) Años antes la divisa,		&	2. Years before the sign,\\
& la destreza en la esperanza,		&	dexterity in hope%
	%
	\footnote{%
	\fq{Destreza}: literally, dexterity; in Golden Age literature the word connotes heroic dexterity in combat, particularly \fq{esgrima} or sworsdmanship.
	Musically, the term suggests \q{virtuosity}.
	}%
	\\
& por sol comienza una gloria,		&	with the sun [on \term{sol}] a \q{glory} begins,\\
& por mi se canta una gracia,		&	upon me [\term{mi}] a \q{grace} is sung,\\
15 & y a medio compás la noche		&	and at the half-measure, the night%
	%
	\footnote{%
	\fq{A medio compás la noche}: That is, at midnight.
	}
	%
	\\
& remeda quiebros del alba.		&	imitates the trills of the dawn.\\ [1ex]
%
& [Respuesta rep.] (ii.) & \\ 
%
\end{tabular}

%==========%

\begin{tabular}{rll} 
%
& [ESTRIBILLO a 6] & \\ [1ex]
%
& Y a trechos las distancias		&	And from afar, the intervals %
	%
	\footnote{%
	\fq{Distancias}: Both musical intervals and astronomical distances between planetary spheres.
	}
	%
	\\
& en uno y otro coro,			&	in one choir and then the other,\\
& grave, suave, y sonoro,			&	serious, mild, and resonant,\\
20 & hombres y brutos y Dios,		&	men, animals, and God,\\
& tres a tres y dos a dos, 			& three by three and two by two,\\
& uno a uno,						& one by one,\\
& y aguardan tiempo oportuno		&	they all await the opportune time,\\
& quien antes del tiempo fue.		&	the one who was before all time.\\
25 & Por el signo a la mi re,		&	Upon the sign of \term{A (la, mi, re)},\\
& puestos los ojos en mi,			&	with eyes placed on me/\term{mi},\\
& a la voz del padre oí			&	at the voice of the Father I heard\\
& cantar por puntos de llanto.		&	singing in tones of weeping.\\
& \hphantom{uno a uno,} ¡O qué canto!		&\hphantom{one by one,}	Oh, what a song!\\
30 & tan de oír y de admirar,		&	as much to hear as to admire,\\
& tan de admirar y de oír.		&	as much to admire as to hear!\\
& Todo en el hombre es subir		&	Everything in Man is to ascend\\
& y todo en Dios es bajar.		&	and everything in God is to descend.\\ [2ex]
%
& COPLAS a 3 & \\ [1ex]
%
& 1. (i.) Daba un niño peregrino		&	1. A baby gave a wandering song%
	%
	\footnote{%
	Or \q{pilgrim song}, \q{wandering song}, or \term{tonus peregrinus}. 
	}%
	\\
%
35 & tono al hombre y subió tanto		&	to the Man, and ascended so high\\
& que en sustenidos de llanto		&	that in sustained weeping%
	%
	\footnote{%
	Musically, \q{in sharps of weeping}.
	}
	%
	\\
& dió octava arriba en un trino.		&	he went up the eighth [day] into the triune.%
	%
	\footnote{%
	Musically, \q{he went up the octave in a trill}.
	}
	%
	\\ [1ex]
%
& 2. (ii.) Hizo alto en lo divino		&	2. From on high in divinity,%
	%
	\footnote{%
	\fq{Alto} also denotes the musical voice part.
	}
	%
	\\
& y de la máxima y breve			&	of the greatest and the least,%
	%
	\footnote{%
	Musically, \q{of the \term{maxima} and the \term{breve}}.
	}
	%
	\\
40 & composición en que pruebe		&	he made a composition in which to prove%
	%
	\footnote{%
	\fq{Pruebe}: Or, test.
	}%
	\\
& de un hombre y Dios consonancias.	&	the consonances of a Man and God.\\ [1ex]
%
& [Estribillo rep.] & \\ 
%
\end{tabular}

