% Suspended, cielos, vuestro dulce canto
% Critical poetry edition
% New version 2016/06/08

\begin{poemtitle}
\poemhead{\worktitle{Suspended, cielos, vuestro dulce canto} (Montserrat, before 1680)}
Anonymous text from setting by Joan Cererols (\signature{E-CAN}{AU/0116})%
  \begin{Footnote}
    The poem as set by Cererols is one variant of a textual tradition extending back as early as a Royal Chapel performance in 1651.
    A distinct branch of later variant versions may be traced to the work of Manuel de León Marchante from 1675.
    Cererols's text incorporates aspects of both the early Royal Chapel tradition and the versions influenced by Marchante.

    This family of villancico poems is attested in the following imprints:
    \begin{tabular}{lll}
      1651 & Madrid & \signature{E-Mn}{R/34199/27}\\
      1668 & Catalayud & \signature{GB-Lbl}{11450.dd.8~(54)}\\
      1675 & Alcalá & Reprinted in Marchante, \worktitle{Obras poéticas} (Madrid, 1733), 139\\
      1680 & Seville & \signature{E-Mn}{VE/83/10}\\
      1681 & Seville & \signature{E-Mn}{VE/79/7}\\
      1683 & Zaragoza & \signature{E-Mn}{VE/129/2}, \signature{GB-Lbl}{1073.k.22~(07)}\\
      1689 & Madrid & \signature{E-Mn}{VE/88/80}\\
    \end{tabular}


    There are two variants of the Cererols musical setting: a complete version with a Christmas theme from Canet de Mar (\signature{E-CAN}{AU/0116}) and an alternate version of the estribillo only with the lyrics altered for a Eucharistic dedication (\signature{E-Bbc}{M/765/25}).
    In a few passages, indicated below, the text of the coplas in the Canet manuscript departs from the consensus of the other poetic imprints from this villancico tradition.
    \end{Footnote}
\end{poemtitle}

\begin{poemtranslation}
\begin{original}

\StanzaSection{13}[\add{Estribillo}]
Suspended, cielos, &
vuestro dulce canto; &
tened, parad, escuchad &
la más nueva consonancia &
que forman en su distancia &
lo eterno y lo temporal. &
Escuchad, &
que entonan las jerarquías &
en sonoras armonías &
contrapunto celestial. &
Y con sollozos tiernos &
\critnote{un niño soberano}
  {In place of ll.~11--12, the Eucharistic \worktitle{Bbc} version has \quotedgloss{y desde un pan divino/ un hombre soberano}{and through divine bread, a sovereign man}.} &
a los ángeles lleva el canto llano.
\SectionBreak

\StanzaSection{4}[Coplas]
1. \critnote{Las fugas que el}
  {\worktitle{CAN} has \textquote{Las fugas del}, but all the poetry imprints have \textquote{que el}.}
     primer hombre &
formó en desatentos pasos &
al compás ajusta un Niño &
de las perlas de su llanto. \&

\Stanza{4}
2. \critnote{Qué mucho si}
  {Corrected after poetry imprints; \worktitle{CAN} has \textquote{Qué mucho que}.} 
 a los despeños &
que le ocasionó un engaño, &
bella corriente de aljófar, &
grillos le previene blandos. \&

\Stanza{4}
3. Una voz que ha dado el cielo, &
de metal más soberano &
a ordenar entra sonora &
la disonancia del barro. \&

\Stanza{4}
4. Concierto tan soberano &
sólo pudo ser reparo, &
con una voz tan humilde, &
de \critnote{un desentono tan vano}
  {\worktitle{CAN}: Tiple I-1 has \quotedgloss{desatento}{inattentiveness} instead of \quotedgloss{desentono}{untunefulness}; both vocal parts have \quoted{tan grande} instead of the metrically correct \quoted{tan vano} in the poetry imprints.}. \&

\Stanza{4}
5. En las pajas \critnote{sustenido}
  {\worktitle{CAN}: Tiple I-1 and 2 have \quotedgloss{susteniendo}{sustaining/sharping}; but Altus I and Tenor I have \textquote{sustenido}, in agreement with the poetry imprints.} &
dulcemente se ha escuchado &
ligar en pajas lo eterno, &
reducir \critnote{lo inmenso a espacio}
  {All the poetry imprints have this text; the \worktitle{CAN} partbooks have \textquote{lo inmenso spacio}, most likely a contraction for the same.}. \&

\Stanza{4}
6. Divina cláusula sea &
deste eterno canto llano, &
que forma en su movimiento &
de cada punto un milagro. \&

\end{original}

\begin{translation}
\StanzaSection{13}
Suspend, O heavens, &
your sweet chant. &
Hold, stop, and listen &
to the newest consonance &
that the eternal and the temporal &
are forming in their distance. &
Listen, &
for the hierarchies are entoning &
in resounding harmonies &
celestial counterpoint. &
And with tender sobs, &
a sovereign baby boy &
bears the plainsong to the angels. \&

\Stanza{4}
1. The flight/fugue that the first man &
made in heedless paces &
is set aright by a baby boy to the measure &
of the pearls of his crying. \&

\Stanza{4}
2. What wonder, if from the falls &
that a deceit caused him, &
the lovely mother-of-pearl stream &
\critnote{gently restrains him with shackles.}{Translation uncertain.} \&

\Stanza{4}
3. A voice that heaven has given, &
of the most sovereign timbre, &
to bring order, enters resounding &
into the dissonance of the clay. \&

\Stanza{4}
4. So sovereign a concord/concerto & 
could only be a restoration, &
with so humble a voice, &
of so vain a discord. \&

\Stanza{4}
5. Upon the straw \critnote{sustained}
  {Musically, \quoted{sharp}.} &
sweetly he has been heard &
\critnote{binding}{Musically, \quoted{tying} or forming a ligature.} 
  in straw the eternal, &
reducing the immense \critnote{to this space}
  {Musically, \quoted{slowly}.}. \&

\Stanza{4}
6. Let there be a divine cadence &
of this eternal plainsong, &
which forms in its movement &
a miracle from each \critnote{note}{Literally, \quoted{point}.}. \&

\end{translation}
\end{poemtranslation}

