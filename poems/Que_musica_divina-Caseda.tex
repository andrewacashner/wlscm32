% Qué música divina (Cáseda)
% Critical poetry edition
% 2016-06-01  New version begun

\begin{poemtitle}
\poemhead{\worktitle{Qué música divina} (Zaragoza and Puebla, \circa{1700})}
\poemsource{Anonymous text from setting by José de Cáseda (\signature{MEX-Mcen}{CSG.256}); coplas attrib. Vicente Sánchez, \worktitle{Lyra poética} (Zaragoza, 1688), 191}
\end{poemtitle}

\begin{poemtranslation}
\begin{original}
\StanzaSection{10}[\add{Estribillo} a 4]
Qué música divina, &
acorde y soberana &
afrenta de las aves &
con tiernas, armoniosas consonancias, &
en quiebros \critnote{suaves, sonoros y graves}
  {Cf. \worktitle{Voces, las de la capilla} (Padilla, in this edition), \quoted{grave, suave y sonoro}.}%
  , &
acordes accentos &
ofrece a los vientos &
y en cláusulas varias &
sentidos eleva, &
\critnote{potencias}
  {Powers or faculties of the \gloss{anima sensitiva}{sensitive soul}, such as the intellective, cogitative, imaginative factulties, and the memory.
  See \fullcite[\X]{LuisdeGranada:Simbolo}.}%
   desmaya.
\SectionBreak

\StanzaSection{4}[Coplas a 4 y Solo]
1. Suenen las dulces cuerdas &
de esa \critnote{divina cítara y humana}
  {The central conceit of the coplas connect Christ, in his Passion, to a string instrument. 
  The Spanish \term{vihuela} is linked to an older symbolic tradition of the \term{cítara} and the \term{lira}.
  The pairs of seven- and eleven-syllable lines evoke the Spanish poetic form known as the \term{lira}.}% \X check
  , &
que \critnote{aún sol}
  {Sánchez edition has \gloss{un son}{a sound}.}
    que es de los cielos, &
forma unida la alta con la baja. \&

\Stanza{4}
2. De la fe es instrumento &
y al oído su música regala &
donde hay por gran misterio &
en cada punto entera consonancia. \&

\Stanza{4}
3. De el lazo a este instrumento &
sirve la unión que sus extremos ata: &
tres clavos son clavijas &
y puente de madera fue una tabla. \&

\Stanza{4}
4. Misteriosa vihuela, &
al \critnote{herirle}
  {Sánchez: \mentioned{herirla}.} 
    sus cuerdas una lanza, &
su sagrada armonía &
\critnote{se vió allí}
  {Sánchez omits \mentioned{allí}, preserving the pattern of eleven-syllable lines.}\
  \critnote{de siete órdenes formada}
    {The seven-course \term{vihuela} as metaphor for the seven sacraments, signified by the blood and water coming from Christ's pierced side (John 19:34).}%
    . \&

\Stanza{4}
5. No son a los sentidos &
lo que suenan sus voces soberanas &
porque de este instrumento &
\critnote{cuantas}
  {Sánchez: \mentioned{quantos}.}
     ellos percibían serían \critnote{falsas}
      {The term may refer to notes that are out of tune, out of temperament, incorrect, or that use \term{musica ficta} accidentals.}%
        . \&

\Stanza{4}
6. Su primor misterioso, &
que a los cielos eleva al que \critnote{lo}{Sánchez: \mentioned{le}.} alcanza &
no lo come el sentido &
porque es pasto su música del alma. \&
\end{original}

%**** TRANSLATION ****
\begin{translation}
\StanzaSection{10}
What divine music, &
tuneful and sovereign, &
rivals that of the birds &
with tender, harmonious consonances, &
in trills mild, sonorous and solemn, &
it offers tuneful accents &
to the winds, &
and in varying cadences &
elevates the senses, &
confounds the \add{mind's} powers. \&

\StanzaSection{4}
1. Let the sweet strings sound &
of that divine and human \term{cithara}, &
who, the very sun/\term{sol} who is in the heavens, &
forms the high \add{string} and the low in unity. \&

\Stanza{4}
2. Of faith he is the instrument, &
and his music regales the ear &
when, by a great mystery, there is &
in every point a perfect consonance. \&

\Stanza{4}
3. Serving as the string on this instrument &
is the union that ties together his extremes: &
there nails are the pegs &
and a crossing of wood was a soundboard. \&

\Stanza{4}
4. Mysterious \term{vihuela}, &
when a lance wounded/plucked your strings, &
your sacred harmony & 
was seen there, formed of seven orders. \&

\Stanza{4}
5. They are not for the senses, &
that which your sovereign notes sound, &
for, of this instrument &
as many notes as they perceived will be false. \&

\Stanza{4}
6. Your mysterious virtuosity, which &
elevates to the heavens the one who achieves it: &
sensation does not eat it, &
for your music is fodder for the soul. \&

\end{translation}
\end{poemtranslation}
