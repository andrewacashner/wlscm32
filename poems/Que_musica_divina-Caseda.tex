% Qué música divina (Cáseda)
% Critical poetry edition
% New version 2016/06/01

\begin{poemtitle}
\poemhead{\worktitle{Qué música divina} (Zaragoza and Puebla, \circa{1700})}
Anonymous text from setting by José de Cáseda (\signature{MEX-Mcen}{CSG.256}); coplas attrib. Vicente Sánchez % more
\end{poemtitle}

\begin{poemtranslation}
\begin{original}
\StanzaSection{10}[\add{Estribillo} a 4]
Qué música divina, &
acorde y soberana &
afrenta de las aves &
con tiernas armoniosas consonancias, &
en quiebros suaves, sonoros y graves, &
acordes accentos &
ofrece a los vientos &
y en cláusulas varias &
sentidos eleva, &
potencias desmaya.
\SectionBreak

\StanzaSection{4}[Coplas a 4 y Solo]
1. Suenen las dulces cuerdas &
de esa divina cítara y humana, &
que aun sol que es de los cielos, & %aún?
forma unida la alta con la baja. \&

\Stanza{4}
2. De la fe es instrumento &
y al oído su música regala &
donde hay por gran misterio &
en cada punto entera consonancia. \&

\Stanza{4}
3. De el lazo a este instrumento &
sirve la unión que sus extremos ata: &
tres clavos son clavijas &
y puente de madera fue una tabla. \&

\Stanza{4}
4. Misteriosa vihuela, &
al herirle sus cuerdas una lanza, &
su sagrada armonía se vió allí & 
de siete órdenes formada. \&

\Stanza{4}
5. No son a los sentidos &
lo que suenan sus voces soberanas &
porque de este instrumento &
cuantas ellos percibían serían falsas. \&

\Stanza{4}
6. Su primor misterioso, &
que a los cielos eleva al que lo alcanza &
no lo come el sentido porque es pasto &
su música del alma. \&
\end{original}

%**** TRANSLATION ****
\begin{translation}
\StanzaSection{10}
What divine music, &
tuneful and sovereign, &
rivals that of the birds &
with tender, harmonious consonances, &
in trills mild, sonorous and solemn, &
it offers tuneful accents &
to the winds, &
and in varying cadences &
elevates the senses, &
confounds the \add{mind's} powers. \&

\StanzaSection{4}
1. Let the sweet strings sound &
of that divine and human \term{cithara}, &
who, the very sun/\term{sol} who is in the heavens, &
forms the high \add{string} and the low in unity. \&

\Stanza{4}
2. Of faith he is the instrument, &
and his music regales the ear &
when, by a great mystery, there is &
in every point a perfect consonance. \&

\Stanza{4}
3. Serving as the string on this instrument &
is the union that ties together his extremes: &
there nails are the pegs &
and a crossing of wood was a soundboard. \&

\Stanza{4}
4. Mysterious \term{vihuela}, &
when a lance wounded/plucked your strings, &
your sacred harmony was seen there, &
formed of seven orders. \&

\Stanza{4}
5. They are not for the senses, &
that which your sovereign notes sound, &
for, of this instrument &
as many notes as they perceived will be false. \&

\Stanza{4}
6. Your mysterious virtuosity, which &
elevates to the heavens the one who achieves it: &
sensation does not eat it, for your music &
is fodder for the soul. \&

\end{translation}
\end{poemtranslation}
