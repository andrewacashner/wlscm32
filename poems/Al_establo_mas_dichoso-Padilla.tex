% Al establo más dichoso
% Critical poetry edition
% New version 2016/06/17

\begin{poemtitle}
\poemhead{\worktitle{Al establo más dichoso} (Puebla, 1652)}
\poemsource{Anonymous, from musical setting by Juan Gutiérrez de Padilla, \worktitle{Navidad del año de 1652} (\signature{MEX-Pc}{Leg.~1/3})}
\end{poemtitle}

\begin{poemtranslation}
\begin{original}

\StanzaSection{4}[\add{Prologue} a 4]
Al establo más dichoso, &
donde triunfa la victoria, &
principio a siglos de gracia, &
la noche más venturosa, \&

\Stanza{4}
\critnote{Buena noche y la más buena}
  {\mentioned{La Nochebuena} is Christmas Eve.}, &
pues a pesar de las sombras &
en su mitad amanece &
quién con tanta luz entolda. \&

\Stanza{4}
Un zagal de aquel contorno, &
en su templada zampoña, & 
tocando el Nuevo Troyano, & 
cantó en la pajiza choza:
\SectionBreak

\StanzaSection{4}[\add{Nuevo Troyano} Solo y a 4]
En Belén cantando están, &
todo es gloria, todo es cielo, &
y en un portalico pobre &
se ha estrechado él que es inmenso. \&

\Stanza{4}
Fuego derrite la nieve, &
y entre tanta nieve el fuego &
a cada llama bosteza, & 
lo acendrado deste estremo. \&

\Stanza{4}
Míranse por todos lados, &
en cada paja un lucero, &
una antorcha a cada viso &
y un Dios grande aunque pequeño.
\SectionBreak

%*** ARRIERO *****
\StanzaSection{4}[\add{Prologue a 4}]
Después Bartholo, él de marras, &
arriero \critnote{de cala y gorra}
  {The manuscripts have \foreign{cala} clearly, but the meaning is unclear. Possibly a mistake for \gloss{de capa y gorra}{in plain, simple clothes} (DRAE).} &
que fue espadachín de antaño, &
y hoy mercader de \critnote{panochas}
  {In Mexican Spanish, slabs of hard brown sugar or candies made from them (DRAE); etymological source for English \mentioned{penuche} (OED).}, \&

\Stanza{4}
En busca de una mulilla &
que se le fue por \critnote{tramoya}
  {Scheme, or a piece of stage machinery.}, &
a darse una buena noche & 
en las pajas misteriosas. \&

\Stanza{4}
Al portal con los pastores & 
se entró arrojando bramonas &
y a quién ocupa el pesebre, &
dice como que se entona:
\SectionBreak

%*******
\StanzaSection{4}[El Arriero: Responsión a Dúo]
\critnote{}[Responsión a Dúo]{Scored for solo Tenor with \term{bajón} accompaniment.}%
  \critnote{Señor niño, voto a San---}
  {Bartholo refers to the Christ-child as \quoted{Sir} or \quoted{Lord}, and
  addresses him with formal \foreign{Usted} forms, but in the same breath begins
  to curse, using a figure of speech that stops short of actual blasphemy.} &
ya lo dije, y esto sobra &
para que entienda que vengo &
puesto a \critnote{lo de aquí fue Troya}
  {Idiom for a disastrous mess; with double meaning referring to Bartholo's mule as a Trojan horse, bringing him unawares to the stable, and referring back to the \quoted{Nuevo Troyano}.} \&

\Stanza{4}
No se me asuste le digo &
ni de inocente se ponga, &
cuando me dicen que sabe &
lo que su padre no ignora. \&

\Stanza{4}
Es bueno que de mis mulas, &
la más lucia y la más gorda &
me la traiga a este pesebre &
\critnote{sin decir esta es mi boca}
  {Idiom for keeping silent.}, \&

\Stanza{4}
Y yo sin haber vendido &
las cargas de mis melcochas, &
\critnote{ande en flores y con flores}
  {\foreign{Andar en flores} is an idiom for refusing to argue; \foreign{con flores} could refer to a type of sweets.} &
pregonándola a mi costa. \&

\Stanza{4}
Si \critnote{arrobar}
  {Play on \gloss{a robar}{to steal}.}
    viene a los hombres, &
paréceme cosa impropia &
dar principio con mi mula, &
si no ha de ocupar carroza. \&

\Stanza{4}
Pero ya he considerado, &
si mi decir no le enoja, &
que por la escarcha pretende &
el aliento de su boca. \& 

\Stanza{4}
Y por vida de Bartholo, &
que en aquestas y en esotras, &
cuando por esto la quiera, &
que aquí se las traiga todas. \&

\Stanza{4}
Abra esa boca de perlas &
con que tanto me enamora, &
y pida que \critnote{estos serranos}
  {The next group to perform.} &
no pretenden otra cosa. \&

\Stanza{4}
Un baile quieren hacerle, &
que \critnote{\term{papalotillo}}
  {Diminutive of \gloss{papalote}{kite or paper toy}, derived from Nahuatl \gloss{papalotl}{butterfly} (RAE).}
   nombran &
y como cantemos todos, &
más que rueden las panochas. 
\SectionBreak

%*** PAPALOTILLO *****
\StanzaSection{2}[Papalotillo: Solo]
Ven y verás un donoso chiquito. &
Míralo bien, que en sus ojos me miro.
\SectionBreak

\StanzaSection{2}[Responsión a 4]
Ven y verás un donoso chiquito. &
Míralo bien, que en sus ojos me miro.
\SectionBreak

\StanzaSection{2}[Coplas]
\critnote{}[Coplas]{Alternating stanzas are scored as a dialogue between two voices.}%
Míralo bien, como llora y suspira, &
siendo del padre la misma alegría. \&

\Stanza{2}
Míralo bien entre pobres alajas, &
grano fecundo escondido entre pajas. \&

\Stanza{2}
Míralo bien que aunque agora se estrecha, &
nos ha de dar una fértil cosecha. \&

\Stanza{2}
Míralo bien con terneza y cuidado, &
que ha de ser pasto y pastor desvelado. \&

\Stanza{2}
Míralo bien, corderito amoroso, &
que ha de huir de las garras del lobo. \&

\Stanza{2}
Míralo bien, pequeñito pastor, &
pues cuando grande será labrador.
\SectionBreak

\StanzaSection{2}[Responsión a 4]
Ven y verás un donoso chiquito. &
Míralo bien, que en sus ojos me miro.
\SectionBreak


%*** NEGRILLA *****
\StanzaSection{4}[\add{Prologue a 4}]
\critnote{El Angola Minguelillo}
  {Diminutive of Miguel, identified as an African of the Angolan \soCalled{nation} or brand, likely a slave.} &
acaudillando su tropa, &
no quiere ser el postrero &
en la fiesta en que se goza. \&

\Stanza{4}
Dejando \critnote{el tumba catumba} 
  {Apparently a nonsense word, possibly imitating African drumming and the sounds of Angolan languages like Kikongo.
  Cf. the refrain of Padilla's 1651 \term{ensaladilla}, \quoted{Tumbu cutu, cutu, cutu}.} &
y gruñendo a lo de Angola, &
desenvainó con la voz, &
\critnote{de su tizón la tizona.}[desenvainó\dots\ de su tizón la tizona]
  {Mocking the voice and singing of this African character. 
  \foreign{Tizona} means sword (after the Cid's weapon), playing on the idea of Minguelillo leading a quasi-military \soCalled{troop}; \foreign{tizón} means a charred log or piece of coal, referring to Minguelillo's dark-skinned, muscular neck, and to the perceived dark, gravelly sound of his voice.}
\SectionBreak

\StanzaSection{4}[Negrilla: \add{Introducción} Dúo y a 6]
\critnote{}[Negrilla]{\term{La negrilla} is the name of a subgenre of villancico
representing black characters.}
\critnote{}[Dúo]{This section scored as dialogue between two voices, each with separate \term{bajón} accompaniment.}%
Diga plimo donde sa? &
la niño, de nacimenta &
pluque samo su palenta &
y la venimo a \critnote{busca}[Diga plimo\dots\ busca]
  {Pseudo-African dialect Spanish, original orthography and punctuation. 
   Possible equivalent in proper Spanish: \quoted{Diga primo, ¿dónde está/ el niño de nacimiento?/ porque sabemos sus parientes/ y lo venimos a buscar}.}. \&

\Stanza{5}
\critnote{Ayta}
  {Also written in MS as \foreign{aytá} and \foreign{a\'yta}; probably for \gloss{ahí está}{there he is}, answering the question \foreign{donde sa}.}%
    , ayta, &
cundiro entle pajita &
su ojo como treyita &
y uno buey y uno mulita &
con su baho, \critnote{cayenta.}[Ayta\dots\ cayenta]
  {\quoted{Ahí está,/ candela entre pajitas,/ su ojo como estrellita,/ y un buey y una mulilla/ con su bajo callentar}.} \&

\Stanza{2}
Turu turu yega, &
ayta \critnote{ayta}[Turu turu yega,/ ayta, ayta]
  {Possibly, \quoted{Todos, todos llegan,/ ahí está}; or pseudo-African nonsense.}. \&

\Stanza{4}
Caya, caya, mila no panta &
que duelme la siguetito. &
Sesu, Sesu, que bonito, &
\critnote{sucucha}
  {Accentuated in the setting as \term{sucuchá}.}%
    , que cantamo lo \critnote{angelito:}[Caya\dots\ lo angelito]
  {Possible equivalent: \quoted{Calla, calla, mira, no le espanta,/ que duerme el chiquitito,/ Jesús, Jesús, qué bonito,/ esuchar, que cantamos a lo del angelito} or \quoted{a lo angélico}.}
\SectionBreak

\StanzaSection{1}[A 3]
\critnote{Gloria en las alturas y en la tierra paz.}[Gloria\dots]  
  {Two new voices join here in contrasting musical meter, with music based on the plainchant \worktitle{Gloria in excelsis}.}

\SectionBreak

\StanzaSection{4}[\add{Estribillo a 6}]
\critnote{Valamindioso que lindo canta}
  {Dubious possible equivalent: \quoted{Para mi Dios, O qúe lindo cantar}.}, &
ayta, ayta, &
sucucha, sucucha, &
ayta, ayta, ayta.
\SectionBreak

\StanzaSection{5}[Coplas a 6]
Caya, caya, chiquito, \emph{ayta}. &
Que tlaemo plecente, \emph{ayta}. &
Mantiya pañalito, \emph{ayta}. &
Y uno \critnote{papagayito}
  {\foreign{Flor de Nochebuena}, poinsettia, native to Central America.}, \emph{ayta}. &
Que savemo \critnote{habra}[Caya\dots\ savemo habra]
  {\quoted{Calla, calla, chiquito,/ que traemos un presente,/ una mantilla, un pañalito,/ y un papagayito,/ que sabemos habrá}.}.
\SectionBreak[\add{Repeat negrilla estribillo}]

\Stanza{6}
Mi siñol Manuele, \emph{ayta}. &
ese papa he sablosa, \emph{ayta}.  &
pluque sa linda cosa, \emph{ayta}.  &
mantequiya con \critnote{mele}[Mi siñol\dots\ mele]
  {\quoted{Mi señor Manuel/, esa papa, qué sabrosa,/ porque está linda cosa,/ mantequilla con mel}.}%
    , \emph{ayta}. &
ay, Sesu, le, le, le, le, \emph{ayta}. &
\critnote{ro}[le, le\dots\ ro, ro]
  {Common nonsense lullaby words.}%
    , ro, ro, ro, caya.
\SectionBreak[\add{Repeat negrilla estribillo}]

\end{original}

%**********************************************
\begin{translation}
\StanzaSection{4}
At the most blessed stable, &
where victory triumphs, &
beginning of centuries of grace, &
the most fortunate night, \&

\Stanza{4}
A merry eve, the best, &
since despite the shadows &
at its midpoint dawns &
one who with so much light overwhelms it. \&

\Stanza{4}
A shepherd-boy from that scene, &
on his tempered panpipes, &
playing the \quoted{New Trojan}, &
sang in the straw-filled hutch: \&

\StanzaSection{4}
In Bethlehem they are singing, &
all is glory, all is heaven, &
and in a poor little stable &
he who is immense has confined himself. \&

\Stanza{4}
Fire melts the snow, &
and among so much snow, the fire &
yawns to each flame, &
that which is purified from this extreme. \&

\Stanza{4}
Look around on all sides: &
in each bit of straw, a blazing star, &
a torch at each glance &
and a God who is great, though little. \&

%*** ARRIERO
\StanzaSection{4}[\add{The New Trojan}]
Next Bartholo---you know the one--- &
a mule skinner in plain clothes, &
who was a swordsman \critnote{in days gone by}
  {Or perhaps in a previous villancico?}, &
and now, a vendor of candies, \&

\Stanza{4}
In search of a little mule &
who went off from him in a scheme &
to give himself a merry eve &
in the mysterious straw. \&

\Stanza{4}
Into the stable with the shepherds &
he entered, braying up a storm, &
and to the one who occupies the manger, &
he says as it is intoned: \&

%*****
\StanzaSection{4}[\add{The Mule Driver}]
Sir Baby, I swear to Saint Somebody, &
well now I've said it, and it's more than enough &
for you to understand that I come &
on account of all this \quoted{Troy}/mess. \&

\Stanza{4}
Don't be afraid of me, I tell you, Sir, &
or play innocent &
when they tell me that you know &
whatever is not unknown to your father. \&

\Stanza{4}
It's just great that of all my mules, &
\critnote{the most brilliant and the most fat}
  {Or, \quoted{the nicest and fattest one}.} &
should bring me to this manger &
\critnote{without telling me anything}
  {Or, referring to Bartholo rather than to the mule, \quoted{without anything to say}.}%
    , \&

\Stanza{4}
And that I, without having sold &
all my stock of sweets, &
should have to play nice, &
hawking it at my own expense. \&

\Stanza{4}
If you come to enrapture men &
it seems to me an improper thing &
to have my mule go first, &
if she's not even going to carry the wagon. \&

\Stanza{4}
But now I've been thinking, &
if my saying so doesn't make you mad, &
that on account of the frost you ought to have &
the breath from her mouth. \&

\Stanza{4}
And upon the life of Bartholo, &
whether in these things or any others, &
if you should want anything, &
they should all be brought here for you. \&

\Stanza{4}
Open that mouth of pearls, &
with which I am so enamored, &
and request that these mountain folk &
don't try anything else. \&

\Stanza{4}
They want to do a dance for you, &
that they call \term{papalotillo}, &
and so, let us all sing, &
and let the candies go round all the more. \&

%*** PAPALOTILLO *****

\StanzaSection{2}[\add{The Butterfly}]
Come and you will see a genteel little boy. &
Look on him well, for in his eyes I see myself. \&

\StanzaSection{2}
Come and you will see a genteel little boy. &
Look on him well, for in his eyes I see myself. \&

\StanzaSection{2}
Look on him well, how he cries and sighs, &
which at the same time is his father's joy. \&

\Stanza{2}
Look on him well: jewels among the poor, &
a fertile seed hidden in the straw. \&

\Stanza{2}
Look on him well, though now \critnote{bundled up}
  {Comparing the swaddled infant to a seed; in a theological sense, \quoted{confines himself}.}%
    , &
he will give us a fertile harvest. \&

\Stanza{2}
Look on him well, with tenderness and care, &
for he will be revealed as \critnote{pasture and pastor}
  {\foreign{Pasto} is livestock feed (and anagogically, the Eucharist); \foreign{pastor} is both shepherd and religious minister.}. \&

\Stanza{2}
Look on him well, a little lamb full of love, &
for he will flee from the claws of the wolf. \&

\Stanza{2}
Look on him well, the tiny shepherd, &
for when he is big he will be a laborer. \&

\StanzaSection{2}
Come and you will see a genteel little boy. &
Look on him well, for in his eyes I see myself. \&


%*** NEGRILLA *****
\StanzaSection{4}
Little Miguel the Angolan, &
marshalling his troop, &
does not wish to be the last one &
at the party that is being enjoyed. \&

\Stanza{4}
Leaving the \quoted{tumba catumba} &
and grunting like the Angolans do &
he unsheathed his voice, &
like pulling a sword from his charred log. \&

\StanzaSection{4}[\add{Little Black Song}]
Tell me, cousin, where is &
the baby who was born? &
for we know his relatives &
and we come to seek him. \&

\Stanza{5}
There he his, &
a candle among the straw, &
his eye like a little star, &
and an ox and a little mule &
with its belly to warm him. \&

\Stanza{2}
Come on, everybody, &
there he is. \&

\Stanza{4}
Hush, hush, look, don't startle him, &
for the tiny boy is sleeping. &
Jesu, Jesu, how lovely, &
listen, for we are singing like the angels: \&

\StanzaSection{1}
Glory in the heights and on earth, peace. \&

\StanzaSection{4}
For my God, O what a lovely song, &
there he is, &
listen, &
there he is. \&

\StanzaSection{5}
Hush, hush, baby boy, &
for we are bringing you a present: &
a little blanket, a diaper, &
and a little poinsettia, &
for we know how things go [with babies]. \&

\StanzaSection{6}
My Lord Emmanuel, &
this potato, how tasty, &
since this is a nice thing, &
butter with honey, &
ay, Jesu, lulla, lulla, &
ro, ro, ro, ro, hush. \&

\end{translation}

\end{poemtranslation}
