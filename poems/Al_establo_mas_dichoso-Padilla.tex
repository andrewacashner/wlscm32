% Al establo más dichoso
% Critical poetry edition
% New version 2016/06/17

\begin{poemtitle}
\poemhead{\worktitle{Al establo más dichoso} (Puebla, 1652)}
Anonymous, from musical setting by Juan Gutiérrez de Padilla, \worktitle{Navidad del año de 1652} (\signature{MEX-Pc}{Leg.~1/3})
\end{poemtitle}

\begin{poemtranslation}
\begin{original}

\StanzaSection{4}[\add{Pt. I: El Nuevo Troyano, Choral Prologue}]
1. Al establo más dichoso, &
donde triunfa la victoria, &
principio a siglos de gracia, &
la noche más venturosa, \&

\Stanza{4}
2. \critnote{Buena noche y la más buena}
  {A play on \mentioned{la Nochebuena}, the Spanish idiom for Christmas Eve.}, &
pues a pesar de las sombras &
en su mitad amanece &
quién con tanta luz entolda. \&

\Stanza{4}
3. Un zagal de aquel contorno, &
en su templada zampoña, & 
tocando el nuevo troyano, & 
cantó en la pajiza choza:
\SectionBreak

\StanzaSection{4}[\add{Song/Dance} Solo y a 4]
1. En Belén cantando están, &
todo es gloria, todo es cielo, &
y en un portalico pobre &
se ha estrechado él que es inmenso. \&

\Stanza{4}
2. Fuego derrite la nieve, &
y entre tanta nieve el fuego &
a cada llama bosteza, & 
lo acendrado deste estremo. \&

\Stanza{4}
3. Míranse por todos lados, &
en cada paja un lucero, &
una antorcha a cada viso &
y un Dios grande aunque pequeño. \&
\SectionBreak

%*** ARRIERO
\StanzaSection{4}[\add{Part II:} El Arriero \add{Choral prologue}]
1. Después Bartholo, él de marras, &
arriero de cala y gorra, &
que fue espadachín de antaño, &
y hoy mercader de \critnote{panochas}
  {In Mexico today, \foreign{panocha} refers to a range of candies; cf. penuche, a brown sugar fudge of the southern United States, and Italian \foreign{panucci}.
  Probably not to be confused with the modern peninsular Spanish meaning (corn cobs), and certainly not with the term's vulgar slang usage.}. \&

\Stanza{4}
2. En busca de una mulilla &
que se le fue por tramoya, &
a darse \critnote{una buena noche}
  {Another play on \foreign{Nochebuena}.}, & 
en las pajas misteriosas. \&

\Stanza{4}
3. Al portal con los pastores & 
\critnote{se entró arrojando bramonas}
  {It is ambiguous (probably on purpose) whether it is the mule or his driver that is bellowing and braying.} &
y a quién ocupa el pesebre, &
dice como que se entona:
\SectionBreak

\StanzaSection{4}[\add{Song/Dance} Responsión Duo \add{Solo with acc.}]
1. \critnote{Señor niño, voto a San}
  {Bartholo refers to the Christ-child as \quoted{Sir} or \quoted{Lord}, and addresses him with formal \foreign{Usted} forms, but in the same breath begins to curse.}--- &
ya lo dije, y esto sobra &
para que entienda que vengo &
puesto a lo de aquí fue \critnote{Troya}
  {\foreign{Algo fue Troya} is an idiom for \quoted{it was a hell of a fuss!} (as the \worktitle{Oxford Spanish Dictionary} puts it); here it has a double meaning, since Bartholo's mule, in a comic inversion of the Trojan horse, brings Bartholo into the manger unawares.}. \&

\Stanza{4}
2. No se me asuste le digo &
ni de inocente se ponga, &
cuando me dicen que sabe &
lo que su padre no ignora. \&

\Stanza{4}
3. Es bueno que de mis mulas, &
la más sucia y la más gorda &
me la traiga a este pesebre &
sin decir esta es mi boca. \&

\Stanza{4}
4. Y yo sin haber vendido &
las cargas de mis melcochas, &
\critnote{ande en flores y con flores}
  {\foreign{Andar en flores} is an idiom for refusing to get into an argument; \foreign{con flores} plays on the first idiom, suggesting perhaps that Bartholo also sells flowers from his cart, which he now reluctantly offers the Christ-child for free.} &
pregonándola a mi costa. \&

\Stanza{4}
5. Si \critnote{arrobar}
  {Clearly spelled as one word in the manuscript (\quoted{aarouar}), but possibly also a play on \gloss{a robar}{to steal}; since Bartholo is indignant at the business he has lost.}
    viene a los hombres, &
paréceme cosa impropia &
dar principio con mi mula, &
si no ha de ocupar carroza. \&

\Stanza{4}
6. Pero ya he considerado, &
si mi decir no le enoja, &
que por la escarcha pretende &
\critnote{el aliento de su boca}
  {That is, Bartholo realizes Christ should have the straw (which the mule is eating) for warmth. 
  This is the turning point for him, after which he begins to offer his own services to the child, and even in the end, sharing his precious wares with the other visitors to the stable.} \&

\Stanza{4}
7. Y por vida de Bartholo, &
que en aquestas y en esotras, &
cuando por esto la quiera, &
que aquí se las traiga todas. \&

\Stanza{4}
8. Abre esa boca de perlas &
con que tanto me enamora, &
y pida que \critnote{estos serranos}
  {Bartholo introduces the next group that is coming to perform before the child.} &
no pretenden otra cosa. \&

\Stanza{4}
9. Un baile quieren hacerle, &
que \critnote{\term{papalotillo}}
  {Diminutive of \gloss{papalote}{kite or paper toy}, derived from Nahuatl \gloss{papalotl}{butterfly} (RAE).
   The name of the dance may indicate that these \quoted{mountain folk} are Indians.}
   nombran &
y como cantemos todos, &
más que rueden las panochas. 
\SectionBreak

\end{original}

%**********************************************
\begin{translation}
\StanzaSection{4}
1. At the most blessed stable, &
where victory triumphs, &
beginning/prince of centuries of grace, &
the most fortunate night, \&

\Stanza{4}
2. A merry eve, the best, &
since despite the shadows &
at its midpoint dawns &
one who with so much light overwhelms it. \&

\Stanza{4}
3. A shepherd-boy from that scene, &
on his tempered panpipes, &
playing the \quoted{New Trojan}, &
sang in the straw-filled hutch:
\SectionBreak

\StanzaSection{4}
1. In Bethlehem they are singing, &
all is glory, all is heaven, &
and in a poor little stable &
he who is immense has confined himself. \&

\StanzaSection{4}
2. Fire melts the snow, &
and among so much snow, the fire &
yawns with each flame, &
having refined it from this extreme. \&

\StanzaSection{4}
3. They are seen on all sides, &
in each bit of straw there is a lantern, &
a torch at each spark &
and a God who is great, though little.
\SectionBreak

%*** ARRIERO
\StanzaSection{4}
1. Next Bartholo---you know the one--- &
a mule skinner \critnote{of the finest pedigree}{Or, \quoted{of real quality and class}.} &
who was a swordsman \critnote{in days gone by}{Or perhaps in a previous villancico?}, &
and now, a vendor of candies. \&

\Stanza{4}
2. In search of a little mule &
who went off from him in a scheme &
to give himself a merry eve &
in the mysterious straw. \&

\Stanza{4}
3. Into the stable with the shepherds &
he entered, braying up a storm, &
and to the one who occupies the manger, &
he says as it is intoned:
\SectionBreak

\StanzaSection{4}
1. Mr. Baby, I swear to Saint--- &
well now I've said it, and it's more than enough &
for you to understand that I come &
on account of all this \quoted{Troy}/mess. \&

\Stanza{4}
2. Don't be afraid of me, I tell you, Sir, &
or play innocent &
when they tell me that you know &
whatever is not unknown to your father. \&

\Stanza{4}
3. It's just about right that of all my mules, &
the dirtiest and the fattest &
should bring me to this manger &
without knowing his mouth from a hole in the wall, \&

\Stanza{4}
4. And that I, without having sold &
all my stock of candies, &
should have to \critnote{be so polite}{Or, \quoted{give up the struggle}.}, carrying all these flowers, &
hawking it at my own expense. \&

\Stanza{4}
5. If you come to enrapture men &
it seems to me an improper thing &
to have my mule go first, &
if she's not even going to carry the wagon. \&

\Stanza{4}
6. But now I've been thinking, &
if my saying so doesn't make you mad, &
that on account of the frost you ought to have &
the feed from her mouth. \&

\Stanza{4}
7. And upon the life of Bartholo, &
whether in these things or any others, &
if you should want anything, &
they should all be brought here for you. \&

\Stanza{4}
8. Open that mouth of pearls, &
with which I am so enamored, &
and request that these mountain folk &
don't try anything else. \&

\Stanza{4}
9. They want to do a dance for you, &
that they call \term{papalotillo}, &
and so, let us all sing, &
and let the candies go round all the more.
\SectionBreak

\end{translation}

\end{poemtranslation}

