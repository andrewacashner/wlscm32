\edsection{Performance Suggestions}

\edsubsection{Spanish Pronunciation}
The editor believes that Spanish-speaking ensembles should feel free to
pronounce the Spanish according to their own accent.
Other ensembles are encouraged to work with local native speakers and experts
whenever possible to shape their pronunciation and understanding, so that they
can perform these pieces in a way that Spanish-speaking audience members will
understand and recognize as a part of their own cultural heritage.

\edsubsection{Instrumentation and Voicing}
These villancicos are scored for an ensemble of voices with instrumental bass 
or continuo groups.
Vocal ensembles varied in size, from one-to-a-part groups to much larger 
polychoral forces.
Most of the pieces also feature prominent solo parts, particularly in the 
\term{coplas}.

The lowest voice parts in these pieces are meant to be performed on instruments. 
They are only provided with short incipits of the text to orient the 
performer, and in several cases instruments like \gloss{bajón}{dulcian, bass 
curtal} or organ are specified.
Though there is need for more research into the specific instrumentation of 
Spanish musical ensembles, it is plausible that the bass line was performed in 
most cases by a continuo group of \term{bajón} doubled by harp, organ, and 
possibly other instruments like the \term{vihuela de mano}.%
\begin{Footnote}
    On the changing instrumentation in one Spanish institution, see 
    \autocite{Torrente:PhD}.
\end{Footnote}
In pieces without figured bass, continuo players---which could include any
polyphonic instruments like keyboard or plucked strings---likely improvised
harmonies to match the other voices.

The upper voices could have been doubled on \term{bajoncillos},
\gloss{chirimías}{shawms}, \gloss{sacabuches}{sackbuts}, and other instruments
according to local resources and suited to the occasion.
There is as yet no clear evidence, though, that church ensembles of 
seventeenth-century Spain or Spanish America included percussion instruments
when performing in the liturgy.%
\begin{Footnote}
    For a critique of exoticizing practices in recent villancico performances,  
    see \autocites{Baker:PerformancePostColonial}{Davies:LocalContent}.
\end{Footnote}

Ensembles should not be deterred by the lack of early instruments or by vocal
ranges outside their resources.
It would be entirely within the spirit of the performing traditions that these
sources represent, for a school or community chorus to substitute modern
instruments for their historic relatives.
At a minimum, it is appropriate to use any keyboard, preferably with bassoon or
cello, for the continuo, and bassoon or cello for the instrumental bass lines.
If more instruments are available, a small organ (or a good digital sample of an 
8$'$ flue-type stop), harp, and classical guitar could be added to the continuo
section.  
Vocal parts could be doubled with bassoons, oboes, trombones, or any other
available instruments.

If possible, it would be appropriate to use soloists or a reduced ensemble for
the first chorus in polychoral pieces, and for the coplas.
In this way a chorus of more modest ability, such as a high school choir, could
be paired with more advanced soloists, such as college students or adult
community members.
If there are more instrumentalists than singers, there should be at least one
singing voice per chorus to present the text.
Instrumental parts and continuo realizations are available from the editor upon
request.

\edsubsection{Pitch Level}
In encountering Hispanic choral music of this period, musicians more familiar
with other repertoires may be surprised by how high the vocal ranges are.
Many of the Tiple (treble) parts have tessituras above \pitch{F\,}{5}, and none of
the pieces have texted bass parts, these parts being played instrumentally
instead.
Either Spanish ensembles performed these pieces at a lower pitch level than
notated (because of a lower general pitch, or through transposition), or Spain
cultivated a lost art of angelically high singing.%
\Autocite[157]{Kendrick:Jeremiah}
Modern ensembles should sing the pieces at a pitch level or transposition that
works for them.
In addition to the two transposed scores already included, other transpositions  
are available from the editor upon request.

\edsubsection{Rhythm, Meter, and Tempo}
The meters and barlines in the edition are only conveniences to make the pieces
plainly legible and performable.
Performers should not always take the barlines as guides to accentuation, nor
should they assume the music lacks natural accentuation.
Mensural meters do not necessarily imply any particular pattern of rhythmic 
accentuation.
Most of the time poetic declamation should be the primary guide for pacing
and emphasis.
In other cases, when a set style of dance or song seems to be evoked, a regular
rhythmic pattern may win out over poetic nuances.
Regarding such rhythmic patterns, it should be noted that no one has yet
provided conclusive evidence for the presence of African or American indigenous
rhythms in villancicos.

The sign \meterC{} indicates a duple meter that should be felt \quoted{in
two}.
The sign \meterCThree{} indicates a ternary meter that should be felt
\quoted{in one}.
Often triple meter is syncopated or altered by hemiola (also called
\term{sesquialtera}) to create patterns of accentuation that differ from the
normal ternary groupings indicated by the barlines.
In José de Cáseda's \worktitle{Qué música divina}, for example, after shifting
to \meterCThree{}, the composer makes novel use of \term{sesquialtera}.
Starting in \measure{19} Cáseda creates a sustained pattern of stresses in
groups of three imperfect semibreves, such that the music could be rebarred in
modern \meter{3}{1}.

In most cases, it seems appropriate to maintain a tempo relationship of three
minims (half notes) in \meterCThree{} to one minim (half note) in \meterC{}.
By maintaining this tempo relationship it is usually possible to maintain a
consistent pulse throughout the whole piece.
A resting heart rate of about sixty beats per minute generally makes a good
tempo, such that in \meterC{}, $\musMinim{} = 60$; while in \meterCThree,
$\musSemibreveDotted = 60$ and $\musMinim{} = 180$.


\edsubsection{Authenticity and Flexibility}
In the editor's opinion, an authentic performance of a seventeenth-century
villancico would be one that is not only meaningful to present-day performers
and their audience, but that also opens a window to experiencing what made the
piece meaningful to its original performers and hearers.
Performers should seek out the distinctive character and significance of each
piece, but should also feel free to adapt the pieces to suit their own resources
and social context.
It would be better to have a spirited, respectful, musically sensitive
performance with modern instruments, for example, than to have no performance
at all because historic instruments were not available.

The sources for this edition are performing parts that, on the one hand,
were used as practical tools for performance in a particular place, and, on the
other hand, represent traditions of performance that cannot be completely fixed
in place or time.
Even within one institution, such as the Conceptionist convent in Puebla from
which come the parts for Salazar's \worktitle{Angélicos coros} and Cáseda's
\worktitle{Qué música divina}, these parts were used and reused possibly over
generations. 
In some cases, later performers made corrections, added barlines, sewed in new
lines of lyrics or even new music to replace certain strophes.
There is no single way that these pieces were performed throughout their terms
of service as part of the local repertoire.

Moreover, these pieces represent single instances of a repertoire that
circulated around the globe. 
José de Cáseda lived in Zaragoza and set a text by a poet from his same
region, Vicente Sánchez; but this setting is only known from the surviving parts
in the Puebla convent.
The spelling in those parts reflects New Spanish, not Zaragozan pronunciation.%
\begin{Footnote}
    For example, \mentioned{consonancias} is spelled \mentioned{consonansias} in
    the Puebla parts, even though Cáseda's ensemble in Zaragoza probably used a
    sound like English TH for the final C.
\end{Footnote}
The piece may have been rearranged or adapted for female ensemble from a lost
original with different scoring.
On some occasions, a particular sister may have fallen ill and her vocal line
may have been played instrumentally.

Historic performers made these pieces their own and performed them in a way
that fit their local needs in terms of personnel, instrumentation, acoustic
space, and other factors.
They performed these pieces in a way that was intelligible and meaningful to
them and to their hearers.
Modern performers are continuing in the same spirit when they make practical
adaptations for their circumstances.

\edsubsection{Ethical Responsibility}
While some amount of adaptation seems appropriate for this repertoire,
performers are urged never to lose sight of the religious, social, and political
contexts of these pieces in their early modern origins.
These villancicos are all devotional pieces, used at some point in liturgical
worship, but they do not fit easily into modern notions of sacred and profane,
and embody both \quoted{piety and play}.%
    \Autocite{Cashner:Cards}
If we perform villancicos with too much solemnity, listeners may miss the
elements of fun and virtuosity;
but if we perform them too flippantly, the audience may fail to recognize them
as expressions of human spirituality and ingenuity.

These pieces cannot be cleanly separated from the social values of the colonial
era that this music both reflected and reinforced.
A piece like Juan Gutiérrez de Padilla's \worktitle{Al establo más dichoso}
bears the imprint of imperial Spain's racial hierarchy. 
It is documented that the composer himself owned an Angolan slave,%
    \Autocite{Mauleon:PadillaCivil}
and the representation of \quoted{Angolans} in the piece caricatures their
bodies and voices as deformed and deficient, even as it perhaps strives to
present them in a sympathetic light as offering devotion to Christ and joining
with the angelic chorus.
As Geoffrey Baker has argued, it would be ethically irresponsible to perform
such a piece merely as an exotic curiosity, or worse, as though it were a
twenty-first century celebration of ethnic diversity.%
    \Autocite{Baker:PerformancePostColonial}

Indeed, performers, scholars, and community members ought to engage in serious
discussions about what performing such a piece might mean in a contemporary
context.  
In the right setting, such as a community workshop with appropriate
opportunities for critique, response, and discussion, the piece might be used 
effectively to raise issues of great contemporary relevance.
In the wrong context, though, the piece could actually perpetuate the negative
racial stereotypes that are built into it.



