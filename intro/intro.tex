\edpart{Introduction}

At the height of the Spanish Empire in the seventeenth century,
villancicos were one of the most widespread forms of religious expression and a central part of social life.%
  \begin{Footnote}
  For an introduction to the genre, see the entries for \quoted{villancico} in \worktitle{Grove Music Online} and the \worktitle{Diccionario de la música española e hispanoamericana}.
  The fundamental studies of villancicos as a musical genre are (in chronological order) \autocites{Rubio:Forma}{CaberoPueyo:PhD}{Laird:VC}{Torrente:PhD}{Illari:Polychoral}, 
  and the introduction and essays in \autocite{Knighton-Torrente:VCs}; 
  see also \autocite{Cashner:Cards}.
  On the villancico as a literary genre, see \autocite{Tenorio:SorJuana}.

  Modern editions of seventeenth-century villancicos include (chronologically) \autocites
{Cererols:MEM-VC}
{Stevenson:Christmas}
{Ruimonte:Parnaso}
{Padilla:Tello}
{Ezquerro:MME55}
{RuizSamaniego:MME63}
{Ezquerro:MME59}
{Ezquerro:MME65}
{Fernandez:Cancionero}
{Torrejon:VCs}.
  \end{Footnote}
These poems in vernacular languages (usually Spanish or Portuguese) celebrated common Catholic beliefs, popular customs, and modes of devotion through metaphorical conceits both earthy and ingenious.
In the form of printed leaflets of the poetry (\term{pliegos sueltos}) and in manuscript performing parts of the music they were passed from hand to hand across oceans in a global network of affiliated musicians and members of the literate elite.%
  \begin{Footnote}
  The largest collections of these leaflets or chapbooks are cataloged in \autocites{BNE:VCs17C}{BNE:VCs18C}{UK:VCs}{US:VCs}.
  \end{Footnote}

The musical settings of these poems occupied the energies of every major chapelmaster and his ensemble on all the highest feast days of the year.
The pieces were interspersed after or in place of the Reponsory chants of the Matins liturgy, as well as for Mass, Forty Hours' Devotion and other Eucharistic adoration, Corpus Christi processions and mystery plays (\term{autos sacramentales}) and oratorios.
The surviving repertoire, even if it is only a fraction of what was composed, is vast and rich, encompassing a broad range of Hispanic devotional life.
From the comic to the learned, there was a villancico for everyone and nearly every occasion.

This edition offers performers and scholars a coherent set of newly edited villancicos, drawn from archives in Spain and Mexico, that share a common theme of \quoted{singing about singing}.
The vernacular poetry of these pieces represents the act of hearing and making music, and the musical settings of these poems, then, become discourses about music, through the medium of music itself.%
  \begin{Footnote}
  These villancicos formed the primary corpus of study for my dissertation, \citemydiss.
  This edition corrects numerous errors in the musical editions of the dissertation, and is intended to replace those editions.
  \end{Footnote}
The thematic organization makes these pieces ideal both for concert programming and for scholarly study.

\section{Editorial Policies}

\paragraph{Sources}

The sources for the each poem and its musical setting are listed in the critical notes.
The music is preserved in individual manuscript performing parts in looseleaf sets or bound partbooks.
For the villancico by Miguel de Irízar, the composer's draft score also survives.

The poetic editions are based on the lyrics of the musical settings.
They have been annotated and sometimes corrected in comparision with the surviving poetry imprints of the same or related villancico poems.
The poems are generally anonymous, but are often adapted from existing poems or poetic types.

These manuscript parts were practical tools for performers.
They all bear evidence of frequent use over a long period: they are soiled along the creases in the paper where performers held them up, and they include the names of multiple performers, corrections in different hands, and added barlines.
There are therefore only a few undoubted errors in the parts, and these are usually of the sort (like missing rests) that performers could easily have overlooked, and for some reason felt no need to correct.
Other aspects that seem ambiguous to a modern scholar were not, apparently, impediments to effective performance from the originals.

The goal of this edition, in keeping with the nature of these sources, is to enable the practical performance and study of these villancicos.
I have endeavored to present the music in a clear and consistent modern notation.


\paragraph{Orthography}
For the sake of consistency and intelligibility, spelling and punctuation have been modernized and standardized.
Doing so means losing some information about the local pronunciation of Spanish in different parts of the empire.
The phonetic orthography in the performing parts demonstrates that \mentioned{ci} and \mentioned{ce} were pronounced like \mentioned{si} and \mentioned{se} in New Spain and Catalonia.
But standardization has the greater benefit of enabling performers to present the works in a way that will be most intelligible to their audiences.

\paragraph{Editorial Text}
Italic text in the lyrics indicate editorial text underlay, usually where there are unambiguous repeat signs in the sources.
Other textual additions by the editor, such as standardized section headings, are enclosed in square brackets.

\paragraph{Pitch Level}
All pieces are transcribed at their original notated pitch level.
The preparatory staves at the beginning of each piece show the original clefs, signatures, and the first note.
While the  general pitch level was likely lower than today, some of the pieces may have been performed at a lower transposition.
I will be happy to make different transpositions available for performers on request.

The critical notes use the International Pitch Notation system for denoting octaves, wherein \soCalled{middle C} is C\octave{4}.



\paragraph{Accidentals}
Accidental placement in the partbooks is contextual and sometimes ambiguous to a modern reader.
The original notation has no \na{} symbol, using B\sh{} and E\sh{} instead.
In a few cases, indicated in the critical notes, scribes use a \sh{} sign as a cautionary accidental, warning the singer \emph{not} to apply a sharp according to \term{ficta} conventions.%
  \autocites{Harran:Cautionary1}{Harran:Cautionary2}

The edition presents the pitches with their accidental inflections when unambiguously specified in the source.
According to modern convention, these accidentals are valid until the next barline.
Thus repeated accidentals in the source are omitted if the modern convention does not require them; and in a few cases accidentals are added on the staff where the original notation is unambiguous about what the pitch should be.

Editorial suggestions for other accidentals, mostly according to \term{musica ficta} conventions, are set above the staff.

\paragraph{Meter, Rhythm, Tempo}
The original music was written in mensural notation, with few barlines in the performing parts. 
The duple-meter sections of these piece were written in mensural \meterC{} meter, which the seventeenth-century Spanish theorists Pedro Cerone and Andrés Lorente refer to as \term{tiempo menor imperfecto} or \term{compasillo} (henceforth, C).%
  \autocites[537]{Cerone:Melopeo}[156, 210]{Lorente:Porque}
In this meter, the \term{compás} or \term{tactus} consisted of a semibreve divided into two minims.%
  \autocites{GonzalezValle:MusicaTexto}{GonzalezValle:CompasCabezon}

The other common meter for seventeenth-century villancicos was notated with the symbol \meterCZ{}, a cursive CZ.
Lorente says that this is a shorthand for \meterCThree{} or \meterCThreeTwo{}.%
  \autocite[165]{Lorente:Porque}
These signs all indicate \term{tiempo menor de proporción menor}, a proportion of C meter.
The \term{compás} consists of one perfect semibreve which is divided into three minims, instead of the two minims of C.

In the sources, deviations from the normal ternary groups are indicated through coloration. 
When noteheads in CZ meter are blackened, this often indicates a shift to \term{sesquialtera} or hemiola.
In \term{sesquialtera} two groups of three minims are exchanged for three groups of two minims; two perfect semibreves are subsituted by three imperfect semibreves.

The edition presents the rhythms of the sources according to modern conventions of meter and barlines.
The original meter signs are shown in preparatory staves or above the staff.
The original note values have not been reduced.
Mensural coloration is indicated with short rectangular brackets above the staff.
Ligatures are indicated by long rectangular brackets.
Beaming is unchanged.

The music has been notated in the modern meters of \meter{2}{2} for C time and \meter{6}{2} for CZ time.
In some cases in ternary meter an extra bar of \meter{3}{2} has to be added, and wherever possible this has been done at the beginning or end of a section.

These barlines and meters are only conveniences for modern performers.
This edition's barring practice allows many of the original durations to be transcribed without reduction into tied notes.

Spanish composers like Miguel de Irízar did use barlines when they notated in score format.
Irízar writes two \term{compases} per bar in both CZ and C meters, and also fits odd \term{compases} within these groups.
Cerone advises students who wish to write out a score from parts to write barlines every two \term{compases}.%
  \begin{Footnote}
  \autocite[745]{Cerone:Melopeo}.
  This practice would suggest using \meter{4}{2} for C meter in the modern edition, but in my opinion this makes the duple-meter sections too difficult to read.
  \end{Footnote}

Performers should not take the barlines as guides to accentuation, nor should they assume the music lacks natural accentuation.
Mensural meters do not necessarily imply any particular pattern of rhythmic accentuation.
The music is often composed to create clearly articulated \soCalled{downbeats}; but many pieces also make much use of syncopation and irregular patterns.

Regarding tempo, the theoretical proportion of $3:2$ does not necessarily imply a tempo relationship.
In actual practice with these pieces, a \emph{tempo} relationship of $3:1$ often makes more musical sense when moving between CZ and C meters.
Thus two \term{compases} of CZ would have about the same duration as one \term{compás} of C.
In this edition, then, one bar of \meter{6}{2} would approximately equal one bar of \meter{2}{2}.
Thus performers can maintain a consistent pulse throughout a single piece.
The edition includes markers of these tempo relationships at each change of meter, but these are only suggestions.

\paragraph{Instrumentation}

These villancicos are scored for an ensemble of voices with instrumental bass or continuo groups.
Vocal ensembles varied in size, from one-to-a-part groups to much larger polychoral forces.
Most of the pieces also feature prominent solo parts, particularly in the \term{coplas}.

The bottom voice parts in these pieces are mean to be performed on instruments. 
They are only provided with short incipits of the lyrics to orient the performer, and in several cases instruments like \term{bajón} (dulcian, bass curtal) or organ are specified.
Though there is need for more research into the specific instrumentation of Spanish musical ensembles, it is plausible that the bass line was performed in most cases by a continuo group of \term{bajón} doubled by harp, organ, and possibly other instruments like the \term{vihuela} or perhaps guitar.%
  \footnote{On the changing instrumentation in one Spanish institution, see \autocite{Torrente:PhD}.}
In pieces without figured bass, continuo players likely improvised harmonies to match the other voices.

The upper voices could have been doubled on \term{chirimías} (shawms) or a variety of other instruments like \term{sacabuches} (sackbuts) according to local resources and suited to the occasion.
There is as yet no clear evidence, though, that church ensembles of seventeenth-century Spain included percussion instruments.%
  \begin{Footnote}
  For a critique of exoticizing practices in recent villancico performances,  see \autocite{Baker:PerformancePostColonial}.
  This is particularly relevant to the performance of \soCalled{ethnic} villancicos such as the \term{negrilla} included in Juan Gutiérrez de Padilla's \worktitle{Al establo más dichoso}, in this edition.
  \end{Footnote}

\section{Abbreviations}

\begin{tabu} to \textwidth{lZ}
A. & Alto, \term{Altus}\\
Ac. & \term{Acompañamiento}: Accompaniment, \term{basso continuo}\\
B. & \term{Bajo}, \term{Bassus}\\
CN & See critical notes\\
DRAE & Real Academía Española, \worktitle{Diccionario de la lengua española}, 23rd ed.\\
Leg. & \term{Legajo} (archival folder)\\
OED & \worktitle{Oxford English Dictionary}\\
S. & Soprano; Used in part listings (e.g., \quoted{SSAT}) for the highest voice part, usually for \term{Tiple}, to avoid confusion between \quoted{Ti.} for Tiple and \quoted{T.} for Tenor\\
T. & Tenor\\
Ti. & \term{Tiple}: Treble, boy soprano\\
Ti. I-1 & Chorus 1, First Tiple\\
\end{tabu}

\subsection{Archival Sigla}

\begin{tabu} to \textwidth{llZ}
\textsc{Siglum} & \textsc{Country} & \textsc{Archive}\\
E-Bbc & Spain & Barcelona, Biblioteca de Catalunya\\
E-CAN &  & Canet de Mar, Arxiu parroquia de Sant Pere i Sant Pau de Canet de Mar, Bisbat de Girnoa, Fons Capella de Música\\
E-Mn & &  Madrid, Biblioteca Nacional de España\\
E-SE & & Segovia, Catedral, Archivo Capitular\\
MEX-Pc & Mexico &  Puebla, Catedral, Archivo Capitular\\
MEX-Mcen & Mexico & Mexico City, CENIDIM (Centro Nacional de Investigación, Documentación e Información Musical Carlos Chávez)\\
GB-Lbl & United Kingdom & London, British Library\\
\end{tabu}


\section{Colophon}

This edition was prepared with free and open-source software, using the \LaTeX{} document-preparation system for the text and Lilypond (version 2.18.2) for the music.
The text typeface is EB Garamond, designed by Georg Duffner, based on 1592 type specimens by Claude Garamont.
The Spanish CZ metrical symbol used in the scores was traced manually in Inkscape from a villancico manuscript by Miguel de Irízar.

\section{Acknowledgments}

I am grateful to the following people and institutions for access to the primary sources on which these editions are based: 
the capitular archive of the Cathedral of Puebla de los Ángeles (P. Francisco Vázquez, rector; the Illmo. Sr. Carlos Ordaz, \term{canónigo archivista});
CENIDIM, the Mexican national center for music research in Mexico City;
the Biblioteca de Catalunya in Barcelona;
the parochial archive of the Church of Saints Peter and Paul, Canet de Mar, in the Archdiocese of Girona, Barcelona province;
the capitular archive of Segovia Cathedral (P. Bonifacio Bartolomé);
the Biblioteca Nacional in Madrid, and
the British Library in London.

Travel for archival research in Mexico and Spain was funded by 
\begin{anonymize}
a Jacob K. Javits Fellowship from the United States Department of Education, 
a Pre-Dissertation Research Fellowship from the Council for European Studies at Columbia University, 
a Eugene K. Wolf Research Travel Grant from the American Musicological Society, 
and grants from the Department of Music and the Center for Latin American Studies at the University of Chicago.
Other research funding was provided by a Dissertation Completion Fellowship from the American Council of Learned Societies and the Mellon Foundation.

I thank my doctoral advisor, Robert Kendrick, and the other members of my dissertation committee, Anne Walters Robertson, Martha Feldman, Frederick de Armas, and María Gembero-Ustárroz.
I am also grateful to the following people for specific help with the poetry and music in this edition:
Stephen Black,
Anita Damjanovic, 
Miguel Martínez, 
Gustavo Mauleón Rodríguez,
James Nemiroff, and
Martha Tenorio.
My wife Ann makes all of this possible.
\end{anonymize}
