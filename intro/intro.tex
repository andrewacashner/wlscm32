At the height of the Spanish Empire in the seventeenth century,
villancicos were one of the most widespread forms of religious expression and a central part of social life.
These poems in vernacular language (usually Castilian Spanish or Portuguese) celebrated common Catholic beliefs, popular customs, and modes of devotion through metaphorical conceits both early and ingenious.
In the form of printed leaflets (\term{pliegos sueltos}) and in manuscript they were passed from hand to hand across oceans in a global network of affiliated musicians and members of the literate elite.
The musical settings of these poems occupied the energies of every parish chapelmaster and his ensemble on all the major feast days, interspersed after or in place of the Reponsory chants of the Matins liturgy, as well as for Mass, Forty Hours' Devotion and other Eucharistic adoration, Corpus Christi processions and mystery plays (\term{autos sacramentales}) and oratorios.
The surviving repertoire, even if it is only a fraction of what was composed, is vast and rich, encompassing the full range of Hispanic devotional life.
From the comic to the learned, there was a villancico for everyone and nearly every occasion.

This edition offers performers and scholars a coherent set of newly edited villancicos, drawn from archives across Mexico and Spain, that share a common theme of musical hearing.
The vernacular poetry of these pieces represents the act of hearing and making music, and the musical settings of these poems, then, become discourses about music, through the medium of music itself.
This collection provides a wider range of composers, regions, and styles than has ever been made available in one volume.
The thematic organization makes these pieces ideal both for concert programming and for scholarly study.%
  \footnote{These villancicos formed the primary corpus of study for the dissertation, \autocite{Cashner:PhD}.
  This edition corrects numerous errors in the musical editions of the dissertation.}



