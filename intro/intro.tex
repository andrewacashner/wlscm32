\edpart{Introduction}

At the height of the Spanish Empire in the seventeenth century,
villancicos were one of the most widespread forms of religious expression and a central part of social life.%
  \begin{Footnote}
  The fundamental sources on the history of villancicos as a musical genre are \autocites{Laird:VC}{Rubio:Forma}{Torrente:PhD}{Illari:Polychoral} and the essays in \autocite{Knighton-Torrente:VCs}.
  On the villancico as a poetic genre, see \autocite{Tenorio:SorJuana}.

  Modern editions of seventeenth-century villancicos include \autocites{Ruimonte:Parnaso}{Fernandez:Cancionero}{RuizSamaniego:MME63}{Ezquerro:MME55}{Ezquerro:MME65}{Ezquerro:MME59}{Padilla:Tello}{Torrejon:VCs}, and \autocite{Stevenson:Christmas} (though the latter should be treated with caution).
  \end{Footnote}
These poems in vernacular language (usually Castilian Spanish or Portuguese) celebrated common Catholic beliefs, popular customs, and modes of devotion through metaphorical conceits both earthy and ingenious.
In the form of printed leaflets (\term{pliegos sueltos}) and in manuscript they were passed from hand to hand across oceans in a global network of affiliated musicians and members of the literate elite.%
  \begin{Footnote}
  The largest collections of these leaflets or chapbooks are cataloged in \autocites{BNE:VCs17C}{BNE:VCs18C}{UK:VCs}{US:VCs}.
  There are also important collections at the Lilly Library of Indiana University, the John Carter Brown Library of Brown University, the Biblioteca José María Lafragua of the Benemérita Universidad Autónoma de Puebla, the Biblioteca Palafoxiana in Puebla, and other collections still in private hands.
  \end{Footnote}

The musical settings of these poems occupied the energies of every parish chapelmaster and his ensemble on all the major feast days, interspersed after or in place of the Reponsory chants of the Matins liturgy, as well as for Mass, Forty Hours' Devotion and other Eucharistic adoration, Corpus Christi processions and mystery plays (\term{autos sacramentales}) and oratorios.
The surviving repertoire, even if it is only a fraction of what was composed, is vast and rich, encompassing the full range of Hispanic devotional life.
From the comic to the learned, there was a villancico for everyone and nearly every occasion.

This edition offers performers and scholars a coherent set of newly edited villancicos, drawn from archives across Mexico and Spain, that share a common theme of \quoted{singing about singing}.
The vernacular poetry of these pieces represents the act of hearing and making music, and the musical settings of these poems, then, become discourses about music, through the medium of music itself.
The thematic organization makes these pieces ideal both for concert programming and for scholarly study.%
  \begin{Footnote}
  These villancicos formed the primary corpus of study for the dissertation, \autocite{Cashner:PhD}.
  This edition corrects numerous errors in the musical editions of the dissertation, and is intended to replace those editions.
  \end{Footnote}

\section{Sources}

The sources for the each poem and its musical setting are listed in the critical notes.
The music is preserved in individual manuscript performing parts in looseleaf sets or bound partbooks.
For the villancico by Miguel de Irízar, the composer's draft score also survives.

The poetic editions are based on the lyrics of the musical settings and surviving poetry imprints of the same or related villancico poems.

\section{Editorial Policies}

\paragraph{Text Orthography}
For the sake of consistency and intelligibility, spelling and punctuation have been modernized and standardized.
Doing so means losing some information about the local pronunciation of Spanish in different parts of the empire.
The phonetic orthography of the performing parts demonstrates that \mentioned{ci} and \mentioned{ce} were pronounced like \mentioned{si} and \mentioned{se} in New Spain and Catalonia.
But standardization has the greater benefit of enabling performers to present the works in a way that will be most intelligible to their audiences.

\paragraph{Lyric Underlay}
Italic text in the lyrics indicate editorial text underlay, usually where there are unambiguous repeat signs in the sources.
Other textual additions by the editor are enclosed in square brackets.

\paragraph{Pitch Level}
All pieces are transcribed at their original notated pitch level.
The preparatory staves at the beginning of each piece show the original clefs, signatures, and the first note.
In addition to a general pitch level that was likely lower than today, some of the pieces may have been performed at a lower transposition.

\paragraph{Accidentals}
Accidentals are placed according to the modern rule in which accidentals last through to the end of the bar.
The edition presents the pitches with their accidental inflections as specified in the score.
Thus repeated accidentals in the source are omitted if the modern convention does not require them; and in a few cases accidentals are added on the staff where the original notation is unambiguous about what the pitch should be.

The original notation has no \na{} symbol, using B\sh{} and E\sh{} instead.
In a few cases, scribes use a \sh{} sign as a cautionary accidental, warning the singer \emph{not} to apply a sharp according to \term{ficta} conventions.%
  \autocites{Harran:Cautionary1}{Harran:Cautionary2}

Editorial suggestions for other accidentals, mostly according to \term{musica ficta} conventions, are set above the staff.

\paragraph{Meter and Rhythm}
The original music was written in mensural notation, with few barlines in the performing parts (composers did use barlines in score format). 
The duple-meter sections of these piece were written in mensural \meterC{} meter, known as \term{tiempo imperfecto menor} or \term{compasillo} in seventeenth-century Spanish music theory.%
  \autocites[537--538,740]{Cerone:Melopeo}[210]{Lorente:Porque}
  {GonzalezValle:CompasCabezon}{GonzalezValle:MusicaTexto}
In this meter, the \term{compás} or \term{tactus} consisted of a semibreve divided into two minims.
The theorist Andrés Lorente says that in each measure the hand falls once and rises once in even time periods.

Most seventeenth-century villancicos were written in a mensural meter notated with the symbol \meterCZ{} (CZ).
Lorente says that this is a cursive shorthand for \meterCThreeTwo{} and is also, by custom, interchangeable with \meterCThree.%
  \autocite[165]{Lorente:Porque}
All of these are labeled \term{tiempo imperfecto menor de proporción menor}, a proportion of \meterC{}.
The \term{compás} consists of one perfect semibreve which is divided into three minims, instead of the two minims of \meterC.
Thus there is a $3:2$ proportion between the two meters.


Deviations from the normal ternary groups are indicated through coloration. 
When noteheads in CZ meter are blackened, this often (but not always) indicates a shift to \term{sesquialtera} or hemiola.
In \term{sesquialtera} two groups of three minims are exchanged for three groups of two minims; two perfect semibreves are subsituted by three imperfect semibreves.

The edition presents the rhythms of the sources according to modern conventions of meter and barlines.
The basic note values are not reduced.
For \meterC, the edition uses \meter{2}{2}.
For \meterCZ, the edition uses \meter{6}{2} or \meter{3}{2}.
Mensural coloration is indicated with short rectangular brackets above the staff.
Ligatures are indicated by long rectangular brackets.

As noted above, villancico composers did use barlines when they wrote in score.
Cerone instructs students who wish to write out a piece from parts in score format that they should insert barlines every two \term{compases}.\X[cite]
That would mean putting barlines every four minims in \meterC{} and every six in \meterCZ{}.

I have adopted Cerone's suggestion for the latter by putting the ternary-meter music in \meter{6}{2} wherever possible.
Composers make such frequent use of \term{sesquialtera} and other rhythmic alterations that using \meter{3}{2} results in constant ties across the barline.
Using \meter{6}{2} makes more of these groupings visible.
Occasionally a single \meter{3}{2} bar is needed when there is an odd number of \term{compases}; I have put these whenever possible at the end of sections or more rarely at the beginning.

Where grouping six minims per bar in triple meter makes the rhythmic more intelligible, in my opinion it has the opposite effect to follow the same rule in duple meter.
Placing barlines every two \term{compases} in \meterC{}---that is, transcribing the duple-meter music in \meter{4}{2}---results in very long measures that are difficult to scan.
Thus I have stuck with \meter{2}{2}.

This seeming inconsistency between the treatment of duple and triple meters has an benefit for performers beyond simple legibility.
The Spanish meters specify a notational proportion, but not necessarily a tempo relationship.
A strict three-to-two proportion of \term{tempo} between these meters does not seem to fit the character of the music in most cases.
Indeed, much of the time it seems more suitable to maintain a three-to-one tempo relationship between minims in CZ and in \meterC.
With CZ notated in \meter{6}{2} and \meterC{} notated in \meter{2}{2}, a conductor can in most cases get a good result by keeping the general duration of the notated bar the same between meters.
In other words, as a tempo relationship, one bar of duple meter should take about the same amount of time as two bars of triple meter.
Thus a basic underlying pulse can be preserved throughout most of these pieces.

The usual cautions apply about using modern barlines with their implied downbeats.
Mensural meters do not necessarily imply any particular pattern of rhythmic accentuation.
The music is often composed to create clearly articulated \soCalled{downbeats}; but many pieces also make much use of syncopation and irregular patterns.
The modern notation is only a convenience; performers should not take the barlines as guides to accentuation, nor should they assume the music lacks natural accentuation.


\section{Instrumentation}

These villancicos are scored for an ensemble of voices with instrumental bass or continuo groups.
Vocal ensembles varied in size, from one-to-a-part groups to much larger polychoral forces.
Most of the pieces also feature prominent solo parts, particularly in the \term{coplas}.

The bottom voice parts in these pieces are mean to be performed on instruments. 
They are only provided with short incipits of the lyrics to orient the performer, and in several cases instruments like \term{bajón} (dulcian, bass curtal) or organ are specified.
Though there is need for more research into the specific instrumentation of Spanish musical ensembles, it seems safe to assume that the bass line was performed in most cases by a continuo group of \term{bajón} doubled by harp, organ, and possibly other instruments like the \term{vihuela} or perhaps guitar.%
  \footnote{On the changing instrumentation in one Spanish institution, see \autocite{Torrente:PhD}.}
In pieces without figured bass, harpists and organists may have improvised the harmonies according to a kind of aural intabulation process.

The upper voices could have been doubled on \term{chirimías} (shawms) or a variety of other instruments like \term{sacabuches} (sackbuts) according to local resources and suited to the occasion.
There is as yet no clear evidence, though, that church ensembles of seventeenth-century Spain included percussion instruments.%
  \begin{Footnote}
  For a critique of exoticizing practices in recent villancico performances, particularly relevant to the performance of \soCalled{ethnic} villancicos such as the \term{negrilla} included in Juan Gutiérrez de Padilla's \worktitle{Al establo más dichoso} (in this edition), see \autocite{Baker:PerformancePostColonial}.
  \end{Footnote}

\section{Colophon}

This edition was prepared with free and open-source software on the Debian GNU/Linux operating system, version 8.
The edition was typeset with the \LaTeX{} document-preparation system, which is based on the work of Donald Knuth and Leslie Lamport.
This document's custom class file builds on Peter Wilson's \worktitle{memoir} class and uses Maïeul Rouquette's \worktitle{reledmac} and \worktitle{reledpar} packages (based on earlier work by Peter Wilson and others) for parallel poetry typesetting.
The music was typeset with Lilypond, version 2.18.2, using a custom header file.

The text typeface is EB Garamond, designed by Georg Duffner, based on Conrad Berner and Christian Egenolff's 1592 specimens of typefaces by Claude Garamont.
The Spanish CZ metrical symbol used in the scores was traced manually in Inkscape from a villancico manuscript by Miguel de Irízar.

\section{Acknowledgments}

I am grateful to the following people and institutions for access to the primary sources on which these editions are based: 
the capitular archive of the Cathedral of Puebla de los Ángeles (P. Francisco Vázquez, rector; the Illmo. Sr. Carlos Ordaz, \term{canónigo archivista});
CENIDIM, the Mexican national center for music research in Mexico City;
the Biblioteca de Catalunya in Barcelona;
the parochial archive of the Church of Saints Peter and Paul, Canet de Mar, in the Archdiocese of Girona, Barcelona province;
the capitular archive of Segovia Cathedral (P. Bonifacio Bartolomé);
the Biblioteca Nacional in Madrid, and
the British Library in London.

Travel for archival research in Mexico and Spain in 2012 was funded by 
a Jacob K. Javits Fellowship from the United States Department of Education, 
a Pre-Dissertation Research Fellowship from the Council for European Studies at Columbia University, 
a Eugene K. Wolf Research Travel Grant from the American Musicological Society, 
and grants from the Department of Music and the Center for Latin American Studies at the University of Chicago.
Other research funding was provided by a Dissertation Completion Fellowship from the American Council of Learned Societies and the Mellon Foundation.

I thank my doctoral advisor, Robert Kendrick, and the other members of my dissertation committee, Anne Walters Robertson, Martha Feldman, Frederick de Armas, and María Gembero-Ustárroz.
I am also grateful to the following people for specific help with the poetry and music in this edition:
Stephen Black,
Anita Damjanovic, 
Miguel Martínez, 
James Nemiroff, and
Martha Tenorio.
My wife Ann makes all of this possible.

