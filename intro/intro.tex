\edpart{Introduction}

At the height of the Spanish Empire in the seventeenth century,
villancicos were one of the most widespread forms of religious expression and a central part of social life.%
  \begin{Footnote}
  For an introduction to the genre, see the entries for \quoted{villancico} in \worktitle{Grove Music Online} and the \worktitle{Diccionario de la música española e hispanoamericana}, and the studies in the bibliography below.
  Other modern editions of seventeenth-century villancicos are listed in the bibliography.
  \end{Footnote}
These poems in vernacular languages (usually Spanish or Portuguese) celebrated common Catholic beliefs, popular customs, and modes of devotion through metaphorical conceits both earthy and ingenious.
In the form of printed leaflets of the poetry (\term{pliegos sueltos}) and in manuscript performing parts of the music they were passed from hand to hand across oceans in a global network of affiliated musicians and members of the literate elite.%
  \begin{Footnote}
  See the bibliography for catalogs of the largest collections of these leaflets or chapbooks.
  \end{Footnote}

The musical settings of these poems occupied the energies of every major chapelmaster and his ensemble on all the highest feast days of the year.
Though the genre originated as a form of courtly entertainment, by the beginning of the seventeenth century most villancicos were sacred both in their themes and in the venues and occasions of their performance.
Sacred villancicos were often composed in sets of eight so that they could be interspersed after or in place of the Reponsory chants of the Matins liturgy, especially at Christmas and Corpus Christi.
Villancicos were also performed in church for Mass, Forty Hours' Devotion and other Eucharistic adoration, and outside of church in Corpus Christi processions and mystery plays (\term{autos sacramentales}).

Villancicos encompass a wide range of formal structures, but most feature an \term{estribillo}, a motet-like section for the full ensemble, and \term{coplas}, strophic verses often scored for soloists or a reduced ensemble.
Typically the \term{estribillo} was repeated after the \term{coplas}.
Many villancicos of this period were scored for large polychoral ensembles of voices, doubled by loud wind instruments and supported by organ, harp, and other plucked strings.
Others, sometimes called \term{tonos divinos}, are scored for a more intimate texture of a few voices with continuo.

The surviving repertoire, even if it is only a fraction of what was composed, is vast and rich, encompassing a broad range of Hispanic devotional life.
With a variety of subgenres and topics from the comic to the learned, there was a villancico for everyone and nearly every occasion.

This edition offers performers and scholars a coherent set of newly edited villancicos, drawn from archives in Spain and Mexico, that share a common theme of \quoted{singing about singing}.
The vernacular poetry of these \soCalled{metamusical} pieces represents the act of hearing and making music, and the musical settings of these poems, then, become discourses about music, through the medium of music itself.%
  \begin{Footnote}
  These villancicos formed the primary corpus of study for my dissertation, \citemydiss.
  This edition corrects numerous errors in the musical editions of the dissertation, and is intended to replace them.
  \end{Footnote}
The thematic organization makes these pieces ideal both for concert programming and for scholarly study.

\section{Interpretive Themes}

Common tropes run through these pieces and demonstrate traditions of poetry and music about musical performance, or about music as an abstract concept.
The pieces embody a Neoplatonic theology of music in which listeners are invited to listen for echoes of higher music in the imperfect earthly music they hear.

\subsection{Cererols and Padilla: Christ as Singer and Song}

The first two pieces, \worktitle{Suspended, cielos, vuestro dulce canto} by the Catalan monk Joan Cererols, and \worktitle{Voces, las de la capilla} by the Andalusian immigrant chapelmaster of Puebla Cathedra, Juan Gutiérrez de Padilla, represent the newborn Christ as both singer and song. 
Extending on an exegetical tradition going back to Bernard of Clairvaux and Augustine, these pieces celebrate Christ as the \term{Verbum infans}, the Word of God made flesh (John~1:1), but as an infant, unable to speak a word (this is the meaning of the Latin \term{in-fans}). 
Since Christ himself is the Word, these pieces portray his inarticulate cries as a form of music, as the tuning note (Padilla's \quoted{sign of \term{A (la, mi, re)}} to which the music of a renewed creation will be harmonized.

The composers match the musical conceits of the poetry with the appropriate musical devices, such as the eight-voice fugue in strict classical counterpoint that Cererols creates for \term{contrapunto celestial}, or Padilla's quotation of the plainchant \term{tonus peregrinus} on the words \term{peregrino tono}.
Cererols even illustrates the idea of Christ as the \term{cantus firmus} for a restored heavenly music by developing the motive of a descending stepwise fifth throughout the \term{estribillo}, culminating in a concluding section in the style of a \term{cantus firmus} motet.
Cererols illustrates \quoted{the newest consonance} by setting the word \mentioned{consonancia} on a prominent, unprepared, and repeated dissonance.
By drawing listeners' attention to the imperfection of worldly music through this ironic symbol, Cererols points them in Neoplatonic fashion past the sounding music, to listen for an unhearable, higher music of Christ the divine Word.

Padilla has half his ensemble exhort the other half to \quoted{keep count with what is sung} while they are literally counting their rests; and the other chorus sings about \quoted{awaiting the thirty-three} (a reference to Christ's Passion) with exactly thirty-three notes.
Both choirs join together to represent the celebration of heavenly beings, humans, and beasts singing in the manger, in the style of a madrigal, scored for voices \quoted{three by three, two by two, one by one}.
Padilla's \term{estribillo} climaxes with an epitome of Incarnational theology, \quoted{Everything in man is to ascend, and everything in God is to descened}, in which Padilla sets the first line to an ascending line in normal triple meter and juxtaposes this against the second phrase, which he sets as a long descending line in \term{sesquialtera} (hemiola), written using all blackened noteheads.
Thus the theological and musical are closely linked in both pieces, so that one's knowledge of theology informs understanding of the musical structure, and one's knowledge of music theory and ability to perceive musical-rhetorical devices gives insight into theology of Incarnation, voice, and hearing.

\subsection{Irízar, Carrión, Cáseda: Hearing and Faith}

Next are two settings of the villancico poem, \worktitle{Si los sentidos queja forman del Pan Divino}, by successive chapelmasters at Segovia Cathedral in the later seventeenth century.
The poem presents a contest of the senses, to be judged by the merits in relationship to faith, similar to that in Pedro Calderón de la Barca's Corpus Christi play \worktitle{En nuevo palacio del Retiro} of 1634.
Hearing is given the first prize because only through believing in what is heard, and not through the other senses, can one rightly perceive Christ's presence in the Eucharist.
Music is used as the paradigmatic example of hearing.
Irízar's festival setting evokes the contest through polychoral dialogue and perhaps evoking the keyboard genre of \term{batalla}.
Carrión's continuo song, by contrast, invites a more personal reflection on the nature of sensation.

Zaragoza composer José de Cáseda's \worktitle{Qué música divina}, a villancico for Eucharistic devotion, intersects both with the metamusical conceits of the pieces by Padilla and Cererols and with the discourse on sensation in the Irízar and Cáseda villancicos.
The central conceit is of Christ in his Passion as a \term{vihuela}, applying early allegorical traditions of the \term{cithara} and \term{lira} to a distinctly Spanish instrument.
The music played on this instrument is \quoted{not for the senses}; it \quoted{elevates the senses} and \quoted{confounds the mind's powers}.
If it could be heard it would sound \quoted{false}---dissonant, out of tune, or \term{musica ficta}. 
Similar to Cererols evoking divine consonance through earthly dissonance, Cáseda appears to emply deliberate solecisms like the parallel fifths on the word \quoted{tuneful}.
He evokes the seven-course vihuela in several ways through the vocal texture, most notably through the evocation of strumming at the end of the \term{estribillo}.
Thus the ensemble, in this case a chorus of New Spanish Conceptionist nuns whose names are preserved in the parts, in a sense \emph{becomes} a \term{vihuela}, embodying an instrument while presenting that instrument as a symbol of Christ's body.
This piece demonstrates a strain of villancico composition quite far removed from the popularizing, folkloric types of villancicos that have become better known.
Instead this is an exercise in contemplative devotion worked out through musical craft that is both ingenious and affectively moving.

\subsection{Padilla and Salazar: Singing in Christ's Stable}

The last two pieces in this edition return to the stable in Bethlehem to join humans and angels in the music of Christ's Incarnation.
Padilla's piece for the newly consecrate Puebla Cathedral calls a colorful host of characters \quoted{to the most blessed stable} to sing and dance for the baby Jesus.
This \term{ensaladilla} is a potpourri of different song and dance styles, probably referencing pre-existing music known to the audience.
A group of shepherds sing something called the \quoted{New Trojan} to the music of \quoted{tempered panpipes}.
A buffoon mule-skinner's mule barges into the stable in search of straw and the befuddled candy vendor tries to excuse himself before the child he obsequiously calls \quoted{Sir Baby} while struggling to control his mule---a struggle evoked through disorderly rhythm.
Next a group of \quoted{mountain folk} whose language marks them as agricultural laborers dances a gentle \quoted{Papalotillo}, the name of which is derived from a Nahuatl word, and which may be meant to represent Indians.

The final section of the piece is a complete, self-contained \term{negrilla} or \soCalled{black villancico}.
In this subgenre of villancicos, ensembles of Spaniards and Spaniard-descended \term{criollos} like Padilla's chapel in Puebla Cathedral represented caricatures of African music-making in pseudo-African dialect Spanish.
In the midst of this section the black characters are suddenly joined by a chorus of angels in singing the \term{Gloria}---but the blacks sing in ternary meter while the angels sing in duple.
Padilla's \quoted{little salad} tosses together characters from different racial and economic strata to present an idealistic vision of the whole colonial society united around the body of Christ.
While this piece and other \soCalled{ethnic villancicos} have much to teach about how Spanish elites perceived their relationships to the other groups under their control, performers should consider seriously how it might be possible to present such a piece today in an ethically responsible manner.%
  \begin{Footnote}
  \autocites{Baker:EthnicVC}{Baker:PerformancePostColonial}. 
  For further on the relationship between Spanish representations of Africans and their actual situation, see \autocites{Molinero:Negros}{Lipski:AfroHispanic}{Fromont:DancingKingCongo}.
  \end{Footnote}

The last piece in the collection is a typical representation of angelic music at Christmas, by Antonio de Salazar, who became chapelmaster of Mexico City Cathedral.
This delicate villancico, with its lilting rhythms, is from the same convent collection as the Cáseda piece, in Puebla, the \quoted{city of the angels}.
Salazar uses imitative counterpoint to represent the angelic chorus coming down to earth, much as Cererols did in his celestial fugue. 
This villancico provides an excellent example of a truly typical villancico, and in particular represents well the affects of wonder and joy that this music was meant to cultivate at the Christmas feast.

\section{Select Bibliography}

\subsection{Studies of Villancicos}
\nocite{Rubio:Forma}
\nocite{Laird:VC}
\nocite{Torrente:PhD}
\nocite{Tenorio:SorJuana}
\nocite{CaberoPueyo:PhD}
\nocite{Illari:Polychoral}
\nocite{Knighton-Torrente:VCs}
\nocite{Cashner:Cards}
\printbibliography[heading=none,filter=villancico-studies]

\subsection{Musical Editions of Villancicos}
\nocite{Cererols:MEM-VC}
\nocite{Stevenson:Christmas}
\nocite{Ruimonte:Parnaso}
\nocite{Padilla:Tello}
\nocite{Ezquerro:MME55}
\nocite{RuizSamaniego:MME63}
\nocite{Ezquerro:MME59}
\nocite{Ezquerro:MME65}
\nocite{Fernandez:Cancionero}
\nocite{Torrejon:VCs}
\printbibliography[heading=none,filter=villancico-editions]

\subsection{Catalogs of Villancico Poetry Imprints}
\nocite{BNE:VCs17C}
\nocite{BNE:VCs18C}
\nocite{UK:VCs}
\nocite{US:VCs}
\printbibliography[heading=none,filter=villancico-imprint-catalogs]

%*******************************
\section{Editorial Policies}

\paragraph{Sources}

The sources for the each poem and its musical setting are listed in the critical notes.
The music is preserved in individual manuscript performing parts in looseleaf sets or bound partbooks.
For the villancico by Miguel de Irízar, the composer's draft score also survives.

The poetic editions are based on the lyrics of the musical settings.
They have been annotated and sometimes corrected in comparision with the surviving poetry imprints of the same or related villancico poems.
The poems are generally anonymous, but are often adapted from existing poems or poetic types.

The manuscript parts were practical tools for performers.
They all bear evidence of frequent use over a long period: they are soiled along the creases in the paper where performers held them up, and they include the names of multiple performers, corrections in different hands, and added accidentals and barlines.
Aspects of notation that seem ambiguous to a modern scholar were not, apparently, impediments to effective performance from the originals.

The goal of this edition, in keeping with the nature of these sources, is to enable the practical performance and study of these villancicos.
I have endeavored to present the music in a clear and consistent modern notation.


\paragraph{Orthography}
For the sake of consistency and intelligibility, spelling and punctuation have been modernized and standardized.
Doing so means losing some information about the local pronunciation of Spanish in different parts of the empire.
The phonetic orthography in the performing parts demonstrates that \mentioned{ci} and \mentioned{ce} were pronounced like \mentioned{si} and \mentioned{se} in New Spain and Catalonia.
But standardization has the greater benefit of enabling performers to present the works in a way that will be most intelligible to their audiences.

The exception to this rule is in the \term{negrilla} of Padilla's \term{Al establo más dichoso}, in which it seemed more responsible to present the pseudo-African dialect in its original orthography.
Possible equivalents in proper Spanish are given in the footnotes.

\paragraph{Editorial Text}
Italic text in the lyrics indicate editorial text underlay, usually where there are lyrical repeat signs in the sources.
Other textual additions by the editor, such as standardized section headings, are enclosed in square brackets.

\paragraph{Pitch Level}
All pieces are transcribed at their original notated pitch level.
The preparatory staves at the beginning of each piece show the original clefs, signatures, and the first note.
While the  general pitch level was likely lower than today, some of the pieces may have been performed at a lower transposition.
I will be happy to make different transpositions available for performers on request.

The critical notes use the International Pitch Notation system for denoting octaves, wherein C\octave{4} is \soCalled{middle C}.



\paragraph{Accidentals}
Accidental placement in the partbooks is contextual and sometimes ambiguous to a modern reader.
The original notation has no \na{} symbol, using B\sh{} and E\sh{} instead.
In a few cases, indicated in the critical notes, scribes use a \sh{} sign as a cautionary accidental, warning the singer \emph{not} to apply a sharp according to \term{musica ficta} conventions.%
  \autocites{Harran:Cautionary1}{Harran:Cautionary2}

The edition presents the pitches with their accidental inflections when unambiguously specified in the source.
According to modern convention, these accidentals are valid until the next barline.
Thus repeated accidentals in the source are omitted if the modern convention does not require them; and in a few cases accidentals are added where modern notation demands.

Editorial suggestions for other accidentals, mostly according to \term{musica ficta} conventions, are set above the staff.

\paragraph{Meter, Rhythm, Tempo}
The original music was written in mensural notation, with few barlines in the performing parts. 
The duple-meter sections of these pieces were written in mensural \meterC{} meter, which the seventeenth-century Spanish theorists Pedro Cerone and Andrés Lorente refer to as \term{tiempo menor imperfecto} or \term{compasillo} (henceforth, C).%
  \autocites[537]{Cerone:Melopeo}[156, 210]{Lorente:Porque}
In this meter, the \term{compás} or \term{tactus} consisted of a semibreve divided into two minims.%
  \autocites{GonzalezValle:MusicaTexto}{GonzalezValle:CompasCabezon}

The other common meter for seventeenth-century villancicos was notated with the symbol \meterCZ{}, a cursive CZ.
Lorente says that this is a shorthand for \meterCThree{} or \meterCThreeTwo{}, where these signs all indicate \term{tiempo menor de proporción menor}, a proportion of C meter.%
  \autocite[165]{Lorente:Porque}
The \term{compás} consists of one perfect semibreve which is divided into three minims, instead of the two minims of C.

In the sources, deviations from the normal ternary groups are indicated through coloration. 
When noteheads in CZ meter are blackened, this often indicates a shift to \term{sesquialtera} or hemiola.
In \term{sesquialtera} two groups of three minims are exchanged for three groups of two minims; two perfect semibreves are subsituted by three imperfect semibreves.

The edition presents the rhythms of the sources according to modern conventions of meter and barlines.
The original meter signs are shown in preparatory staves or above the staff.
The original note values have not been reduced.
Mensural coloration is indicated with short rectangular brackets above the staff.
Ligatures are indicated by long rectangular brackets.
Beaming is unchanged.

The music has been notated in the modern meters of \meter{2}{2} for C time and \meter{6}{2} for CZ time.
In some cases in ternary meter an extra bar of \meter{3}{2} has to be added, and wherever possible this has been done at the beginning or end of a section.

These barlines and meters are only conveniences to make the pieces easiliy legible and performable.
The chief advantage is that this system allows many of the original durations to be transcribed without reduction into tied notes.

Spanish composers like Miguel de Irízar did use barlines when they notated in score format.
Irízar writes two \term{compases} per bar in both CZ and C meters, and also fits odd \term{compases} within these groups.
Cerone advises students who wish to write out a score from parts to write barlines every two \term{compases}.%
  \begin{Footnote}
  \autocite[745]{Cerone:Melopeo}.
  This practice would suggest using \meter{4}{2} for C meter in the modern edition, but in my opinion this makes the duple-meter sections too difficult to read.
  \end{Footnote}

Performers should not take the barlines as guides to accentuation, nor should they assume the music lacks natural accentuation.
Mensural meters do not necessarily imply any particular pattern of rhythmic accentuation.
The music is often composed to create clearly articulated \soCalled{downbeats}; but many pieces also make much use of syncopation and irregular patterns.

Regarding tempo, the theoretical proportion of $3:2$ between CZ and C does not necessarily imply a tempo relationship.
In actual practice with these pieces, a \emph{tempo} relationship of $3:1$ often makes more musical sense.
Thus two \term{compases} of CZ would have about the same duration as one \term{compás} of C.
In this edition, then, one bar of \meter{6}{2} would approximately equal one bar of \meter{2}{2}.
Thus performers can maintain a consistent pulse throughout a single piece.
The edition includes markers of these tempo relationships at each change of meter, but these are only suggestions.

\paragraph{Instrumentation}

These villancicos are scored for an ensemble of voices with instrumental bass or continuo groups.
Vocal ensembles varied in size, from one-to-a-part groups to much larger polychoral forces.
Most of the pieces also feature prominent solo parts, particularly in the \term{coplas}.

The bottom voice parts in these pieces are mean to be performed on instruments. 
They are only provided with short incipits of the lyrics to orient the performer, and in several cases instruments like \term{bajón} (dulcian, bass curtal) or organ are specified.
Though there is need for more research into the specific instrumentation of Spanish musical ensembles, it is plausible that the bass line was performed in most cases by a continuo group of \term{bajón} doubled by harp, organ, and possibly other instruments like the \term{vihuela} or perhaps guitar.%
  \footnote{On the changing instrumentation in one Spanish institution, see \autocite{Torrente:PhD}.}
In pieces without figured bass, continuo players likely improvised harmonies to match the other voices.

The upper voices could have been doubled on \term{chirimías} (shawms) or a variety of other instruments like \term{sacabuches} (sackbuts) according to local resources and suited to the occasion.
There is as yet no clear evidence, though, that church ensembles of seventeenth-century Spain included percussion instruments.%
  \begin{Footnote}
  For a critique of exoticizing practices in recent villancico performances,  see \autocite{Baker:PerformancePostColonial}.
  \end{Footnote}

\section{Abbreviations}

\begin{tabu} to \textwidth{lZ}
A. & Alto, \term{Altus}\\
Ac. & \term{Acompañamiento}: Accompaniment, \term{basso continuo}\\
B. & \term{Bajo}, \term{Bassus}\\
CN & See critical notes\\
DRAE & Real Academía Española, \worktitle{Diccionario de la lengua española}, 23rd ed.\\
Leg. & \term{Legajo} (archival folder)\\
OED & \worktitle{Oxford English Dictionary}\\
S. & Soprano; Used in part listings (e.g., \quoted{SSAT}) for the highest voice part, usually for \term{Tiple}, to avoid confusion between \quoted{Ti.} for Tiple and \quoted{T.} for Tenor\\
T. & Tenor\\
Ti. & \term{Tiple}: Treble, boy soprano\\
Ti. I-1 & Chorus 1, First Tiple\\
\end{tabu}

\subsection{Archival Sigla}

\begin{tabu} to \textwidth{llZ}
\textsc{Siglum} & \textsc{Country} & \textsc{Archive}\\
E-Bbc & Spain & Barcelona, Biblioteca de Catalunya\\
E-CAN &  & Canet de Mar, Arxiu Parròquia de Sant Pere i Sant Pau de Canet de Mar, Bisbat de Girona, Fons Capella de Música\\
E-Mn & &  Madrid, Biblioteca Nacional de España\\
E-SE & & Segovia, Catedral, Archivo Capitular\\
MEX-Pc & Mexico &  Puebla, Catedral, Archivo Capitular\\
MEX-Mcen & Mexico & Mexico City, CENIDIM (Centro Nacional de Investigación, Documentación e Información Musical Carlos Chávez)\\
GB-Lbl & United Kingdom & London, British Library\\
\end{tabu}


\section{Acknowledgments}

This edition was prepared with free and open-source software, using the \LaTeX{} document-preparation system (based on the work of Donald Knuth and Leslie Lamport) for the text and Lilypond (version 2.18.2) for the music, on the Debian GNU/Linux operating system, release 8. 
I am grateful to the hundreds of volunteers who have built and maintained these systems.

The text typeface is EB Garamond, designed by Georg Duffner, based on 1592 type specimens by Claude Garamont.
The Spanish CZ metrical symbol used in the scores was traced manually in Inkscape from a villancico manuscript by Miguel de Irízar.

I am grateful to the following people and institutions for access to the primary sources on which these editions are based: 
the capitular archive of the Cathedral of Puebla de los Ángeles (P. Francisco Vázquez, rector; the Illmo. Sr. Carlos Ordaz, \term{canónigo archivista});
CENIDIM, the Mexican national center for music research in Mexico City;
the Biblioteca de Catalunya in Barcelona;
the parochial archive of the Church of Saints Peter and Paul, Canet de Mar, in the Archdiocese of Girona, Barcelona province;
the capitular archive of Segovia Cathedral (P. Bonifacio Bartolomé);
the Biblioteca Nacional in Madrid, and
the British Library in London.

Travel for archival research in Mexico and Spain was funded by 
\begin{anonymize}
a Jacob K. Javits Fellowship from the United States Department of Education, 
a Pre-Dissertation Research Fellowship from the Council for European Studies at Columbia University, 
a Eugene K. Wolf Research Travel Grant from the American Musicological Society, 
and grants from the Department of Music and the Center for Latin American Studies at the University of Chicago.
Other research funding was provided by a Dissertation Completion Fellowship from the American Council of Learned Societies and the Mellon Foundation.

I thank my doctoral advisor, Robert Kendrick, and the other members of my dissertation committee, Anne Walters Robertson, Martha Feldman, Frederick de Armas, and María Gembero-Ustárroz.
I am also grateful to the following people for specific help with the poetry and music in this edition:
Stephen Black,
Anita Damjanovic, 
Miguel Martínez, 
Gustavo Mauleón Rodríguez,
James Nemiroff, and
Martha Tenorio.
My wife Ann makes all of this possible.
\end{anonymize}
