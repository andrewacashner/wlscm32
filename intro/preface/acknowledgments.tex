% Acknowledgments
Acknowledgments are due first to Janette Tilley, editor-in-chief of the Web
Library of Seventeenth-Century Music; Kimberly Hieb, associate editor; and the
anonymous reviewers, all of whom greatly improved the clarity and rigor of the
edition. 

I am grateful to the following people and institutions for access to the 
primary sources on which these editions are based: 
the capitular archive of the Cathedral of Puebla de los Ángeles (P. Francisco 
Vázquez, rector; the Illmo. Sr. Carlos Ordaz, \term{canónigo archivista});
CENIDIM, the Mexican national center for music research in Mexico City;
the Biblioteca de Catalunya in Barcelona;
the parochial archive of the Church of Saints Peter and Paul, Canet de Mar, in 
the Archdiocese of Girona, Barcelona province;
the capitular archive of Segovia Cathedral (P. Bonifacio Bartolomé);
the Biblioteca Nacional in Madrid; and
the British Library in London.

Travel for archival research in Mexico and Spain was funded by 
a Jacob K. Javits Fellowship from the United States Department of Education, 
a Pre-Dissertation Research Fellowship from the Council for European Studies at 
Columbia University, 
a Eugene K. Wolf Research Travel Grant from the American Musicological Society, 
and grants from the Department of Music and the Center for Latin American 
Studies at the University of Chicago.
Other research funding was provided by a Dissertation Completion Fellowship 
from the American Council of Learned Societies and the Mellon Foundation.

I am grateful to my doctoral advisor, Robert Kendrick, along with the other
members of my dissertation committee---Anne Walters Robertson, Martha Feldman,
Frederick de Armas, and María Gembero-Ustárroz---and my master's advisor, Mary
Frandsen.
This project has benefited from the help and critical insights of 
Stephen Black,
Ireri Chávez-Bárcenas,
Anita Damjanovic, 
Dianne Goldmann,
Miguel Martínez, 
Gustavo Mauleón Rodríguez,
James Nemiroff, 
Martha Tenorio,
Álvaro Torrente, 
John Swadley,
and especially Devin Burke.
My wife Ann makes all of this possible and my children Ben and Joy make it
worthwhile.

\edsection{Colophon}

This edition was produced with free and open-source software on Fedora and
Debian Linux operating systems.
The text was typeset with the \LaTeX{} document-preparation system, based on 
the work of Donald Knuth and Leslie Lamport.
The music was typeset with Lilypond, version 2.19, created by Han-Wen Nienhuys
and Jan Nieuwenhuizen.
I am grateful to the hundreds of volunteers who build and maintain these 
systems, and to those who provided specific help in programming this document.

The text typeface is EB Garamond, designed by Georg Duffner, based on 1592 type 
specimens by Claude Garamont.
The Spanish CZ metrical symbol used in the scores was traced manually by the
editor in Inkscape from a villancico manuscript by Miguel de Irízar.

The complete source code for this edition is in a Git repository at
\url{http://www.andrewcashner.com/villancicos/}.
Please use that site to report errors or request alternate versions, such as
transposed scores, keyboard reductions, or instrumental performing parts.

\edsection{Version History}

\begin{tabular}{rlp{0.75\textwidth}}
    1.1 & December 2018 & Error corrections, terminology changes (see below) \\
   1.0 & December 2017 & Expanded bibliography and minor corrections\\
    0.1 & November 2017 & Initial test release\\
\end{tabular}

\edsubsection{Version 1.1 Revisions}

Ellen Hargis of the Newberry Consort identified several errors and colleagues
raised concerns about some terminology.

\begin{enumerate}
    \item Introduction
        \begin{itemize}
            \item Surname \term{Gutiérrez de Padilla} used instead of just
                \term{Padilla} (The manuscripts refer to the composer
                as \quoted{Maestro Padilla}, but the full surname is now
                standard in Spanish musicology)
            \item Term \quoted{Hispanic} replaced with alternatives (The
                term was meant as \quoted{in the cultural world of
                Spanish-speaking people, not just in Spain}, but some
                readers perceived it as an ethnic or racial term)
        \end{itemize}
    \item  Gutiérrez de Padilla, \wtitle{Al establo más dichoso} (poem)
        \begin{itemize}
            \item Accent: Papalotillo, l. 13
            \item Translation: Papalotillo, l. 88: \term{Labrador} is farm
                owner, not laborer (Thank you to Prof. Pablo Sierra of the
                University of Rochester for this insight.)
        \end{itemize}
    \item Gutiérrez de Padilla, \wtitle{Al establo más dichoso} (score)
        \begin{itemize}
            \item Pitches: B. II, m. 172
            \item Lyrics: Ch. I, mm. 18, 20; T. II, m. 93; Ti. I, mm. 220--223
        \end{itemize}
    \item Salazar, \wtitle{Angélicos coros} (score), 
        \begin{itemize}
            \item Pitches: Ti. II, m. 42; T. II, mm. 61, 63 
        \end{itemize}
\end{enumerate}
\endinput
