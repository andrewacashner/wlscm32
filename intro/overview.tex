\edpart{Introduction}

At the height of the Spanish Empire in the seventeenth century,
villancicos were one of the most widespread forms of religious expression and a 
central part of social life.%
  \begin{Footnote}
      For an introduction to the genre, see the entries for \quoted{villancico} in 
      \worktitle{Grove Music Online} and the \worktitle{Diccionario de la música 
      española e hispanoamericana}; \autocites{Laird:VC}{Knighton-Torrente:VCs};
      and the other studies in the bibliography below.
      The bibliography also lists other modern editions of seventeenth-century
      villancicos.  
  \end{Footnote}
These poems in vernacular languages (usually Spanish or Portuguese) celebrated 
common Catholic beliefs, popular customs, and modes of devotion through 
metaphorical conceits both earthy and ingenious.
A few villancico poets are known by name from published collections, like
Vicente Sánchez (author of \worktitle{Qué música divina}, in this edition),
Manuel de León Marchante, and Sor Juana Inés de la Cruz; but most villancico
poems are anonymous, and in many cases were probably adapted and reworked by the
composers themselves.
In the form of printed leaflets of the poetry (\term{pliegos sueltos}) and in 
manuscript performing parts of the music they were passed from hand to hand 
across oceans in a global network of affiliated musicians and members of the 
literate elite.%
  \begin{Footnote}
      \Autocite{BNE:VCs17C}; see the bibliography for other catalogs of the
      largest collections of these leaflets or chapbooks and editions of
      villancico poetry.
  \end{Footnote}

The musical settings of these poems occupied the energies of every major 
chapelmaster and his ensemble on all the highest feast days of the year.
Though the genre originated as a form of courtly entertainment, by the 
beginning of the seventeenth century most villancicos were sacred both in their 
themes and in the venues and occasions of their performance.
Sacred villancicos were often composed in sets of eight so that they could be 
interspersed after or in place of the Responsory chants of the Matins liturgy, 
especially at Christmas and Corpus Christi.%
    \begin{Footnote}
        On the villancico's sister genre, the Responsory, see
        \autocite{Goldman:Responsory}.
    \end{Footnote}
Villancicos were also performed in church for Mass, Forty Hours' Devotion and 
other Eucharistic adoration, and outside of church in Corpus Christi 
processions and mystery plays (\term{autos sacramentales}).

Villancicos encompass a wide range of formal structures, but most feature an 
\term{estribillo}, a motet-like section for the full ensemble, and 
\term{coplas}, strophic verses often scored for soloists or a reduced ensemble.
Many of the musical sources specify that the \term{estribillo} was repeated
after the \term{coplas}.

Many villancicos of this period were scored for large polychoral ensembles of 
voices, probably doubled by loud wind instruments and supported by organ, harp, 
and other plucked strings.
Others, sometimes called \term{tonos divinos}, are scored for a more intimate 
texture of a few voices with continuo.
The first type are public, festival pieces; the second offer more private, 
contemplative experiences.

Though many of the sources have been lost, the surviving repertoire is vast and 
rich, encompassing a broad range of Hispanic devotional life.
With a variety of subgenres and topics from the comic to the learned, there was 
a villancico for everyone and nearly every occasion.

This edition offers performers and scholars a coherent set of newly edited 
villancicos, drawn from archives in Spain and Mexico, that share a common theme 
of \quoted{singing about singing}.
The vernacular poetry of these \soCalled{metamusical} pieces represents the act 
of hearing and making music.
The musical settings of these poems, then, become discourses about music, 
through the medium of music itself.%
  \begin{Footnote}
      These villancicos formed the primary corpus of study for my dissertation, 
      \autocite{Cashner:PhD}.
      This edition corrects and supersedes the musical editions in the dissertation.
  \end{Footnote}
The thematic organization makes these pieces ideal both for concert programming 
and for scholarly study.

\section*{Interpretive Themes}

The villancicos in this collection present a complex and multilayered discourse 
of music and theology.
My monograph now in preparation develops these interpretations fully by reading 
these villancicos in the context of contemporary theological literature.
The following brief interpretive notes may serve as an initial guide to 
understanding these fascinating pieces.

Common tropes run through these pieces and demonstrate traditions of poetry and 
music about musical performance, or about music as an abstract concept.
The pieces embody a Neoplatonic theology of music in which listeners are 
invited to listen for echoes of higher music in the imperfect earthly music 
they hear.%
    \Autocite[108--132]{Cashner:PhD}


\subsection*{Cererols and Gutiérrez de Padilla: Christ as Singer and Song}

The first two pieces, \worktitle{Suspended, cielos, vuestro dulce canto} by Joan
Cererols and \worktitle{Voces, las de la capilla} by Juan Gutiérrez de Padilla,
are villancicos for Christmas that represent the newborn Christ as both singer
and song.%
    \Autocite[133--284]{Cashner:PhD}
Extending on an exegetical tradition going back to Bernard of Clairvaux and 
Augustine, these pieces celebrate Christ as the \term{Verbum infans}, the Word 
of God made flesh (John~1:1), but as an infant, unable to speak a word. 
Since Christ in his incarnate body is himself the Word, these pieces portray 
his inarticulate cries as a form of music, as the tuning note---the  
\quoted{sign of \term{A (la, mi, re)}}---to which the music of a renewed 
creation will be harmonized.
Joan Cererols, monk and chapelmaster of the choir school of the Benedictine
Abbey of Our Lady of Montserrat near Barcelona, has his ensemble bid the
heavenly spheres themselves to cease their imperfect music and \quoted{listen to
the newest consonance} of Christ.
Juan Gutiérrez de Padilla, priest and chapelmaster of Puebla Cathedral in New
Spain (Mexico), presents Christ as the heir of the musician-king David, the
masterpiece of the divine chapelmaster who puts God and Man in harmony through
his Incarnate body, which is made known through his infant voice.%
    \Autocite{Cashner:ChristSingerRSA}

The composers match the musical conceits of the poetry with the appropriate 
musical devices, such as the eight-voice fugue in strict counterpoint in late
sixteenth-century style (like that of Palestrina, Morales, and Guerrero) that
Cererols creates for \term{contrapunto celestial}.
Cererols even illustrates the idea of Christ as the \term{cantus firmus} for a 
restored heavenly music by developing the motive of a descending stepwise fifth 
throughout the estribillo, culminating in a concluding section in the style of 
a cantus-firmus motet.
Cererols illustrates \quoted{the newest consonance} by setting the word 
\mentioned{consonancia} on a prominent, unprepared, and repeated dissonance.
By drawing listeners' attention to the imperfection of worldly music through 
this ironic symbol, Cererols points them in Neoplatonic fashion past the 
sounding music, to listen for an unhearable, higher music of Christ the divine 
Word.

Padilla also creates musical devices to illustrate the arcane music-theoretical 
and theological references in his poem.
He quotes the plainchant \term{tonus peregrinus} on the words \term{peregrino 
tono}.
Padilla has half his ensemble exhort the other half to \quoted{keep count with 
what is sung} while they are literally counting their rests.
Then the other chorus sings about \quoted{awaiting the thirty-three} (a 
reference to Christ's Passion) with exactly thirty-three notes.
Both choirs join together to represent the celebration of heavenly beings, 
humans, and beasts singing in the manger, in the style of a madrigal, scored 
for voices \quoted{three by three, two by two, one by one}.
Padilla's \term{estribillo} climaxes with an epitome of Catholic belief about
Christ's Incarnation, \quoted{Everything in man is to ascend, and everything in
God is to descend}.
Padilla sets the first line to an ascending line in normal triple meter and 
juxtaposes this against the second phrase, which he sets as a long descending 
line in \gloss{sesquialtera}{hemiola}, written using all blackened noteheads.
Thus the theological and musical are closely linked in both pieces, so that 
one's knowledge of theology informs understanding of the musical structure, and 
one's knowledge of music theory and ability to perceive musical-rhetorical 
devices gives insight into theological conceptions of Incarnation, voice, and
hearing.

\subsection*{Irízar, Carrión, Cáseda: Hearing and Faith}

Next are two settings of the villancico poem, \worktitle{Si los sentidos queja 
forman del Pan Divino}, by successive chapelmasters at Segovia Cathedral in the 
later seventeenth century.%
    \Autocite[285--338]{Cashner:PhD}
The poem, attributed to poet Vicente Sánchez of Zaragoza, presents a contest 
of the senses, to be judged by their merits in relationship to faith.
The contest is similar to that in Pedro Calderón de la Barca's Corpus Christi 
play \worktitle{En nuevo palacio del Retiro} of 1634.%
    \Autocites{Calderon:Retiro}{Cashner:FaithHearingUCSB}[52--107]{Cashner:PhD}

The coplas articulate commonly held beliefs about the powers of the senses and 
emphasize that the mystery of the Eucharist confounds every sense.
Hearing is given the first prize because only through believing in what is 
heard, and not through the other senses, can one rightly perceive Christ's 
presence in the sacrament.
The poem uses music to exemplify the sense of hearing.

Irízar's festival setting evokes the contest through polychoral dialogue and 
perhaps evoking the keyboard genre of \term{batalla}.%
    \Autocite{Sutton:IberianBatalla}
Carrión's continuo song, by contrast, invites a more personal reflection on the 
nature of sensation.

José de Cáseda's setting of \worktitle{Qué música divina} intersects 
both with the metamusical conceits of the pieces by Padilla and Cererols and 
with the discourse on sensation in the Irízar and Carrión villancicos.
The central conceit of this piece for Eucharistic devotion presents Christ in 
his Passion as a \term{vihuela}.%
    \Autocite[375--405]{Cashner:PhD}
The poem applies patristic allegorical traditions of the \term{cithara} and 
\term{lira} to a distinctly Spanish instrument.
The music played on this instrument is \quoted{not for the senses}; it 
\quoted{elevates the senses} and \quoted{confounds the mind's powers}.
If it could be heard it would sound \quoted{false}---dissonant, out of tune, or 
as \term{musica ficta}. 
Similar to Cererols evoking divine consonance through earthly dissonance, 
Cáseda appears to employ deliberate solecisms to represent this \quoted{false}
music, like the parallel fifths and direct octaves on the 
word \quoted{tuneful}, or the cadential patterns on \quoted{various cadences}
that tempt singers to add accidentals in the wrong places.
He evokes the seven-course vihuela in several ways through the vocal texture, 
most notably through the strumming texture at the end of the estribillo.

Though Cáseda lived and worked in Zaragoza, this piece survives only in the
collection of the Conceptionist Convento de la Santísima Trinidad in Puebla.%
    \Autocite{Favila:PhD}
In performing this piece, the chorus of nuns whose names are preserved in the 
parts would in a sense \emph{become} a vihuela, embodying an instrument while 
presenting that instrument as a symbol of Christ's body.

This piece demonstrates a strain of villancico composition quite removed from 
the popularizing, folkloric types of villancicos that have become better known,
such as the pieces that follow in this edition.  
Instead this is an exercise in contemplative devotion worked out through a
musical craft that emphasizes both ingenuity and affective power.

\subsection*{Padilla and Salazar: Singing in Christ's Stable}

The last two pieces in this edition return to the stable in Bethlehem to unite 
humans and angels in the music of Christ's Incarnation.
In a piece for the new cathedral of Puebla (consecrated three years earlier in 
1649), Padilla and his ensemble call up a colorful host of characters 
\quoted{to the most blessed stable} to sing and dance for the baby Jesus.%
    \Autocites{Cashner:PadillaRhythm-AMS}[406--467]{Cashner:PhD}
This \term{ensaladilla} is a potpourri of different song and dance styles, 
probably referencing pre-existing music known to the hearers.
A group of shepherds sing something called the \quoted{New Trojan} to the music 
of \quoted{tempered panpipes}.
A buffoon mule-skinner's mule barges into the stable in search of straw; the 
befuddled candy vendor tries to excuse himself before the Christ-child, whom he 
obsequiously calls \quoted{Sir Baby}, while struggling to control his mule---a 
struggle evoked through disorderly rhythm.
Next a group of \quoted{mountain folk}, whose language marks them as 
agricultural laborers, dances a gentle \quoted{Papalotillo}.
This name is derived from a Nahuatl word, and these characters may be meant to 
represent indigenous people.

The final section of the piece is a complete, self-contained \term{negrilla} or 
\soCalled{black villancico}, a common subgenre.
Here Padilla's ensemble of Spaniards and Spaniard-descended \term{criollos} 
presented caricatures of Africans and their music, in a mocking imitation of 
African speech.
In the midst of a pseudo-African dance, the black characters are suddenly 
joined by a chorus of angels in singing \term{Gloria}---but the blacks sing in 
ternary meter while the angels sing in duple.

Padilla's \quoted{little salad} tosses together characters from different 
racial and economic strata to present an idealistic vision of the whole 
colonial society united around the body of Christ.
This composer, who was both a university-educated priest and a slave-owner,
brings the highest and lowest beings together in harmony while paradoxically
keeping them apart, reflecting a Neoplatonic concept of the social hierarchy.%
    \Autocite{Mauleon:PadillaPalafox}
While this piece and other \soCalled{ethnic villancicos} have much to teach 
about how Spanish elites perceived their relationships to the other groups 
under their control, performers should consider seriously how it might be 
possible to present such a piece today in an ethically responsible manner.%
  \begin{Footnote}
      \Autocites{Baker:EthnicVC}{Baker:PerformancePostColonial}. 
      For further on the relationship between Spanish representations of Africans 
      and their actual situation, see \autocites{Molinero:Negros}{Lipski:AfroHispanic}
      {Fromont:DancingKingCongo}.
  \end{Footnote}

The last piece in the collection is a typical representation of angelic music 
at Christmas, by Antonio de Salazar, who became chapelmaster of Mexico City 
Cathedral.%
    \Autocite[29--34]{Cashner:PhD}
This delicate villancico, with its lilting rhythms, is from the same convent 
collection as the Cáseda piece, in Puebla de los Ángeles, the original
\quoted{city of the angels}.  
Salazar uses imitative counterpoint to represent the angelic chorus coming down 
to earth, much as Cererols did in his celestial fugue. 
The convent sisters who sang this piece at Christmas embodied and incited the 
affects of wonder and joy that theologians considered most characteristic of 
this feast.


