\section*{Editorial Policies}

\paragraph{Sources}
The sources for the each poem and its musical setting are listed in the 
critical notes.
The music is preserved in individual manuscript performing parts in looseleaf 
sets or bound partbooks.
For the villancico by Miguel de Irízar, the composer's draft score also 
survives.

The texts and translations are based on the poetic text in the musical settings.
They have been annotated and sometimes corrected in comparision with the 
surviving poetry imprints of the same or related villancico poems.
The poems are generally anonymous, but are often adapted from existing poems or 
poetic types.

The manuscript parts were practical tools for performers.
They all bear evidence of frequent use over a long period: they are soiled 
along the creases in the paper where performers held them up, and they include 
the names of multiple performers, corrections in different hands, and added 
accidentals and barlines.
Aspects of notation that seem ambiguous to a modern scholar were not, 
apparently, impediments to effective performance from the originals.
The goal of this edition, in keeping with the nature of its sources, is to 
enable the practical performance and study of these villancicos through a clear
and consistent notation.


\paragraph{Orthography}
Spelling and punctuation have been modernized and standardized.
Though in doing this some information about historic local pronunciation is
lost, a standard orthography allows performers to present the works in a way
that will be most intelligible to their audiences.%
\begin{Footnote}
    The phonetic orthography in the performing parts does suggest that
    \mentioned{ci} and \mentioned{ce} were pronounced like \mentioned{si} and
    \mentioned{se} in New Spain and Catalonia, rather than with the TH sound in
    modern peninsular Spanish (as in \mentioned{thick}).
\end{Footnote}
The exception to this rule is in the \term{negrilla} of Padilla's \term{Al 
establo más dichoso}, in which it seemed more responsible to present the 
pseudo-African dialect in its original orthography.
Possible equivalents in proper Spanish are given in the footnotes.

\paragraph{Voice and Instruments}
The original names for voices and instruments have been preserved. 
\mentioned{Tiple} refers to a treble singer, usually a boy.
Several terms are used for continuo parts, such as \term{Acompañamiento},
\term{General}, or \term{Guión}.

The edition preserves indications of solo and instrumental parts when they
appear in the original.
Original figured bass is preserved, but continuo realizations are left to the
discretion and creativity of the performer.
Separate instrumental parts and realized keyboard parts are available on request
from the editor.

\paragraph{Editorial Text}
Italic text indicates editorial underlay, usually where there are signs
(\MSrepeat{}) in the sources that specify that the preceding text should
be repeated.
Other textual additions by the editor, such as standardized section headings, 
are enclosed in square brackets.

\paragraph{Pitch Level}
All pieces are transcribed at their original notated pitch level.
The preparatory staves at the beginning of each piece show the original clefs, 
signatures, and the first note.

\paragraph{Accidentals}
Accidental placement in the partbooks is contextual and sometimes ambiguous to 
a modern reader.
The original notation has no \na{} symbol, using B\sh{} and E\sh{} instead.
In a few cases, indicated in the critical notes, scribes use a \sh{} sign as a 
cautionary accidental.
One common use was to warn the singer \emph{not} to apply a sharp according 
to \term{musica ficta} conventions.%
  \autocites{Harran:Cautionary1}{Harran:Cautionary2}

The edition presents the pitches with their accidental inflections when 
unambiguously specified in at least one source.
According to modern convention, these accidentals are valid until the next 
barline.
Thus repeated accidentals in the source are omitted if the modern convention 
does not require them; and in a few cases accidentals are added where modern 
notation demands.
Editorial suggestions for other accidentals, mostly according to \term{musica 
ficta} conventions, are set above the staff.

\paragraph{Repeats}
Some of the sources indicate repeated sections barlines with dots (like modern
repeats), or by giving the incipit of the music and text to be repeated; often
there is also a \term{signum congruentiae} at the point of repetition or a
textual note.
In most cases, the estribillo was reprised after the last copla was sung (more
like a psalm antiphon than a \quoted{refrain} as the term might imply).
Some pieces call for a reprise after each copla or after certain groups of
coplas.
In many sources, the repeat of the estribillo is not specified, and it is
possible that it was not always reprised, especially as villancicos became
longer and more complex.%
    \Autocite{Torrente:Estribillo}

This edition uses modern repeat barlines for short repeated sections and
indications of \quoted{D.C. al Fine} or \quoted{D.S. al Fine}. % XXX continue




\paragraph{Rhythm, Meter, Tempo}
The original music was written in mensural notation, with few barlines in the 
performing parts.%
\begin{Footnote}
    Spanish composers like Miguel de Irízar did use barlines when they notated in 
    score format.
    Irízar writes two \term{compases} per bar in both triple and duple meters,
    occasionally squeezing in a third \term{compás} if there was an odd number
    of groups.
    Cerone advises students who wish to write out a score from parts to write 
    barlines every two \term{compases}; \textcite[745]{Cerone:Melopeo}.
\end{Footnote}
The duple-meter sections of these pieces were written in mensural \meterC{} 
meter, which the seventeenth-century Spanish theorists Pedro Cerone and Andrés 
Lorente refer to as \term{tiempo menor imperfecto} or \term{compasillo}.%
  \autocites[537]{Cerone:Melopeo}[156, 210]{Lorente:Porque}
In this meter, the \term{compás} or \term{tactus} consisted of a semibreve 
divided into two minims.%
  \autocites{GonzalezValle:MusicaTexto}{GonzalezValle:CompasCabezon}

The other common meter for seventeenth-century villancicos was notated with the 
symbol \meterCZorig{}, a cursive \meterCZ{}.
Lorente says that this is a shorthand for \meterCThreeTwo{} or \meterCThree{},
where these signs all indicate \term{tiempo menor de proporción menor}, a 
proportion of C meter.%
  \autocite[165]{Lorente:Porque}
The \term{compás} consists of one perfect semibreve which is divided into three 
minims, instead of the two minims of C.

In the sources, deviations from the normal ternary groups are indicated through 
coloration. 
When noteheads in CZ meter are blackened, this often indicates a shift to 
\term{sesquialtera} or hemiola.
In \term{sesquialtera} two groups of three minims are exchanged for three 
groups of two minims; and three imperfect semibreves take the place of two
perfect semibreves.

The edition presents the rhythms of the sources according to modern conventions 
of meter and barlines.
The music has been notated in the \meterC{} for duple meter and \meterCThree{}
for triple meter.
The original meter signs are shown in preparatory staves or above the staff.
The original note values have not been reduced.
Mensural coloration is indicated with short rectangular brackets above the 
staff.
Ligatures are indicated by long rectangular brackets.
Beaming is unchanged.

Regarding tempo, the theoretical $3:2$ proportion of minims between
\meterCThreeTwo{} and \meterC{} does not necessarily imply the same proportion of
\emph{tempo}.  
In actual practice, a $3:1$ tempo relationship often makes more musical sense,
so that three minims in triple meter together take the same amount of time as
one minim in duple meter.
Thus two \term{compases} of CZ would have about the same duration as one 
\term{compás} of C.

\section*{Performance Suggestions}

\paragraph{Spanish Pronunciation}
Spanish-speaking ensembles should feel free to pronounce the Spanish according
to their own accent.
Other ensembles should work with local native speakers and experts whenever
possible to shape their pronunciation and understanding, so that they can
perform these pieces in a way that Spanish-speaking audience members will
understand and recognize as a part of their own cultural heritage.

\paragraph{Instrumentation and Voicing}
These villancicos are scored for an ensemble of voices with instrumental bass 
or continuo groups.
Vocal ensembles varied in size, from one-to-a-part groups to much larger 
polychoral forces.
Most of the pieces also feature prominent solo parts, particularly in the 
\term{coplas}.

The lowest voice parts in these pieces are meant to be performed on instruments. 
They are only provided with short incipits of the text to orient the 
performer, and in several cases instruments like \gloss{bajón}{dulcian, bass 
curtal} or organ are specified.
Though there is need for more research into the specific instrumentation of 
Spanish musical ensembles, it is plausible that the bass line was performed in 
most cases by a continuo group of \term{bajón} doubled by harp, organ, and 
possibly other instruments like the \term{vihuela de mano}.%
\begin{Footnote}
    On the changing instrumentation in one Spanish institution, see 
    \autocite{Torrente:PhD}.
\end{Footnote}
In pieces without figured bass, continuo players---which could include any
polyphonic instruments like keyboard or plucked strings---likely improvised
harmonies to match the other voices.

The upper voices could have been doubled on \term{bajoncillos},
\gloss{chirimías}{shawms}, \gloss{sacabuches}{sackbuts}, and other instruments
according to local resources and suited to the occasion.
There is as yet no clear evidence, though, that church ensembles of 
seventeenth-century Spain or Spanish America included percussion instruments.%
\begin{Footnote}
  For a critique of exoticizing practices in recent villancico performances,  
  see \autocite{Baker:PerformancePostColonial}.
\end{Footnote}

Ensembles should not be deterred by the lack of early instruments or by by vocal
ranges outside their resources.
It would be entirely within the spirit of the performing traditions that these
sources represent, for a school or community chorus to substitute modern
instruments for their historic relatives.

At a minimum, it is appropriate to use a contrapuntal instrument like keyboard
for the continuo and a melodic instrument like the historic \term{bajón} or
modern bassoon or cello for the instrumental bass lines.
If more instruments are available, add other instruments to the continuo
section like historic \term{bajón}, \term{vihuela}, Spanish harp, and organ;
or modern alternatives like bassoon, cello, modern harp, guitar, digital
organ (with a good sample of an 8$'$ flue stop), or piano.
Double vocal parts freely with historic or modern reeds, winds, and brass, or
possibly bowed strings.

If possible, use soloists or a reduced ensemble for the first chorus in
polychoral pieces, and for the coplas.
A chorus of more modest ability, such as a high school choir, could be paired
with more advanced soloists, such as college students or adult community
members.
Make sure there is at least one singing voice per chorus to present the text;
the other voices could be played instrumentally.

Instrumental parts and continuo realizations are available from the editor upon
request.

\paragraph{Pitch Level}
One of the features of seventeeth-century Spanish choral music that is most
surprising to those more familiar with other repertoires is how high the vocal
ranges are.
Many of the Tiple (treble) parts have tessituras above \pitch{F}{5}; and none of
the pieces have texted bass parts, these parts being played instrumentally
instead.
Either Spanish ensembles performed these pieces at a lower pitch level than
notated (because of a lower general pitch, or through transposition), or Spain
cultivated a lost art of angelically high singing.
Modern ensembles should sing the pieces at a pitch level or transposition that
works for them.

Transposed scores are available from the editor upon request.

\paragraph{Rhythm, Meter, and Tempo}
The meters and barlines in the edition are only conveniences to make the pieces
plainly legible and performable.
Performers should not take the barlines as guides to accentuation, nor should 
they assume the music lacks natural accentuation.
Mensural meters do not necessarily imply any particular pattern of rhythmic 
accentuation.
Most of the time poetic declamation should be the primary guide for pacing
and emphasis.
In other cases, when a set style of dance or song seems to be evoked, a regular
rhythmic pattern may win out over poetic nuances.
Regarding such rhythmic patterns, it should be noted that no one has yet
provided conclusive evidence for the presence of African or American indigenous
rhythms in villancicos.

The sign \meterC{} indicates a duple meter that should be felt \quoted{in
two}.
The sign \meterCThree{} indicates a ternary meter that should be felt
\quoted{in one}.
Often triple meter is syncopated or altered by hemiola (also called
\term{sesquialtera}) to create patterns of accentuation that differ from the
normal ternary groupings indicated by the barlines.

In most cases, it seems appropriate to maintain a tempo relationship of three
minims in \meterCThree{} to one minim in \meterC{}.
By maintaining this tempo relationship it is usually possible to maintain a
consistent pulse throughout the whole piece.
A resting heart rate of about sixty beats per minute generally makes a good
tempo, such that in \meterC{}, $\musMinim{} = 60$; while in \meterCThree,
$\musSemibreveDotted = 60$ and $\musMinim{} = 180$.


\paragraph{Authenticity and Flexibility}
In the editor's opinion, an authentic performance of a seventeenth-century
villancico would be one that is meaningful to present-day performers and their
audience, but which also opens a window to experiencing what made the piece
meaningful to its original performers and hearers.
Performers should seek out the distinctive character and significance of each
piece, but should also feel free to adapt the pieces to suit their own resources
and social context.
It would be better to have a spirited, respectful, musically sensitive
performance with modern instruments than to have no performance at all because
historic instruments were not available.

The sources for this edition are performing parts that, on the one hand,
were used as practical tools for performance in a particular place, and, on the
other hand, represent traditions of performance that cannot be completely fixed
in place or time.
In other words, even within one institution, such as the Conceptionist convent
in Puebla from which come the parts for Salazar's \worktitle{Angélicos coros}
and Cáseda's \worktitle{Qué música divina}, these parts were used and reused
possibly over generations. 
In some cases, later performers made corrections, added barlines, sewed in new
lines of lyrics or even new music to replace certain strophes.
There is no single way that these pieces were performed throughout their terms
of service as part of the local repertoire.

Moreover, these pieces represent single instances of a repertoire that
circulated around the globe. 
José de Cáseda lived in Zaragoza and set a poem by a composer from his same
region, Vicente Sánchez; but his setting is only known from the surviving parts
in the Puebla convent.
The spelling in those parts reflects New Spanish, not Zaragozan pronunciation
(e.g., \mentioned{consonancias} is spelled \mentioned{consonansias} in the
Puebla parts, even though the final C would probably have been pronounced like an
English TH sound in Zaragoza).
The piece may have been rearranged or adapted for female ensemble from a lost
original with different scoring.
On some occasions, a particular sister may have fallen ill and her vocal line
may have been played instrumentally.

The starting point for considering modern performance of these pieces, then, is
that historic performers made these pieces their own and performed them in a way
that fit their local needs in terms of personnel, instrumentation, rehearsal
time, acoustic space, and other factors; and in a way that was intelligible and
meaningful to them and to their hearers.

\paragraph{Ethical Responsibility}
While some amount of adaptation seems appropriate for this repertoire,
performers are urged never to lose site of the religious, social, and political
contexts of these pieces in their early modern origins.
These pieces cannot be cleanly separated from the social values of the colonial
era that this music both reflected and reinforced.
A piece like Juan Gutiérrez de Padilla's \worktitle{Al establo más dichoso}
bears the imprint of imperial Spain's racial hierarchy: it is documented that
the composer himself owned an Angolan slave, and the representation of
\quoted{Angolans} in the piece caricatures their bodies and voices as deformed
and deficient, even as it perhaps strives to present them in a sympathetic light
as offering devotion to Christ and joining with the angelic chorus.

It would be ethically irresponsibly to perform such a piece merely as an exotic
curiosity, or worse, as though it were a twenty-first century celebration of
ethnic diversity. 
Indeed, performers, scholars, and community members ought to engage in serious
discussions about what performing a piece might mean in a contemporary context.
In the right setting, such as a community workshop with appropriate
opportunities for critique, response, and discussion, the piece might be used 
effectively to raise issues of tremendous contemporary relevance; but in the
wrong context the piece could actually perpetuate the negative racial
stereotypes that are built into it.


