\section*{Editorial Policies}

\paragraph{Sources}

The sources for the each poem and its musical setting are listed in the 
critical notes.
The music is preserved in individual manuscript performing parts in looseleaf 
sets or bound partbooks.
For the villancico by Miguel de Irízar, the composer's draft score also 
survives.

The poetic editions are based on the lyrics of the musical settings.
They have been annotated and sometimes corrected in comparision with the 
surviving poetry imprints of the same or related villancico poems.
The poems are generally anonymous, but are often adapted from existing poems or 
poetic types.

The manuscript parts were practical tools for performers.
They all bear evidence of frequent use over a long period: they are soiled 
along the creases in the paper where performers held them up, and they include 
the names of multiple performers, corrections in different hands, and added 
accidentals and barlines.
Aspects of notation that seem ambiguous to a modern scholar were not, 
apparently, impediments to effective performance from the originals.

The goal of this edition, in keeping with the nature of these sources, is to 
enable the practical performance and study of these villancicos.
I have endeavored to present the music in a clear and consistent modern 
notation.


\paragraph{Orthography}
Spelling and punctuation have been modernized and standardized.
Though some information about historic local pronunciation is lost, a standard 
orthography allows performers to present the works in a way that will be most 
intelligible to their audiences.%
  \begin{Footnote}
      The phonetic orthography in the performing parts does suggest that 
      \mentioned{ci} and \mentioned{ce} were pronounced like \mentioned{si} and 
      \mentioned{se} in New Spain and Catalonia.
  \end{Footnote}

The exception to this rule is in the \term{negrilla} of Padilla's \term{Al 
establo más dichoso}, in which it seemed more responsible to present the 
pseudo-African dialect in its original orthography.
Possible equivalents in proper Spanish are given in the footnotes.

\paragraph{Editorial Text}
Italic text in the lyrics indicate editorial text underlay, usually where there 
are lyrical repeat signs in the sources.
Other textual additions by the editor, such as standardized section headings, 
are enclosed in square brackets.

\paragraph{Pitch Level}
All pieces are transcribed at their original notated pitch level.
The preparatory staves at the beginning of each piece show the original clefs, 
signatures, and the first note.
While the  general pitch level was likely lower than today, some of the pieces 
may have been performed at a lower transposition.
I will be happy to make different transpositions available for performers on 
request.

The critical notes use the International Pitch Notation system for denoting 
octaves, wherein C\octave{4} is \soCalled{middle C}.



\paragraph{Accidentals}
Accidental placement in the partbooks is contextual and sometimes ambiguous to 
a modern reader.
The original notation has no \na{} symbol, using B\sh{} and E\sh{} instead.
In a few cases, indicated in the critical notes, scribes use a \sh{} sign as a 
cautionary accidental, warning the singer \emph{not} to apply a sharp according 
to \term{musica ficta} conventions.%
  \autocites{Harran:Cautionary1}{Harran:Cautionary2}

The edition presents the pitches with their accidental inflections when 
unambiguously specified in the source.
According to modern convention, these accidentals are valid until the next 
barline.
Thus repeated accidentals in the source are omitted if the modern convention 
does not require them; and in a few cases accidentals are added where modern 
notation demands.

Editorial suggestions for other accidentals, mostly according to \term{musica 
ficta} conventions, are set above the staff.

\paragraph{Meter, Rhythm, Tempo}
The original music was written in mensural notation, with few barlines in the 
performing parts. 
The duple-meter sections of these pieces were written in mensural \meterC{} 
meter, which the seventeenth-century Spanish theorists Pedro Cerone and Andrés 
Lorente refer to as \term{tiempo menor imperfecto} or \term{compasillo} 
(henceforth, C).%
  \autocites[537]{Cerone:Melopeo}[156, 210]{Lorente:Porque}
In this meter, the \term{compás} or \term{tactus} consisted of a semibreve 
divided into two minims.%
  \autocites{GonzalezValle:MusicaTexto}{GonzalezValle:CompasCabezon}

The other common meter for seventeenth-century villancicos was notated with the 
symbol \meterCZorig{}, a cursive \meterCZ{}.
Lorente says that this is a shorthand for \meterCThree{} or \meterCThreeTwo{}, 
where these signs all indicate \term{tiempo menor de proporción menor}, a 
proportion of C meter.%
  \autocite[165]{Lorente:Porque}
The \term{compás} consists of one perfect semibreve which is divided into three 
minims, instead of the two minims of C.

In the sources, deviations from the normal ternary groups are indicated through 
coloration. 
When noteheads in CZ meter are blackened, this often indicates a shift to 
\term{sesquialtera} or hemiola.
In \term{sesquialtera} two groups of three minims are exchanged for three 
groups of two minims; two perfect semibreves are subsituted by three imperfect 
semibreves.

The edition presents the rhythms of the sources according to modern conventions 
of meter and barlines.
The original meter signs are shown in preparatory staves or above the staff.
The original note values have not been reduced.
Mensural coloration is indicated with short rectangular brackets above the 
staff.
Ligatures are indicated by long rectangular brackets.
Beaming is unchanged.

The music has been notated in the modern meters of \meter{2}{2} for C time and 
\meter{6}{2} for CZ time.
In some cases in ternary meter an extra bar of \meter{3}{2} has to be added, 
and wherever possible this has been done at the beginning or end of a section.

These barlines and meters are only conveniences to make the pieces plainly 
legible and performable.
The chief advantage is that this system allows many of the original durations 
to be transcribed without reduction into tied notes.

Spanish composers like Miguel de Irízar did use barlines when they notated in 
score format.
Irízar writes two \term{compases} per bar in both CZ and C meters, and also 
fits odd \term{compases} within these groups.
Cerone advises students who wish to write out a score from parts to write 
barlines every two \term{compases}.%
  \begin{Footnote}
      \autocite[745]{Cerone:Melopeo}.
      This practice would suggest using \meter{4}{2} for C meter in the modern 
      edition, but in my opinion this makes the duple-meter sections too difficult to 
      read.
  \end{Footnote}

Performers should not take the barlines as guides to accentuation, nor should 
they assume the music lacks natural accentuation.
Mensural meters do not necessarily imply any particular pattern of rhythmic 
accentuation.
The music is often composed to create clearly articulated \soCalled{downbeats}; 
but many pieces also make much use of syncopation and irregular patterns.

Regarding tempo, the theoretical proportion of $3:2$ between CZ and C does not 
necessarily imply a tempo relationship.
In actual practice with these pieces, a \emph{tempo} relationship of $3:1$ 
often makes more musical sense.
Thus two \term{compases} of CZ would have about the same duration as one 
\term{compás} of C.
In this edition, then, one bar of \meter{6}{2} would approximately equal one 
bar of \meter{2}{2}.
Thus performers can maintain a consistent pulse throughout a single piece.
The edition includes markers of these tempo relationships at each change of 
meter, but these are only suggestions.

\paragraph{Instrumentation}

These villancicos are scored for an ensemble of voices with instrumental bass 
or continuo groups.
Vocal ensembles varied in size, from one-to-a-part groups to much larger 
polychoral forces.
Most of the pieces also feature prominent solo parts, particularly in the 
\term{coplas}.

The bottom voice parts in these pieces are mean to be performed on instruments. 
They are only provided with short incipits of the lyrics to orient the 
performer, and in several cases instruments like \term{bajón} (dulcian, bass 
curtal) or organ are specified.
Though there is need for more research into the specific instrumentation of 
Spanish musical ensembles, it is plausible that the bass line was performed in 
most cases by a continuo group of \term{bajón} doubled by harp, organ, and 
possibly other instruments like the \term{vihuela} or perhaps guitar.%
  \footnote{On the changing instrumentation in one Spanish institution, see 
\autocite{Torrente:PhD}.}
In pieces without figured bass, continuo players likely improvised harmonies to 
match the other voices.

The upper voices could have been doubled on \term{chirimías} (shawms) or a 
variety of other instruments like \term{sacabuches} (sackbuts) according to 
local resources and suited to the occasion.
There is as yet no clear evidence, though, that church ensembles of 
seventeenth-century Spain included percussion instruments.%
  \begin{Footnote}
      For a critique of exoticizing practices in recent villancico performances,  
      see \autocite{Baker:PerformancePostColonial}.
  \end{Footnote}

I believe that musicians without access to early instruments could render 
creditable performances on modern ones.
A school or community mixed chorus could certainly perform these pieces with 
the instrumental bass parts played on modern bassoon or cello.
Continuo harmonies could be played on organ (with a soft 8$'$ flue), modern 
harp, or classical guitar; in several pieces this could be left out altogether.
If a wind ensemble is available, the other vocal parts could be doubled with 
modern oboes, bassoons, and trombones.


