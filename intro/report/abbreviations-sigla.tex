\edsection{Abbreviations}

\begin{multicols}{2}
    \setlength{\parindent}{0pt}
    \begin{tabu} to \linewidth{lZ}
        A. & Alto, \term{Altus}\\
        Ac. & \term{Acompañamiento}: Accompaniment, \term{basso continuo}\\
        B. & \term{Bajo}, \term{Bassus}\\
        Ch. & Chorus\\
        CN & See critical notes\\
        Corr. & Editorial correction\\
        \worktitle{DRAE} & Real Academía Española, \worktitle{Diccionario de la
        lengua española}, 23rd ed.\\
        Imprints & Reading based on consensus of extant poetry imprints\\
    \end{tabu}

    \begin{tabu} to \linewidth{lZ}
        Leg. & \term{Legajo} (archival folder)\\
        \worktitle{OED} & \worktitle{Oxford English Dictionary Online} (accessed 2017)\\
        S. & Soprano; Used in part listings for highest voice, avoids
        confusion between Tiple and Tenor (e.g., \quoted{SSAT})\\
        Sugg. & Editorial suggestion\\
        T. & Tenor\\
        Ti. & \term{Tiple}: Treble, boy soprano\\
        Ti. I-1 & Chorus 1, First Tiple\\
    \end{tabu}
\end{multicols}


\edsubsection{Pitch and Octave Designations}

This edition uses upper- and lowercase pitch names together with prime symbols
to indicate specific pitches.
These designations map onto Helmholtz octave numbers as follows:

\begin{center}
\begin{tabular}{ll@{\hspace{2em}}ll}
    \octaveprimes\pitch{C}{1} & \octavenumbers\pitch{C}{1} & 
    \octaveprimes\pitch{C}{4} & \octavenumbers\pitch{C}{4} \\

    \octaveprimes\pitch{C}{2} & \octavenumbers\pitch{C}{2} & 
    \octaveprimes\pitch{C}{5} & \octavenumbers\pitch{C}{5} \\

    \octaveprimes\pitch{C}{3} & \octavenumbers\pitch{C}{3} &
    \octaveprimes\pitch{C}{6} & \octavenumbers\pitch{C}{6} \\
\end{tabular}
\end{center}




\edsubsection{Archival Sigla}

\begin{tabu} to \textwidth{llZ}
    \textsc{Siglum} & \textsc{Country} & \textsc{Archive}\\
    \siglum{E-Bbc} & Spain & Barcelona, Biblioteca de Catalunya\\
    \siglum{E-CAN} &  & Canet de Mar, Arxiu Parròquia de Sant Pere i Sant Pau de Canet de 
    Mar, Bisbat de Girona, Fons Capella de Música\\
    \siglum{E-Mn} & &  Madrid, Biblioteca Nacional de España\\
    \siglum{E-SE} & & Segovia, Catedral, Archivo Capitular\\
    \siglum{MEX-Pc} & Mexico &  Puebla, Catedral, Archivo Capitular\\
    \siglum{MEX-Mcen} & Mexico & 
    Mexico City, CENIDIM \foreignlanguage{spanish}{%
        (Centro Nacional de Investigación, Documentación e Información Musical
        Carlos Chávez)%
    }\\
    \siglum{GB-Lbl} & United Kingdom & London, British Library\\
\end{tabu}





