\edsection{Abbreviations}

\begin{tabu} to \textwidth{lZ}
    A. & Alto, \term{Altus}\\
    Ac. & \term{Acompañamiento}: Accompaniment, \term{basso continuo}\\
    B. & \term{Bajo}, \term{Bassus}\\
    Ch. & Chorus\\
    CN & See critical notes\\
    Corr. & Editorial correction\\
    \worktitle{DRAE} & Real Academía Española, \worktitle{Diccionario de la lengua española}, 
    23rd ed.\\
    Imprints & Reading based on consensus of extant poetry imprints\\
    Leg. & \term{Legajo} (archival folder)\\
    \worktitle{OED} & \worktitle{Oxford English Dictionary Online} (accessed 2017)\\
    S. & Soprano; Used in part listings (e.g., \quoted{SSAT}) for the highest voice 
    part, usually for \term{Tiple}, to avoid confusion between \quoted{Ti.} for 
    Tiple and \quoted{T.} for Tenor\\
    Sugg. & Editorial suggestion\\
    T. & Tenor\\
    Ti. & \term{Tiple}: Treble, boy soprano\\
    Ti. I-1 & Chorus 1, First Tiple\\
\end{tabu}


\edsubsection{Pitch and Octave Designations}

This edition uses upper- and lowercase pitch names together with prime symbols
to indicate specific pitches.
These designations map onto Helmholtz octave numbers as follows:

\octavetable

\edsubsection{Archival Sigla}

\begin{tabu} to \textwidth{llZ}
    \textsc{Siglum} & \textsc{Country} & \textsc{Archive}\\
    \siglum{E-Bbc} & Spain & Barcelona, Biblioteca de Catalunya\\
    \siglum{E-CAN} &  & Canet de Mar, Arxiu Parròquia de Sant Pere i Sant Pau de Canet de 
    Mar, Bisbat de Girona, Fons Capella de Música\\
    \siglum{E-Mn} & &  Madrid, Biblioteca Nacional de España\\
    \siglum{E-SE} & & Segovia, Catedral, Archivo Capitular\\
    \siglum{MEX-Pc} & Mexico &  Puebla, Catedral, Archivo Capitular\\
    \siglum{MEX-Mcen} & Mexico & 
    Mexico City, CENIDIM \foreignlanguage{spanish}{%
        (Centro Nacional de Investigación, Documentación e Información Musical
        Carlos Chávez)%
    }\\
    \siglum{GB-Lbl} & United Kingdom & London, British Library\\
\end{tabu}





