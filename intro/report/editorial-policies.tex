% Editorial report
\edsection{Editorial Policies}

\edsubsection{Sources}
The sources for the each poem and its musical setting are listed in the 
critical notes.
The music is preserved in individual manuscript performing parts in looseleaf 
sets or bound partbooks.
For the villancico by Miguel de Irízar, the composer's draft score also 
survives.

The texts and translations are based on the poetic text in the musical settings.
They have been annotated and sometimes corrected in comparision with the 
surviving poetry imprints of the same or related villancico poems.
The poems are generally anonymous, but are often adapted from existing poems or 
poetic types.

The manuscript parts were practical tools for performers.
They all bear evidence of frequent use over a long period: they are soiled 
along the creases in the paper where performers held them up, and they include 
the names of multiple performers, corrections in different hands, and added 
accidentals and barlines.
Aspects of notation that seem ambiguous to a modern scholar were not, 
apparently, impediments to effective performance from the originals.
The goal of this edition, in keeping with the nature of its sources, is to 
enable the practical performance and study of these villancicos through a clear
and consistent notation.


\edsubsection{Orthography}
Spelling and punctuation have been modernized and standardized.
Though in doing this some information about historic local pronunciation is
lost, a standard orthography allows performers to present the works in a way
that will be most intelligible to their audiences.%
\begin{Footnote}
    The phonetic orthography in the performing parts does suggest that
    \mentioned{ci} and \mentioned{ce} were pronounced like \mentioned{si} and
    \mentioned{se} in New Spain and Catalonia, rather than with the TH sound in
    modern peninsular Spanish (as in \mentioned{thick}).
\end{Footnote}
The exception to this rule is in the \term{negrilla} of Padilla's \term{Al 
establo más dichoso}, in which it seemed more responsible to present the 
pseudo-African dialect in its original orthography.
Possible equivalents in proper Spanish are given in the footnotes.

\edsubsection{Translations}
The villancico poems in this edition are complex examples of the Spanish
literary technique of \term{conceptismo}, in which the poem is governed by a
central conceit that links two (or more) ideas together in an extended
metaphor.%
    \Autocite{Gaylord:Poetry}
In these poems, music forms one side of the conceit, and a theological concept
like Christ's Incarnation or Passion forms the other side, though this is an
oversimplification.
The wording of the Spanish is deliberately ambiguous so that one can read the
poems concentrating on either or both sides of the metaphor.
This means that it nearly impossible to translate the poetry into English and
preserve the delicate balance of double and sometimes triple meanings.
For Spanish words with two meanings, English equivalents with a similar range of
meaning were chosen; but in other cases multiple alternatives had to be
provided.
The translations are as literal as possible while still conveying at least one
level of the original sense.
In some cases, the meaning of a cryptic phrase only becomes clear when read in
the context of contemporary theological and devotional literature.
Perplexed readers are urged to consult the detailed exegesis of these poems in
the editor's dissertation, as the translations are based on rigorous textual
criticism and historically grounded contextual interpretation.

\edsubsection{Voice and Instruments}
The original names for voices and instruments have been preserved. 
\mentioned{Tiple} refers to a treble singer, usually a boy.
Several terms are used for continuo parts, such as \term{Acompañamiento},
\term{General}, or \term{Guión}.

The edition preserves indications of solo and instrumental parts when they
appear in the original.
Original figured bass is preserved, but continuo realizations are left to the
discretion and creativity of the performer.
Separate instrumental parts and realized keyboard parts are available on request
from the editor.

\edsubsection{Editorial Text}
Italic text indicates editorial underlay, usually where there are signs
(\MSrepeat{}) in the sources that specify that the preceding text should
be repeated.
Other textual additions by the editor, such as standardized section headings, 
are enclosed in square brackets.

\edsubsection{Pitch Level}
All pieces are transcribed at their original notated pitch level.
The preparatory staves at the beginning of each piece show the original clefs, 
signatures, and the first note.

\edsubsection{Accidentals}
Accidental placement in the partbooks is contextual and sometimes ambiguous to 
a modern reader.
The original notation has no \na{} symbol, using B\sh{} and E\sh{} instead.
In a few cases, indicated in the critical notes, scribes use a \sh{} sign as a 
cautionary accidental.
One common use was to warn the singer \emph{not} to apply a sharp according 
to \term{musica ficta} conventions.%
  \autocites{Harran:Cautionary1}{Harran:Cautionary2}

The edition presents the pitches with their accidental inflections when 
unambiguously specified in at least one source.
According to modern convention, these accidentals are valid until the next 
barline.
Thus repeated accidentals in the source are omitted if the modern convention 
does not require them; and in a few cases accidentals are added where modern 
notation demands.
Editorial suggestions for other accidentals, mostly according to \term{musica 
ficta} conventions, are set above the staff.

\edsubsection{Repeats}
Some of the sources indicate repeated sections barlines with dots (like modern
repeats), or by giving the incipit of the music and text to be repeated; often
there is also a \term{signum congruentiae} at the point of repetition or a
textual note.
In most cases, the estribillo was reprised after the last copla was sung (more
like a psalm antiphon than a \quoted{refrain} as the term might imply).
Some pieces call for a reprise after each copla or after certain groups of
coplas.
In many sources, the repeat of the estribillo is not specified, and it is
possible that it was not always reprised, especially as villancicos became
longer and more complex.%
    \Autocite{Torrente:Estribillo}
This edition uses modern repeat barlines for short repeated sections and
indications of \quoted{D.C. al Fine} or \quoted{D.S. al Fine} with the
\musSegno{} symbol, though these Italian texts are not used in the originals.

\edsubsection{Rhythm, Meter, Tempo}
The original music was written in mensural notation, with few barlines in the 
performing parts.%
\begin{Footnote}
    Spanish composers like Miguel de Irízar did use barlines when they notated in 
    score format.
    Irízar writes two \term{compases} per bar in both triple and duple meters,
    occasionally squeezing in a third \term{compás} if there was an odd number
    of groups.
    Cerone advises students who wish to write out a score from parts to write 
    barlines every two \term{compases}; \textcite[745]{Cerone:Melopeo}.
\end{Footnote}
The duple-meter sections of these pieces were written in \meterC{} meter, which
the seventeenth-century Spanish theorists Pedro Cerone and Andrés Lorente refer
to as \term{tiempo menor imperfecto} or \term{compasillo}.%
  \Autocites[537]{Cerone:Melopeo}[156, 210]{Lorente:Porque}
In this meter, the \term{compás} or \term{tactus} consisted of a semibreve 
divided into two minims.%
  \Autocites{GonzalezValle:MusicaTexto}{GonzalezValle:CompasCabezon}

The other common meter for seventeenth-century villancicos was notated with the 
symbol \meterCZorig{}, a cursive \meterCZ{}.
Lorente says that this is a shorthand for \meterCThreeTwo{} or \meterCThree{},
where these signs all indicate \term{tiempo menor de proporción menor}, a 
proportion of \meterC{} meter.%
  \autocite[165]{Lorente:Porque}
The \term{compás} consists of one perfect semibreve which is divided into three 
minims, instead of the two minims of \meterC{}.

In the sources, deviations from the normal ternary groups are indicated through 
coloration. 
When noteheads in \meterCThree{} meter are blackened, this often indicates a
shift to \term{sesquialtera} or hemiola.
In \term{sesquialtera} two groups of three minims are exchanged for three 
groups of two minims; and three imperfect semibreves take the place of two
perfect semibreves.

The edition presents the rhythms of the sources according to modern conventions 
of meter and barlines.
The music has been notated in the \meterC{} for duple meter and \meterCThree{}
for triple meter.
The original meter signs are shown in preparatory staves or above the staff.
The original note values have not been reduced.
Mensural coloration is indicated with short rectangular brackets above the 
staff.
Ligatures are indicated by long rectangular brackets.
Beaming is unchanged.

Regarding tempo, the theoretical $3:2$ proportion of minims between
\meterCThreeTwo{} and \meterC{} does not necessarily imply the same proportion of
\emph{tempo}.  
In actual practice, a $3:1$ tempo relationship often makes more musical sense,
so that three minims in triple meter together take the same amount of time as
one minim in duple meter.
Thus two \term{compases} of CZ would have about the same duration as one 
\term{compás} of C.


