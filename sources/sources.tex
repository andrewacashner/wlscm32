\chapter{Source Images}

\clearpage
\begin{figure}
    \includeSource{Cererols/Cererols-CAN-SI1}
    \caption{%
        Cererols, \worktitle{Suspended, cielos}, \signature{E-CAN}{AU/0116},
        Tiple I-1 performing part, beginning
    }
    \label{Cererols-1}
\end{figure}

\begin{figure}
    \includeSource{Cererols/Cererols-CAN-Ac-Coplas}
    \caption{%
        Cererols, \worktitle{Suspended, cielos}, \signature{E-CAN}{AU/0116},
        \term{Acompañamiento}, beginning of coplas with incipits for lyrical
        text
    }
    \label{Cererols-2}
\end{figure}

\begin{figure}
    \includeSource{Cererols/Cererols-Bbc-SI1}
    \caption{%
        Cererols, \worktitle{Suspended, cielos}, \signature{E-Bbc}{M/765/25}
        (alternate version, previously unattributed), Tiple I-1 performing part,
        beginning, with \term{eco} markings
    }
    \label{Cererols-3}
\end{figure}

\begin{figure}
    \includeSource{Cererols/Cererols-poem-1651-Madrid}
    \caption{%
        \worktitle{Suspended, cielos}, earliest known version, in poetry imprint
        from the Madrid Royal Chapel, Christmas 1651,
        \signature{E-Mn}{R/34199/27}
    }
    \label{Cererols-5}
\end{figure}

\begin{figure}
    \includeSource{Padilla-Voces/Padilla-Voces-TI-intro}
        \caption{%
            Gutiérrez de Padilla, \worktitle{Voces, las de la capilla}, 
            \signature{MEX-Pc}{Leg. 3/3}, Tenor I partbook, \term{introducción}
            and beginning of \term{estribillo}
        }
        \label{Padilla-1}
\end{figure}

\begin{figure}
        \includeSource{Padilla-Voces/Padilla-Voces-AI-copla}
        \caption{%
            Gutiérrez de Padilla, \worktitle{Voces, las de la capilla},
            \signature{MEX-Pc}{Leg. 3/3}, Alto I partbook, end
            of \term{estribillo} with symbolic coloration, and
            \term{copla} 1 with quotation of \term{tonus peregrinus} and
            cautionary sharps 
        }
        \label{Padilla-2}
\end{figure}

\begin{figure}
    \includeSource{Caseda/Caseda-S2-estr-end}
    \caption{%
        Cáseda, \worktitle{Qué música divina}, 
        \signature{MEX-Mcen}{CSG.154}, Tiple 2 performing part, end of
        \term{estribillo}, implicit B flats on \foreign{potensias desmaya}
    }
    \label{Caseda-1}
\end{figure}

\begin{figure}
    \includeSource{Caseda/Caseda-T-estr-end}
    \caption{%
        Cáseda, \worktitle{Qué música divina}, 
        \signature{MEX-Mcen}{CSG.154}, Tenor performing part, end of
        \term{estribillo}, explicit B sharp (natural) on \foreign{potensias
        desmaya}
    }
    \label{Caseda-2}
\end{figure}

\begin{figure}
    \includeSource{Caseda/Caseda-T-copla-above}
    \includeSource{Caseda/Caseda-T-copla-under}
    \caption{
        Cáseda, \worktitle{Qué música divina}, 
        \signature{MEX-Mcen}{CSG.154}, Tenor performing part, 
        replacement \term{copla} sewn over original, and original beneath
    }
    \label{Caseda-3}
\end{figure}

\begin{figure}
    \includeSource{Padilla-Al_establo/Padilla-Al_establo-SI-1}
    \caption{%
        Gutiérrez de Padilla, \worktitle{Al establo más dichoso},
        \signature{MEX-Pc}{Leg. 1/3}, Tiple I-1 partbook, beginning of
        \term{Negrilla}
    }
    \label{Padilla-Establo-1}
\end{figure}

\begin{figure}
    \includeSource{Padilla-Al_establo/Padilla-Al_establo-BI-bajon}
    \caption{%
        Gutiérrez de Padilla, \worktitle{Al establo más dichoso},
        \signature{MEX-Pc}{Leg. 1/3}, Bassus I partbook, with incipit of lyrical
        text and beneath, a note to the performer, implying that both Bassus
        parts were performed on \term{bajón}: \quoted{Before the
        papalotillo speaks the mule-driver with the other \term{bajón}}
    }
    \label{Padilla-Establo-2}
\end{figure}

\begin{figure}
    \includeSource{Padilla-Al_establo/Padilla-Al_establo-SI-Gloria}
    \caption{%
        Gutiérrez de Padilla, \worktitle{Al establo más dichoso},
        \signature{MEX-Pc}{Leg. 1/3}, Tiple I partbook, \term{Negrilla}, with
        corrections to music and text underlay for the polymetrical
        \term{Gloria} \quoted{a 3} 
    }
    \label{Padilla-Establo-3}
\end{figure}

\begin{figure}
    \includeSource{Padilla-Al_establo/Padilla-Al_establo-SII-Gloria}
    \caption{%
        Gutiérrez de Padilla, \worktitle{Al establo más dichoso},
        \signature{MEX-Pc}{Leg. 1/3}, the polymetrical \term{Gloria} in Tiple II
        partbook, including an earlier draft in CZ meter
    }
    \label{Padilla-Establo-4}
\end{figure}


\begin{figure}
    \includeSource{Salazar/Salazar-SI}
    \caption{%
        Salazar, \worktitle{Angélicos coros}, \signature{MEX-Mcen}{CSG.256},
        Tiple I performing part with the names of two performers, sisters in the
        Convento de la Santísima Trinidad, Puebla
    }
    \label{Salazar-1}
\end{figure}
