\chapter{Source Images}
\begin{figure}[ht!]
    \includeSource{Cererols/Cererols-CAN-SI1}
    \def\CererolsCaption{%
        Cererols, \worktitle{Suspended, cielos}, \signature{E-CAN}{AU/0116},
        Tiple I-1 part (image courtesy Diocese of Girona)%
    }
    \caption[\CererolsCaption]{\CererolsCaption\footnotemark}
    \label{Cererols-CAN}
\end{figure}

\footnotetext{%
    The inclusion of these images is deemed to constitute fair use. 
    The original sources are in the public domain, and the images are 
    edited and presented in an interpretive context, 
    in a scholarly edition whose license proscribes commercial use.%
}

\begin{figure}[hb!]
    \includeSource{Cererols/Cererols-poem-1651-Madrid}
    \caption{%
        \worktitle{Suspended, cielos}, poetry imprint of earliest known version, 
        Madrid, Royal Chapel, Christmas 1651, \signature{E-Mn}{R/34199/27} 
        (image courtesy Biblioteca Nacional de España, Madrid)%
    }
    \label{Cererols-Poem}
\end{figure}


 
%*********************

\begin{figure}
    \includeSource{Padilla-Voces/Padilla-Voces-TI-intro}
        \caption{%
            Gutiérrez de Padilla, \worktitle{Voces, las de la capilla}, 
            \signature{MEX-Pc}{Leg. 3/3}, Tenor I partbook, \term{introducción}
            and beginning of \term{estribillo} 
            (microfilm image, courtesy Archdiocese of Puebla)%
        }
        \label{Padilla-Voces} 
\end{figure}

\begin{figure}
    \includeSource{Padilla-Al_establo/Padilla-Al_establo-SI-Gloria}
    \caption{%
        Gutiérrez de Padilla, \worktitle{Al establo más dichoso},
        \signature{MEX-Pc}{Leg. 1/3}, Tiple I partbook, \term{Negrilla}, with
        corrections to music and text underlay for the polymetrical
        \term{Gloria} \quoted{a 3} 
        (microfilm image, courtesy Archdiocese of Puebla)%
    }
    \label{Padilla-Establo-GloriaSI}
\end{figure}

\begin{figure}
    \includeSource{Padilla-Al_establo/Padilla-Al_establo-SII-Gloria}
    \caption{%
        Gutiérrez de Padilla, \worktitle{Al establo más dichoso},
        \signature{MEX-Pc}{Leg. 1/3}, the polymetrical \term{Gloria} in Tiple II
        partbook, including an earlier draft in CZ meter
        (microfilm image, courtesy Archdiocese of Puebla)%
    }
    \label{Padilla-Establo-GloriaSII}
\end{figure}

\begin{figure}
    \includeSource{Salazar/Salazar-SI}
    \caption{%
        Salazar, \worktitle{Angélicos coros}, \signature{MEX-Mcen}{CSG.256},
        Tiple I performing part with the names of two performers, sisters in the
        Convento de la Santísima Trinidad, Puebla
        (photograph by Andrew Cashner, courtesy CENIDIM, Mexico City)%
    }
    \label{Salazar}
\end{figure}

%****************

\begin{figure}
    \includeSource{Caseda/Caseda-T-estr-end}
    \caption{%
        Cáseda, \worktitle{Qué música divina}, 
        \signature{MEX-Mcen}{CSG.154}, Tenor performing part, end of
        \term{estribillo}, with explicit B sharp (natural) on \foreign{potensias
        desmaya}
        (photograph by Andrew Cashner, courtesy CENIDIM, Mexico City)%
    }
    \label{Caseda-Estribillo}
\end{figure}

\begin{figure}
    \includeSource{Caseda/Caseda-T-copla-above}
    \includeSource{Caseda/Caseda-T-copla-under}
    \caption{
        Cáseda, \worktitle{Qué música divina}, 
        \signature{MEX-Mcen}{CSG.154}, Tenor performing part:
        replacement \term{copla} sewn over original music (top),
        and view of the original beneath
        (photographs by Andrew Cashner, courtesy CENIDIM, Mexico City)%
    }
    \label{Caseda-Coplas}
\end{figure}


