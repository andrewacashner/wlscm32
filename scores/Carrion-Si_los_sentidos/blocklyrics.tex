\begin{blocklyrics}
2. Toca el tacto pero hierra,\\
que si en que es pan se equivoca,\\
aunque todo un cielo toca,\\
no toca en cielo ni en tierra,\\
toca misterio y si encierra\\
portentos no oídos,\\
\emph{no se den por sentidos los sentidos.}

3. Que tenga voto no es justo\\
el gusto en este manjar,\\
que el gusto en él no ha de entrar,\\
aunque el manjar entre en gusto,\\
mas si les causa disgusto\\
no ser admitidos,\\
\emph{no se den por sentidos los sentidos.}

4. Si el olfato se le humilla\\
con fe a entenderle la flor\\
le maravilla su olor\\
porque guele a maravilla,\\
mas si para percibilla\\
no llegan rendidos,\\
\emph{no se den por sentidos los sentidos.}

5. Porque a Dios puedan gustar\\
en los puntos sus concentos,\\
todos sus cinco instrumentos\\
la fe los ha de templar,\\
sino los puede ajustar\\
para ser oídos,\\
\emph{no se den por sentidos los sentidos.}
\end{blocklyrics}
