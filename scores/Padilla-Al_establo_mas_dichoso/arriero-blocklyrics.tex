\documentclass{vcscores}
\pagestyle{empty}
\begin{document}
\begin{multicols}{2}
\begin{verse}
2. No se me asuste le digo \\
ni de inocente se ponga, \\
cuando me dicen que sabe \\
lo que su padre no ignora. 

3. Es bueno que de mis mulas, \\
la más lucia y la más gorda \\
me la traiga a este pesebre \\
sin decir esta es mi boca,

4. Y yo sin haber vendido \\
las cargas de mis melcochas, \\
ande en flores y con flores \\
pregonándola a mi costa. 

5. Si arrobar viene a los hombres, \\
paréceme cosa impropia \\
dar principio con mi mula, \\
si no ha de ocupar carroza. 

6. Pero ya he considerado, \\
si mi decir no le enoja, \\
que por la escarcha pretende \\
el aliento de su boca.  

7. Y por vida de Bartolo, \\
que en aquestas y en esotras, \\
cuando por esto la quiera, \\
que aquí se las traiga todas. 

8. Abra esa boca de perlas \\
con que tanto me enamora, \\
y pida que estos serranos \\
no pretenden otra cosa. 

9. Un baile quieren hacerle, \\
que \emph{papalotillo} nombran \\
y como cantemos todos, \\
más que rueden las panochas.
\end{verse}
\end{multicols}
\end{document}
