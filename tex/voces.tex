\documentclass[12pt]{article}
\usepackage{xparse}
\usepackage{ebgaramond}
\usepackage[T1]{fontenc}
\usepackage[utf8]{inputenc}
\usepackage{microtype}
\usepackage[spanish,american]{babel}
\usepackage{csquotes}
\usepackage{geometry}
\usepackage[series={A},noend,nofamiliar,noeledsec,noledgroup]{reledmac}
\usepackage{reledpar}

% Title block at beginning of poem
\NewDocumentEnvironment{poemtitle}{}
  {}
  {\vspace{2em}}


% Set size of verse indent
\setlength{\stanzaindentbase}{1em}

% Wrap every dual-column poem in 'poemtranslation' environment
% This environment must include 'original' and 'translation' environments
% These set up numbering and appropriate languages
\NewDocumentEnvironment{poemtranslation}{}
  {\begin{pairs}}
  {\end{pairs}\Columns}

\NewDocumentEnvironment{original}{}
  {%
    \begin{Leftside}
      \selectlanguage{spanish}
      \raggedright
      \beginnumbering
  }
  {%
      \endnumbering
    \end{Leftside}
  }

\NewDocumentEnvironment{translation}{}
  {%
    \begin{Rightside}
      \raggedright
      \beginnumbering
  }
  {%
      \endnumbering
     \end{Rightside}
   }

% Before each call of \stanza we must \setstanzaindents.
% This code allows this to be done automatically based on the 
% number of lines in the stanza.
% \indentvalues is defined for a hanging indent, and the 
% optional argument specifies the indent for the first line of the stanza.
% The others are all given 0 indent through a loop.
\newcounter{stanzalines}
\NewDocumentCommand{\writestanzaindents}{ o m }{%
  \setcounter{stanzalines}{#2}%
  \addtocounter{stanzalines}{-1}%
  \def\indentvalues{1,#1}
  \loop
    \edef\indentvalues{\indentvalues,0}%
    \addtocounter{stanzalines}{-1}%
    \ifnum\value{stanzalines} > 0
  \repeat
}
    
% Our replacement for \stanza sets up automatic indenting of the 
% first line, and no others. 
% This is based on the mandatory argument, which specifies the 
% number of lines in the stanza.
\NewDocumentCommand{\Stanza}{ m }{%
  \writestanzaindents[1]{#1}%
  \expandafter\setstanzaindents\expandafter{\indentvalues}
  \stanza
}
% If the stanza begins a section with a header, we use \StanzaSection.
% This sets up the indents so that the first line is *not* indented.
% If a header text (the optional second argument) is provided, 
% it is printed in the specified font.
% Normal usage will be like this:
%   In the original text:       \StanzaSection{4}[Estribillo]
%   In the translation of same: \StanzaSection{4}
\NewDocumentCommand{\StanzaHeaderFont}{}{\scshape}
\NewDocumentCommand{\StanzaSection}{m o}{%
  \writestanzaindents[0]{#1}%
  \expandafter\setstanzaindents\expandafter{\indentvalues}
  \IfNoValueTF{#2}
    {\stanza}
    {\stanza[{\StanzaHeaderFont #2}]}
}

% To provide extra vertical space before section headings, 
% we use \SectionBreak at the end of the preceding stanza instead of \&.
% The optional argument allows for a text to be inserted after the 
% stanza, before the added space.
% The text is printed in the specified font.
% For example: last line of poem \SectionBreak[Repeat first section] 
\newlength{\PreSectionSkip}
\setlength{\PreSectionSkip}{1.5em}
\NewDocumentCommand{\SectionBreak}{o}{%
  \IfNoValueTF{#1}
    {\&[\vspace{1em}]}
    {\&[{\PostStanzaFont #1}\vspace{\PreSectionSkip}]}
}

% If we only want to add a note after a stanza (as in, between stanzas 
% rather than StanzaSections), we use \AfterStanza, which sets the text 
% in the specified font and adds a specified amount of space.
\NewDocumentCommand{\PostStanzaFont}{}{\itshape}
\newlength{\AfterStanzaSkip}
\setlength{\AfterStanzaSkip}{0.5\PreSectionSkip}
\NewDocumentCommand{\AfterStanza}{m}{%
  \&[{\PostStanzaFont #1}\vspace{\AfterStanzaSkip}]
}

% This is a shorthand for reledmac's \edtext macro, using just one 
% series of footnotes automatically.
\NewDocumentCommand{\fn}{ m m }{%
  \edtext{#1}{\Afootnote{#2}}%
}

% Semantic markup commands
%  \add for editorial additions in brackets
%  \term for technical or foreign terms in italics
\NewDocumentCommand{\add}{m}{[#1]}
\NewDocumentCommand{\term}{}{\emph}


\begin{document}

\begin{poemtitle}
\section*{\emph{Voces, las de la capilla} (Puebla, 1657)}
Anonymous text from musical setting by Juan Gutiérrez de Padilla, 
  \emph{Navidad del año de 1657}, no.~4 (MEX-Pc: Leg.~3/3)
\end{poemtitle}

\begin{poemtranslation}
\begin{original}

\StanzaSection{6}[\add{Introducción}]
1. Voces, las de la capilla, &
\fn{cuenta}{Pay attention to.} con lo que se canta, &
que es músico el rey, y \fn{nota}{Takes note of.} &
las más leves disonancias &
a lo de Jesús infante &
y a lo de David monarca.
\SectionBreak

\StanzaSection{4}[Respuesta]
Puntos ponen a sus letras &
los siglos de sus hazañas. &
La clave que sobre el hombro &
para el treinta y tres se aguarda.
\SectionBreak

\StanzaSection{6}[\add{Introducción} cont.]
2. Años antes la divisa, &
la destreza en la esperanza, &
por sol comienza una gloria, &
por mi se canta una gracia, &
y a medio compás la noche &
remeda quiebros del alba
\SectionBreak[\add{Respuesta rep.}]

\StanzaSection{15}[\add{Estribillo}]
Y a trechos las distancias &
en uno y otro coro, &
grave, suave y sonoro, &
hombres y brutos y Dios, &
tres a tres y dos a dos, &
uno a uno, &
y aguardan tiempo oportuno, &
quién antes del tiempo fue. &
Por el signo a la mi re, &
puestos los ojos en mi, &
a la voz del padre oí &
cantar por puntos de llanto. &
\hphantom{uno a uno,} ¡O qué canto! &
tan de oír y de admirar, &
tan de admirar y de oír. \&

\Stanza{2}
Todo en el hombre es subir &
y todo en Dios es bajar.
\SectionBreak

\StanzaSection{4}[Coplas]
1. Daba un niño peregrino &
tono al hombre y subió tanto &
que en sustenidos de llanto &
dió octava arriba en un trino. \&

\Stanza{4}
2. Hizo alto en lo divino &
y de la máxima y breve &
composición en que pruebe &
de un hombre y Dios consonancias. \&

\end{original}

\begin{translation}
\StanzaSection{6}
1. Voices, those of the chapel choir &
keep count with what is sung, &
for the king is a musician, and notes &
even the most venial dissonances, &
in the manner of Jesus \fn{the infant prince}{\term{Infante} means both infant and prince.}, &
as in the manner of David the monarch. \&

\StanzaSection{4}
The centuries of his heroic exploits &
are putting notes to his lyrics. &
The \fn{key}{Or clef.} that upon his shoulder &
awaits the thirty-three. \&

\StanzaSection{6}
2. Years before the sign, &
\fn{dexterity in hope}
  {In Golden Age literature \term{destreza} connotes heroic skill in combat, particularly in \term{esgrima} or sworsdmanship. 
Musically, the term suggests virtuosity. 
The whole phrase sounds like a heraldic device (\term{divisa}) or motto, summing up Christ's mission.} &
\fn{with the sun}
  {Here begins a series of musical plays on words: \term{sol} and \term{mi} are solmization syllables with double meanings; \term{gloria} and \term{gracia} probably refer to the songs of Christmas in both history and liturgy like the \term{Gloria in excelsis}.}
  [on \term{sol}] a \textquote{glory} begins, &
upon me [\term{mi}] a \textquote{grace} is sung, &
and at the half-measure, the night &
imitates the trills of the dawn. \&

\StanzaSection{15}
And from afar, the \fn{intervals}
  {Both musical intervals and astronomical distances between planetary spheres.} &
in one choir and then the other, &
solemn, mild, and resonant, &
men, animals, and God, &
three by three and two by two, &
one by one, &
they all await the opportune time, &
the one who was before all time. &
Upon the sign of \term{A (la, mi, re)}, &
with eyes placed on me [\term{mi}] &
at the voice of the Father I heard &
singing in tones of weeping--- &
\hphantom{one by one,} Oh, what a song! &
as much to hear as to admire, &
as much to admire as to hear! \&

\Stanza{2}
Everything in Man is to ascend &
and everything in God is to descend. \&

% COPLAS
\StanzaSection{4}
1. A baby gave a \fn{wandering song}
  {Or \textquote{pilgrim song}, or the musical \term{tonus peregrinus}.} &
to the Man, and ascended so high &
that in \fn{sustained weeping}
  {Musically, \textquote{sharps of weeping}} &
\fn{he went up the eighth \add{day} into the triune.}
  {Musically, \textquote{he went up the octave in a trill.}} \&

\Stanza{4}
2. From \fn{on high}
  {\term{Alto} also denotes the musical voice part.} in divinity, &
\fn{of the greatest and least}
  {A play on the name of very long and short music notes.}, &
he made a composition in which to \fn{prove}{Or \textquote{test}.} &
the consonances of a Man and God. \&
\end{translation}
\end{poemtranslation}

\end{document}