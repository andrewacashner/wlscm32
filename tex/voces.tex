\documentclass[12pt]{article}
\usepackage{ebgaramond}
\usepackage[T1]{fontenc}
\usepackage[utf8]{inputenc}
\usepackage{microtype}
\usepackage[spanish,american]{babel}
\usepackage{csquotes}
\usepackage{geometry}
\usepackage[series={A},noend,nofamiliar,noeledsec,noledgroup]{reledmac}
\usepackage{reledpar}

\newenvironment{poemtranslation}
  {\vspace{2em}
   \begin{pairs}}
  {\end{pairs}
   \Columns}

\newenvironment{original}
  {%
    \selectlanguage{spanish}
    \begin{Leftside}
      \beginnumbering
  }
  {%
      \endnumbering
    \end{Leftside}%
  }
\newenvironment{translation}
  {%
    \begin{Rightside}
      \beginnumbering
  }
  {%
      \endnumbering
     \end{Rightside}%
  }

% \newcommand{\setupindents}{%
%   \setstanzaindents{1,0}
%   \setcounter{stanzaindentsrepetition}{1}%
% }
   
\newcommand{\poemsectionfont}{\scshape}
\newcommand{\stanzasection}[1]{\stanza[{\poemsectionfont #1}]}
\newcommand{\afterstanza}[1]{\&[\emph{#1}]}

% \newcommand{\doafterstanza}[1]{\&[\emph{#1}\vspace{\afterstanzaskip}]\par}

% \newlength{\afterstanzaskip}
% \setlength{\afterstanzaskip}{1em}
% \newcommand{\stanzabreak}[1]{\&[\vspace{\afterstanzaskip}]}
% \newcommand{\sectionbreak}[1]{\&[\vspace{2\afterstanzaskip}]}


\newcommand{\add}[1]{[#1]}
\newcommand{\othersense}{\add}
\newcommand{\term}{\emph}

\newcommand{\fn}[2]{\edtext{#1}{\Afootnote{#2}}}

\begin{document}

\section*{\emph{Voces, las de la capilla} (Puebla, 1657)}
Poet anonymous; Text from musical setting by Juan Gutiérrez de Padilla, 
\emph{Navidad del año de 1657}, no. 4 (MEX-Pc: Leg. 3/3)

\begin{poemtranslation}
\begin{original}
\setstanzaindents{1,1,0,0,0,0,0}
\stanzasection{\add{Introducción} a 6}
1. Voces, las de la capilla,&
\fn{cuenta}{Pay attention to.} con lo que se canta,&
que es músico el rey, y \fn{nota}{Takes note of.}&
las más leves disonancias&
a lo de Jesús infante&
y a lo de David monarca.\&

\setstanzaindents{1,1,0,0,0}
\stanzasection{Respuesta a 3}
Puntos ponen a sus letras&
los siglos de sus hazañas.&
La clave que sobre el hombro&
para el treinta y tres se aguarda.\&

\setstanzaindents{1,1,0,0,0,0,0}
\stanzasection{\add{Introducción} cont.}
2. Años antes la divisa,&
la destreza en la esperanza,&
por sol comienza una gloria,&
por mi se canta una gracia,&
y a medio compás la noche&
remeda quiebros del alba\afterstanza{\add{Respuesta rep.}}

\setstanzaindents{1,1,0,0,0,0,0}
\stanzasection{\add{Estribillo a 6}}
Y a trechos las distancias&
en uno y otro coro,&
grave, suave y sonoro,&
hombres y brutos y Dios,&
tres a tres y dos a dos,&
uno a uno,\&

\setstanzaindents{1,0,0,0,0,0,0,0,0,0}
\stanza
y aguardan tiempo oportuno,&
quién antes del tiempo fue.&
Por el signo a la mi re,&
puestos los ojos en mi,&
a la voz del padre oí&
cantar por puntos de llanto.&
\hphantom{uno a uno,} ¡O qué canto!&
tan de oír y de admirar,&
tan de admirar y de oír.\&

\setstanzaindents{1,1,0}
\stanza
Todo en el hombre es subir&
y todo en Dios es bajar.\&


\setstanzaindents{1,1,0,0,0}
\stanzasection{Coplas a 3}
1. Daba un niño peregrino&
tono al hombre y subió tanto&
que en sustenidos de llanto&
dió octava arriba en un trino.\&

\stanza
2. Hizo alto en lo divino&
y de la máxima y breve&
composición en que pruebe&
de un hombre y Dios consonancias.\afterstanza{\add{Estribillo rep.}}

\end{original}

\begin{translation}
\setstanzaindents{1,1,0,0,0,0,0}
\stanza
1. Voices, those of the chapel choir&
keep count with what is sung,&
for the king is a musician, and notes&
even the most venial dissonances,&
in the manner of Jesus \fn{the infant prince}{\term{Infante} means both infant and prince.},&
as in the manner of David the monarch.\&

\setstanzaindents{1,1,0,0,0}
\stanza
The centuries of his heroic exploits&
are putting notes to his lyrics.&
The \fn{key}{Or clef.} that upon his shoulder&
awaits the thirty-three.\&

\setstanzaindents{1,1,0,0,0,0,0}
\stanza
2. Years before the sign,&
\fn{dexterity in hope}
  {In Golden Age literature \term{destreza} connotes heroic skill in combat, particularly in \term{esgrima} or sworsdmanship. 
Musically, the term suggests virtuosity. 
The whole phrase sounds like a heraldic device (\term{divisa}) or motto, summing up Christ's mission.}&
\fn{with the sun}
  {Here begins a series of musical plays on words: \term{sol} and \term{mi} are solmization syllables with double meanings; \term{gloria} and \term{gracia} probably refer to the songs of Christmas in both history and liturgy like the \term{Gloria in excelsis}.}
  \othersense{on \term{sol}} a \textquote{glory} begins,&
upon me \othersense{\term{mi}} a \textquote{grace} is sung,&
and at the half-measure, the night&
imitates the trills of the dawn.\&

\setstanzaindents{1,1,0,0,0,0}
\stanza
And from afar, the \fn{intervals}
  {Both musical intervals and astronomical distances between planetary spheres.}&
in one choir and then the other,&
solemn, mild, and resonant,&
men, animals, and God,&
three by three and two by two,&
one by one,\&

\setstanzaindents{1,0,0,0,0,0,0,0,0,0}
\stanza
they all await the opportune time,&
the one who was before all time.&
Upon the sign of \term{A (la, mi, re)},&
with eyes placed on me \othersense{\term{mi}}&
at the voice of the Father I heard&
singing in tones of weeping---&
\hphantom{one by one,} Oh, what a song!&
as much to hear as to admire,&
as much to admire as to hear!\&

\setstanzaindents{1,1,0}
\stanza
Everything in Man is to ascend&
and everything in God is to descend.\&

% COPLAS
\setstanzaindents{1,1,0,0,0}
\stanza
1. A baby gave a \fn{wandering song}
  {Or \textquote{pilgrim song}, or the musical \term{tonus peregrinus}.}&
to the Man, and ascended so high&
that in \fn{sustained weeping}
  {Musically, \textquote{sharps of weeping}}&
\fn{he went up the eighth \add{day} into the triune.}
  {Musically, \textquote{he went up the octave in a trill.}}\&

\stanza
2. From \fn{on high}
  {\term{Alto} also denotes the musical voice part.} in divinity,&
\fn{of the greatest and least}
  {A play on the name of very long and short music notes.},&
he made a composition in which to \fn{prove}{Or \textquote{test}.}&
the consonances of a Man and God.\&
\end{translation}
\end{poemtranslation}

\end{document}