\documentclass{article}
\usepackage{ebgaramond}
\usepackage[T1]{fontenc}
\usepackage[utf8]{inputenc}
\usepackage{microtype}
\usepackage[spanish,american]{babel}
\usepackage{csquotes}
\usepackage{reledmac,reledpar}

\newenvironment{poemtranslation}
  {\vspace{2em}
   \begin{pairs}}
  {\end{pairs}
   \Columns}
\newcommand{\setupcolumn}{%
  \beginnumbering
  \setstanzaindents{1,0}
  \setcounter{stanzaindentsrepetition}{1}%
}
\newenvironment{original}
  {\selectlanguage{spanish}
   \begin{Leftside}
   \setupcolumn}
  {\endnumbering
   \end{Leftside}}
\newenvironment{translation}
  {\begin{Rightside}
   \setupcolumn}
  {\endnumbering
   \end{Rightside}}
   
\newcommand{\stanzatitlefont}{\scshape}
\newcommand{\titlestanza}[1]{\stanza[{\stanzatitlefont #1}]}

\newcommand{\fn}[2]{\edtext{#1}{\Afootnote{#2}}}

\begin{document}

\section*{\emph{Voces, las de la capilla} (Puebla, 1657)}
As set by Juan Gutiérrez de Padilla, MEX-Pc: Leg. 2/1

\begin{poemtranslation}
\begin{original}
\titlestanza{Introducción}
Voces, las de la capilla,&
\fn{cuenta}{Pay attention to.} con lo que se canta,&
que es músico el rey,&
y \fn{nota}{Takes note of.} las más leves disonancias,&
a lo de Jesús infante&
y a lo de David monarca.\&

\titlestanza{Respuesta}
Puntos ponen a sus letras&
los siglos de sus hazañas.&
La clave que sobre el hombro&
para el treinta y tres se aguarda.\&
\end{original}

\begin{translation}
\stanza
Voices, those of the chapel choir&
keep count with what is sung,&
for the king is a musician,&
and notes even the most venial dissonances,&
in the manner of Jesus \fn{the infant prince}{\emph{Infante} means both infant and prince.},&
as in the manner of David the monarch.\&

\stanza
The centuries of his heroic exploits&
are putting notes to his lyrics.&
The \fn{key}{Or clef.} that upon his shoulder&
awaits the thirty-three.\&
\end{translation}
\end{poemtranslation}

\end{document}