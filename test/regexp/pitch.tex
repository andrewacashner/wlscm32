\documentclass{article}
\usepackage{semantic-markup}
\usepackage{octave}

\begin{document}

\section{Numbers}

The gamut runs from \pitch{G}{2} to \pitch{E}{5}.
Middle C is \pitch{C}{4}.
Accidentals are like this:
\pitch{A}[\sh]{4}--\pitch{B}[\fl]{5}--\pitch{C}[\na]{0}.

\begin{tabular}{ll}
    Octave & Designation\\
    -1 & \pitch{C}[\sh]{-1}\\
    0 & \pitch{C}[\sh]{0}\\
    1 & \pitch{C}[\sh]{1}\\
    2 & \pitch{C}[\sh]{2}\\
    3 & \pitch{C}[\sh]{3}\\
    4 & \pitch{C}[\sh]{4}\\
    5 & \pitch{C}[\sh]{5}\\
    6 & \pitch{C}[\sh]{6}\\
    7 & \pitch{C}[\sh]{7}\\
\end{tabular}

\section{Primes}

\octaveprimes
       
The gamut runs from \pitch{G}{2} to \pitch{E}{5}.
Middle C is \pitch{C}{4}.
Accidentals are like this:
\pitch{A}[\sh]{4}--\pitch{B}[\fl]{5}--\pitch{C}[\na]{0}.

\begin{tabular}{ll}
    Octave & Designation\\
    -1 & \pitch{C}[\sh]{-1}\\
    0 & \pitch{C}[\sh]{0}\\
    1 & \pitch{C}[\sh]{1}\\
    2 & \pitch{C}[\sh]{2}\\
    3 & \pitch{C}[\sh]{3}\\
    4 & \pitch{C}[\sh]{4}\\
    5 & \pitch{C}[\sh]{5}\\
    6 & \pitch{C}[\sh]{6}\\
    7 & \pitch{C}[\sh]{7}\\
\end{tabular}


\end{document}
