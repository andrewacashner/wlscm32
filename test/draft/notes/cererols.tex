% Cererols critical notes first version (deprecated)

\begin{criticalnotes}
1 & A. I 
  & \term{Bbc} starts coloration on C, matching other Chorus I parts.\\

13 & T. II, third note 
  & \term{Bbc} has F\octave{4}; edition follows \term{CAN} and puts A\octave{4}, matching the contour of the A. II (\measures{13--14}) and of Ti. I-2 and A. I in \measures{12--13}.\\

13--14 & A. II, lyrics 
  & \term{CAN} has \quoted{tened \MSrepeat{} parad}, but the other two voices (Ti. II and T. II) have \quoted{tened} for the final word in this phrase.\\

16--18 & Chorus I, lyrics 
  & \term{Bbc} has \quoted{tened parad parad escuchad} in Ti.~I-1 and A.~I, and \quoted{parad parad parad escuchad} in Ti.~I-2.
  Edition follows \term{CAN}.\\

19--20 & Ti. I.-2 
  & This is one of numerous places with this rising and falling stepwise motive, in which the usage of \term{ficta} accidentals is uncertain.\\

33 & T. II 
  & \term{Bbc} starts coloration in \measure{33} like the other Chorus II parts in both sources.\\

40 & B. II, third note 
  & F is sharped in \term{CAN} only; certainly a cautionary accidental (that is, it means F natural), as \term{MEM} also determines. 
  This figure, with F natural, is the subject of the eight-voice fugue, and the Ti.~I-2 has F natural on the same beat.\\

45--46 & Ti. I-2 
  & \term{Bbc} places beginning and end of slur one note earlier.\\

47--59 & All voices, lyrics 
  & \term{Bbc} substitutes the following Eucharistic lyrics in place of those in \term{CAN} and in the other poetic sources: \quoted{y desde un pan divino/ un hombre soberano}.\\

57 & A. II 
  & Slur in \term{CAN} only.\\

58 & T. I, first note 
  & \term{CAN} (the only source of the T. I voice) has B\octave{3}, which must be an error. 
  Correction to A\octave{3} follows \term{MEM}.\\

77--80 & Ti. I-1 
  & \term{Bbc} has different conclusion shown in small staff above.\\

78 & Ti. I-2, first note 
  & In \term{Bbc}, the D\octave{5} is a semibreve, resulting in an extra minim and misalignment with the other voices. Edition follows \term{CAN}, which has a minim.\\

79 & Ti. 1-2, A. II 
  & MSS do not indicate F sharp until the final note in \measure{80}, but perhaps both of these Fs should be sharp as well.\\

81 & Ac. 
  & The accompaniment part is the same for each pair of coplas, so \term{CAN} only writes it out once.\\

81 & Lyrics 
  & \quoted{Las fugas del primer hombre formó} amended to \quoted{Las fugas que el primer hombre formó}, after consensus of poetry imprints.\\

92 & Ti. I-2 
  & \term{CAN} omits fermata.\\

94 & Lyrics 
  & \quoted{Qué mucho que} amended to \quoted{Qué mucho si} after imprints.\\

96 & Ti. I-1, last two notes 
  & The first of several places where the scribe confuses rhythmic values for the two-seminimim pickup gesture. 
  The scribe notes the pickup as two \term{corcheas}, which are too fast to sing in a reasonable tempo; the values do not add up to a full measure.
  Edition follows \term{MEM} in treating these pickup notes as semiminims.\\

105 & Ac., fourth note 
  & Note obscured by a hole in the paper in \term{CAN}, on the fold.\\

110 & All voices 
  & Cf. \measure{96}: scribe writes a semiminim rest and two corcheas.
  The breath mark preserves the sense of phrasing indicated by the rest.\\

116--117 & A. I, lyrics 
  & \quoted{Las disonancias} amended to singular in accord with the other voices and the imprints.\\

131--135 & Lyrics
  & \quoted{Desatento} in Ti. I-1 amended to \quoted{desentono}, to match the Ti. I-2 and 1651 Madrid poetry imprint.
  \quoted{Tan grande} amended to \quoted{tan vano} after the poetry imprint, preserving the poetic meter of \term{romance} in \term{-a -o}.\\

137 & All voices 
  & Same problem as \measure{96} and \measure{110}.\\

142--146 & Lyrics 
  & \quoted{Lo inmenso spacio} amended to \quoted{Lo inmenso a espacio} following the poetry imprints.\\

135--137 & Lyrics 
  & \quoted{Susteniendo} in the two Tiple parts amended to \quoted{sustenido}, in accord with A. I, T. I, and poetry imprints.\\

152 & Ti. I-2 
  & \term{CAN} omits the semibreve rest (compare \measures{98, 125}).\\

161 & Ti. I-1, last note 
  & The last note of all the even-numbered coplas is explicitly written as F\sh, but here the sharp is omitted.
  Either the sharp is assumed, or it is a deliberate F\na, to create a contrast with F\sh{} upon repeat of the estribillo.\\

\end{criticalnotes}
